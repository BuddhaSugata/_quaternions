\documentclass[11pt,a4paper]{article}

% ====== Packages (adapt / sync with the rev.2 paper as needed) ======
\usepackage[utf8]{inputenc}
\usepackage[T1]{fontenc}
\usepackage[english]{babel}

\usepackage{lmodern}
\usepackage{microtype}

\usepackage{geometry}
\geometry{margin=2.5cm}

\usepackage{amsmath,amssymb,amsfonts,mathtools}
\usepackage{bm}
\usepackage{physics}
\usepackage{siunitx}
\usepackage{tensor}
\usepackage{csquotes}

\usepackage{graphicx}
\usepackage{xcolor}
\usepackage{tikz}
\usepackage{tikz-cd}

\usepackage{hyperref}
\hypersetup{
  colorlinks=true,
  linkcolor=blue!40!black,
  citecolor=blue!40!black,
  urlcolor=blue!60!black
}

\usepackage{amsthm} % если ещё не подключен

\theoremstyle{definition}
\newtheorem{definition}{Definition}[section] % или без [section], как тебе надо

\usepackage[nameinlink,capitalize]{cleveref}

\usepackage{enumitem}

% ====== Common macros (adapt from rev.2 as needed) ======
% TODO: copy or sync macros from the rev.2 unimetry paper here.
\newcommand{\clight}{c}
\newcommand{\GNewton}{G}
\newcommand{\epszero}{\varepsilon_0}
\newcommand{\muzero}{\mu_0}

\newcommand{\R}{\mathbb{R}}
\newcommand{\Espace}{\mathcal{E}}

\newcommand{\qh}{\hat h}  % temporal unit
\newcommand{\qvec}[1]{\bm{#1}}

\title{Unimetry: A Quaternionic Gravito--Electromagnetic Formulation}
\author{Timur Abizgeldin%
  \thanks{Email: \texttt{foo@example.org}. TODO: update contact/affiliation.}}
\date{\today}

\begin{document}

\maketitle

\begin{abstract}
% TODO: Briefly describe the goal of the paper and the main idea:
% quaternionic representation of flows in unimetry, metrically dressed body quaternions,
% bilinear forms A,B,C, D-rotors, static limit (Newton + Coulomb),
% the role of C_{\text{phys}} = \clight\,\mathbf B and of \GNewton, \epszero, \muzero, \clight.
\end{abstract}

\tableofcontents

%==================== 1. Introduction ====================

\section{Introduction}
\section{Introduction}

\subsection{Context and motivation}

Unimetry is a proposed phase--geometric framework in which physical systems are
described in terms of stationary flows in an underlying Euclidean proto-space
\(\Espace\).
Rather than postulating space--time as a primary arena, unimetry treats the
observed space--time geometry, relativistic kinematics, and field interactions
as effective structures derived from the orientation and coupling of such flows.
A dimensionless scalar phase potential \(\Phi:\Espace\to\mathbb{R}\) and its
gradient define a normalized flow direction; the familiar Minkowski metric and
Lorentzian phenomena then appear as particular projections of this underlying
flow geometry.

In this sense, special relativity (SR) is not the endpoint, but the first
benchmark for the framework: unimetry aims at a unified phase-based description
of kinematics, gravity and gauge interactions, with SR recovered as a specific
limit of the general construction.
The present paper develops one important sector of this programme, namely a
quaternionic gravito--electromagnetic (GEM) formulation built on top of the
unimetrical flow picture.

At the classical level, gravito--electromagnetic analogies are well known:
in the weak-field, slow-motion limit of general relativity, the Einstein
equations can be cast into a Maxwell-like form, and moving masses generate a
``gravitomagnetic'' field.
Quaternions and related algebras have also long been used to encode rotations
and the Maxwell equations in a compact way.
What unimetry adds to this landscape is a concrete phase-geometric
interpretation: a single quaternionic object encodes both the temporal and
spatial parts of a flow, and bilinear forms of such objects naturally split into
scalar, symmetric vector, and axial (vorticity-like) channels.
This suggests that gravity and electromagnetism might be viewed as different
faces of the same bilinear structure acting on suitably dressed flow
quaternions.

Our goal here is to make this statement precise.
We construct a quaternionic GEM formalism in which gravitational and
electromagnetic interactions originate from the \emph{same} bilinear machinery
applied to metrically dressed ``body quaternions''.
In particular, we show that Newton and Coulomb potentials arise as two branches
of a single scalar form, while the magnetic and gravitomagnetic sectors are
associated with a vortical bilinear form whose physical calibration reveals a
natural role for the constants \(\epszero\), \(\muzero\), \(\GNewton\) and
\(\clight\).
The resulting description remains Euclidean at the level of the proto-space,
yet reproduces relativistic kinematics and GEM fields in the observable
three-space.

\subsection{Relation to the base unimetry paper}

This work is a direct sequel to the base unimetry paper,
\emph{``Unimetry: A Phase-Space Reformulation of Special Relativity''}
(henceforth ``Paper~I'').
Paper~I develops the core phase/flow structure: the phase potential \(\Phi\),
the phase 1-form \(\alpha = d\Phi\), the normalized flow
\(\widehat{\boldsymbol\chi}\), and the calibration
\(\boldsymbol\chi = \clight\,\widehat{\boldsymbol\chi}\), together with the
derivation of the Minkowski interval and standard SR effects from a Euclidean
proto-space.
It also introduces the unimetrical D-rotation, which encodes Lorentz boosts as
Euclidean rotations in a suitable plane of \(\Espace\).

From the unimetry viewpoint, however, these SR results are only the first 
consistency test of a broader phase-based paradigm.
The present paper assumes familiarity with the conceptual setting of Paper~I,
but is written to be as self-contained as reasonably possible.
We briefly recall the key definitions of the phase proto-space, the flow vector,
and the two calibrations of the flow that lead to kinematic and energetic
interpretations.
All constructions that are essential for the GEM sector are reproduced or
adapted here; more detailed discussions of SR and cosmological applications
remain in Paper~I and are only referenced when needed.

\subsection{Main results}

The main technical contributions of this paper can be summarized as follows.

\begin{itemize}[leftmargin=*]
  \item We introduce \emph{metrically dressed body quaternions}
  \(\widetilde{\mathsf q}_i = L_{E,i}\,\qh + L_{G,i}\,\widehat{\bm n}_i\),
  whose components have the dimension of length.
  The ``electric'' and ``gravitational'' lengths
  \[
    L_{E,i} = \sqrt{\frac{\GNewton}{4\pi \epszero \clight^4}}\,Q_i,
    \qquad
    L_{G,i} = \frac{\GNewton}{\clight^2}\,m_i
  \]
  encode the charge \(Q_i\) and mass \(m_i\) of the body in a unified geometric
  fashion.
  The unit vector \(\widehat{\bm n}_i\) represents the spatial flow direction
  associated with the body.

  \item We show that the scalar bilinear form
  \[
    A(\widetilde{\mathsf q}_1,\widetilde{\mathsf q}_2)
    = L_{E,1}L_{E,2} - \bm S_1\cdot\bm S_2
  \]
  (with \(\bm S_i = L_{G,i}\widehat{\bm n}_i\)) yields, after a single global
  calibration by \(\clight^4/\GNewton\) and a geometric \(1/r\) factor, the
  combined Newton--Coulomb potential:
  \[
    U(r) = \frac{\clight^4}{\GNewton}\,\frac{A}{r}
    = \frac{1}{4\pi\epszero}\,\frac{Q_1 Q_2}{r}
      - \GNewton\,\frac{m_1 m_2}{r}.
  \]
  Thus gravity and electrostatics arise as two channels of a single invariant
  scalar form.

  \item We identify two vector-valued bilinear forms,
  \(\bm B(\widetilde{\mathsf q}_1,\widetilde{\mathsf q}_2)\) and
  \(\bm C(\widetilde{\mathsf q}_1,\widetilde{\mathsf q}_2)\), corresponding to
  the symmetric and axial parts of the quaternion product.
  In the dressed setting these naturally describe current-like and vortical
  channels.
  In particular, the vortical form \(\bm C\) reproduces the geometry of
  magnetic and gravitomagnetic fields generated by moving charges and masses.

  \item We construct a quaternionic GEM field
  \(\mathcal F_{\rm GEM}(\bm x)\) over the observable three-space by combining
  dressed source quaternions with purely imaginary distance quaternions.
  Its scalar channel reproduces the gravitational and electrostatic potentials,
  while its vortical channel yields a physically natural ``phase-vortical''
  field \(C_{\text{phys}}\) with the same dimension as \(\bm E\).
  The standard magnetic field \(\bm B\) in SI units then appears as
  \[
    \bm B = \frac{1}{\clight}\,C_{\text{phys}},
  \]
  so that the familiar \(\muzero\) and \(\epszero\) can be interpreted in terms
  of linear and areal stiffness of the vacuum, combined into an effective
  volumetric stiffness proportional to \(1/(\epszero \clight^3)\).

  \item We analyze the action of unimetrical D-rotations and ordinary spatial
  rotors on dressed quaternions.
  Pure spatial rotations act in the usual way on the vector channels and leave
  the scalar form \(A\) invariant, while D-rotations mix the scalar channel and
  the longitudinal component of \(\bm B\) in a two-dimensional
  ``energy--current'' plane.
  This provides a quaternionic encoding of relativistic kinematics in the GEM
  setting, with Lorentz-consistent transformation properties of the fields.

  \item Finally, we outline a Hamiltonian and Lagrangian formulation of the
  quaternionic GEM theory in terms of the self-form \(A\) and the norm-squares
  of \(\bm B\) and \(\bm C\), and discuss how the standard Maxwell Lagrangian
  and linearized GEM equations arise in appropriate limits.
\end{itemize}

\subsection{Structure of the paper}

The paper is organized as follows.
In \cref{sec:quat-algebra} we recall the basic quaternion algebra and introduce
the bilinear forms \(A\), \(\bm B\), and \(\bm C\) that arise from the
quaternion product, together with their matrix representation and geometric
interpretation.
\Cref{sec:phase-protospace} provides a brief overview of the unimetrical phase
proto-space, the phase potential, the flow vector, and the two calibrations of
the flow that lead to kinematic and energetic interpretations.

In \cref{sec:dressed-quats} we introduce metrically dressed body quaternions and
define the electric and gravitational lengths \(L_E\) and \(L_G\).
\Cref{sec:bilinear-static-gem} shows how the scalar form \(A\) for dressed
quaternions reproduces the static Newton and Coulomb potentials.
In \cref{sec:quat-gem-field} we construct a quaternionic GEM field over the
observable three-space and identify the scalar and vortical channels with
gravitational, electric, and magnetic sectors.

\Cref{sec:drotors-dynamics} analyzes the action of spatial rotors and D-rotors
on dressed quaternions and on the GEM field, clarifying the relativistic
transformation properties of the scalar, current-like, and vortical channels.
\Cref{sec:calibration-EB-Cphys} is devoted to the calibration of \(\bm E\) and
\(\bm B\), to the definition of the phase-vortical field \(C_{\text{phys}}\),
and to the interpretation of \(\epszero\), \(\muzero\), and \(\clight\) in
terms of vacuum stiffness.

In \cref{sec:hamiltonians-lagrangians} we outline Hamiltonian and Lagrangian
formulations of quaternionic GEM, and in
\cref{sec:maxwell-gem-comparison} we compare the resulting equations with the
standard Maxwell and linearized GEM formalisms.
Finally, \cref{sec:discussion-conclusions} discusses limitations and open
questions, and sketches possible extensions towards non-Abelian interactions
and cosmological applications.

%==================== 2. Quaternion algebra ====================

\section{Quaternion algebra and bilinear forms}
\label{sec:quat-algebra}

\subsection{Basic notation and conventions}

We denote by \(\mathbb{H}\) the real quaternion algebra, viewed as a
four-dimensional real vector space
\[
  \mathbb{H} \simeq \mathbb{R}^4(\qh,\hat{\imath},\hat{\jmath},\hat{k}),
\]
where \(\qh\) is a distinguished real (``temporal'') basis element and
\(\hat{\imath},\hat{\jmath},\hat{k}\) are purely imaginary basis elements.
A general quaternion is written as
\[
  q = T\,\qh + \bm S,
  \qquad
  \bm S = S^1 \hat{\imath} + S^2 \hat{\jmath} + S^3 \hat{k},
\]
with \(T\in\mathbb{R}\) and \(\bm S \in \Im\mathbb{H} \simeq \mathbb{R}^3\).
We call \(T\) the scalar part and \(\bm S\) the vector part of \(q\).

The imaginary basis satisfies the usual quaternion relations
\[
  \hat{\imath}^2 = \hat{\jmath}^2 = \hat{k}^2
  = \hat{\imath}\hat{\jmath}\hat{k} = -\qh,
\]
and the mixed products reproduce the three-dimensional cross product structure,
e.g.\ \(\hat{\imath}\hat{\jmath} = \hat{k}\), \(\hat{\jmath}\hat{k} = \hat{\imath}\),
\(\hat{k}\hat{\imath} = \hat{\jmath}\), with antisymmetry under exchange of
factors.
We use the standard identification \(\Im\mathbb{H} \simeq \mathbb{R}^3\) with
its Euclidean inner product \(\bm S_1\cdot\bm S_2\) and cross product
\(\bm S_1\times\bm S_2\).

Quaternionic conjugation is defined by
\[
  \overline{q} := T\,\qh - \bm S,
\]
and the norm is \(\|q\|^2 = q\,\overline{q} = \overline{q}\,q
= T^2 + \|\bm S\|^2\).
The real and imaginary parts are
\[
  \mathrm{Re}(q) = T,\qquad \mathrm{Im}(q) = \bm S.
\]

In what follows we will systematically write
\[
  q = (T,\bm S)
\]
when it is convenient to emphasize the split into scalar and vector parts.
We will also denote vectors in \(\Im\mathbb{H}\) in boldface (e.g.\ \(\bm S\))
and reserve \(\qh\) for the distinguished scalar basis element.
This choice aligns with the unimetrical interpretation, where the scalar part
will later be associated with the temporal channel of a flow, and the vector
part with its spatial channel.

\subsection{Quaternion product and decomposition into A, B, C forms}

Let \(\{e_\mu\}_{\mu=0}^3 = \{\qh,\hat{\imath},\hat{\jmath},\hat{k}\}\) be the
standard quaternion basis, with
\[
  \qh^2 = \qh,\qquad
  \hat{\imath}^2 = \hat{\jmath}^2 = \hat{k}^2 = -\qh,
\]
and the usual multiplication table
\(\hat{\imath}\hat{\jmath}=\hat{k}\), \(\hat{\jmath}\hat{k}=\hat{\imath}\),
\(\hat{k}\hat{\imath}=\hat{\jmath}\), with antisymmetry under exchange of
factors.
Any quaternion can be written as
\[
  q = x^\mu e_\mu
  = x^0 \qh + x^1 \hat{\imath} + x^2 \hat{\jmath} + x^3 \hat{k},
\]
with real components \(x^\mu \in \mathbb{R}\).

Let
\[
  \mathbf{q}_1 = x^\mu e_\mu, \qquad
  \mathbf{q}_2 = y^\nu e_\nu,
  \qquad
  \mathbf{q}_1,\mathbf{q}_2 \in \mathbb{H}.
\]
Then their quaternion product can be expanded as
\begin{align}
  \mathbf{q}_1 \circ \mathbf{q}_2
  &= (x^0 \qh + x^1 \hat{\imath} + x^2 \hat{\jmath} + x^3 \hat{k})
     (y^0 \qh + y^1 \hat{\imath} + y^2 \hat{\jmath} + y^3 \hat{k}) \nonumber\\
  &= ( \mathbf{q}_1 * \mathbf{q}_2 )
   + ( \mathbf{q}_1 \diamond \mathbf{q}_2 )
   + ( \mathbf{q}_1 \times \mathbf{q}_2 ),
\end{align}
where we have grouped terms according to three bilinear structures:
\begin{itemize}[leftmargin=*]
  \item a scalar (``Minkowski-like'') channel \(\mathbf{q}_1 * \mathbf{q}_2\),
  \item a symmetric vector channel \(\mathbf{q}_1 \diamond \mathbf{q}_2\),
  \item an axial/vortical vector channel \(\mathbf{q}_1 \times \mathbf{q}_2\).
\end{itemize}
Explicitly,
\begin{align}
  \mathbf{q}_1 \circ \mathbf{q}_2
  &= (x^0 y^0 \qh^2 + x^1 y^1 \hat{\imath}^2 + x^2 y^2 \hat{\jmath}^2
      + x^3 y^3 \hat{k}^2) \nonumber\\
  &\quad
   + (x^0 y^1 \qh \hat{\imath} + x^0 y^2 \qh \hat{\jmath}
      + x^0 y^3 \qh \hat{k}
      + x^1 y^0 \hat{\imath} \qh + x^2 y^0 \hat{\jmath} \qh
      + x^3 y^0 \hat{k} \qh) \nonumber\\
  &\quad
   + (x^1 y^2 \hat{\imath} \hat{\jmath}
      + x^1 y^3 \hat{\imath} \hat{k}
      + x^2 y^1 \hat{\jmath} \hat{\imath}
      + x^2 y^3 \hat{\jmath} \hat{k}
      + x^3 y^1 \hat{k} \hat{\imath}
      + x^3 y^2 \hat{k} \hat{\jmath}).
  \label{eq:q-product-expanded}
\end{align}
Using the multiplication rules, this can be written in a compact tensor-like form
\[
  \mathbf{q}_1 \circ \mathbf{q}_2
  = \sum_{\mu,\nu=0}^3
    \bigl(
      \color{red}{A_{\mu\nu}}\,\qh
      + \color{green}{B_{\mu\nu}}
      + \color{blue}{C_{\mu\nu}}
    \bigr)\,x^\mu y^\nu,
\]
where the three \(4\times 4\) coefficient matrices are
\begin{align}
  \color{red}{A_{\mu\nu}}
  &= \begin{pmatrix}
       1 & 0 & 0 & 0 \\
       0 & -1 & 0 & 0 \\
       0 & 0 & -1 & 0 \\
       0 & 0 & 0 & -1
     \end{pmatrix},
  \label{eq:A-matrix}\\[0.5em]
  \color{green}{B_{\mu\nu}}
  &= \begin{pmatrix}
       0 & \hat{\imath} & \hat{\jmath} & \hat{k} \\
       \hat{\imath} & 0 & 0 & 0 \\
       \hat{\jmath} & 0 & 0 & 0 \\
       \hat{k} & 0 & 0 & 0
     \end{pmatrix},
  \label{eq:B-matrix}\\[0.5em]
  \color{blue}{C_{\mu\nu}}
  &= \begin{pmatrix}
       0 & 0 & 0 & 0 \\
       0 & 0 & \hat{k} & -\hat{\jmath} \\
       0 & -\hat{k} & 0 & \hat{\imath} \\
       0 & \hat{\jmath} & -\hat{\imath} & 0
     \end{pmatrix}.
  \label{eq:C-matrix}
\end{align}

Equivalently, if we collect the scalar and vector parts explicitly, we recover
the more familiar invariant decomposition
\begin{align}
  \mathbf{q}_1 \circ \mathbf{q}_2
  &= \Bigl(
       \color{red}{x^0 y^0 - x^1 y^1 - x^2 y^2 - x^3 y^3}
     \Bigr)\,\qh
   \nonumber\\
  &\quad
   + \color{green}{\bigl(x^0 \bm y + y^0 \bm x\bigr)}
   + \color{blue}{\bigl(\bm x \times \bm y\bigr)},
  \label{eq:q-product-ABC-coords}
\end{align}
where \(\bm x = (x^1,x^2,x^3)\), \(\bm y = (y^1,y^2,y^3)\),
and the colour coding is as in \cref{eq:A-matrix,eq:B-matrix,eq:C-matrix}:
\begin{itemize}[leftmargin=*]
  \item \(\color{red}{A_{\mu\nu}}\) is the Minkowski-like bilinear form
  \(\mathrm{diag}(1,-1,-1,-1)\) acting on the coordinate vectors \(x^\mu,y^\nu\);
  \item \(\color{green}{B_{\mu\nu}}\) collects the symmetric mixed products
  between the scalar and vector components;
  \item \(\color{blue}{C_{\mu\nu}}\) collects the antisymmetric products between
  the spatial components, encoding the cross product \(\bm x \times \bm y\).
\end{itemize}

This tensor-like presentation makes explicit that the quaternion product can be
viewed as the contraction of a rank-\((0,2)\) object with two four-vectors:
\[
  \mathbf{q}_1 \circ \mathbf{q}_2
  = \bigl(\color{red}{A} + \color{green}{B} + \color{blue}{C}\bigr)_{\mu\nu}
    x^\mu y^\nu,
\]
with a scalar (red), a symmetric vector-valued (green), and an antisymmetric
vector-valued (blue) block.
In the unimetrical setting we will reinterpret the scalar block
\(\color{red}{A_{\mu\nu}}\) as the basic energy-like invariant, the symmetric
block \(\color{green}{B_{\mu\nu}}\) as a current-like coupling, and the
antisymmetric block \(\color{blue}{C_{\mu\nu}}\) as the vortical channel that
underlies magnetic and gravitomagnetic fields.

\subsection{Geometric interpretation of A, B, C}

The decomposition \eqref{eq:q-product-ABC} plays a central role in what
follows:
\begin{itemize}[leftmargin=*]
  \item The scalar form \(\color{red}{A}\) is a symmetric bilinear form of
  signature \((+,-,-,-)\) when restricted to suitable unimetrical subspaces and
  will be used as the basic energy-like invariant.
  In the dressed setting it will produce both Newtonian and Coulomb potentials
  from a single scalar channel.

  \item The symmetric vector form \(\color{green}{\bm B}\) behaves like a
  current-like quantity: it is linear in each argument and symmetric under
  exchange.
  When applied to dressed body quaternions, it will encode the coupling between
  scalar (temporal) and vector (spatial) parts, and will be directly related to
  mass and charge currents in the GEM field.

  \item The axial vector form \(\color{blue}{\bm C}\) is purely imaginary and
  antisymmetric.
  It is built from the cross product of the spatial parts and therefore encodes
  vorticity-like structures.
  In the GEM interpretation it will be responsible for the magnetic and
  gravitomagnetic sectors.

\end{itemize}

Thus the elementary quaternion product already contains, in a rigid algebraic
way, the three channels that we will later reinterpret as (i) an energy-like
scalar invariant, (ii) a current-like symmetric vector channel, and
(iii) a vortical (axial) channel.
In the next section we recall how unimetry associates physical flows and
space--time structure to quaternions, so that the forms
\(\color{red}{A}\), \(\color{green}{\bm B}\), and \(\color{blue}{\bm C}\) can
be given a gravito--electromagnetic meaning.


%==================== 3. Phase proto-space in unimetry ====================

\section{Phase proto-space and flow: brief unimetry overview}

\subsection{Proto-space, phase potential and 1-form}
% TODO: Introduce (\Espace,\langle\cdot,\cdot\rangle), the phase potential \Phi,
% the phase 1-form \alpha = d\Phi, and the gradient \nabla\Phi.

\subsection{Flow vector and normalization}
% TODO: Introduce \widehat{\boldsymbol\chi} = \nabla\Phi / \|\nabla\Phi\|,
% then \boldsymbol\chi = \clight\,\widehat{\boldsymbol\chi}, with \|\boldsymbol\chi\| = \clight.

\subsection{Interval, proper time and correspondence with SR}
% TODO: Briefly indicate how the flow \boldsymbol\chi gives rise to the interval,
% proper time, and how this corresponds to the Minkowski metric in SR.

\subsection{Notation table}
% TODO: Insert a compact table of main symbols (\Phi, \alpha, \boldsymbol\chi, \zeta, etc.),
% adapted from Paper I (rev.2).

%==================== 4. Calibrations of the flow and energy ====================

\section{Flow calibrations and energy-like functionals}

\subsection{Phase (kinematic) calibration}
% TODO: Describe calibration in terms of phase/velocity,
% the relation between \zeta and \beta = v/\clight,
% and how Lorentz boosts appear geometrically in this picture.

\subsection{Proper-time (energetic) calibration}
% TODO: Introduce calibration with respect to proper flow (proper time or phase parameter),
% and relate it to an energy-like functional.

\subsection{Energy of a flow and the self-form A(q,q)}
% TODO: Show that A(q,q) naturally plays the role of a free Hamiltonian
% / energy invariant for a single flow, and relate this to proper frequency
% and to an E \sim \kappa \clight^3 scaling in unimetry.

%==================== 5. Metrically dressed body quaternions ====================

\section{Metrically dressed body quaternions}

\subsection{Free unit imaginary vector for isotropic Newtonian bodies}
\label{subsec:free-unit-vector}

In the metrically dressed setting we write the spatial part of a body quaternion
as
\[
  \bm S_i = L_{G,i}\,\widehat{\bm n}_i,
\]
where \(\widehat{\bm n}_i\) is a unit spatial direction associated with body \(i\).
For charged bodies it is natural to interpret \(\widehat{\bm n}_i\) as an
intrinsic flow direction (e.g.\ an orientation of the underlying streamlet
structure), which will in general produce non-trivial contributions in both
the symmetric vector form \(\bm B\) and the vortical form \(\bm C\).

For purely Newtonian, isotropic mass distributions, however, we may and should
distinguish between an intrinsic direction and the \emph{interaction} direction.
To reflect this, we introduce the notion of a \emph{free unit imaginary vector}
for the gravitational channel.

\begin{definition}[Free unit imaginary vector]
  A free unit imaginary vector is a unit spatial quaternion
  \(\widehat{\bm u}\in \Im\mathbb H\) whose orientation is not fixed by the
  internal structure of the body, but is freely assigned at the level of the
  interaction.
  For an isotropic Newtonian body \(i\) we take the gravitational spatial part
  of its dressed quaternion to be
  \[
    \bm S^{(G)}_i(\bm x)
    := L_{G,i}\,\widehat{\bm u}_i(\bm x),
  \]
  where, for a field point \(\bm x\), the free unit vector is chosen to be
  aligned with the radius vector from the body to that point,
  \[
    \widehat{\bm u}_i(\bm x) := \widehat{\bm r}_i(\bm x)
    = \frac{\bm x - \bm x_i}{\|\bm x - \bm x_i\|}.
  \]
\end{definition}

In other words, for an isotropic Newtonian source the gravitational channel of
the dressed quaternion is always taken to be \emph{radial} with respect to the
field point, and carries no intrinsic ``spin'' information. This has an
important structural consequence for the bilinear forms.

Consider two isotropic masses \(m_1,m_2\) at positions \(\bm x_1,\bm x_2\), and
evaluate their gravitational spatial parts at a common field point \(\bm x\).
By construction,
\[
  \bm S^{(G)}_1(\bm x) \parallel \widehat{\bm r}_1(\bm x),
  \qquad
  \bm S^{(G)}_2(\bm x) \parallel \widehat{\bm r}_2(\bm x).
\]
In the static two-body configuration the interaction is along the line joining
the bodies, so that effectively
\[
  \bm S^{(G)}_1 \parallel \bm S^{(G)}_2,
\]
and therefore their contribution to the vortical form
\[
  \bm C(\widetilde{\mathsf q}_1,\widetilde{\mathsf q}_2)
  = \bm S_1 \times \bm S_2
\]
vanishes in the purely gravitational, isotropic limit:
\[
  \bm C^{(G)} = \bm S^{(G)}_1 \times \bm S^{(G)}_2 = \bm 0.
\]

Thus, by assigning the gravitational spatial part of an isotropic body to a
free unit imaginary vector that is always chosen radial, we ensure that:
\begin{itemize}[leftmargin=*]
  \item the gravitational interaction of isotropic masses is purely scalar and
  radial, as in Newtonian gravity;
  \item there is \emph{no} spurious contribution of the gravitational channel
  to the vortical form \(\bm C\) in the static limit;
  \item all non-trivial vortical contributions in \(\bm C\) are genuinely
  associated with anisotropy and/or motion (currents), i.e.\ with the EM and
  gravitomagnetic sectors rather than with static isotropic gravity.
\end{itemize}

In contrast, for charged bodies we will keep an intrinsic unit direction
\(\widehat{\bm n}_i\) in the electric channel, which can contribute to both the
symmetric form \(\bm B\) and the vortical form \(\bm C\).
This separation between a free gravitational direction and an intrinsic
electromagnetic direction will be important when we analyze the GEM field and
its magnetic and gravitomagnetic components in \cref{sec:quat-gem-field,sec:drotors-dynamics}.


\subsection{Dimensionless flow quaternion and dressed body quaternion}
% TODO: Distinguish between a dimensionless flow quaternion \widehat{\mathsf q}
% and a metrically dressed body quaternion \widetilde{\mathsf q} with dimension of length.

\subsection{Lengths L\_E and L\_G: definitions and dimensional analysis}
% TODO: Define
% L_{E,i} = \sqrt{\GNewton/(4\pi\epszero \clight^4)}\,Q_i,
% L_{G,i} = (\GNewton/\clight^2)\,m_i,
% and discuss dimensions and physical meaning (``electric length'' and gravitational radius).

\subsection{Self-form of a dressed quaternion and rest energy}
% TODO: Consider A(\widetilde{\mathsf q}_i,\widetilde{\mathsf q}_i)
% and relate it to rest energy / the mass channel.

%==================== 6. Bilinear forms and static GEM ====================

\section{Bilinear forms for dressed bodies and static GEM}

\subsection{Forms A, B, C for dressed body quaternions}
% TODO: Write A,B,C explicitly for \widetilde{\mathsf q}_1,\widetilde{\mathsf q}_2
% and discuss the interpretation of each component in the dressed context.

\subsection{Static interaction energy of two bodies}
% TODO: Derive A_0 = A(\widetilde{\mathsf q}_1,\widetilde{\mathsf q}_2),
% define U(r) = (\clight^4/\GNewton)\,A_0 / r,
% and show that this reproduces the Coulomb + Newton potentials.

\subsection{Multi-body field and linear regime}
% TODO: Discuss superposition of many sources and the linear classical limit.

%==================== 7. Quaternionic GEM field in R^3 ====================

\section{Quaternionic GEM field in the observable three-space}

\subsection{Distance as a purely imaginary quaternion}
% TODO: Introduce R_j(\bm x) = \bm r_j and R_j^{-1} = -\widehat{\bm r}_j / r_j.

\subsection{Potential-like object \texorpdfstring{$\mathcal U_{\rm GEM}$}{U\_GEM}}
% TODO: Define \mathcal U_{\rm GEM}(\bm x) = \sum_j R_j^{-1}\,\widetilde{\mathsf q}_j.

\subsection{Field \texorpdfstring{$\mathcal F_{\rm GEM}$}{F\_GEM} and extraction of potentials}
% TODO: Define \mathcal F_{\rm GEM}(\bm x) = \sum_j R_j^{-1}\,\widetilde{\mathsf q}_j\,R_j^{-1},
% and explain how its A-channel yields the gravitational and electrostatic potentials.

%==================== 8. D-rotors and dynamics ====================

\section{D-rotors and relativistic dynamics of GEM}

\subsection{Definition of the D-rotation and its action on a flow}
% TODO: Introduce d(\zeta) = \cos(\zeta/2) + \widehat{\bm u}\sin(\zeta/2),
% and the action q' = d q d,
% with explicit formulas for T', S_u', \bm S_\perp'.

\subsection{Spatial rotors and invariance of A}
% TODO: Show that a purely spatial rotor R
% leaves A(q_1,q_2) invariant,
% while \bm B and \bm C transform as ordinary 3-vectors.

\subsection{Mixing of A and B channels under a D-rotation}
% TODO: Show how the pair (U,B_u) = (T_1T_2 - S_{u,1}S_{u,2}, B_u)
% rotates by an angle 2\zeta in the (energy–current) plane,
% and interpret this as the dynamical mixing of scalar and current-like channels.

\subsection{Example configurations}
% TODO: Discuss:
% - a purely gravitational channel (L_E = 0),
% - a purely electric channel (L_G \to 0),
% - transverse motion of a charge and the emergence of a vortical EM part.

%==================== 9. Calibration of E, B and C_phys ====================

\section{Calibration of fields \texorpdfstring{$\bm E$}{E}, \texorpdfstring{$\bm B$}{B}
and the phase-vortical object \texorpdfstring{$C_{\text{phys}}$}{Cphys}}

\subsection{Electric field from the A-channel}
% TODO: Derive \bm E(\bm r) = (1/(4\pi\epszero))\,Q\,\widehat{\bm r} / r^2
% from the scalar A-channel in the static limit.

\subsection{The vortical form C and the geometry of \texorpdfstring{$C_{\text{geom}}$}{Cgeom}}
% TODO: Show that C_{\text{geom}} \sim (Q/r^2)\,\beta\,(\widehat{\bm v} \times \widehat{\bm r})
% for a moving charge.

\subsection{Definition of \texorpdfstring{$C_{\text{phys}}$}{Cphys} and relation to \texorpdfstring{$\bm B$}{B}}
% TODO: Define C_{\text{phys}} := (1/(4\pi\epszero))\,C_{\text{geom}},
% note that its dimension matches that of \bm E,
% and write \bm B = C_{\text{phys}} / \clight in SI units.

\subsection{Interpretation of \texorpdfstring{$\epszero$}{eps0}, \texorpdfstring{$\muzero$}{mu0}
and volumetric stiffness of the vacuum}
% TODO: Discuss \muzero = 1/(\epszero \clight^2)
% and \muzero/\clight = 1/(\epszero \clight^3) as an effective ``volumetric stiffness''
% of the vortical channel of the vacuum.

%==================== 10. Hamiltonians and Lagrangians ====================

\section{Hamiltonians and Lagrangians of quaternionic GEM}

\subsection{Free-body Hamiltonian}
% TODO: Introduce H_{\rm free} = \kappa A(\widetilde{\mathsf q}_i,\widetilde{\mathsf q}_i),
% and relate it to the rest energy and proper frequency of a body.

\subsection{Interaction Hamiltonian}
% TODO: Introduce H_{\rm int} = (\clight^4/\GNewton)\,
% A(\widetilde{\mathsf q}_1,\widetilde{\mathsf q}_2)/r
% and decompose it into gravitational and EM contributions.

\subsection{Field Lagrangian in terms of A, B, C}
% TODO: Sketch a field Lagrangian in terms of A(Q,Q), \|B(Q,Q)\|^2, \|C(Q,Q)\|^2,
% and relate it to the standard EM Lagrangian and linearized GEM.

%==================== 11. Relation to Maxwell and GEM ====================

\section{Relation to Maxwell's equations and linearized GEM}

\subsection{Static equations and Poisson-type equations}
% TODO: Show how Poisson's equation for \phi_{\rm EM}
% and its gravitational analogue arise from the quaternionic field.

\subsection{Quasi-stationary regime and continuity equations}
% TODO: Show how, in the quasi-stationary regime,
% one recovers \nabla \cdot \bm E, \nabla \times \bm B,
% and continuity equations for charge and mass currents.

\subsection{Comparison with the classical GEM formalism}
% TODO: Discuss the limits L_E = 0 (pure gravity),
% L_G = 0 (pure EM), and correspondence with standard GEM formulations.

%==================== 12. Discussion and conclusions ====================

\section{Discussion and conclusions}

\subsection{Summary of results}
% TODO: Summarize the main achievements and emphasize
% the unified quaternionic language for gravity and EM.

\subsection{Limitations and open questions}
% TODO: Discuss assumptions (classical limit, linearity, etc.)
% and highlight open problems.

\subsection{Perspectives}
% TODO: Sketch possible extensions:
% inclusion of weak/strong interactions,
% quantization of the quaternionic field,
% links to cosmology and possible evolution of \GNewton, etc.

%==================== Appendices ====================

\appendix

\section{Quaternion algebra and matrix representation (details)}
% TODO: Provide more detailed algebraic and matrix derivations.

\section{Extended notation table}
% TODO: Full table of symbols used in the paper.

\section{Dimensional analysis and numerical estimates}
% TODO: Work out explicit examples for the electron, proton, etc.

\section{D-rotors, Lorentz transformations and Wigner--Thomas rotation}
% TODO: Show in detail how the D-rotor formalism reproduces Lorentz boosts
% and the Wigner--Thomas rotation.

\end{document}
