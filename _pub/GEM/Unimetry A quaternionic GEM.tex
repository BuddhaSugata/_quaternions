\documentclass[11pt,a4paper]{article}

% ====== Packages (adapt / sync with the rev.2 paper as needed) ======
\usepackage[utf8]{inputenc}
\usepackage[T1]{fontenc}
\usepackage[english]{babel}

\usepackage{lmodern}
\usepackage{microtype}

\usepackage{geometry}
\geometry{margin=2.5cm}

\usepackage{amsmath,amssymb,amsfonts,mathtools}
\usepackage{bm}
\usepackage{physics}
\usepackage{siunitx}
\usepackage{tensor}
\usepackage{csquotes}

\usepackage{graphicx}
\usepackage{xcolor}
\usepackage{tikz}
\usepackage{tikz-cd}

\usepackage{hyperref}
\hypersetup{
  colorlinks=true,
  linkcolor=blue!40!black,
  citecolor=blue!40!black,
  urlcolor=blue!60!black
}

\usepackage{amsthm} % если ещё не подключен

\theoremstyle{definition}
\newtheorem{definition}{Definition}[section] % или без [section], как тебе надо

\usepackage[nameinlink,capitalize]{cleveref}

\usepackage{enumitem}

% ====== Common macros (adapt from rev.2 as needed) ======
% TODO: copy or sync macros from the rev.2 unimetry paper here.
\newcommand{\clight}{c}
\newcommand{\GNewton}{G}
\newcommand{\epszero}{\varepsilon_0}
\newcommand{\muzero}{\mu_0}

\newcommand{\R}{\mathbb{R}}
\newcommand{\Espace}{\mathcal{E}}

\newcommand{\qh}{\hat h}  % temporal unit
\newcommand{\qvec}[1]{\bm{#1}}

\newcommand{\uhat}{\hat{\boldsymbol{\ell}}}

\title{Unimetry: A Quaternionic Gravito--Electromagnetic Formulation}
\author{Timur Abizgeldin%
  \thanks{Email: \texttt{foo@example.org}. TODO: update contact/affiliation.}}
\date{\today}

\begin{document}

\maketitle

\begin{abstract}
% TODO: Briefly describe the goal of the paper and the main idea:
% quaternionic representation of flows in unimetry, metrically dressed body quaternions,
% bilinear forms A,B,C, D-rotors, static limit (Newton + Coulomb),
% the role of C_{\text{phys}} = \clight\,\mathbf B and of \GNewton, \epszero, \muzero, \clight.
\end{abstract}

\tableofcontents

%==================== 1. Introduction ====================

\section{Introduction}
\section{Introduction}

\subsection{Context and motivation}

Unimetry is a proposed phase--geometric framework in which physical systems are
described in terms of stationary flows in an underlying Euclidean proto-space
\(\Espace\).
Rather than postulating space--time as a primary arena, unimetry treats the
observed space--time geometry, relativistic kinematics, and field interactions
as effective structures derived from the orientation and coupling of such flows.
A dimensionless scalar phase potential \(\Phi:\Espace\to\mathbb{R}\) and its
gradient define a normalized flow direction; the familiar Minkowski metric and
Lorentzian phenomena then appear as particular projections of this underlying
flow geometry.

In this sense, special relativity (SR) is not the endpoint, but the first
benchmark for the framework: unimetry aims at a unified phase-based description
of kinematics, gravity and gauge interactions, with SR recovered as a specific
limit of the general construction.
The present paper develops one important sector of this programme, namely a
quaternionic gravito--electromagnetic (GEM) formulation built on top of the
unimetrical flow picture.

At the classical level, gravito--electromagnetic analogies are well known:
in the weak-field, slow-motion limit of general relativity, the Einstein
equations can be cast into a Maxwell-like form, and moving masses generate a
``gravitomagnetic'' field.
Quaternions and related algebras have also long been used to encode rotations
and the Maxwell equations in a compact way.
What unimetry adds to this landscape is a concrete phase-geometric
interpretation: a single quaternionic object encodes both the temporal and
spatial parts of a flow, and bilinear forms of such objects naturally split into
scalar, symmetric vector, and axial (vorticity-like) channels.
This suggests that gravity and electromagnetism might be viewed as different
faces of the same bilinear structure acting on suitably dressed flow
quaternions.

Our goal here is to make this statement precise.
We construct a quaternionic GEM formalism in which gravitational and
electromagnetic interactions originate from the \emph{same} bilinear machinery
applied to metrically dressed ``body quaternions''.
In particular, we show that Newton and Coulomb potentials arise as two branches
of a single scalar form, while the magnetic and gravitomagnetic sectors are
associated with a vortical bilinear form whose physical calibration reveals a
natural role for the constants \(\epszero\), \(\muzero\), \(\GNewton\) and
\(\clight\).
The resulting description remains Euclidean at the level of the proto-space,
yet reproduces relativistic kinematics and GEM fields in the observable
three-space.

\subsection{Relation to the base unimetry paper}

This work is a direct sequel to the base unimetry paper,
\emph{``Unimetry: A Phase-Space Reformulation of Special Relativity''}
(henceforth ``Paper~I'').
Paper~I develops the core phase/flow structure: the phase potential \(\Phi\),
the phase 1-form \(\alpha = d\Phi\), the normalized flow
\(\widehat{\boldsymbol\chi}\), and the calibration
\(\boldsymbol\chi = \clight\,\widehat{\boldsymbol\chi}\), together with the
derivation of the Minkowski interval and standard SR effects from a Euclidean
proto-space.
It also introduces the unimetrical D-rotation, which encodes Lorentz boosts as
Euclidean rotations in a suitable plane of \(\Espace\).

From the unimetry viewpoint, however, these SR results are only the first 
consistency test of a broader phase-based paradigm.
The present paper assumes familiarity with the conceptual setting of Paper~I,
but is written to be as self-contained as reasonably possible.
We briefly recall the key definitions of the phase proto-space, the flow vector,
and the two calibrations of the flow that lead to kinematic and energetic
interpretations.
All constructions that are essential for the GEM sector are reproduced or
adapted here; more detailed discussions of SR and cosmological applications
remain in Paper~I and are only referenced when needed.

\subsection{Main results}

The main technical contributions of this paper can be summarized as follows.

\begin{itemize}[leftmargin=*]
  \item We introduce \emph{metrically dressed body quaternions}
  \(\widetilde{\mathsf q}_i = L_{E,i}\,\qh + L_{G,i}\,\widehat{\bm n}_i\),
  whose components have the dimension of length.
  The ``electric'' and ``gravitational'' lengths
  \[
    L_{E,i} = \sqrt{\frac{\GNewton}{4\pi \epszero \clight^4}}\,Q_i,
    \qquad
    L_{G,i} = \frac{\GNewton}{\clight^2}\,m_i
  \]
  encode the charge \(Q_i\) and mass \(m_i\) of the body in a unified geometric
  fashion.
  The unit vector \(\widehat{\bm n}_i\) represents the spatial flow direction
  associated with the body.

  \item We show that the scalar bilinear form
  \[
    A(\widetilde{\mathsf q}_1,\widetilde{\mathsf q}_2)
    = L_{E,1}L_{E,2} - \bm S_1\cdot\bm S_2
  \]
  (with \(\bm S_i = L_{G,i}\widehat{\bm n}_i\)) yields, after a single global
  calibration by \(\clight^4/\GNewton\) and a geometric \(1/r\) factor, the
  combined Newton--Coulomb potential:
  \[
    U(r) = \frac{\clight^4}{\GNewton}\,\frac{A}{r}
    = \frac{1}{4\pi\epszero}\,\frac{Q_1 Q_2}{r}
      - \GNewton\,\frac{m_1 m_2}{r}.
  \]
  Thus gravity and electrostatics arise as two channels of a single invariant
  scalar form.

  \item We identify two vector-valued bilinear forms,
  \(\bm B(\widetilde{\mathsf q}_1,\widetilde{\mathsf q}_2)\) and
  \(\bm C(\widetilde{\mathsf q}_1,\widetilde{\mathsf q}_2)\), corresponding to
  the symmetric and axial parts of the quaternion product.
  In the dressed setting these naturally describe current-like and vortical
  channels.
  In particular, the vortical form \(\bm C\) reproduces the geometry of
  magnetic and gravitomagnetic fields generated by moving charges and masses.

  \item We construct a quaternionic GEM field
  \(\mathcal F_{\rm GEM}(\bm x)\) over the observable three-space by combining
  dressed source quaternions with purely imaginary distance quaternions.
  Its scalar channel reproduces the gravitational and electrostatic potentials,
  while its vortical channel yields a physically natural ``phase-vortical''
  field \(C_{\text{phys}}\) with the same dimension as \(\bm E\).
  The standard magnetic field \(\bm B\) in SI units then appears as
  \[
    \bm B = \frac{1}{\clight}\,C_{\text{phys}},
  \]
  so that the familiar \(\muzero\) and \(\epszero\) can be interpreted in terms
  of linear and areal stiffness of the vacuum, combined into an effective
  volumetric stiffness proportional to \(1/(\epszero \clight^3)\).

  \item We analyze the action of unimetrical D-rotations and ordinary spatial
  rotors on dressed quaternions.
  Pure spatial rotations act in the usual way on the vector channels and leave
  the scalar form \(A\) invariant, while D-rotations mix the scalar channel and
  the longitudinal component of \(\bm B\) in a two-dimensional
  ``energy--current'' plane.
  This provides a quaternionic encoding of relativistic kinematics in the GEM
  setting, with Lorentz-consistent transformation properties of the fields.

  \item Finally, we outline a Hamiltonian and Lagrangian formulation of the
  quaternionic GEM theory in terms of the self-form \(A\) and the norm-squares
  of \(\bm B\) and \(\bm C\), and discuss how the standard Maxwell Lagrangian
  and linearized GEM equations arise in appropriate limits.
\end{itemize}

\subsection{Structure of the paper}

The paper is organized as follows.
In \cref{sec:quat-algebra} we recall the basic quaternion algebra and introduce
the bilinear forms \(A\), \(\bm B\), and \(\bm C\) that arise from the
quaternion product, together with their matrix representation and geometric
interpretation.
\Cref{sec:phase-protospace} provides a brief overview of the unimetrical phase
proto-space, the phase potential, the flow vector, and the two calibrations of
the flow that lead to kinematic and energetic interpretations.

In \cref{sec:dressed-quats} we introduce metrically dressed body quaternions and
define the electric and gravitational lengths \(L_E\) and \(L_G\).
\Cref{sec:bilinear-static-gem} shows how the scalar form \(A\) for dressed
quaternions reproduces the static Newton and Coulomb potentials.
In \cref{sec:quat-gem-field} we construct a quaternionic GEM field over the
observable three-space and identify the scalar and vortical channels with
gravitational, electric, and magnetic sectors.

\Cref{sec:drotors-dynamics} analyzes the action of spatial rotors and D-rotors
on dressed quaternions and on the GEM field, clarifying the relativistic
transformation properties of the scalar, current-like, and vortical channels.
\Cref{sec:calibration-EB-Cphys} is devoted to the calibration of \(\bm E\) and
\(\bm B\), to the definition of the phase-vortical field \(C_{\text{phys}}\),
and to the interpretation of \(\epszero\), \(\muzero\), and \(\clight\) in
terms of vacuum stiffness.

In \cref{sec:hamiltonians-lagrangians} we outline Hamiltonian and Lagrangian
formulations of quaternionic GEM, and in
\cref{sec:maxwell-gem-comparison} we compare the resulting equations with the
standard Maxwell and linearized GEM formalisms.
Finally, \cref{sec:discussion-conclusions} discusses limitations and open
questions, and sketches possible extensions towards non-Abelian interactions
and cosmological applications.

%==================== 2. Quaternion algebra ====================

\section{Quaternion algebra and bilinear forms}
\label{sec:quat-algebra}

\subsection{Basic notation and conventions}

We denote by \(\mathbb{H}\) the real quaternion algebra, viewed as a
four-dimensional real vector space
\[
  \mathbb{H} \simeq \mathbb{R}^4(\qh,\hat{\imath},\hat{\jmath},\hat{k}),
\]
where \(\qh\) is the distinguished scalar basis element and
\(\hat{\imath},\hat{\jmath},\hat{k}\) are purely imaginary basis elements.
A general quaternion is written as
\[
  \mathbf{q} = x^\mu e_\mu
  = x^0 \qh + x^1 \hat{\imath} + x^2 \hat{\jmath} + x^3 \hat{k},
\]
with real components \(x^\mu \in \mathbb{R}\) and basis
\(e_0 := \qh\), \(e_1 := \hat{\imath}\), \(e_2 := \hat{\jmath}\),
\(e_3 := \hat{k}\).

The imaginary basis satisfies the usual quaternion relations
\[
  \hat{\imath}^2 = \hat{\jmath}^2 = \hat{k}^2 = -\qh,
  \qquad
  \hat{\imath}\hat{\jmath}=\hat{k},\quad
  \hat{\jmath}\hat{k}=\hat{\imath},\quad
  \hat{k}\hat{\imath}=\hat{\jmath},
\]
with antisymmetry under exchange of factors.
We identify \(\Im\mathbb{H}\simeq\mathbb{R}^3\) with its Euclidean inner
product \(\bm x\cdot\bm y\) and cross product \(\bm x\times\bm y\), so that
\[
  \bm x = x^1 \hat{\imath} + x^2 \hat{\jmath} + x^3 \hat{k}
  \quad\longleftrightarrow\quad
  (x^1,x^2,x^3) \in \mathbb{R}^3.
\]

Quaternionic conjugation is defined by
\[
  \overline{\mathbf{q}} := x^0 \qh - x^1 \hat{\imath} - x^2 \hat{\jmath}
                           - x^3 \hat{k},
\]
and the norm is \(\|\mathbf{q}\|^2 = \mathbf{q}\,\overline{\mathbf{q}}
= \overline{\mathbf{q}}\,\mathbf{q} = (x^0)^2 + (x^1)^2 + (x^2)^2 + (x^3)^2\).

When convenient we will write a quaternion as \(\mathbf{q}=(T,\bm S)\) with
\[
  T := x^0,\qquad \bm S := x^1 \hat{\imath} + x^2 \hat{\jmath} + x^3 \hat{k},
\]
emphasizing the split into scalar and vector parts.

\paragraph{Universal imaginary unit.}

Besides the fixed imaginary basis \(\{\hat{\imath},\hat{\jmath},\hat{k}\}\)
it will be convenient to introduce a ``universal'' unit imaginary quaternion
\(\uhat\), defined abstractly by
\begin{equation}
  \uhat^2 = -\qh,
  \qquad
  \|\uhat\| = 1.
  \label{eq:u-hat-def}
\end{equation}
Geometrically, \(\uhat\) should be understood as a \emph{joker} direction: in
any concrete configuration it is identified with the unit vector along the
relevant interaction axis (for instance, the radial direction between two
approximately isotropic Newtonian bodies).
Algebraically, \(\uhat\) behaves as any other unit imaginary quaternion, and
all scalar invariants such as \(A(q,q)\) remain well defined when the spatial
part of \(q\) is restricted to the one-dimensional subspace \(\mathbb{R}\uhat\).

When describing isotropic Newtonian bodies we will often use quaternions of the
form
\begin{equation}
  q = T\,\qh + S\,\uhat,
  \label{eq:radial-quat}
\end{equation}
with \(T,S\in\mathbb{R}\), so that all interaction channels reduce to the
radial line spanned by \(\uhat\).
This eliminates spurious transverse contributions in the vortical form
\(\bm C\) for purely radial gravito--electric configurations.


\subsection{Quaternion product and decomposition into A, B, C forms}

Let
\[
  \mathbf{q}_1 = x^\mu e_\mu
  = x^0 \qh + x^1 \hat{\imath} + x^2 \hat{\jmath} + x^3 \hat{k},
\]
\[
  \mathbf{q}_2 = y^\nu e_\nu
  = y^0 \qh + y^1 \hat{\imath} + y^2 \hat{\jmath} + y^3 \hat{k},
\]
with \(\mathbf{q}_1,\mathbf{q}_2 \in \mathbb{H}\).
Their quaternion product can be expanded term by term as
\begin{align}
  \mathbf{q}_1 \circ \mathbf{q}_2
  &= (x^0 \qh + x^1 \hat{\imath} + x^2 \hat{\jmath} + x^3 \hat{k})
     (y^0 \qh + y^1 \hat{\imath} + y^2 \hat{\jmath} + y^3 \hat{k}) \nonumber\\
  &= (x^0 y^0 \qh^2 + x^1 y^1 \hat{\imath}^2 + x^2 y^2 \hat{\jmath}^2
      + x^3 y^3 \hat{k}^2) \nonumber\\
  &\quad
   + (x^0 y^1 \qh \hat{\imath} + x^0 y^2 \qh \hat{\jmath}
      + x^0 y^3 \qh \hat{k}
      + x^1 y^0 \hat{\imath} \qh + x^2 y^0 \hat{\jmath} \qh
      + x^3 y^0 \hat{k} \qh) \nonumber\\
  &\quad
   + (x^1 y^2 \hat{\imath} \hat{\jmath}
      + x^1 y^3 \hat{\imath} \hat{k}
      + x^2 y^1 \hat{\jmath} \hat{\imath}
      + x^2 y^3 \hat{\jmath} \hat{k}
      + x^3 y^1 \hat{k} \hat{\imath}
      + x^3 y^2 \hat{k} \hat{\jmath}).
  \label{eq:q-product-expanded-coords}
\end{align}

Using the multiplication rules, this can be organised into three bilinear
contributions:
\begin{align}
  \mathbf{q}_1 \circ \mathbf{q}_2
  &= ( \mathbf{q}_1 * \mathbf{q}_2 )
   + ( \mathbf{q}_1 \diamond \mathbf{q}_2 )
   + ( \mathbf{q}_1 \times \mathbf{q}_2 ),
\end{align}
where:
\begin{itemize}[leftmargin=*]
  \item \(\mathbf{q}_1 * \mathbf{q}_2\) collects the purely scalar terms,
  \item \(\mathbf{q}_1 \diamond \mathbf{q}_2\) collects the mixed scalar--vector
  terms,
  \item \(\mathbf{q}_1 \times \mathbf{q}_2\) collects the purely vector--vector
  terms.
\end{itemize}

Explicitly, one finds the familiar invariant decomposition
\begin{align}
  \mathbf{q}_1 \circ \mathbf{q}_2
  &= \Bigl(
       \color{red}{x^0 y^0 - x^1 y^1 - x^2 y^2 - x^3 y^3}
     \Bigr)\,\qh
   \nonumber\\
  &\quad
   + \color{green}{\bigl(x^0 \bm y + y^0 \bm x\bigr)}
   + \color{blue}{\bigl(\bm x \times \bm y\bigr)},
  \label{eq:q-product-ABC-coords}
\end{align}
where
\[
  \bm x := x^1 \hat{\imath} + x^2 \hat{\jmath} + x^3 \hat{k},
  \qquad
  \bm y := y^1 \hat{\imath} + y^2 \hat{\jmath} + y^3 \hat{k}.
\]
This suggests three natural bilinear maps:
\begin{align}
  A(\mathbf{q}_1,\mathbf{q}_2)
  &:= x^0 y^0 - x^1 y^1 - x^2 y^2 - x^3 y^3,
  \label{eq:A-form-coords}\\
  \bm B(\mathbf{q}_1,\mathbf{q}_2)
  &:= x^0 \bm y + y^0 \bm x,
  \label{eq:B-form-coords}\\
  \bm C(\mathbf{q}_1,\mathbf{q}_2)
  &:= \bm x \times \bm y.
  \label{eq:C-form-coords}
\end{align}
In terms of these,
\begin{equation}
  \mathbf{q}_1 \circ \mathbf{q}_2
  = \bigl(\color{red}{A(\mathbf{q}_1,\mathbf{q}_2)}\bigr)\,\qh
    + \color{green}{\bm B(\mathbf{q}_1,\mathbf{q}_2)}
    + \color{blue}{\bm C(\mathbf{q}_1,\mathbf{q}_2)}.
  \label{eq:q-product-ABC}
\end{equation}

It is often convenient to view \eqref{eq:q-product-ABC-coords} in a tensor-like
form.
We can write
\[
  \mathbf{q}_1 \circ \mathbf{q}_2
  = \sum_{\mu,\nu=0}^3
    \Bigl(
      \color{red}{A_{\mu\nu}}\,\qh
      + \color{green}{B_{\mu\nu}}
      + \color{blue}{C_{\mu\nu}}
    \Bigr)\,x^\mu y^\nu,
\]
with three \(4\times 4\) coefficient matrices:
\begin{align}
  \color{red}{A_{\mu\nu}}
  &= \begin{pmatrix}
       1 & 0 & 0 & 0 \\
       0 & -1 & 0 & 0 \\
       0 & 0 & -1 & 0 \\
       0 & 0 & 0 & -1
     \end{pmatrix},
  \label{eq:A-matrix}\\[0.5em]
  \color{green}{B_{\mu\nu}}
  &= \begin{pmatrix}
       0 & \hat{\imath} & \hat{\jmath} & \hat{k} \\
       \hat{\imath} & 0 & 0 & 0 \\
       \hat{\jmath} & 0 & 0 & 0 \\
       \hat{k} & 0 & 0 & 0
     \end{pmatrix},
  \label{eq:B-matrix}\\[0.5em]
  \color{blue}{C_{\mu\nu}}
  &= \begin{pmatrix}
       0 & 0 & 0 & 0 \\
       0 & 0 & \hat{k} & -\hat{\jmath} \\
       0 & -\hat{k} & 0 & \hat{\imath} \\
       0 & \hat{\jmath} & -\hat{\imath} & 0
     \end{pmatrix}.
  \label{eq:C-matrix}
\end{align}
Here \(\color{red}{A_{\mu\nu}}\) is the usual Minkowski-like bilinear form
\(\mathrm{diag}(1,-1,-1,-1)\) acting on the coordinate components, while
\(\color{green}{B_{\mu\nu}}\) and \(\color{blue}{C_{\mu\nu}}\) collect the
symmetric and antisymmetric vector-valued pieces of the product.

\subsection{Geometric interpretation and tensor structure of A, B, C}

The decomposition \eqref{eq:q-product-ABC} and the matrices
\eqref{eq:A-matrix}--\eqref{eq:C-matrix} make explicit that the quaternion
product can be regarded as the contraction of a rank-\((0,2)\) object with two
four-vectors:
\[
  \mathbf{q}_1 \circ \mathbf{q}_2
  = \bigl(
      \color{red}{A} + \color{green}{B} + \color{blue}{C}
    \bigr)_{\mu\nu}\,x^\mu y^\nu,
\]
with three structurally distinct blocks:
\begin{itemize}[leftmargin=*]
  \item \(\color{red}{A_{\mu\nu}}\) is a symmetric scalar bilinear form of
  signature \((+,-,-,-)\).
  In the unimetrical context it will play the role of an energy-like invariant
  and will generate both Newtonian and Coulomb potentials once we pass to
  dressed body quaternions.

  \item \(\color{green}{B_{\mu\nu}}\) is symmetric and vector-valued; it couples
  the scalar component to the spatial components.
  It will later be interpreted as a current-like channel, encoding the coupling
  between temporal and spatial parts of dressed flow quaternions.

  \item \(\color{blue}{C_{\mu\nu}}\) is antisymmetric and vector-valued; it
  encodes the cross product \(\bm x\times\bm y\) of the spatial parts and thus
  represents a vorticity (axial) channel.
  This will underlie the magnetic and gravitomagnetic sectors of the GEM field.
\end{itemize}

In summary, the elementary quaternion product already contains, in a rigid
algebraic way, the three channels that we will later reinterpret as
\begin{enumerate}[label=(\roman*),leftmargin=*]
  \item an energy-like scalar invariant \(A\),
  \item a current-like symmetric vector channel \(\bm B\),
  \item a vortical (axial) vector channel \(\bm C\).
\end{enumerate}
In the next section we recall how unimetry associates physical flows and an
effective space--time structure to quaternions, so that these three forms can
be given a gravito--electromagnetic meaning.



%==================== 3. Phase proto-space in unimetry ====================

\section{Phase proto-space and flow: brief unimetry overview}
\label{sec:phase-protospace}

\subsection{Proto-space, phase potential and phase 1-form}

In unimetry the basic kinematical arena is a Euclidean (or, more generally,
Hilbert) proto-space \((\Espace,\langle\cdot,\cdot\rangle)\).
Points of \(\Espace\) will be denoted by \(X\), and the inner product
\(\langle\cdot,\cdot\rangle\) is used to identify tangent and cotangent spaces
via the Riesz isomorphism.
For the purposes of this paper one may think of \(\Espace\) as a finite- or
countable-dimensional real Hilbert space.

The fundamental scalar field of unimetry is a dimensionless \emph{phase
potential}
\[
  \Phi : \Espace \to \mathbb{R}.
\]
From \(\Phi\) we obtain the \emph{phase 1-form}
\[
  \alpha := d\Phi,
\]
which is a smooth 1-form on \(\Espace\).
At each point \(X\in\Espace\), the value \(\alpha_X\) is a linear functional on
the tangent space \(T_X\Espace\):
\[
  \alpha_X : T_X\Espace \to \mathbb{R},
  \qquad
  \alpha_X(V) = d\Phi_X(V).
\]
Using the inner product, we define the gradient \(\nabla\Phi(X)\in T_X\Espace\)
by the standard relation
\[
  \alpha_X(V)
  = d\Phi_X(V)
  = \langle \nabla\Phi(X), V\rangle,
  \qquad
  \forall\,V \in T_X\Espace.
\]
Thus \(\alpha\) and \(\nabla\Phi\) carry the same information; the former is
covariant, the latter contravariant.

Physically, the phase potential \(\Phi\) encodes the global phase structure of
the underlying flow, while the phase 1-form \(\alpha\) and the gradient
\(\nabla\Phi\) encode local directions in which the phase changes most rapidly.
The key idea of unimetry is to use this structure to define a canonical flow
through \(\Espace\).

\subsection{Flow vector and normalization}
\label{subsec:flow-normalization}

Whenever \(\nabla\Phi(X)\neq 0\), we define the \emph{normalized flow direction}
at \(X\) by
\begin{equation}
  \widehat{\boldsymbol\chi}(X)
  := \frac{\nabla\Phi(X)}{\|\nabla\Phi(X)\|},
  \qquad
  \|\nabla\Phi(X)\|
  := \sqrt{\langle\nabla\Phi(X),\nabla\Phi(X)\rangle}.
  \label{eq:chi-hat-def}
\end{equation}
Thus \(\widehat{\boldsymbol\chi}(X)\) is a unit vector in \(T_X\Espace\) pointing
along the steepest phase ascent.
We then introduce the \emph{physical flow vector} by a global calibration
\begin{equation}
  \boldsymbol\chi(X)
  := \clight\,\widehat{\boldsymbol\chi}(X),
  \qquad
  \|\boldsymbol\chi(X)\| \equiv \clight,
  \label{eq:chi-phys-def}
\end{equation}
where \(\clight\) is the speed of light.
In other words, in unimetry the physical flow is a unit-speed curve in
\(\Espace\) with respect to the fixed scale \(\clight\).

A flow line (or \emph{stream}) is then a curve
\(\gamma : \lambda \mapsto X(\lambda)\in\Espace\) whose tangent vector is
everywhere aligned with the physical flow:
\begin{equation}
  \dot{X}(\lambda)
  := \frac{dX}{d\lambda}
  = \boldsymbol\chi\bigl(X(\lambda)\bigr),
  \qquad
  \|\dot{X}(\lambda)\| = \clight.
  \label{eq:flow-line-def}
\end{equation}
The parameter \(\lambda\) is a proto-space parameter, not yet identified with
any observed time.
The geometric content of \eqref{eq:flow-line-def} is simply that physical
objects are represented by flows of constant Euclidean speed \(\clight\) in the
proto-space.

\subsection{Intrinsic angle, proper time and correspondence with SR}
\label{subsec:interval-SR}

In unimetry a macroscopic body \(B\) is represented not by a single flow line,
but by an ensemble of streamlets with weights \(w_a\) and tilt angles
\(\Theta_a\) relative to the body's self–time fibre.\footnote{For the detailed
construction see Paper~I, \S\ref{sec:intrinsic-angle} there.}
On this ensemble one defines the temporal second moment
and the spatial shape tensor as
\begin{equation}
  T_B:=\sum_a w_a \cos^2\Theta_a,
  \qquad
  \mathbf C_B:=\sum_a w_a \sin^2\Theta_a\;
               \mathbf u_a\!\otimes\!\mathbf u_a,
  \label{eq:TB-CB-here}
\end{equation}
where \(0<T_B\le 1\), \(\mathbf C_B\) is a symmetric positive semidefinite
tensor on the body's three-surface, and \(\mathbf u_a\) are unit spatial
directions of the streamlets’ projections.
Operationally, \(T_B\) captures the aggregate fraction of flow carried in the
orthogonal (self–time) fibre, while \(\mathbf C_B\) encodes the anisotropic
distribution of spatial projections across the body.

From these second moments one can define an \emph{intrinsic angle}
\(\zeta\in[0,\tfrac{\pi}{2}]\) as an effective statistical parameter of the
ensemble.
Introducing
\[
  C:=\sum_a w_a\cos 2\Theta_a,\qquad
  S:=\sum_a w_a\sin 2\Theta_a,
\]
there exists a unique \(\zeta\) such that
\begin{equation}
  (\cos 2\zeta,\ \sin 2\zeta) = (C,S)
  \quad\Longleftrightarrow\quad
  T_B = \tfrac12(1+C) = \cos^2\!\zeta,
  \qquad
  \operatorname{tr}\mathbf C_B
    = \tfrac12(1-C) = \sin^2\!\zeta .
  \label{eq:zeta-def-here}
\end{equation}
We call \(\zeta\) the \emph{intrinsic angle} of the body.
It aggregates the second–moment information \((T_B,\mathbf C_B)\) into a single
scalar and should be thought of as a \emph{statistical} internal parameter:
it is \emph{not} a geometric direction and is not attached to any particular
flow line.

The intrinsic angle controls the rate at which the body’s own proper time
\(\tau_B\) accumulates with respect to the phase parameter \(\chi\) used to
parametrize the flow in proto-space.
In the calibrated gauge \(\|\boldsymbol\chi\|=\clight\), one has
\begin{equation}
  d\tau_B = \cos\zeta\,d\chi,
  \label{eq:dtau-zeta}
\end{equation}
so that the temporal second moment \(T_B\) appears as
\(T_B=\cos^2\!\zeta = (d\tau_B/d\chi)^2\).
The corresponding intrinsic metric of the body,
as a quadratic form on \((d\chi,d\boldsymbol\ell)\), reads
\begin{equation}
  ds_B^2
  := \clight^2\,d\tau_B^2 - d\boldsymbol\ell^{\!\top}\mathbf C_B\,d\boldsymbol\ell
   = \clight^2 T_B\,d\chi^2
     - d\boldsymbol\ell^{\!\top}\mathbf C_B\,d\boldsymbol\ell .
  \label{eq:body-metric-here}
\end{equation}
For an isotropic texture one has
\(\mathbf C_B = \tfrac{\sin^2\!\zeta}{3}\,\mathbf I_S\), and with the rest
gauge \(T_B\equiv 1\) this reduces to the familiar Minkowski form in the
body’s rest frame (up to the overall phase gauge \(d\chi\)).

In the full unimetrical construction the intrinsic angle \(\zeta\) is combined
with a kinematic angle \(\vartheta\) (associated with the relative motion
between bodies) and, when present, with a gravitational angle \(\phi\)
(associated with an external tilt field).
The resulting time-rate factor factorises into intrinsic, kinematic, and
gravitational contributions.
For the purposes of the present GEM paper, we only need the following
structural facts:

\begin{itemize}[leftmargin=*]
  \item The intrinsic angle \(\zeta\) is a scalar \emph{second-moment} parameter
  of a body, not a direction: it encodes how the flow budget is split between
  self–time and spatial channels in the ensemble of streamlets.

  \item The proper time \(\tau_B\) along the body’s worldline is related to the
  phase parameter \(\chi\) by \eqref{eq:dtau-zeta}, and the body’s intrinsic
  metric takes the Minkowski form \eqref{eq:body-metric-here} once the rest
  gauge is fixed.

  \item The relativistic kinematics of unimetry can therefore be formulated
  entirely in terms of phase flow and second-moment data, with the usual SR
  interval emerging as a derived object; we will reuse this structure when
  interpreting the scalar form \(A\) as an energy-like invariant for dressed
  quaternions.
\end{itemize}

\subsection{Notation table}
\label{subsec:notation-table}

For reference, we collect here the main unimetrical symbols used in the
remainder of the paper.  A more extensive table can be found in Paper~I; the
subset below is chosen to make the present text self-contained.

\begin{table}[h]
  \centering
  \begin{tabular}{ll}
    \hline
    Symbol & Meaning \\ \hline
    \(\Espace\) & Euclidean/Hilbert proto-space with inner product
                   \(\langle\cdot,\cdot\rangle\) \\
    \(\Phi : \Espace\to\mathbb{R}\) & dimensionless phase potential \\
    \(\alpha = d\Phi\) & phase 1-form, \(\alpha_X(V)=\langle\nabla\Phi(X),V\rangle\) \\
    \(\nabla\Phi(X)\) & gradient of \(\Phi\) at \(X\), defined via the inner product \\
    \(\widehat{\boldsymbol\chi}(X)\) &
      normalized flow direction, \(\widehat{\boldsymbol\chi}=\nabla\Phi/\|\nabla\Phi\|\) \\
    \(\boldsymbol\chi(X)\) &
      physical flow vector, \(\boldsymbol\chi=\clight\,\widehat{\boldsymbol\chi}\),
      \(\|\boldsymbol\chi\|=\clight\) \\
    \(\gamma(\lambda)\) &
      flow line in \(\Espace\) with tangent \(\dot{X}=\boldsymbol\chi\) \\
    \(\widehat{\bm u}\) &
      unit rest direction associated with an observer (local temporal axis) \\
    \(\widehat{\bm n}\) &
      unit spatial direction orthogonal to \(\widehat{\bm u}\) \\
    \(\zeta\) &
      flow angle between \(\boldsymbol\chi\) and \(\widehat{\bm u}\), see
      \eqref{eq:chi-angle-decomp} \\
    \(\delta T\) &
      effective temporal increment for the observer, see \eqref{eq:deltaT-def} \\
    \(\delta\bm x\) &
      effective spatial increment in the observer's rest space,
      see \eqref{eq:deltax-def} \\
    \(\delta s^2\) &
      effective interval, \(\delta s^2 = \clight^2\delta T^2 - \|\delta\bm x\|^2\),
      see \eqref{eq:interval-zeta} \\
    \hline
  \end{tabular}
  \caption{Key unimetrical quantities used in the quaternionic GEM construction.}
  \label{tab:unimetry-notation}
\end{table}

In the next section we introduce two calibrations of the flow --- one
kinematic and one energetic --- which will allow us to interpret the scalar
form \(A\) as an energy-like invariant and to define metrically dressed body
quaternions suitable for the gravito--electromagnetic setting.


%==================== 4. Calibrations of the flow and energy ====================
\section{Flow calibrations and energy-like functionals}
\label{sec:flow-calibrations}

In the unimetrical picture, physical bodies are represented by flows of
constant Euclidean speed \(\clight\) in the proto-space \(\Espace\), while the
phase potential \(\Phi\) provides a dimensionless scalar structure along these
flows.
To connect this geometric description to observable kinematics and to
energy-like quantities, we need to relate the flow parameter to time
coordinates and to introduce a structural scale.
This section introduces two complementary calibrations of the flow parameter
and shows how they lead to a factorized expression for energy, which will later
be recast in quaternionic terms.

\subsection{Flow parameter and phase frequencies}
\label{subsec:chi-and-phi}

Let \(\gamma:\chi\mapsto X(\chi)\) be a flow line in \(\Espace\).
We take the flow parameter \(\chi\) to have units of time,
\[
  [\chi] = \mathrm{s},
\]
and we assume that along the worldline of a given body \(B\) the phase
potential \(\Phi\) depends smoothly on \(\chi\).
We define the \emph{phase frequency} per unit \(\chi\) by
\begin{equation}
  \omega_\chi(\chi)
  := \frac{d\Phi}{d\chi},
  \qquad
  [\omega_\chi] = \mathrm{s}^{-1}.
  \label{eq:omega-chi-def}
\end{equation}
In many situations it is convenient to consider a gauge in which
\(\omega_\chi\) is constant along the worldline of a given body; in that case
\(\Phi(\chi)=\omega_\chi\,\chi\).
The product \(\omega_\chi \chi\) is always dimensionless.

The geometric flow vector \(\boldsymbol\chi\) is normalized by
\(\|\boldsymbol\chi\|\equiv\clight\), cf.\ \eqref{eq:chi-phys-def}, so that the
flow is a unit-speed (with respect to \(\clight\)) curve in proto-space.
The different calibrations introduced below relate \(\chi\) to observable
time parameters and thus furnish a bridge between the proto-space flow and
physical time measurements.

\subsection{Cyclic origin of local time and proto-parameter speed}
\label{subsec:cyclic-time}

We now make explicit a key unimetrical assumption: local time arises from a
cyclic action of the flow along a compactified temporal axis \(x_0\).
For simplicity we model the geometry of this axis by a circle of variable
radius and show how the ratio between the proto-parameter \(\chi\) and the
proper time \(\tau\) emerges as a frequency.

Consider two circles in the \((\chi,\tau)\)-plane:
\begin{itemize}[leftmargin=*]
  \item a ``phase circle'' of radius \(R_1\) in the \(\chi\)-direction,
  representing one phase step \(\Delta\chi = 2\pi R_1\);
  \item a ``time circle'' of radius \(R_2\) in the \(\tau\)-direction, along
  which the flow advances with a characteristic proto-space speed
  component \(\widetilde X_0\) along the temporal axis.
\end{itemize}
One full turn along the time circle then corresponds to a proper-time
increment
\begin{equation}
  \Delta\tau = \frac{2\pi R_2}{\widetilde X_0}.
\end{equation}
The frequency of phase ticks with respect to proper time is
\begin{equation}
  \nu
  := \frac{\Delta\chi}{\Delta\tau}
   = \frac{2\pi R_1}{2\pi R_2/\widetilde X_0}
   = \underbrace{\frac{R_1}{R_2}}_{=:k}\,\widetilde X_0.
  \label{eq:nu-k-X0tilde}
\end{equation}
In the high-frequency limit \(R_1\to 0\) the discrete ratio becomes a
derivative and we obtain
\begin{equation}
  \nu \;\xrightarrow{R_1\to 0}\;
  \frac{d\chi}{d\tau}
  =: \dot\chi
  \quad\Rightarrow\quad
  \boxed{\dot\chi = k\,\widetilde X_0},
  \label{eq:chi-dot-k-X0tilde}
\end{equation}
with dimensionless rate \(\dot\chi\) and dimensionless structural ratio
\(k=R_1/R_2\).
This is the fundamental relation between the accumulation of the proto-parameter
\(\chi\) and the proper time \(\tau\): local time is the frequency of phase
ticks generated by the cyclic flow along the compactified temporal axis.

The same geometric picture yields a conversion factor between proto-space
velocities (measured per unit \(\chi\)) and velocities measured per unit
proper time.
Let \(\widetilde{\mathcal H}\) be the Euclidean norm of the proto-space flow
per unit \(\chi\), with components \(\widetilde X_\mu\).
Using the chain rule we have
\begin{equation}
  \dot{\mathcal H}
  := \frac{d\mathcal H}{d\tau}
   = \frac{d\mathcal H}{d\chi}\,\frac{d\chi}{d\tau}
   = \widetilde{\mathcal H}\,\dot\chi.
  \label{eq:Hdot-chain-rule}
\end{equation}
Imposing the unimetrical calibration in which the magnitude of the proto-space
flow equals its temporal component,
\begin{equation}
  \widetilde{\mathcal H} = \widetilde X_0,
\end{equation}
and using \eqref{eq:chi-dot-k-X0tilde} we obtain
\begin{equation}
  \dot{\mathcal H}
  = k\,\widetilde X_0^2.
  \label{eq:Hdot-k-X0tilde2}
\end{equation}
This quantity \(\dot{\mathcal H}\) can be interpreted as the full flow
magnitude in local time: it measures how fast the flow advances in proto-space
per unit proper time.

Requiring that the full magnitude of the flow in local time be preserved under
rotations of the proto-space velocity (i.e.\ under changes of the temporal
projection \(\widetilde X_0\)) leads to the constraint
\begin{equation}
  k = \widetilde X_0^{-2},
  \label{eq:k-X0tilde-invariance}
\end{equation}
so that \(\dot{\mathcal H}\) becomes a universal constant.
In physical units this constant is identified with the speed of light:
\begin{equation}
  \dot{\mathcal H} = \clight
  \quad\Longleftrightarrow\quad
  \text{the speed of light is the invariant magnitude of the flow in local
  time.}
\end{equation}
In this sense the constancy of \(\clight\) is reinterpreted as the constancy
of the Euclidean flow speed in proto-space, while variations of its projection
between temporal and spatial directions are captured by the kinematic angle
\(\vartheta\).

The relation
\begin{equation}
  d\chi = k\,\widetilde X_0\,d\tau
\end{equation}
shows that the proto-parameter \(\chi\) plays the role of a global invariant
for transformations mixing temporal and spatial channels.
Expressing the interval \(dS^2\) in terms of \(d\chi\) and the components
\(\widetilde X_\mu\) one recovers the Minkowski form
\begin{equation}
  dS^2 = \dot X_0^2\,d\tau^2
          - dx_1^2 - dx_2^2 - dx_3^2,
\end{equation}
i.e.\ the ``inverted'' Euclidean norm of the flow once the temporal geometry
is taken into account.
For a body of fixed structural complexity (fixed \(k\)) the quantity
\(\widetilde X_0\) must compensate changes in the local rate of time, whereas
for a fixed full magnitude \(\dot{\mathcal H}\) the ratio \(k\) must adjust so
that the cyclic construction of time remains compatible with the invariance of
the interval.


\subsection{Kinematic (phase) calibration}
\label{subsec:phase-calibration}

Let us choose an inertial laboratory frame with coordinate time \(t\) and
three-position \(\bm x\).
In the \emph{kinematic calibration} we identify the flow parameter \(\chi\)
with the lab time:
\begin{equation}
  \chi = t,
  \qquad
  \frac{d\chi}{dt} = 1.
  \label{eq:chi-equals-t}
\end{equation}
Along the worldline of a body \(B\) we may then write
\[
  \Phi(t) = \omega_\chi\,t,
\]
with \(\omega_\chi = d\Phi/dt\) in that frame.

The normalized flow direction can be decomposed with respect to the
lab-frame temporal unit vector \(\widehat{\bm u}\) and a spatial unit vector
\(\widehat{\bm n}\) in the rest space:
\begin{equation}
  \widehat{\boldsymbol\chi}
  = \cos\vartheta\,\widehat{\bm u}
    + \sin\vartheta\,\widehat{\bm n},
  \qquad
  \|\widehat{\bm n}\|=1,
  \label{eq:chi-split-theta}
\end{equation}
where \(\vartheta\in[0,\pi/2]\) is a \emph{kinematic angle}.
For a body moving with constant lab three-velocity
\(\bm v = d\bm x/dt\) we impose
\begin{equation}
  \sin\vartheta = \beta := \frac{\|\bm v\|}{\clight},
  \qquad
  \cos\vartheta = \sqrt{1-\beta^2}.
  \label{eq:v-theta-calibration}
\end{equation}
This calibration ensures that the spatial projection of the flow has magnitude
\(\|\bm v\|=\clight\sin\vartheta\), while the temporal projection matches the
usual factor \(\sqrt{1-\beta^2}\) that appears in time dilation.

Using \eqref{eq:dtau-zeta} specialized to the kinematic sector and
\eqref{eq:chi-equals-t}, we obtain the standard relation between proper time
\(\tau_B\) and lab time:
\[
  \frac{d\tau_B}{dt}
  = \cos\vartheta
  = \sqrt{1-\beta^2},
\]
with
\begin{equation}
  \gamma(\vartheta)
  := \frac{1}{\sqrt{1-\beta^2}}
   = \frac{1}{\cos\vartheta}
  \label{eq:gamma-theta}
\end{equation}
the usual Lorentz factor.
Thus the kinematic angle \(\vartheta\) encodes the purely kinematic tilt of the
flow relative to the chosen inertial frame.

\subsection{Structural parameter and volumetric coefficient}
\label{subsec:struct-kappa}

Besides kinematic tilts, unimetry allows for an internal rescaling of the
cyclic structure of the flow.
A convenient way to visualize this is to consider a pair of circles in the
\((\chi,\tau)\)-plane:

\begin{itemize}[leftmargin=*]
  \item a ``phase circle'' of radius \(R_1\) in the \(\chi\)-direction, which
  measures the size of a single phase step \(\Delta\chi\);
  \item a ``time circle'' of radius \(R_2\) in the \(\tau\)-direction, along
  which the flow executes a cyclic motion with some characteristic proper-time
  angular frequency \(\omega_\ast\) (units \(\mathrm{s}^{-1}\)).
\end{itemize}

One full turn on the phase circle corresponds to a phase increment
\(\Delta\chi = 2\pi R_1\).
One full turn on the time circle corresponds to a proper-time increment
\begin{equation}
  \Delta\tau = \frac{2\pi R_2}{\omega_\ast},
\end{equation}
since the circumference is \(2\pi R_2\) and the ``speed'' along the time
circle is set by \(\omega_\ast\).

In the high-frequency limit, when the phase step becomes small
(\(R_1\to 0\)) and many cycles are accumulated, the effective frequency of
phase ticks with respect to proper time is
\begin{equation}
  \nu
  := \frac{\Delta\chi}{\Delta\tau}
   = \frac{2\pi R_1}{2\pi R_2/\omega_\ast}
   = \underbrace{\frac{R_1}{R_2}}_{=:k}\,\omega_\ast.
  \label{eq:nu-k-omega}
\end{equation}
Passing to the differential limit, this gives
\begin{equation}
  \dot\chi
  := \frac{d\chi}{d\tau}
  = k\,\omega_\ast,
  \label{eq:chi-dot-k-omega}
\end{equation}
with a dimensionless structural parameter
\begin{equation}
  k := \frac{R_1}{R_2}.
  \label{eq:k-R1-R2}
\end{equation}
Equation \eqref{eq:chi-dot-k-omega} expresses the rate at which the flow
parameter \(\chi\) is accumulated per unit proper time \(\tau\) as the product
of a geometric ratio \(k\) and a characteristic internal proper-time
frequency \(\omega_\ast\).

We model the cumulative effect of this structural rescaling on energy by a
\emph{volumetric structural coefficient} \(\kappa(k)\).
Since a rescaling of a characteristic length by a factor \(k\) changes a
three-dimensional volume by \(k^3\), we take
\begin{equation}
  \kappa(k)
  = \kappa_* \left(\frac{k}{k_*}\right)^{\!3},
  \label{eq:kappa-of-k}
\end{equation}
where \(\kappa_*\) is a reference value at some fiducial structural state
\(k=k_*\).
Dimensional analysis then dictates that \(\kappa\) has units
\begin{equation}
  [\kappa] = \mathrm{kg\,s\,m^{-1}},
  \label{eq:kappa-dim}
\end{equation}
so that the combination \(\kappa c^3\) has the dimension of energy:
\[
  [\kappa c^3] = \mathrm{kg\,m^2\,s^{-2}} = \mathrm{J}.
\]

We therefore define the \emph{structural rest energy} and associated rest mass
by
\begin{equation}
  E_0(k)
  := \kappa(k)\,\clight^3,
  \qquad
  m_0(k) := \frac{E_0(k)}{\clight^2} = \kappa(k)\,\clight.
  \label{eq:E0-and-m0}
\end{equation}
For a fixed structural state \(k\) these reduce to constants
\(E_0=\kappa(k)\clight^3\), \(m_0=\kappa(k)\clight\), while spatial variations
of \(k\) would correspond to variations of the local rest energy and rest mass
scale.

\subsection{Proper-time calibration and energy factorization}
\label{subsec:E-factorization}

In the \emph{proper-time calibration} we parametrize the flow by the body's
proper time \(\tau_B\).
In the unimetrical framework the rate at which proper time accumulates with
respect to the flow parameter \(\chi\) is controlled by the body's intrinsic
angle \(\zeta_B\), introduced in \cref{subsec:interval-SR} as a characteristic
of the second-moment structure of the streamlet ensemble:
\begin{equation}
  d\tau_B = \cos\zeta_B\,d\chi,
  \qquad
  \frac{d\tau_B}{d\chi} = \cos\zeta_B.
  \label{eq:dtau-dchi-zetaB}
\end{equation}
Both \(\chi\) and \(\tau_B\) carry units of time, so \(\cos\zeta_B\) is a
dimensionless factor determined by the internal flow texture of the body.

A key assumption of unimetry in this paper is that, in the body's own rest
frame and in the absence of dissipation or absorption/emission of radiation,
the intrinsic angle \(\zeta_B\) is \emph{fixed}:
\begin{equation}
  \zeta_B = \mathrm{const}
  \quad\text{for a given body with fixed internal energy.}
  \label{eq:zetaB-const}
\end{equation}
In other words, purely kinematic changes of motion relative to an external
observer do not alter \(\zeta_B\); they act ``outside'' the internal flow
texture.

By contrast, the kinematic calibration of \cref{subsec:phase-calibration}
relates proper time to the laboratory time \(t\) via the kinematic angle
\(\vartheta\):
\begin{equation}
  \frac{d\tau_B}{dt}
  = \cos\vartheta
  = \sqrt{1-\beta^2},
  \qquad
  \gamma(\vartheta) = \frac{1}{\cos\vartheta}.
  \label{eq:dtau-dt-theta}
\end{equation}
Here \(\vartheta\) encodes the tilt of the macroscopic body flow relative to
the lab frame and is meaningful only for an external observer.
It is conceptually distinct from \(\zeta_B\): the intrinsic angle describes
the internal streamlet ensemble in the body's own frame, whereas
\(\vartheta\) describes relative motion between frames.

Using the structural rest-energy scale
\begin{equation}
  E_0(\zeta_B,k) := \kappa(k;\zeta_B)\,\clight^3,
  \qquad
  m_0(\zeta_B,k) := \kappa(k;\zeta_B)\,\clight,
  \label{eq:E0-m0-zetaB}
\end{equation}
we \emph{define} the energy of the body in the state \((\vartheta;\zeta_B,k)\)
as
\begin{equation}
  E(\vartheta;\zeta_B,k)
  := \gamma(\vartheta)\,E_0(\zeta_B,k)
   = \frac{1}{\cos\vartheta}\,\kappa(k;\zeta_B)\,\clight^3,
  \label{eq:E-theta-zetaB-k}
\end{equation}
or equivalently
\[
  E(\vartheta;\zeta_B,k)
  = \gamma(\vartheta)\,m_0(\zeta_B,k)\,\clight^2.
\]
Thus the energy factorizes into a \emph{purely kinematic} factor
\(\gamma(\vartheta)\), coming from the macroscopic tilt of the body's flow in
the laboratory frame, and a \emph{purely intrinsic-structural} factor
\(E_0(\zeta_B,k)=\kappa(k;\zeta_B)c^3\), which encodes the internal flow
texture (via \(\zeta_B\)) and the internal cyclic geometry (via \(k\)) in the
body's own frame.

For small velocities \(\beta\ll 1\), using
\(\cos\vartheta=\sqrt{1-\beta^2}\) and \(\gamma(\vartheta)=1/\cos\vartheta\) we
recover the usual expansion
\begin{equation}
  E(\vartheta;\zeta_B,k)
  = m_0(\zeta_B,k)\,\clight^2
    + \tfrac12 m_0(\zeta_B,k)\,v^2
    + O(\beta^4),
  \qquad
  v = \clight\sin\vartheta.
  \label{eq:E-expansion-zetaB}
\end{equation}
Hence the structurally generalized energy retains the standard
non-relativistic limit, while allowing the rest energy scale to vary with the
intrinsic flow texture (through \(\zeta_B\)) and with the internal cyclic
geometry (through \(k\)), without ever identifying \(\zeta_B\) with the
kinematic angle \(\vartheta\).


\subsection{From flow vector to flow quaternion}
\label{subsec:flow-to-quat}

So far the flow has been represented as a unit vector
\(\widehat{\boldsymbol\chi}\in T_X\Espace\) with a split into temporal and
spatial channels controlled by the kinematic angle \(\vartheta\), see
\eqref{eq:chi-split-theta}.
To connect this to the quaternion algebra of \cref{sec:quat-algebra}, we now
introduce a simple representation map from flow directions to unit quaternions.

Fix a body \(B\) and a local observer comoving with \(B\).
At a point \(X\in\Espace\) on the body's flow choose an orthonormal frame
\(\{\widehat{\bm u},\widehat{\bm e}_1,\widehat{\bm e}_2,\widehat{\bm e}_3\}\)
in \(T_X\Espace\), where \(\widehat{\bm u}\) is the observer's temporal
direction and \(\widehat{\bm e}_i\) span the three-dimensional rest space.
The normalized flow direction can then be written as
\begin{equation}
  \widehat{\boldsymbol\chi}
  = \cos\vartheta\,\widehat{\bm u}
    + \sin\vartheta\,\widehat{\bm n},
  \qquad
  \widehat{\bm n}
    := n^1 \widehat{\bm e}_1
      + n^2 \widehat{\bm e}_2
      + n^3 \widehat{\bm e}_3,
  \quad
  \|\widehat{\bm n}\|=1.
  \label{eq:chi-decomp-frame}
\end{equation}

On the quaternion side we have the basis
\(\{\qh,\hat{\imath},\hat{\jmath},\hat{k}\}\) introduced in
\cref{sec:quat-algebra}.
We fix a local identification between the observer's frame and the quaternion
basis by declaring
\begin{equation}
  \widehat{\bm u}
  \;\longleftrightarrow\; \qh,
  \qquad
  \widehat{\bm e}_i
  \;\longleftrightarrow\; \text{an orthonormal triple in }
  \{\hat{\imath},\hat{\jmath},\hat{k}\}.
  \label{eq:frame-to-quat-basis}
\end{equation}
In the simplest case one may align \(\widehat{\bm e}_1,\widehat{\bm e}_2,
\widehat{\bm e}_3\) with \(\hat{\imath},\hat{\jmath},\hat{k}\) respectively,
so that the spatial unit vector \(\widehat{\bm n}\) is represented by a unit
imaginary quaternion with the same components \(n^i\).

\begin{definition}[Flow quaternion]
  Given a normalized flow direction \(\widehat{\boldsymbol\chi}\) with
  kinematic angle \(\vartheta\) and spatial unit vector \(\widehat{\bm n}\) in
  the observer's rest space, the associated \emph{flow quaternion} is the unit
  quaternion
  \begin{equation}
    \widehat{\mathsf q}(\vartheta,\widehat{\bm n})
    := \cos\vartheta\,\qh + \sin\vartheta\,\widehat{\bm n},
    \qquad
    \|\widehat{\bm n}\|=1.
    \label{eq:flow-quaternion-def}
  \end{equation}
\end{definition}

By construction \(\|\widehat{\mathsf q}\|=1\).
The scalar part of \(\widehat{\mathsf q}\) encodes the temporal fraction of the
flow, while the vector part encodes the spatial fraction in the chosen rest
space.
Different choices of the spatial orthonormal frame correspond to spatial
rotations of the imaginary basis and do not affect scalar invariants such as
\(A(\widehat{\mathsf q},\widehat{\mathsf q})\).

\subsection{Scalar self-form \texorpdfstring{$A(q,q)$}{A(q,q)} as kinematic
factor}
\label{subsec:A-kinematic-factor}

We can now relate the kinematic factor in \eqref{eq:E-theta-k} to the
quaternionic self-form \(A(q,q)\) introduced in
\cref{sec:quat-algebra}.
For brevity we suppress the explicit dependence on \(\widehat{\bm n}\) and
write \(\widehat{\mathsf q}(\vartheta)\) when no confusion can arise:
\[
  \widehat{\mathsf q}(\vartheta)
  = \cos\vartheta\,\qh + \sin\vartheta\,\widehat{\bm n},
  \qquad
  \|\widehat{\bm n}\|=1.
\]
The scalar self-form \(\color{red}{A}\) applied to this unit quaternion gives
\begin{equation}
  A\bigl(\widehat{\mathsf q}(\vartheta),
         \widehat{\mathsf q}(\vartheta)\bigr)
  = \cos^2\vartheta - \sin^2\vartheta
  = \cos 2\vartheta.
  \label{eq:A-self-theta}
\end{equation}
Solving for \(\cos^2\vartheta\) we obtain
\begin{equation}
  1-\beta^2
  = \cos^2\vartheta
  = \frac{1 + A(\widehat{\mathsf q},\widehat{\mathsf q})}{2},
  \label{eq:1-minus-beta-from-A}
\end{equation}
and hence the Lorentz factor can be expressed as
\begin{equation}
  \gamma(\vartheta)
  = \frac{1}{\sqrt{1-\beta^2}}
  = \sqrt{\frac{2}{1 + A(\widehat{\mathsf q},\widehat{\mathsf q})}}.
  \label{eq:gamma-from-A}
\end{equation}

Substituting \eqref{eq:gamma-from-A} into the factorized energy
\eqref{eq:E-theta-k}, we find
\begin{equation}
  E(\widehat{\mathsf q},k)
  = \gamma(\vartheta)\,E_0(k)
  = \sqrt{\frac{2}{1 + A(\widehat{\mathsf q},\widehat{\mathsf q})}}\;
    \kappa(k)\,\clight^3.
  \label{eq:E-from-A-and-k}
\end{equation}
Thus:
\begin{itemize}[leftmargin=*]
  \item the scalar self-form \(A(\widehat{\mathsf q},\widehat{\mathsf q})\)
  is a \emph{purely kinematic, dimensionless} invariant of the flow
  quaternion;
  \item the structural coefficient \(\kappa(k)\) carries the absolute energy
  scale, via \(E_0(k)=\kappa(k)\clight^3\);
  \item energy emerges as the product of a quaternionic kinematic factor (a
  function of \(A\)) and a structural scale.
\end{itemize}

In the gravito--electromagnetic construction below we will apply the same
scalar form \(A\) not to unit flow quaternions, but to \emph{metrically
dressed} body quaternions \(\widetilde{\mathsf q}_i\) that package mass and
charge into effective lengths.
For isotropic Newtonian bodies the spatial parts of these quaternions will be
further restricted to multiples of the universal unit imaginary \(\uhat\)
introduced in \cref{sec:quat-algebra}, so that all interaction channels reduce
to a single radial direction.
The vector forms \(\bm B\) and \(\bm C\) will then provide, respectively, the
current-like and vortical channels of the gravito--electromagnetic field.

%==================== 5. Metrically dressed body quaternions ====================

\section{Metrically dressed body quaternions}

\subsection{Free unit imaginary vector for isotropic Newtonian bodies}
\label{subsec:free-unit-vector}

In the metrically dressed setting we write the spatial part of a body quaternion
as
\[
  \bm S_i = L_{G,i}\,\widehat{\bm n}_i,
\]
where \(\widehat{\bm n}_i\) is a unit spatial direction associated with body \(i\).
For charged bodies it is natural to interpret \(\widehat{\bm n}_i\) as an
intrinsic flow direction (e.g.\ an orientation of the underlying streamlet
structure), which will in general produce non-trivial contributions in both
the symmetric vector form \(\bm B\) and the vortical form \(\bm C\).

For purely Newtonian, isotropic mass distributions, however, we may and should
distinguish between an intrinsic direction and the \emph{interaction} direction.
To reflect this, we introduce the notion of a \emph{free unit imaginary vector}
for the gravitational channel.

\begin{definition}[Free unit imaginary vector]
  A free unit imaginary vector is a unit spatial quaternion
  \(\widehat{\bm u}\in \Im\mathbb H\) whose orientation is not fixed by the
  internal structure of the body, but is freely assigned at the level of the
  interaction.
  For an isotropic Newtonian body \(i\) we take the gravitational spatial part
  of its dressed quaternion to be
  \[
    \bm S^{(G)}_i(\bm x)
    := L_{G,i}\,\widehat{\bm u}_i(\bm x),
  \]
  where, for a field point \(\bm x\), the free unit vector is chosen to be
  aligned with the radius vector from the body to that point,
  \[
    \widehat{\bm u}_i(\bm x) := \widehat{\bm r}_i(\bm x)
    = \frac{\bm x - \bm x_i}{\|\bm x - \bm x_i\|}.
  \]
\end{definition}

In other words, for an isotropic Newtonian source the gravitational channel of
the dressed quaternion is always taken to be \emph{radial} with respect to the
field point, and carries no intrinsic ``spin'' information. This has an
important structural consequence for the bilinear forms.

Consider two isotropic masses \(m_1,m_2\) at positions \(\bm x_1,\bm x_2\), and
evaluate their gravitational spatial parts at a common field point \(\bm x\).
By construction,
\[
  \bm S^{(G)}_1(\bm x) \parallel \widehat{\bm r}_1(\bm x),
  \qquad
  \bm S^{(G)}_2(\bm x) \parallel \widehat{\bm r}_2(\bm x).
\]
In the static two-body configuration the interaction is along the line joining
the bodies, so that effectively
\[
  \bm S^{(G)}_1 \parallel \bm S^{(G)}_2,
\]
and therefore their contribution to the vortical form
\[
  \bm C(\widetilde{\mathsf q}_1,\widetilde{\mathsf q}_2)
  = \bm S_1 \times \bm S_2
\]
vanishes in the purely gravitational, isotropic limit:
\[
  \bm C^{(G)} = \bm S^{(G)}_1 \times \bm S^{(G)}_2 = \bm 0.
\]

Thus, by assigning the gravitational spatial part of an isotropic body to a
free unit imaginary vector that is always chosen radial, we ensure that:
\begin{itemize}[leftmargin=*]
  \item the gravitational interaction of isotropic masses is purely scalar and
  radial, as in Newtonian gravity;
  \item there is \emph{no} spurious contribution of the gravitational channel
  to the vortical form \(\bm C\) in the static limit;
  \item all non-trivial vortical contributions in \(\bm C\) are genuinely
  associated with anisotropy and/or motion (currents), i.e.\ with the EM and
  gravitomagnetic sectors rather than with static isotropic gravity.
\end{itemize}

In contrast, for charged bodies we will keep an intrinsic unit direction
\(\widehat{\bm n}_i\) in the electric channel, which can contribute to both the
symmetric form \(\bm B\) and the vortical form \(\bm C\).
This separation between a free gravitational direction and an intrinsic
electromagnetic direction will be important when we analyze the GEM field and
its magnetic and gravitomagnetic components in \cref{sec:quat-gem-field,sec:drotors-dynamics}.


\subsection{Dimensionless flow quaternion and dressed body quaternion}
% TODO: Distinguish between a dimensionless flow quaternion \widehat{\mathsf q}
% and a metrically dressed body quaternion \widetilde{\mathsf q} with dimension of length.

\subsection{Lengths L\_E and L\_G: definitions and dimensional analysis}
% TODO: Define
% L_{E,i} = \sqrt{\GNewton/(4\pi\epszero \clight^4)}\,Q_i,
% L_{G,i} = (\GNewton/\clight^2)\,m_i,
% and discuss dimensions and physical meaning (``electric length'' and gravitational radius).

\subsection{Self-form of a dressed quaternion and rest energy}
% TODO: Consider A(\widetilde{\mathsf q}_i,\widetilde{\mathsf q}_i)
% and relate it to rest energy / the mass channel.

%==================== 6. Bilinear forms and static GEM ====================

\section{Bilinear forms for dressed bodies and static GEM}

\subsection{Forms A, B, C for dressed body quaternions}
% TODO: Write A,B,C explicitly for \widetilde{\mathsf q}_1,\widetilde{\mathsf q}_2
% and discuss the interpretation of each component in the dressed context.

\subsection{Static interaction energy of two bodies}
% TODO: Derive A_0 = A(\widetilde{\mathsf q}_1,\widetilde{\mathsf q}_2),
% define U(r) = (\clight^4/\GNewton)\,A_0 / r,
% and show that this reproduces the Coulomb + Newton potentials.

\subsection{Multi-body field and linear regime}
% TODO: Discuss superposition of many sources and the linear classical limit.

%==================== 7. Quaternionic GEM field in R^3 ====================

\section{Quaternionic GEM field in the observable three-space}

\subsection{Distance as a purely imaginary quaternion}
% TODO: Introduce R_j(\bm x) = \bm r_j and R_j^{-1} = -\widehat{\bm r}_j / r_j.

\subsection{Potential-like object \texorpdfstring{$\mathcal U_{\rm GEM}$}{U\_GEM}}
% TODO: Define \mathcal U_{\rm GEM}(\bm x) = \sum_j R_j^{-1}\,\widetilde{\mathsf q}_j.

\subsection{Field \texorpdfstring{$\mathcal F_{\rm GEM}$}{F\_GEM} and extraction of potentials}
% TODO: Define \mathcal F_{\rm GEM}(\bm x) = \sum_j R_j^{-1}\,\widetilde{\mathsf q}_j\,R_j^{-1},
% and explain how its A-channel yields the gravitational and electrostatic potentials.

%==================== 8. D-rotors and dynamics ====================

\section{D-rotors and relativistic dynamics of GEM}

\subsection{Definition of the D-rotation and its action on a flow}
% TODO: Introduce d(\zeta) = \cos(\zeta/2) + \widehat{\bm u}\sin(\zeta/2),
% and the action q' = d q d,
% with explicit formulas for T', S_u', \bm S_\perp'.

\subsection{Spatial rotors and invariance of A}
% TODO: Show that a purely spatial rotor R
% leaves A(q_1,q_2) invariant,
% while \bm B and \bm C transform as ordinary 3-vectors.

\subsection{Mixing of A and B channels under a D-rotation}
% TODO: Show how the pair (U,B_u) = (T_1T_2 - S_{u,1}S_{u,2}, B_u)
% rotates by an angle 2\zeta in the (energy–current) plane,
% and interpret this as the dynamical mixing of scalar and current-like channels.

\subsection{Example configurations}
% TODO: Discuss:
% - a purely gravitational channel (L_E = 0),
% - a purely electric channel (L_G \to 0),
% - transverse motion of a charge and the emergence of a vortical EM part.

%==================== 9. Calibration of E, B and C_phys ====================

\section{Calibration of fields \texorpdfstring{$\bm E$}{E}, \texorpdfstring{$\bm B$}{B}
and the phase-vortical object \texorpdfstring{$C_{\text{phys}}$}{Cphys}}

\subsection{Electric field from the A-channel}
% TODO: Derive \bm E(\bm r) = (1/(4\pi\epszero))\,Q\,\widehat{\bm r} / r^2
% from the scalar A-channel in the static limit.

\subsection{The vortical form C and the geometry of \texorpdfstring{$C_{\text{geom}}$}{Cgeom}}
% TODO: Show that C_{\text{geom}} \sim (Q/r^2)\,\beta\,(\widehat{\bm v} \times \widehat{\bm r})
% for a moving charge.

\subsection{Definition of \texorpdfstring{$C_{\text{phys}}$}{Cphys} and relation to \texorpdfstring{$\bm B$}{B}}
% TODO: Define C_{\text{phys}} := (1/(4\pi\epszero))\,C_{\text{geom}},
% note that its dimension matches that of \bm E,
% and write \bm B = C_{\text{phys}} / \clight in SI units.

\subsection{Interpretation of \texorpdfstring{$\epszero$}{eps0}, \texorpdfstring{$\muzero$}{mu0}
and volumetric stiffness of the vacuum}
% TODO: Discuss \muzero = 1/(\epszero \clight^2)
% and \muzero/\clight = 1/(\epszero \clight^3) as an effective ``volumetric stiffness''
% of the vortical channel of the vacuum.

%==================== 10. Hamiltonians and Lagrangians ====================

\section{Hamiltonians and Lagrangians of quaternionic GEM}

\subsection{Free-body Hamiltonian}
% TODO: Introduce H_{\rm free} = \kappa A(\widetilde{\mathsf q}_i,\widetilde{\mathsf q}_i),
% and relate it to the rest energy and proper frequency of a body.

\subsection{Interaction Hamiltonian}
% TODO: Introduce H_{\rm int} = (\clight^4/\GNewton)\,
% A(\widetilde{\mathsf q}_1,\widetilde{\mathsf q}_2)/r
% and decompose it into gravitational and EM contributions.

\subsection{Field Lagrangian in terms of A, B, C}
% TODO: Sketch a field Lagrangian in terms of A(Q,Q), \|B(Q,Q)\|^2, \|C(Q,Q)\|^2,
% and relate it to the standard EM Lagrangian and linearized GEM.

%==================== 11. Relation to Maxwell and GEM ====================

\section{Relation to Maxwell's equations and linearized GEM}

\subsection{Static equations and Poisson-type equations}
% TODO: Show how Poisson's equation for \phi_{\rm EM}
% and its gravitational analogue arise from the quaternionic field.

\subsection{Quasi-stationary regime and continuity equations}
% TODO: Show how, in the quasi-stationary regime,
% one recovers \nabla \cdot \bm E, \nabla \times \bm B,
% and continuity equations for charge and mass currents.

\subsection{Comparison with the classical GEM formalism}
% TODO: Discuss the limits L_E = 0 (pure gravity),
% L_G = 0 (pure EM), and correspondence with standard GEM formulations.

%==================== 12. Discussion and conclusions ====================

\section{Discussion and conclusions}

\subsection{Summary of results}
% TODO: Summarize the main achievements and emphasize
% the unified quaternionic language for gravity and EM.

\subsection{Limitations and open questions}
% TODO: Discuss assumptions (classical limit, linearity, etc.)
% and highlight open problems.

\subsection{Perspectives}
% TODO: Sketch possible extensions:
% inclusion of weak/strong interactions,
% quantization of the quaternionic field,
% links to cosmology and possible evolution of \GNewton, etc.

%==================== Appendices ====================

\appendix

\section{Quaternion algebra and matrix representation (details)}
% TODO: Provide more detailed algebraic and matrix derivations.

\section{Extended notation table}
% TODO: Full table of symbols used in the paper.

\section{Dimensional analysis and numerical estimates}
% TODO: Work out explicit examples for the electron, proton, etc.

\section{D-rotors, Lorentz transformations and Wigner--Thomas rotation}
% TODO: Show in detail how the D-rotor formalism reproduces Lorentz boosts
% and the Wigner--Thomas rotation.

\end{document}
