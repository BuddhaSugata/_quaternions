\documentclass[11pt,a4paper]{article}

% ====== Packages (adapt / sync with the rev.2 paper as needed) ======
\usepackage[utf8]{inputenc}
\usepackage[T1]{fontenc}
\usepackage[english]{babel}

\usepackage{lmodern}
\usepackage{microtype}

\usepackage{geometry}
\geometry{margin=2.5cm}

\usepackage{amsmath,amssymb,amsfonts,mathtools}
\usepackage{bm}
\usepackage{physics}
\usepackage{siunitx}
\usepackage{tensor}
\usepackage{csquotes}

\usepackage{graphicx}
\usepackage{xcolor}
\usepackage{tikz}
\usepackage{tikz-cd}

\usepackage{hyperref}
\hypersetup{
  colorlinks=true,
  linkcolor=blue!40!black,
  citecolor=blue!40!black,
  urlcolor=blue!60!black
}

\usepackage{amsthm} % если ещё не подключен

\theoremstyle{definition}
\newtheorem{definition}{Definition}[section] % или без [section], как тебе надо

\usepackage[nameinlink,capitalize]{cleveref}

\usepackage{enumitem}

% ====== Common macros (adapt from rev.2 as needed) ======
% TODO: copy or sync macros from the rev.2 unimetry paper here.
\newcommand{\clight}{c}
\newcommand{\GNewton}{G}
\newcommand{\epszero}{\varepsilon_0}
\newcommand{\muzero}{\mu_0}

\newcommand{\R}{\mathbb{R}}
\newcommand{\Espace}{\mathcal{E}}

\newcommand{\qh}{\hat h}  % temporal unit
\newcommand{\qvec}[1]{\bm{#1}}

\newcommand{\uhat}{\hat{\boldsymbol{\ell}}}

\title{Unimetry: A Quaternionic Gravito--Electromagnetic Formulation}
\author{Timur Abizgeldin%
  \thanks{Email: \texttt{foo@example.org}. TODO: update contact/affiliation.}}
\date{\today}

\begin{document}

\maketitle

\begin{abstract}
% TODO: Briefly describe the goal of the paper and the main idea:
% quaternionic representation of flows in unimetry, metrically dressed body quaternions,
% bilinear forms A,B,C, D-rotors, static limit (Newton + Coulomb),
% the role of C_{\text{phys}} = \clight\,\mathbf B and of \GNewton, \epszero, \muzero, \clight.
\end{abstract}

\tableofcontents

%==================== 1. Introduction ====================

\section{Introduction}

\subsection{Context and motivation}

Unimetry is a proposed phase--geometric framework in which physical systems are
described in terms of stationary flows in an underlying Euclidean proto-space
\(\Espace\).
Rather than postulating space--time as a primary arena, unimetry treats the
observed space--time geometry, relativistic kinematics, and field interactions
as effective structures derived from the orientation and coupling of such flows.
A dimensionless scalar phase potential \(\Phi:\Espace\to\mathbb{R}\) and its
gradient define a normalized flow direction; the familiar Minkowski metric and
Lorentzian phenomena then appear as particular projections of this underlying
flow geometry.

In this sense, special relativity (SR) is not the endpoint, but the first
benchmark for the framework: unimetry aims at a unified phase-based description
of kinematics, gravity and gauge interactions, with SR recovered as a specific
limit of the general construction.
The present paper develops one important sector of this programme, namely a
quaternionic gravito--electromagnetic (GEM) formulation built on top of the
unimetrical flow picture.

At the classical level, gravito--electromagnetic analogies are well known:
in the weak-field, slow-motion limit of general relativity, the Einstein
equations can be cast into a Maxwell-like form, and moving masses generate a
``gravitomagnetic'' field.
Quaternions and related algebras have also long been used to encode rotations
and the Maxwell equations in a compact way.
What unimetry adds to this landscape is a concrete phase-geometric
interpretation: a single quaternionic object encodes both the temporal and
spatial parts of a flow, and bilinear forms of such objects naturally split into
scalar, symmetric vector, and axial (vorticity-like) channels.
This suggests that gravity and electromagnetism might be viewed as different
faces of the same bilinear structure acting on suitably dressed flow
quaternions.

Our goal here is to make this statement precise.
We construct a quaternionic GEM formalism in which gravitational and
electromagnetic interactions originate from the \emph{same} bilinear machinery
applied to metrically dressed ``body quaternions''.
In particular, we show that Newton and Coulomb potentials arise as two branches
of a single scalar form, while the magnetic and gravitomagnetic sectors are
associated with a vortical bilinear form whose physical calibration reveals a
natural role for the constants \(\epszero\), \(\muzero\), \(\GNewton\) and
\(\clight\).
The resulting description remains Euclidean at the level of the proto-space,
yet reproduces relativistic kinematics and GEM fields in the observable
three-space.

\subsection{Relation to the base unimetry paper}

This work is a direct sequel to the base unimetry paper,
\emph{``Unimetry: A Phase-Space Reformulation of Special Relativity''}
(henceforth ``Paper~I'').
Paper~I develops the core phase/flow structure: the phase potential \(\Phi\),
the phase 1-form \(\alpha = d\Phi\), the normalized flow
\(\widehat{\boldsymbol\chi}\), and the calibration
\(\boldsymbol\chi = \clight\,\widehat{\boldsymbol\chi}\), together with the
derivation of the Minkowski interval and standard SR effects from a Euclidean
proto-space.
It also introduces the unimetrical D-rotation, which encodes Lorentz boosts as
Euclidean rotations in a suitable plane of \(\Espace\).

From the unimetry viewpoint, however, these SR results are only the first 
consistency test of a broader phase-based paradigm.
The present paper assumes familiarity with the conceptual setting of Paper~I,
but is written to be as self-contained as reasonably possible.
We briefly recall the key definitions of the phase proto-space, the flow vector,
and the two calibrations of the flow that lead to kinematic and energetic
interpretations.
All constructions that are essential for the GEM sector are reproduced or
adapted here; more detailed discussions of SR and cosmological applications
remain in Paper~I and are only referenced when needed.

\subsection{Main results}

The main technical contributions of this paper can be summarized as follows.

\begin{itemize}[leftmargin=*]
  \item We introduce \emph{metrically dressed body quaternions}
  \(\widetilde{\mathsf q}_i = L_{E,i}\,\qh + L_{G,i}\,\widehat{\bm n}_i\),
  whose components have the dimension of length.
  The ``electric'' and ``gravitational'' lengths
  \[
    L_{E,i} = \sqrt{\frac{\GNewton}{4\pi \epszero \clight^4}}\,Q_i,
    \qquad
    L_{G,i} = \frac{\GNewton}{\clight^2}\,m_i
  \]
  encode the charge \(Q_i\) and mass \(m_i\) of the body in a unified geometric
  fashion.
  The unit vector \(\widehat{\bm n}_i\) represents the spatial flow direction
  associated with the body.

  \item We show that the scalar bilinear form
  \[
    A(\widetilde{\mathsf q}_1,\widetilde{\mathsf q}_2)
    = L_{E,1}L_{E,2} - \bm S_1\cdot\bm S_2
  \]
  (with \(\bm S_i = L_{G,i}\widehat{\bm n}_i\)) yields, after a single global
  calibration by \(\clight^4/\GNewton\) and a geometric \(1/r\) factor, the
  combined Newton--Coulomb potential:
  \[
    U(r) = \frac{\clight^4}{\GNewton}\,\frac{A}{r}
    = \frac{1}{4\pi\epszero}\,\frac{Q_1 Q_2}{r}
      - \GNewton\,\frac{m_1 m_2}{r}.
  \]
  Thus gravity and electrostatics arise as two channels of a single invariant
  scalar form.

  \item We identify two vector-valued bilinear forms,
  \(\bm B(\widetilde{\mathsf q}_1,\widetilde{\mathsf q}_2)\) and
  \(\bm C(\widetilde{\mathsf q}_1,\widetilde{\mathsf q}_2)\), corresponding to
  the symmetric and axial parts of the quaternion product.
  In the dressed setting these naturally describe current-like and vortical
  channels.
  In particular, the vortical form \(\bm C\) reproduces the geometry of
  magnetic and gravitomagnetic fields generated by moving charges and masses.

  \item We construct a quaternionic GEM field
  \(\mathcal F_{\rm GEM}(\bm x)\) over the observable three-space by combining
  dressed source quaternions with purely imaginary distance quaternions.
  Its scalar channel reproduces the gravitational and electrostatic potentials,
  while its vortical channel yields a physically natural ``phase-vortical''
  field \(C_{\text{phys}}\) with the same dimension as \(\bm E\).
  The standard magnetic field \(\bm B\) in SI units then appears as
  \[
    \bm B = \frac{1}{\clight}\,C_{\text{phys}},
  \]
  so that the familiar \(\muzero\) and \(\epszero\) can be interpreted in terms
  of linear and areal stiffness of the vacuum, combined into an effective
  volumetric stiffness proportional to \(1/(\epszero \clight^3)\).

  \item We analyze the action of unimetrical D-rotations and ordinary spatial
  rotors on dressed quaternions.
  Pure spatial rotations act in the usual way on the vector channels and leave
  the scalar form \(A\) invariant, while D-rotations mix the scalar channel and
  the longitudinal component of \(\bm B\) in a two-dimensional
  ``energy--current'' plane.
  This provides a quaternionic encoding of relativistic kinematics in the GEM
  setting, with Lorentz-consistent transformation properties of the fields.

  \item Finally, we outline a Hamiltonian and Lagrangian formulation of the
  quaternionic GEM theory in terms of the self-form \(A\) and the norm-squares
  of \(\bm B\) and \(\bm C\), and discuss how the standard Maxwell Lagrangian
  and linearized GEM equations arise in appropriate limits.
\end{itemize}

\subsection{Structure of the paper}

The paper is organized as follows.
In \cref{sec:quat-algebra} we recall the basic quaternion algebra and introduce
the bilinear forms \(A\), \(\bm B\), and \(\bm C\) that arise from the
quaternion product, together with their matrix representation and geometric
interpretation.
\Cref{sec:phase-protospace} provides a brief overview of the unimetrical phase
proto-space, the phase potential, the flow vector, and the two calibrations of
the flow that lead to kinematic and energetic interpretations.

In \cref{sec:dressed-quats} we introduce metrically dressed body quaternions and
define the electric and gravitational lengths \(L_E\) and \(L_G\).
\Cref{sec:bilinear-static-gem} shows how the scalar form \(A\) for dressed
quaternions reproduces the static Newton and Coulomb potentials.
In \cref{sec:quat-gem-field} we construct a quaternionic GEM field over the
observable three-space and identify the scalar and vortical channels with
gravitational, electric, and magnetic sectors.

\Cref{sec:drotors-dynamics} analyzes the action of spatial rotors and D-rotors
on dressed quaternions and on the GEM field, clarifying the relativistic
transformation properties of the scalar, current-like, and vortical channels.
\Cref{sec:calibration-EB-Cphys} is devoted to the calibration of \(\bm E\) and
\(\bm B\), to the definition of the phase-vortical field \(C_{\text{phys}}\),
and to the interpretation of \(\epszero\), \(\muzero\), and \(\clight\) in
terms of vacuum stiffness.

In \cref{sec:hamiltonians-lagrangians} we outline Hamiltonian and Lagrangian
formulations of quaternionic GEM, and in
\cref{sec:maxwell-gem-comparison} we compare the resulting equations with the
standard Maxwell and linearized GEM formalisms.
Finally, \cref{sec:discussion-conclusions} discusses limitations and open
questions, and sketches possible extensions towards non-Abelian interactions
and cosmological applications.

%==================== 2. Quaternion algebra ====================

\section{Quaternion algebra and bilinear forms}
\label{sec:quat-algebra}

\subsection{Basic notation and conventions}

We denote by \(\mathbb{H}\) the real quaternion algebra, viewed as a
four-dimensional real vector space
\[
  \mathbb{H} \simeq \mathbb{R}^4(\qh,\hat{\imath},\hat{\jmath},\hat{k}),
\]
where \(\qh\) is the distinguished scalar basis element and
\(\hat{\imath},\hat{\jmath},\hat{k}\) are purely imaginary basis elements.
A general quaternion is written as
\[
  \mathbf{q} = x^\mu e_\mu
  = x^0 \qh + x^1 \hat{\imath} + x^2 \hat{\jmath} + x^3 \hat{k},
\]
with real components \(x^\mu \in \mathbb{R}\) and basis
\(e_0 := \qh\), \(e_1 := \hat{\imath}\), \(e_2 := \hat{\jmath}\),
\(e_3 := \hat{k}\).

The imaginary basis satisfies the usual quaternion relations
\[
  \hat{\imath}^2 = \hat{\jmath}^2 = \hat{k}^2 = -\qh,
  \qquad
  \hat{\imath}\hat{\jmath}=\hat{k},\quad
  \hat{\jmath}\hat{k}=\hat{\imath},\quad
  \hat{k}\hat{\imath}=\hat{\jmath},
\]
with antisymmetry under exchange of factors.
We identify \(\Im\mathbb{H}\simeq\mathbb{R}^3\) with its Euclidean inner
product \(\bm x\cdot\bm y\) and cross product \(\bm x\times\bm y\), so that
\[
  \bm x = x^1 \hat{\imath} + x^2 \hat{\jmath} + x^3 \hat{k}
  \quad\longleftrightarrow\quad
  (x^1,x^2,x^3) \in \mathbb{R}^3.
\]

Quaternionic conjugation is defined by
\[
  \overline{\mathbf{q}} := x^0 \qh - x^1 \hat{\imath} - x^2 \hat{\jmath}
                           - x^3 \hat{k},
\]
and the norm is \(\|\mathbf{q}\|^2 = \mathbf{q}\,\overline{\mathbf{q}}
= \overline{\mathbf{q}}\,\mathbf{q} = (x^0)^2 + (x^1)^2 + (x^2)^2 + (x^3)^2\).

When convenient we will write a quaternion as \(\mathbf{q}=(T,\bm S)\) with
\[
  T := x^0,\qquad \bm S := x^1 \hat{\imath} + x^2 \hat{\jmath} + x^3 \hat{k},
\]
emphasizing the split into scalar and vector parts.

\paragraph{Universal imaginary unit.}

Besides the fixed imaginary basis \(\{\hat{\imath},\hat{\jmath},\hat{k}\}\)
it will be convenient to introduce a ``universal'' unit imaginary quaternion
\(\uhat\), defined abstractly by
\begin{equation}
  \uhat^2 = -\qh,
  \qquad
  \|\uhat\| = 1.
  \label{eq:u-hat-def}
\end{equation}
Geometrically, \(\uhat\) should be understood as a \emph{joker} direction: in
any concrete configuration it is identified with the unit vector along the
relevant interaction axis (for instance, the radial direction between two
approximately isotropic Newtonian bodies).
Algebraically, \(\uhat\) behaves as any other unit imaginary quaternion, and
all scalar invariants such as \(A(q,q)\) remain well defined when the spatial
part of \(q\) is restricted to the one-dimensional subspace \(\mathbb{R}\uhat\).

When describing isotropic Newtonian bodies we will often use quaternions of the
form
\begin{equation}
  q = T\,\qh + S\,\uhat,
  \label{eq:radial-quat}
\end{equation}
with \(T,S\in\mathbb{R}\), so that all interaction channels reduce to the
radial line spanned by \(\uhat\).
This eliminates spurious transverse contributions in the vortical form
\(\bm C\) for purely radial gravito--electric configurations.


\subsection{Quaternion product and decomposition into A, B, C forms}

Let
\[
  \mathbf{q}_1 = x^\mu e_\mu
  = x^0 \qh + x^1 \hat{\imath} + x^2 \hat{\jmath} + x^3 \hat{k},
\]
\[
  \mathbf{q}_2 = y^\nu e_\nu
  = y^0 \qh + y^1 \hat{\imath} + y^2 \hat{\jmath} + y^3 \hat{k},
\]
with \(\mathbf{q}_1,\mathbf{q}_2 \in \mathbb{H}\).
Their quaternion product can be expanded term by term as
\begin{align}
  \mathbf{q}_1 \circ \mathbf{q}_2
  &= (x^0 \qh + x^1 \hat{\imath} + x^2 \hat{\jmath} + x^3 \hat{k})
     (y^0 \qh + y^1 \hat{\imath} + y^2 \hat{\jmath} + y^3 \hat{k}) \nonumber\\
  &= (x^0 y^0 \qh^2 + x^1 y^1 \hat{\imath}^2 + x^2 y^2 \hat{\jmath}^2
      + x^3 y^3 \hat{k}^2) \nonumber\\
  &\quad
   + (x^0 y^1 \qh \hat{\imath} + x^0 y^2 \qh \hat{\jmath}
      + x^0 y^3 \qh \hat{k}
      + x^1 y^0 \hat{\imath} \qh + x^2 y^0 \hat{\jmath} \qh
      + x^3 y^0 \hat{k} \qh) \nonumber\\
  &\quad
   + (x^1 y^2 \hat{\imath} \hat{\jmath}
      + x^1 y^3 \hat{\imath} \hat{k}
      + x^2 y^1 \hat{\jmath} \hat{\imath}
      + x^2 y^3 \hat{\jmath} \hat{k}
      + x^3 y^1 \hat{k} \hat{\imath}
      + x^3 y^2 \hat{k} \hat{\jmath}).
  \label{eq:q-product-expanded-coords}
\end{align}

Using the multiplication rules, this can be organised into three bilinear
contributions:
\begin{align}
  \mathbf{q}_1 \circ \mathbf{q}_2
  &= ( \mathbf{q}_1 * \mathbf{q}_2 )
   + ( \mathbf{q}_1 \diamond \mathbf{q}_2 )
   + ( \mathbf{q}_1 \times \mathbf{q}_2 ),
\end{align}
where:
\begin{itemize}[leftmargin=*]
  \item \(\mathbf{q}_1 * \mathbf{q}_2\) collects the purely scalar terms,
  \item \(\mathbf{q}_1 \diamond \mathbf{q}_2\) collects the mixed scalar--vector
  terms,
  \item \(\mathbf{q}_1 \times \mathbf{q}_2\) collects the purely vector--vector
  terms.
\end{itemize}

Explicitly, one finds the familiar invariant decomposition
\begin{align}
  \mathbf{q}_1 \circ \mathbf{q}_2
  &= \Bigl(
       \color{red}{x^0 y^0 - x^1 y^1 - x^2 y^2 - x^3 y^3}
     \Bigr)\,\qh
   \nonumber\\
  &\quad
   + \color{green}{\bigl(x^0 \bm y + y^0 \bm x\bigr)}
   + \color{blue}{\bigl(\bm x \times \bm y\bigr)},
  \label{eq:q-product-ABC-coords}
\end{align}
where
\[
  \bm x := x^1 \hat{\imath} + x^2 \hat{\jmath} + x^3 \hat{k},
  \qquad
  \bm y := y^1 \hat{\imath} + y^2 \hat{\jmath} + y^3 \hat{k}.
\]
This suggests three natural bilinear maps:
\begin{align}
  A(\mathbf{q}_1,\mathbf{q}_2)
  &:= x^0 y^0 - x^1 y^1 - x^2 y^2 - x^3 y^3,
  \label{eq:A-form-coords}\\
  \bm B(\mathbf{q}_1,\mathbf{q}_2)
  &:= x^0 \bm y + y^0 \bm x,
  \label{eq:B-form-coords}\\
  \bm C(\mathbf{q}_1,\mathbf{q}_2)
  &:= \bm x \times \bm y.
  \label{eq:C-form-coords}
\end{align}
In terms of these,
\begin{equation}
  \mathbf{q}_1 \circ \mathbf{q}_2
  = \bigl(\color{red}{A(\mathbf{q}_1,\mathbf{q}_2)}\bigr)\,\qh
    + \color{green}{\bm B(\mathbf{q}_1,\mathbf{q}_2)}
    + \color{blue}{\bm C(\mathbf{q}_1,\mathbf{q}_2)}.
  \label{eq:q-product-ABC}
\end{equation}

It is often convenient to view \eqref{eq:q-product-ABC-coords} in a tensor-like
form.
We can write
\[
  \mathbf{q}_1 \circ \mathbf{q}_2
  = \sum_{\mu,\nu=0}^3
    \Bigl(
      \color{red}{A_{\mu\nu}}\,\qh
      + \color{green}{B_{\mu\nu}}
      + \color{blue}{C_{\mu\nu}}
    \Bigr)\,x^\mu y^\nu,
\]
with three \(4\times 4\) coefficient matrices:
\begin{align}
  \color{red}{A_{\mu\nu}}
  &= \begin{pmatrix}
       1 & 0 & 0 & 0 \\
       0 & -1 & 0 & 0 \\
       0 & 0 & -1 & 0 \\
       0 & 0 & 0 & -1
     \end{pmatrix},
  \label{eq:A-matrix}\\[0.5em]
  \color{green}{B_{\mu\nu}}
  &= \begin{pmatrix}
       0 & \hat{\imath} & \hat{\jmath} & \hat{k} \\
       \hat{\imath} & 0 & 0 & 0 \\
       \hat{\jmath} & 0 & 0 & 0 \\
       \hat{k} & 0 & 0 & 0
     \end{pmatrix},
  \label{eq:B-matrix}\\[0.5em]
  \color{blue}{C_{\mu\nu}}
  &= \begin{pmatrix}
       0 & 0 & 0 & 0 \\
       0 & 0 & \hat{k} & -\hat{\jmath} \\
       0 & -\hat{k} & 0 & \hat{\imath} \\
       0 & \hat{\jmath} & -\hat{\imath} & 0
     \end{pmatrix}.
  \label{eq:C-matrix}
\end{align}
Here \(\color{red}{A_{\mu\nu}}\) is the usual Minkowski-like bilinear form
\(\mathrm{diag}(1,-1,-1,-1)\) acting on the coordinate components, while
\(\color{green}{B_{\mu\nu}}\) and \(\color{blue}{C_{\mu\nu}}\) collect the
symmetric and antisymmetric vector-valued pieces of the product.

\subsection{Geometric interpretation and tensor structure of A, B, C}

The decomposition \eqref{eq:q-product-ABC} and the matrices
\eqref{eq:A-matrix}--\eqref{eq:C-matrix} make explicit that the quaternion
product can be regarded as the contraction of a rank-\((0,2)\) object with two
four-vectors:
\[
  \mathbf{q}_1 \circ \mathbf{q}_2
  = \bigl(
      \color{red}{A} + \color{green}{B} + \color{blue}{C}
    \bigr)_{\mu\nu}\,x^\mu y^\nu,
\]
with three structurally distinct blocks:
\begin{itemize}[leftmargin=*]
  \item \(\color{red}{A_{\mu\nu}}\) is a symmetric scalar bilinear form of
  signature \((+,-,-,-)\).
  In the unimetrical context it will play the role of an energy-like invariant
  and will generate both Newtonian and Coulomb potentials once we pass to
  dressed body quaternions.

  \item \(\color{green}{B_{\mu\nu}}\) is symmetric and vector-valued; it couples
  the scalar component to the spatial components.
  It will later be interpreted as a current-like channel, encoding the coupling
  between temporal and spatial parts of dressed flow quaternions.

  \item \(\color{blue}{C_{\mu\nu}}\) is antisymmetric and vector-valued; it
  encodes the cross product \(\bm x\times\bm y\) of the spatial parts and thus
  represents a vorticity (axial) channel.
  This will underlie the magnetic and gravitomagnetic sectors of the GEM field.
\end{itemize}

In summary, the elementary quaternion product already contains, in a rigid
algebraic way, the three channels that we will later reinterpret as
\begin{enumerate}[label=(\roman*),leftmargin=*]
  \item an energy-like scalar invariant \(A\),
  \item a current-like symmetric vector channel \(\bm B\),
  \item a vortical (axial) vector channel \(\bm C\).
\end{enumerate}
In the next section we recall how unimetry associates physical flows and an
effective space--time structure to quaternions, so that these three forms can
be given a gravito--electromagnetic meaning.



%==================== 3. Phase proto-space in unimetry ====================

\section{Phase proto-space and flow: brief unimetry overview}
\label{sec:phase-protospace}

\subsection{Proto-space, phase potential and phase 1-form}

In unimetry the basic kinematical arena is a Euclidean (or, more generally,
Hilbert) proto-space \((\Espace,\langle\cdot,\cdot\rangle)\).
Points of \(\Espace\) will be denoted by \(X\), and the inner product
\(\langle\cdot,\cdot\rangle\) is used to identify tangent and cotangent spaces
via the Riesz isomorphism.
For the purposes of this paper one may think of \(\Espace\) as a finite- or
countable-dimensional real Hilbert space.

The fundamental scalar field of unimetry is a dimensionless \emph{phase
potential}
\[
  \Phi : \Espace \to \mathbb{R}.
\]
From \(\Phi\) we obtain the \emph{phase 1-form}
\[
  \alpha := d\Phi,
\]
which is a smooth 1-form on \(\Espace\).
At each point \(X\in\Espace\), the value \(\alpha_X\) is a linear functional on
the tangent space \(T_X\Espace\):
\[
  \alpha_X : T_X\Espace \to \mathbb{R},
  \qquad
  \alpha_X(V) = d\Phi_X(V).
\]
Using the inner product, we define the gradient \(\nabla\Phi(X)\in T_X\Espace\)
by the standard relation
\[
  \alpha_X(V)
  = d\Phi_X(V)
  = \langle \nabla\Phi(X), V\rangle,
  \qquad
  \forall\,V \in T_X\Espace.
\]
Thus \(\alpha\) and \(\nabla\Phi\) carry the same information; the former is
covariant, the latter contravariant.

Physically, the phase potential \(\Phi\) encodes the global phase structure of
the underlying flow, while the phase 1-form \(\alpha\) and the gradient
\(\nabla\Phi\) encode local directions in which the phase changes most rapidly.
The key idea of unimetry is to use this structure to define a canonical flow
through \(\Espace\).

\subsection{Flow vector and normalization}
\label{subsec:flow-normalization}

Whenever \(\nabla\Phi(X)\neq 0\), we define the \emph{normalized flow direction}
at \(X\) by
\begin{equation}
  \widehat{\boldsymbol\chi}(X)
  := \frac{\nabla\Phi(X)}{\|\nabla\Phi(X)\|},
  \qquad
  \|\nabla\Phi(X)\|
  := \sqrt{\langle\nabla\Phi(X),\nabla\Phi(X)\rangle}.
  \label{eq:chi-hat-def}
\end{equation}
Thus \(\widehat{\boldsymbol\chi}(X)\) is a unit vector in \(T_X\Espace\) pointing
along the steepest phase ascent.
We then introduce the \emph{physical flow vector} by a global calibration
\begin{equation}
  \boldsymbol\chi(X)
  := \clight\,\widehat{\boldsymbol\chi}(X),
  \qquad
  \|\boldsymbol\chi(X)\| \equiv \clight,
  \label{eq:chi-phys-def}
\end{equation}
where \(\clight\) is the speed of light.
In other words, in unimetry the physical flow is a unit-speed curve in
\(\Espace\) with respect to the fixed scale \(\clight\).

A flow line (or \emph{stream}) is then a curve
\(\gamma : \lambda \mapsto X(\lambda)\in\Espace\) whose tangent vector is
everywhere aligned with the physical flow:
\begin{equation}
  \dot{X}(\lambda)
  := \frac{dX}{d\lambda}
  = \boldsymbol\chi\bigl(X(\lambda)\bigr),
  \qquad
  \|\dot{X}(\lambda)\| = \clight.
  \label{eq:flow-line-def}
\end{equation}
The parameter \(\lambda\) is a proto-space parameter, not yet identified with
any observed time.
The geometric content of \eqref{eq:flow-line-def} is simply that physical
objects are represented by flows of constant Euclidean speed \(\clight\) in the
proto-space.

\subsection{Intrinsic angle, proper time and correspondence with SR}
\label{subsec:interval-SR}

In unimetry a macroscopic body \(B\) is represented not by a single flow line,
but by an ensemble of streamlets with weights \(w_a\) and tilt angles
\(\Theta_a\) relative to the body's self–time fibre.\footnote{For the detailed
construction see Paper~I, \S\ref{sec:intrinsic-angle} there.}
On this ensemble one defines the temporal second moment
and the spatial shape tensor as
\begin{equation}
  T_B:=\sum_a w_a \cos^2\Theta_a,
  \qquad
  \mathbf C_B:=\sum_a w_a \sin^2\Theta_a\;
               \mathbf u_a\!\otimes\!\mathbf u_a,
  \label{eq:TB-CB-here}
\end{equation}
where \(0<T_B\le 1\), \(\mathbf C_B\) is a symmetric positive semidefinite
tensor on the body's three-surface, and \(\mathbf u_a\) are unit spatial
directions of the streamlets’ projections.
Operationally, \(T_B\) captures the aggregate fraction of flow carried in the
orthogonal (self–time) fibre, while \(\mathbf C_B\) encodes the anisotropic
distribution of spatial projections across the body.

From these second moments one can define an \emph{intrinsic angle}
\(\zeta\in[0,\tfrac{\pi}{2}]\) as an effective statistical parameter of the
ensemble.
Introducing
\[
  C:=\sum_a w_a\cos 2\Theta_a,\qquad
  S:=\sum_a w_a\sin 2\Theta_a,
\]
there exists a unique \(\zeta\) such that
\begin{equation}
  (\cos 2\zeta,\ \sin 2\zeta) = (C,S)
  \quad\Longleftrightarrow\quad
  T_B = \tfrac12(1+C) = \cos^2\!\zeta,
  \qquad
  \operatorname{tr}\mathbf C_B
    = \tfrac12(1-C) = \sin^2\!\zeta .
  \label{eq:zeta-def-here}
\end{equation}
We call \(\zeta\) the \emph{intrinsic angle} of the body.
It aggregates the second–moment information \((T_B,\mathbf C_B)\) into a single
scalar and should be thought of as a \emph{statistical} internal parameter:
it is \emph{not} a geometric direction and is not attached to any particular
flow line.

The intrinsic angle controls the rate at which the body’s own proper time
\(\tau_B\) accumulates with respect to the phase parameter \(\chi\) used to
parametrize the flow in proto-space.
In the calibrated gauge \(\|\boldsymbol\chi\|=\clight\), one has
\begin{equation}
  d\tau_B = \cos\zeta\,d\chi,
  \label{eq:dtau-zeta}
\end{equation}
so that the temporal second moment \(T_B\) appears as
\(T_B=\cos^2\!\zeta = (d\tau_B/d\chi)^2\).
The corresponding intrinsic metric of the body,
as a quadratic form on \((d\chi,d\boldsymbol\ell)\), reads
\begin{equation}
  ds_B^2
  := \clight^2\,d\tau_B^2 - d\boldsymbol\ell^{\!\top}\mathbf C_B\,d\boldsymbol\ell
   = \clight^2 T_B\,d\chi^2
     - d\boldsymbol\ell^{\!\top}\mathbf C_B\,d\boldsymbol\ell .
  \label{eq:body-metric-here}
\end{equation}
For an isotropic texture one has
\(\mathbf C_B = \tfrac{\sin^2\!\zeta}{3}\,\mathbf I_S\), and with the rest
gauge \(T_B\equiv 1\) this reduces to the familiar Minkowski form in the
body’s rest frame (up to the overall phase gauge \(d\chi\)).

In the full unimetrical construction the intrinsic angle \(\zeta\) is combined
with a kinematic angle \(\vartheta\) (associated with the relative motion
between bodies) and, when present, with a gravitational angle \(\phi\)
(associated with an external tilt field).
The resulting time-rate factor factorises into intrinsic, kinematic, and
gravitational contributions.
For the purposes of the present GEM paper, we only need the following
structural facts:

\begin{itemize}[leftmargin=*]
  \item The intrinsic angle \(\zeta\) is a scalar \emph{second-moment} parameter
  of a body, not a direction: it encodes how the flow budget is split between
  self–time and spatial channels in the ensemble of streamlets.

  \item The proper time \(\tau_B\) along the body’s worldline is related to the
  phase parameter \(\chi\) by \eqref{eq:dtau-zeta}, and the body’s intrinsic
  metric takes the Minkowski form \eqref{eq:body-metric-here} once the rest
  gauge is fixed.

  \item The relativistic kinematics of unimetry can therefore be formulated
  entirely in terms of phase flow and second-moment data, with the usual SR
  interval emerging as a derived object; we will reuse this structure when
  interpreting the scalar form \(A\) as an energy-like invariant for dressed
  quaternions.
\end{itemize}

\subsection{Notation table}
\label{subsec:notation-table}

For reference, we collect here the main unimetrical symbols used in the
remainder of the paper.  A more extensive table can be found in Paper~I; the
subset below is chosen to make the present text self-contained.

\begin{table}[h]
  \centering
  \begin{tabular}{ll}
    \hline
    Symbol & Meaning \\ \hline
    \(\Espace\) & Euclidean/Hilbert proto-space with inner product
                   \(\langle\cdot,\cdot\rangle\) \\
    \(\Phi : \Espace\to\mathbb{R}\) & dimensionless phase potential \\
    \(\alpha = d\Phi\) & phase 1-form, \(\alpha_X(V)=\langle\nabla\Phi(X),V\rangle\) \\
    \(\nabla\Phi(X)\) & gradient of \(\Phi\) at \(X\), defined via the inner product \\
    \(\widehat{\boldsymbol\chi}(X)\) &
      normalized flow direction, \(\widehat{\boldsymbol\chi}=\nabla\Phi/\|\nabla\Phi\|\) \\
    \(\boldsymbol\chi(X)\) &
      physical flow vector, \(\boldsymbol\chi=\clight\,\widehat{\boldsymbol\chi}\),
      \(\|\boldsymbol\chi\|=\clight\) \\
    \(\gamma(\lambda)\) &
      flow line in \(\Espace\) with tangent \(\dot{X}=\boldsymbol\chi\) \\
    \(\widehat{\bm u}\) &
      unit rest direction associated with an observer (local temporal axis) \\
    \(\widehat{\bm n}\) &
      unit spatial direction orthogonal to \(\widehat{\bm u}\) \\
    \(\zeta\) &
      flow angle between \(\boldsymbol\chi\) and \(\widehat{\bm u}\), see
      \eqref{eq:chi-angle-decomp} \\
    \(\delta T\) &
      effective temporal increment for the observer, see \eqref{eq:deltaT-def} \\
    \(\delta\bm x\) &
      effective spatial increment in the observer's rest space,
      see \eqref{eq:deltax-def} \\
    \(\delta s^2\) &
      effective interval, \(\delta s^2 = \clight^2\delta T^2 - \|\delta\bm x\|^2\),
      see \eqref{eq:interval-zeta} \\
    \hline
  \end{tabular}
  \caption{Key unimetrical quantities used in the quaternionic GEM construction.}
  \label{tab:unimetry-notation}
\end{table}

In the next sections we introduce two calibrations of the flow --- one
kinematic and one energetic --- which will allow us to interpret the scalar
form \(A\) as an energy-like invariant and to define metrically dressed body
quaternions suitable for the gravito--electromagnetic setting.


%==================== 4. Calibrations of the flow and energy ====================

\section{Flow calibrations and energy-like functionals}
\label{sec:flow-calibrations}

In the unimetrical picture, physical bodies are represented by flows of
constant Euclidean speed \(\clight\) in the proto-space \(\Espace\), while the
phase potential \(\Phi\) provides a dimensionless scalar structure along these
flows.
To connect this geometric description to observable kinematics and to
energy-like quantities, we need to relate the flow parameter to time
coordinates and to introduce a structural scale.
This section introduces two complementary calibrations of the flow parameter
and shows how they lead to a factorized expression for energy, which will later
be recast in quaternionic terms.

\subsection{Flow parameter and phase frequencies}
\label{subsec:chi-and-phi}

Let \(\gamma:\chi\mapsto X(\chi)\) be a flow line in \(\Espace\).
We take the flow parameter \(\chi\) to have units of time,
\[
  [\chi] = \mathrm{s},
\]
and we assume that along the worldline of a given body \(B\) the phase
potential \(\Phi\) depends smoothly on \(\chi\).
We define the \emph{phase frequency} per unit \(\chi\) by
\begin{equation}
  \omega_\chi(\chi)
  := \frac{d\Phi}{d\chi},
  \qquad
  [\omega_\chi] = \mathrm{s}^{-1}.
  \label{eq:omega-chi-def}
\end{equation}
In many situations it is convenient to consider a gauge in which
\(\omega_\chi\) is constant along the worldline of a given body; in that case
\(\Phi(\chi)=\omega_\chi\,\chi\).
The product \(\omega_\chi \chi\) is always dimensionless.

The geometric flow vector \(\boldsymbol\chi\) is normalized by
\(\|\boldsymbol\chi\|\equiv\clight\), cf.\ \eqref{eq:chi-phys-def}, so that the
flow is a unit-speed (with respect to \(\clight\)) curve in proto-space.
The different calibrations introduced below relate \(\chi\) to observable
time parameters and thus furnish a bridge between the proto-space flow and
physical time measurements.

\subsection{Kinematic vs structural calibration of the flow}
\label{subsec:kin-vs-struct}

The unimetrical framework admits two complementary ways of calibrating the
flow, which correspond to two different physical questions.

\paragraph{Kinematic calibration.}
In the first calibration we fix the magnitude of the flow vector in proto-space
by \eqref{eq:chi-phys-def}.

Geometrically this means that the flow moves with Euclidean speed \(c\) in
\(\Espace\), and only its direction is allowed to change.
The kinematic angle \(\vartheta\) then describes how the flow is tilted between
the temporal axis and a single spatial direction, and the resulting family of
quaternionic \(D\)-rotations reproduces the usual Lorentz kinematics of special
relativity: time dilation, length contraction, and Wigner--Thomas rotations.

This calibration effectively collapses the internal structure of a Newtonian
body: the body is treated as a featureless point flow.
We see how the total flow is redistributed between ``time'' and one spatial
direction, but we do not resolve the internal ensemble of streamlets inside the
body.

\paragraph{Structural (energetic) calibration.}
To access the internal structure of a Newtonian body --- viewed as a collection
of streamlets --- we adopt a different calibration.
Instead of fixing $\|\boldsymbol\chi\|$, we treat the full \emph{cyclic flow}
of the body as invariant and let the geometry of its internal cycle vary.

Conceptually, one may introduce a primary flow parameter $s$ along the
streamlets. The global phase $\chi=\chi(s)$ and the proper time
$\tau=\tau(s)$ are then two cyclic coordinates extracted from the same flow.
In the cyclic picture there are two characteristic radii: a ``phase''
radius $R_1$ for the $\chi$--cycle and a ``temporal'' radius $R_2$ for the
local-time cycle. The normalized proto-velocity has temporal share
$\widetilde X_0=\cos\zeta_B$, so that the component of the flow tangent to
either cycle is proportional to $\widetilde X_0$.

Operationally, local time is defined through the rate at which the
phase coordinate advances:
\begin{equation}
  \dot\chi := \frac{d\chi}{d\tau}
  = \frac{d\chi/ds}{d\tau/ds},
\end{equation}
where $\tau$ is the body's proper time.
When both the phase cycle and the time cycle are taken along the flow
direction, the same projection factor $\widetilde X_0$ affects them in
\emph{dual} ways:

\begin{itemize}
  \item the phase increment per full flow loop scales as
        $\Delta\chi \propto R_1\,\widetilde X_0$, because the angular
        speed along the phase circle is proportional to the tangential
        component of the proto-velocity;
  \item the proper-time increment is defined as the \emph{period} of the
        internal clock, $\Delta\tau = 1/\nu$, with frequency
        $\nu \propto \widetilde X_0$; hence
        $\Delta\tau \propto R_2 / \widetilde X_0$.
\end{itemize}

Taking the ratio we obtain
\begin{equation}
  \dot\chi
  = \frac{d\chi}{d\tau}
  \sim \frac{\Delta\chi}{\Delta\tau}
  \propto
  \frac{R_1\,\widetilde X_0}{R_2/\widetilde X_0}
  = \frac{R_1}{R_2}\,\widetilde X_0^2.
\end{equation}
Thus, in the structural calibration we write
\begin{equation}
  \dot\chi = k\,\widetilde X_0^2,
\end{equation}
where the dimensionless structural parameter
$k := R_1/R_2$ characterizes the geometry of the internal cycle.
The quantity $\dot\chi$ measures how many ``phase steps'' $\chi$ are
performed per unit proper time and serves as a natural structural invariant
of the body.

We now demand that the full magnitude of the flow in local time,
$\dot{\mathcal H}$, remain invariant under changes of the temporal projection
$\widetilde X_0$ of the proto-velocity.
This condition fixes the relation between $k$ and $\widetilde X_0$ and allows
us to interpret $\dot\chi$ (up to a universal conversion factor) as a
mass-like parameter of the body.
The kinematic redistribution of the flow between temporal and spatial
directions of observer is then encoded by the $D$-rotations, while the structural
parameter $\dot\chi$ captures the invariant ``cyclic strength'' of the
Newtonian body.

In the quaternionic language that follows, this calibration is reflected in the
fact that the scalar self-form $A(q,q)$ on a dressed body quaternion plays
the role of a Lagrangian-like functional, whereas the Euclidean self-form
$E(q,q)$ yields a Hamiltonian-like functional, i.e.\ the full energy of the
body in this structural channel.



\subsection{Cyclic origin of local time and proto-parameter speed}
\label{subsec:cyclic-time}

We now make explicit a key unimetrical assumption: local time arises from a
cyclic action of the flow along a compactified temporal axis \(x_0\).
For simplicity we model the geometry of this axis by a circle of variable
radius and show how the ratio between the proto-parameter \(\chi\) and the
proper time \(\tau\) emerges as a frequency.

Consider two circles in the \((\chi,\tau)\)-plane:
\begin{itemize}[leftmargin=*]
  \item a ``phase circle'' of radius \(R_1\) in the \(\chi\)-direction,
  representing one phase step \(\Delta\chi = 2\pi R_1\);
  \item a ``time circle'' of radius \(R_2\) in the \(\tau\)-direction, along
  which the flow advances with a characteristic proto-space speed
  component \(\widetilde X_0\) along the temporal axis.
\end{itemize}
One full turn along the time circle then corresponds to a proper-time
increment
\begin{equation}
  \Delta\tau = \frac{2\pi R_2}{\widetilde X_0}.
\end{equation}
The frequency of phase ticks with respect to proper time is
\begin{equation}
  \nu
  := \frac{\Delta\chi}{\Delta\tau}
   = \frac{2\pi R_1}{2\pi R_2/\widetilde X_0}
   = \underbrace{\frac{R_1}{R_2}}_{=:k}\,\widetilde X_0.
  \label{eq:nu-k-X0tilde}
\end{equation}
In the high-frequency limit \(R_1\to 0\) the discrete ratio becomes a
derivative and we obtain
\begin{equation}
  \nu \;\xrightarrow{R_1\to 0}\;
  \frac{d\chi}{d\tau}
  =: \dot\chi
  \quad\Rightarrow\quad
  \boxed{\dot\chi = k\,\widetilde X_0},
  \label{eq:chi-dot-k-X0tilde}
\end{equation}
with dimensionless rate \(\dot\chi\) and dimensionless structural ratio
\(k=R_1/R_2\).
This is the fundamental relation between the accumulation of the proto-parameter
\(\chi\) and the proper time \(\tau\): local time is the frequency of phase
ticks generated by the cyclic flow along the compactified temporal axis.

The same geometric picture yields a conversion factor between proto-space
velocities (measured per unit \(\chi\)) and velocities measured per unit
proper time.
Let \(\widetilde{\mathcal H}\) be the Euclidean norm of the proto-space flow
per unit \(\chi\), with components \(\widetilde X_\mu\).
Using the chain rule we have
\begin{equation}
  \dot{\mathcal H}
  := \frac{d\mathcal H}{d\tau}
   = \frac{d\mathcal H}{d\chi}\,\frac{d\chi}{d\tau}
   = \widetilde{\mathcal H}\,\dot\chi.
  \label{eq:Hdot-chain-rule}
\end{equation}
Imposing the unimetrical calibration in which the magnitude of the proto-space
flow equals its temporal component,
\begin{equation}
  \widetilde{\mathcal H} = \widetilde X_0,
\end{equation}
and using \eqref{eq:chi-dot-k-X0tilde} we obtain
\begin{equation}
  \dot{\mathcal H}
  = k\,\widetilde X_0^2.
  \label{eq:Hdot-k-X0tilde2}
\end{equation}
This quantity \(\dot{\mathcal H}\) can be interpreted as the full flow
magnitude in local time: it measures how fast the flow advances in proto-space
per unit proper time.

Requiring that the full magnitude of the flow in local time be preserved under
rotations of the proto-space velocity (i.e.\ under changes of the temporal
projection \(\widetilde X_0\)) leads to the constraint
\begin{equation}
  k = \widetilde X_0^{-2},
  \label{eq:k-X0tilde-invariance}
\end{equation}
so that \(\dot{\mathcal H}\) becomes a universal constant.
In physical units this constant is identified with the speed of light:
\begin{equation}
  \dot{\mathcal H} = \clight
  \quad\Longleftrightarrow\quad
  \text{the speed of light is the invariant magnitude of the flow in local
  time.}
\end{equation}
In this sense the constancy of \(\clight\) is reinterpreted as the constancy
of the Euclidean flow speed in proto-space, while variations of its projection
between temporal and spatial directions are captured by the kinematic angle
\(\vartheta\).

The relation
\begin{equation}
  d\chi = k\,\widetilde X_0\,d\tau
\end{equation}
shows that the proto-parameter \(\chi\) plays the role of a global invariant
for transformations mixing temporal and spatial channels.
Expressing the interval \(dS^2\) in terms of \(d\chi\) and the components
\(\widetilde X_\mu\) one recovers the Minkowski form
\begin{equation}
  dS^2 = \dot X_0^2\,d\tau^2
          - dx_1^2 - dx_2^2 - dx_3^2,
\end{equation}
i.e.\ the ``inverted'' Euclidean norm of the flow once the temporal geometry
is taken into account.
For a body of fixed structural complexity (fixed \(k\)) the quantity
\(\widetilde X_0\) must compensate changes in the local rate of time, whereas
for a fixed full magnitude \(\dot{\mathcal H}\) the ratio \(k\) must adjust so
that the cyclic construction of time remains compatible with the invariance of
the interval.


\subsection{Kinematic (phase) calibration}
\label{subsec:phase-calibration}

Let us choose an inertial laboratory frame with coordinate time \(t\) and
three-position \(\bm x\).
In the \emph{kinematic calibration} we identify the flow parameter \(\chi\)
with the lab time:
\begin{equation}
  \chi = t,
  \qquad
  \frac{d\chi}{dt} = 1.
  \label{eq:chi-equals-t}
\end{equation}
Along the worldline of a body \(B\) we may then write
\[
  \Phi(t) = \omega_\chi\,t,
\]
with \(\omega_\chi = d\Phi/dt\) in that frame.

The normalized flow direction can be decomposed with respect to the
lab-frame temporal unit vector \(\widehat{\bm u}\) and a spatial unit vector
\(\widehat{\bm n}\) in the rest space:
\begin{equation}
  \widehat{\boldsymbol\chi}
  = \cos\vartheta\,\widehat{\bm u}
    + \sin\vartheta\,\widehat{\bm n},
  \qquad
  \|\widehat{\bm n}\|=1,
  \label{eq:chi-split-theta}
\end{equation}
where \(\vartheta\in[0,\pi/2]\) is a \emph{kinematic angle}.
For a body moving with constant lab three-velocity
\(\bm v = d\bm x/dt\) we impose
\begin{equation}
  \sin\vartheta = \beta := \frac{\|\bm v\|}{\clight},
  \qquad
  \cos\vartheta = \sqrt{1-\beta^2}.
  \label{eq:v-theta-calibration}
\end{equation}
This calibration ensures that the spatial projection of the flow has magnitude
\(\|\bm v\|=\clight\sin\vartheta\), while the temporal projection matches the
usual factor \(\sqrt{1-\beta^2}\) that appears in time dilation.

Using \eqref{eq:dtau-zeta} specialized to the kinematic sector and
\eqref{eq:chi-equals-t}, we obtain the standard relation between proper time
\(\tau_B\) and lab time:
\[
  \frac{d\tau_B}{dt}
  = \cos\vartheta
  = \sqrt{1-\beta^2},
\]
with
\begin{equation}
  \gamma(\vartheta)
  := \frac{1}{\sqrt{1-\beta^2}}
   = \frac{1}{\cos\vartheta}
  \label{eq:gamma-theta}
\end{equation}
the usual Lorentz factor.
Thus the kinematic angle \(\vartheta\) encodes the purely kinematic tilt of the
flow relative to the chosen inertial frame.

\subsection{Structural parameter and volumetric coefficient}
\label{subsec:struct-kappa}

Besides kinematic tilts, unimetry allows for an internal rescaling of the
cyclic structure of the flow.
A convenient way to visualize this is to consider a pair of circles in the
\((\chi,\tau)\)-plane:

\begin{itemize}[leftmargin=*]
  \item a ``phase circle'' of radius \(R_1\) in the \(\chi\)-direction, which
  measures the size of a single phase step \(\Delta\chi\);
  \item a ``time circle'' of radius \(R_2\) in the \(\tau\)-direction, along
  which the flow executes a cyclic motion with some characteristic proper-time
  angular frequency \(\omega_\ast\) (units \(\mathrm{s}^{-1}\)).
\end{itemize}

One full turn on the phase circle corresponds to a phase increment
\(\Delta\chi = 2\pi R_1\).
One full turn on the time circle corresponds to a proper-time increment
\begin{equation}
  \Delta\tau = \frac{2\pi R_2}{\omega_\ast},
\end{equation}
since the circumference is \(2\pi R_2\) and the ``speed'' along the time
circle is set by \(\omega_\ast\).

In the high-frequency limit, when the phase step becomes small
(\(R_1\to 0\)) and many cycles are accumulated, the effective frequency of
phase ticks with respect to proper time is
\begin{equation}
  \nu
  := \frac{\Delta\chi}{\Delta\tau}
   = \frac{2\pi R_1}{2\pi R_2/\omega_\ast}
   = \underbrace{\frac{R_1}{R_2}}_{=:k}\,\omega_\ast.
  \label{eq:nu-k-omega}
\end{equation}
Passing to the differential limit, this gives
\begin{equation}
  \dot\chi
  := \frac{d\chi}{d\tau}
  = k\,\omega_\ast,
  \label{eq:chi-dot-k-omega}
\end{equation}
with a dimensionless structural parameter
\begin{equation}
  k := \frac{R_1}{R_2}.
  \label{eq:k-R1-R2}
\end{equation}
Equation \eqref{eq:chi-dot-k-omega} expresses the rate at which the flow
parameter \(\chi\) is accumulated per unit proper time \(\tau\) as the product
of a geometric ratio \(k\) and a characteristic internal proper-time
frequency \(\omega_\ast\).

We model the cumulative effect of this structural rescaling on energy by a
\emph{volumetric structural coefficient} \(\kappa(k)\).
Since a rescaling of a characteristic length by a factor \(k\) changes a
three-dimensional volume by \(k^3\), we take
\begin{equation}
  \kappa(k)
  = \kappa_* \left(\frac{k}{k_*}\right)^{\!3},
  \label{eq:kappa-of-k}
\end{equation}
where \(\kappa_*\) is a reference value at some fiducial structural state
\(k=k_*\).
Dimensional analysis then dictates that \(\kappa\) has units
\begin{equation}
  [\kappa] = \mathrm{kg\,s\,m^{-1}},
  \label{eq:kappa-dim}
\end{equation}
so that the combination \(\kappa c^3\) has the dimension of energy:
\[
  [\kappa c^3] = \mathrm{kg\,m^2\,s^{-2}} = \mathrm{J}.
\]

We therefore define the \emph{structural rest energy} and associated rest mass
by
\begin{equation}
  E_0(k)
  := \kappa(k)\,\clight^3,
  \qquad
  m_0(k) := \frac{E_0(k)}{\clight^2} = \kappa(k)\,\clight.
  \label{eq:E0-and-m0}
\end{equation}
For a fixed structural state \(k\) these reduce to constants
\(E_0=\kappa(k)\clight^3\), \(m_0=\kappa(k)\clight\), while spatial variations
of \(k\) would correspond to variations of the local rest energy and rest mass
scale.

\subsection{Proper-time calibration and energy factorization}
\label{subsec:E-factorization}

In the \emph{proper-time calibration} we parametrize the flow by the body's
proper time \(\tau_B\).
In the unimetrical framework the rate at which proper time accumulates with
respect to the flow parameter \(\chi\) is controlled by the body's intrinsic
angle \(\zeta_B\), introduced in \cref{subsec:interval-SR} as a characteristic
of the second-moment structure of the streamlet ensemble:
\begin{equation}
  d\tau_B = \cos\zeta_B\,d\chi,
  \qquad
  \frac{d\tau_B}{d\chi} = \cos\zeta_B.
  \label{eq:dtau-dchi-zetaB}
\end{equation}
Both \(\chi\) and \(\tau_B\) carry units of time, so \(\cos\zeta_B\) is a
dimensionless factor determined by the internal flow texture of the body.

A key assumption of unimetry in this paper is that, in the body's own rest
frame and in the absence of dissipation or absorption/emission of radiation,
the intrinsic angle \(\zeta_B\) is \emph{fixed}:
\begin{equation}
  \zeta_B = \mathrm{const}
  \quad\text{for a given body with fixed internal energy.}
  \label{eq:zetaB-const}
\end{equation}
In other words, purely kinematic changes of motion relative to an external
observer do not alter \(\zeta_B\); they act ``outside'' the internal flow
texture.

By contrast, the kinematic calibration of \cref{subsec:phase-calibration}
relates proper time to the laboratory time \(t\) via the kinematic angle
\(\vartheta\):
\begin{equation}
  \frac{d\tau_B}{dt}
  = \cos\vartheta
  = \sqrt{1-\beta^2},
  \qquad
  \gamma(\vartheta) = \frac{1}{\cos\vartheta}.
  \label{eq:dtau-dt-theta}
\end{equation}
Here \(\vartheta\) encodes the tilt of the macroscopic body flow relative to
the lab frame and is meaningful only for an external observer.
It is conceptually distinct from \(\zeta_B\): the intrinsic angle describes
the internal streamlet ensemble in the body's own frame, whereas
\(\vartheta\) describes relative motion between frames.

Using the structural rest-energy scale
\begin{equation}
  E_0(\zeta_B,k) := \kappa(k;\zeta_B)\,\clight^3,
  \qquad
  m_0(\zeta_B,k) := \kappa(k;\zeta_B)\,\clight,
  \label{eq:E0-m0-zetaB}
\end{equation}
we \emph{define} the energy of the body in the state \((\vartheta;\zeta_B,k)\)
as
\begin{equation}
  E(\vartheta;\zeta_B,k)
  := \gamma(\vartheta)\,E_0(\zeta_B,k)
   = \frac{1}{\cos\vartheta}\,\kappa(k;\zeta_B)\,\clight^3,
  \label{eq:E-theta-zetaB-k}
\end{equation}
or equivalently
\[
  E(\vartheta;\zeta_B,k)
  = \gamma(\vartheta)\,m_0(\zeta_B,k)\,\clight^2.
\]
Thus the energy factorizes into a \emph{purely kinematic} factor
\(\gamma(\vartheta)\), coming from the macroscopic tilt of the body's flow in
the laboratory frame, and a \emph{purely intrinsic-structural} factor
\(E_0(\zeta_B,k)=\kappa(k;\zeta_B)c^3\), which encodes the internal flow
texture (via \(\zeta_B\)) and the internal cyclic geometry (via \(k\)) in the
body's own frame.

For small velocities \(\beta\ll 1\), using
\(\cos\vartheta=\sqrt{1-\beta^2}\) and \(\gamma(\vartheta)=1/\cos\vartheta\) we
recover the usual expansion
\begin{equation}
  E(\vartheta;\zeta_B,k)
  = m_0(\zeta_B,k)\,\clight^2
    + \tfrac12 m_0(\zeta_B,k)\,v^2
    + O(\beta^4),
  \qquad
  v = \clight\sin\vartheta.
  \label{eq:E-expansion-zetaB}
\end{equation}
Hence the structurally generalized energy retains the standard
non-relativistic limit, while allowing the rest energy scale to vary with the
intrinsic flow texture (through \(\zeta_B\)) and with the internal cyclic
geometry (through \(k\)), without ever identifying \(\zeta_B\) with the
kinematic angle \(\vartheta\).


\subsection{From flow vector to flow quaternion}
\label{subsec:flow-to-quat}

So far the flow has been represented as a unit vector
\(\widehat{\boldsymbol\chi}\in T_X\Espace\) with a split into temporal and
spatial channels controlled by the kinematic angle \(\vartheta\), see
\eqref{eq:chi-split-theta}.
To connect this to the quaternion algebra of \cref{sec:quat-algebra}, we now
introduce a simple representation map from flow directions to unit quaternions.

Fix a body \(B\) and a local observer comoving with \(B\).
At a point \(X\in\Espace\) on the body's flow choose an orthonormal frame
\(\{\widehat{\bm u},\widehat{\bm e}_1,\widehat{\bm e}_2,\widehat{\bm e}_3\}\)
in \(T_X\Espace\), where \(\widehat{\bm u}\) is the observer's temporal
direction and \(\widehat{\bm e}_i\) span the three-dimensional rest space.
The normalized flow direction can then be written as
\begin{equation}
  \widehat{\boldsymbol\chi}
  = \cos\vartheta\,\widehat{\bm u}
    + \sin\vartheta\,\widehat{\bm n},
  \qquad
  \widehat{\bm n}
    := n^1 \widehat{\bm e}_1
      + n^2 \widehat{\bm e}_2
      + n^3 \widehat{\bm e}_3,
  \quad
  \|\widehat{\bm n}\|=1.
  \label{eq:chi-decomp-frame}
\end{equation}

On the quaternion side we have the basis
\(\{\qh,\hat{\imath},\hat{\jmath},\hat{k}\}\) introduced in
\cref{sec:quat-algebra}.
We fix a local identification between the observer's frame and the quaternion
basis by declaring
\begin{equation}
  \widehat{\bm u}
  \;\longleftrightarrow\; \qh,
  \qquad
  \widehat{\bm e}_i
  \;\longleftrightarrow\; \text{an orthonormal triple in }
  \{\hat{\imath},\hat{\jmath},\hat{k}\}.
  \label{eq:frame-to-quat-basis}
\end{equation}
In the simplest case one may align \(\widehat{\bm e}_1,\widehat{\bm e}_2,
\widehat{\bm e}_3\) with \(\hat{\imath},\hat{\jmath},\hat{k}\) respectively,
so that the spatial unit vector \(\widehat{\bm n}\) is represented by a unit
imaginary quaternion with the same components \(n^i\).

\begin{definition}[Flow quaternion]
  Given a normalized flow direction \(\widehat{\boldsymbol\chi}\) with
  kinematic angle \(\vartheta\) and spatial unit vector \(\widehat{\bm n}\) in
  the observer's rest space, the associated \emph{flow quaternion} is the unit
  quaternion
  \begin{equation}
    \widehat{\mathsf q}(\vartheta,\widehat{\bm n})
    := \cos\vartheta\,\qh + \sin\vartheta\,\widehat{\bm n},
    \qquad
    \|\widehat{\bm n}\|=1.
    \label{eq:flow-quaternion-def}
  \end{equation}
\end{definition}

By construction \(\|\widehat{\mathsf q}\|=1\).
The scalar part of \(\widehat{\mathsf q}\) encodes the temporal fraction of the
flow, while the vector part encodes the spatial fraction in the chosen rest
space.
Different choices of the spatial orthonormal frame correspond to spatial
rotations of the imaginary basis and do not affect scalar invariants such as
\(A(\widehat{\mathsf q},\widehat{\mathsf q})\).

\subsection{Scalar self-form \texorpdfstring{$A(q,q)$}{A(q,q)} as kinematic
factor}
\label{subsec:A-kinematic-factor}

We can now relate the kinematic factor in \eqref{eq:E-theta-k} to the
quaternionic self-form \(A(q,q)\) introduced in
\cref{sec:quat-algebra}.
For brevity we suppress the explicit dependence on \(\widehat{\bm n}\) and
write \(\widehat{\mathsf q}(\vartheta)\) when no confusion can arise:
\[
  \widehat{\mathsf q}(\vartheta)
  = \cos\vartheta\,\qh + \sin\vartheta\,\widehat{\bm n},
  \qquad
  \|\widehat{\bm n}\|=1.
\]
The scalar self-form \(\color{red}{A}\) applied to this unit quaternion gives
\begin{equation}
  A\bigl(\widehat{\mathsf q}(\vartheta),
         \widehat{\mathsf q}(\vartheta)\bigr)
  = \cos^2\vartheta - \sin^2\vartheta
  = \cos 2\vartheta.
  \label{eq:A-self-theta}
\end{equation}
Solving for \(\cos^2\vartheta\) we obtain
\begin{equation}
  1-\beta^2
  = \cos^2\vartheta
  = \frac{1 + A(\widehat{\mathsf q},\widehat{\mathsf q})}{2},
  \label{eq:1-minus-beta-from-A}
\end{equation}
and hence the Lorentz factor can be expressed as
\begin{equation}
  \gamma(\vartheta)
  = \frac{1}{\sqrt{1-\beta^2}}
  = \sqrt{\frac{2}{1 + A(\widehat{\mathsf q},\widehat{\mathsf q})}}.
  \label{eq:gamma-from-A}
\end{equation}

Substituting \eqref{eq:gamma-from-A} into the factorized energy
\eqref{eq:E-theta-k}, we find
\begin{equation}
  E(\widehat{\mathsf q},k)
  = \gamma(\vartheta)\,E_0(k)
  = \sqrt{\frac{2}{1 + A(\widehat{\mathsf q},\widehat{\mathsf q})}}\;
    \kappa(k)\,\clight^3.
  \label{eq:E-from-A-and-k}
\end{equation}
Thus:
\begin{itemize}[leftmargin=*]
  \item the scalar self-form \(A(\widehat{\mathsf q},\widehat{\mathsf q})\)
  is a \emph{purely kinematic, dimensionless} invariant of the flow
  quaternion;
  \item the structural coefficient \(\kappa(k)\) carries the absolute energy
  scale, via \(E_0(k)=\kappa(k)\clight^3\);
  \item energy emerges as the product of a quaternionic kinematic factor (a
  function of \(A\)) and a structural scale.
\end{itemize}

In the gravito--electromagnetic construction below we will apply the same
scalar form \(A\) not to unit flow quaternions, but to \emph{metrically
dressed} body quaternions \(\widetilde{\mathsf q}_i\) that package mass and
charge into effective lengths.
For isotropic Newtonian bodies the spatial parts of these quaternions will be
further restricted to multiples of the universal unit imaginary \(\uhat\)
introduced in \cref{sec:quat-algebra}, so that all interaction channels reduce
to a single radial direction.
The vector forms \(\bm B\) and \(\bm C\) will then provide, respectively, the
current-like and vortical channels of the gravito--electromagnetic field.

%==================== 5. Metrically dressed body quaternions ====================

\section{Metric dressing of body quaternions and gravito--electromagnetic
self-forms}
\label{sec:metric-dressing-GEM}

In the previous sections we treated the flow quaternion as a dimensionless
unit object encoding only the \emph{direction} of the proto-space flow.
This is sufficient to reproduce the kinematics of special relativity.
To model gravito--electromagnetic (GEM) interactions between extended
Newtonian bodies, however, we must also incorporate their intrinsic strengths
(mass, charge) in a way that respects the metric character of space.

In this section we introduce \emph{metrically dressed} body quaternions,
whose components carry the dimension of length, and show how the scalar and
vector self-forms \(A,B,C,E\) generate GEM-like interaction channels in the
static and slowly moving regimes.

\subsection{Metric dressing: from mass and charge to effective lengths}
\label{subsec:metric-dressing}

Let a Newtonian body \(B_i\) be characterized, in the usual description, by a
gravitational mass \(m_i\) and an electric charge \(q_i\).
Instead of treating \(m_i\) and \(q_i\) as separate scalar labels, we encode
them in a single \emph{metrically dressed body quaternion}
\(\widetilde{\mathsf q}_i\) with units of length:
\begin{equation}
  [\widetilde{\mathsf q}_i] = \mathrm{m}.
\end{equation}
For isotropic bodies interacting radially it is natural to restrict the
spatial part to the universal unit imaginary \(\uhat\) introduced in
\cref{sec:quat-algebra}.
We thus take
\begin{equation}
  \widetilde{\mathsf q}_i
  := T_i\,\qh + S_i\,\uhat,
  \qquad
  T_i,S_i \in \mathbb{R},
  \qquad
  [T_i] = [S_i] = \mathrm{m}.
  \label{eq:body-quat-dressed}
\end{equation}
The scalar component \(T_i\) encodes the \emph{temporal} (electric-like)
channel of the body, while the spatial component \(S_i\,\uhat\) encodes the
\emph{spatial} (gravitational-like) channel.
Both are measured in meters so that the quaternion norm and the bilinear
forms are metrically homogeneous.

We express \(T_i\) and \(S_i\) as linear combinations of mass and charge via
two calibration constants \(\lambda_G\) and \(\lambda_E\) with units chosen so
that \(\lambda_G m_i\) and \(\lambda_E q_i\) are lengths:
\begin{equation}
  T_i = \lambda_E\,q_i,
  \qquad
  S_i = \lambda_G\,m_i.
  \label{eq:Ti-Si-def}
\end{equation}
The exact values of \(\lambda_G\) and \(\lambda_E\) will be fixed later by
matching the resulting interaction to Newton's and Coulomb's laws; for now
we only require that they are universal constants of the theory.

With this dressing, the norm-like self-form \(E(\widetilde{\mathsf q}_i,
\widetilde{\mathsf q}_i)\) carries the dimension of length squared and will
become proportional to the body's rest energy, while bilinear combinations
between different bodies will generate interaction energies.

\subsection{Single body: rest energy from the Euclidean self-form}
\label{subsec:single-body-E}

For a single body \(B_i\) the Euclidean self-form of its dressed quaternion is
\begin{equation}
  E(\widetilde{\mathsf q}_i,\widetilde{\mathsf q}_i)
  = T_i^2 + S_i^2
  = \lambda_E^2 q_i^2 + \lambda_G^2 m_i^2,
  \label{eq:E-self-body}
\end{equation}
since \(\qh^2=+1\), \(\uhat^2=-1\) and \(\uhat\) is orthogonal to \(\qh\) in
the Euclidean sense.

We postulate that the rest energy of the body is proportional to the square
root of this Euclidean self-form:
\begin{equation}
  E_{\text{rest},i}
  = \kappa_{\text{rest}}\,
    \sqrt{E(\widetilde{\mathsf q}_i,\widetilde{\mathsf q}_i)},
  \label{eq:E-rest-from-E}
\end{equation}
where \(\kappa_{\text{rest}}\) is a universal constant with units chosen so
that \([\kappa_{\text{rest}} \sqrt{E}] = \mathrm{J}\).
In the purely gravitational limit \(q_i=0\) this reduces to
\begin{equation}
  E_{\text{rest},i}^{(G)}
  = \kappa_{\text{rest}}\,|\lambda_G|\,m_i,
\end{equation}
which can be matched to \(m_i c^2\) by choosing
\(\kappa_{\text{rest}}|\lambda_G|\equiv c^2\).
In the full GEM setting \eqref{eq:E-self-body} and
\eqref{eq:E-rest-from-E} provide a unified rest-energy scale that depends on
both mass and charge but remains a scalar under quaternionic rotations.

In the dynamical setting, the kinematic factor \(\gamma(\vartheta)\) from
\cref{subsec:E-factorization} multiplies this rest scale so that the total
energy reads
\begin{equation}
  E_i(\vartheta_i)
  = \gamma(\vartheta_i)\,E_{\text{rest},i},
\end{equation}
in direct analogy with the relativistic point-particle Hamiltonian.

\subsection{Two-body interaction and scalar self-form}
\label{subsec:two-body-A}

Consider now two bodies \(B_1\) and \(B_2\) with dressed quaternions
\(\widetilde{\mathsf q}_1\) and \(\widetilde{\mathsf q}_2\) as in
\eqref{eq:body-quat-dressed}.
Their scalar self-form \(A(\widetilde{\mathsf q}_1,\widetilde{\mathsf q}_2)\)
is
\begin{equation}
  A(\widetilde{\mathsf q}_1,\widetilde{\mathsf q}_2)
  = T_1 T_2 - S_1 S_2
  = \lambda_E^2 q_1 q_2 - \lambda_G^2 m_1 m_2.
  \label{eq:A-two-body}
\end{equation}
Thus the scalar channel naturally separates into a temporal (electric-like)
part and a spatial (gravitational-like) part, with opposite signs, reflecting
the empirical fact that electrostatic forces between like charges are
repulsive, while gravitational forces are attractive.

To turn this scalar quantity into an interaction energy we introduce a
radius vector \(\bm r\) from \(B_1\) to \(B_2\) with magnitude
\(r = \|\bm r\|\), and define a scalar interaction potential of the form
\begin{equation}
  V_{12}(r)
  = -\,\frac{\kappa_{\text{int}}}{r}\,
     A(\widetilde{\mathsf q}_1,\widetilde{\mathsf q}_2),
  \label{eq:V12-from-A}
\end{equation}
where \(\kappa_{\text{int}}\) is another universal constant.

Expanding \eqref{eq:V12-from-A} we obtain
\begin{equation}
  V_{12}(r)
  = -\,\frac{\kappa_{\text{int}}\lambda_E^2}{r}\,q_1 q_2
    + \frac{\kappa_{\text{int}}\lambda_G^2}{r}\,m_1 m_2.
  \label{eq:V12-expanded}
\end{equation}
By choosing the calibration constants \(\kappa_{\text{int}}\), \(\lambda_E\)
and \(\lambda_G\) so that
\begin{equation}
  \kappa_{\text{int}}\lambda_G^2 = G,
  \qquad
  \kappa_{\text{int}}\lambda_E^2 = \frac{1}{4\pi\epsilon_0},
  \label{eq:calibration-Newton-Coulomb}
\end{equation}
we recover the usual Newton and Coulomb potentials:
\begin{equation}
  V_{12}(r)
  = -\,\frac{G\,m_1 m_2}{r}
    + \frac{1}{4\pi\epsilon_0}\,\frac{q_1 q_2}{r}.
  \label{eq:V12-Newton-Coulomb}
\end{equation}
Structurally, however, both contributions arise from the \emph{same} scalar
bilinear form \(A\) acting on the dressed body quaternions; the difference
between gravity and electrostatics is encoded entirely in the dressing
\eqref{eq:Ti-Si-def} and in the signs inside \eqref{eq:A-two-body}.

\subsection{Vector self-forms and field-like channels}
\label{subsec:vector-forms-fields}

The scalar form \(A\) encodes the static interaction energy between two bodies
once the radius \(r\) is specified.
When the bodies move or when we consider the field created by a body at a
point in space, we must also account for the vectorial self-forms \(B\) and
\(C\).

For dressed quaternions of the form \eqref{eq:body-quat-dressed} with a
single spatial direction \(\uhat\), the \(B\)-form between two bodies is
\begin{equation}
  \bm B(\widetilde{\mathsf q}_1,\widetilde{\mathsf q}_2)
  = B(\widetilde{\mathsf q}_1,\widetilde{\mathsf q}_2)
  = (T_1 S_2 + T_2 S_1)\,\uhat
  = \lambda_E\lambda_G\,(q_1 m_2 + q_2 m_1)\,\uhat.
  \label{eq:B-two-body}
\end{equation}
This vector lies along the universal direction \(\uhat\), which in an
isotropic Newtonian configuration is identified with the radial unit vector
\(\widehat{\bm r}\).
The \(C\)-form vanishes identically in this one-dimensional spatial sector,
since it measures transverse, vortical components that require at least two
independent spatial directions:
\begin{equation}
  \bm C(\widetilde{\mathsf q}_1,\widetilde{\mathsf q}_2) = \bm 0.
\end{equation}

Thus in the strictly radial, static configuration we have:
\begin{itemize}[leftmargin=*]
  \item \(A\) generates the scalar interaction potential
  \eqref{eq:V12-from-A},
  \item \(B\) is non-zero but degenerate (collinear with \(\uhat\)),
  \item \(C\) vanishes.
\end{itemize}
The emergence of genuinely vortical (magnetic-like) contributions is therefore
tied to configurations where the flow quaternions of the bodies acquire
independent spatial directions, e.g.\ through relative motion and
quaternionic \(D\)-rotations.

\subsection{From interaction to field: local GEM quaternion}
\label{subsec:local-GEM-field}

The potential \eqref{eq:V12-from-A} between two bodies suggests a local
field-like object attached to a single source body \(B_1\).
At a point with radius vector \(\bm r\) from \(B_1\) we define the local
\emph{gravito--electromagnetic field quaternion} by
\begin{equation}
  \mathsf F_1(\bm r)
  := -\,\frac{\kappa_{\text{int}}}{r}\,
      \frac{\widetilde{\mathsf q}_1}{L_r},
  \qquad
  L_r := r,
  \label{eq:F1-local}
\end{equation}
so that
\begin{equation}
  A\bigl(\mathsf F_1(\bm r),\widetilde{\mathsf q}_2\bigr)
  = -\,\frac{\kappa_{\text{int}}}{r}\,
      A(\widetilde{\mathsf q}_1,\widetilde{\mathsf q}_2)
  = V_{12}(r),
\end{equation}
i.e.\ the scalar self-form of the local field quaternion with the test body
\(B_2\) reproduces the interaction potential.

The vector parts of \(\mathsf F_1(\bm r)\) can then be interpreted as
gravito--electric and gravito--magnetic fields in the usual sense:
\begin{equation}
  \mathsf F_1(\bm r)
  = \Phi_1(r)\,\qh + \bm{\mathcal E}_1(r)\cdot\uhat
\end{equation}
in the static, radial case, with \(\bm{\mathcal E}_1(r)\parallel\widehat{\bm r}\)
and vanishing vortical part.
When the source body and/or the test body move, the dressing can be acted upon
by \(D\)-rotations, and the bilinear forms \(B\) and \(C\) between the
rotated quaternions generate current-like and vortical contributions
analogous to the electric and magnetic fields of Maxwell's theory.
We will develop this dynamical picture in the next section.

In summary, metrically dressed body quaternions provide a compact way to
encode mass and charge as effective lengths in a single quaternionic object.
The scalar self-form \(A\) then generates Newton and Coulomb potentials in a
unified fashion, while the vector forms \(B\) and \(C\) supply the linear and
vortical channels needed to describe, respectively, gravito--electric and
gravito--magnetic fields once quaternionic rotations are taken into account.

\section{Mass as a functional of streamlet structure in Unimetry}

\emph{Draft derivation in the ``streamlet + second moment'' picture, without using mass to define the angle \(\zeta\).}

\subsection{Streamlets, normalization and second moments}

Consider a massive object as an ensemble of streamlets indexed by \(a\) with weights \(w_a\), \(\sum_a w_a = 1\).

Each streamlet has a normalized proto–flow direction \(\tilde{\boldsymbol\chi}_a\) in the proto–space \(\mathcal E\), decomposed with respect to the object's own temporal axis \(\hat e_\tau\) and a spatial triad \(\{\hat e_i\}\):
\[
\tilde{\boldsymbol\chi}_a
 = \tilde X^0_a\,\hat e_\tau + \tilde{\mathbf X}_a,
 \qquad
 \|\tilde{\boldsymbol\chi}_a\|^2
 = (\tilde X^0_a)^2 + \|\tilde{\mathbf X}_a\|^2
 = 1.
\]

We parameterize each streamlet by an angle \(\Theta_a\) between its flow and the temporal axis:
\[
\tilde X^0_a = \cos\Theta_a,\qquad
\|\tilde{\mathbf X}_a\| = \sin\Theta_a.
\]

The rest configuration of the object is defined as that frame in which the first spatial moment vanishes:
\[
\sum_a w_a \tilde{\mathbf X}_a = 0.
\]

However, the second moment in general does not vanish and contains structural information about the object.

Define the second moments:
\[
T_B := \sum_a w_a \cos^2\Theta_a,
\qquad
\mathbf C_B := \sum_a w_a \sin^2\Theta_a\;\mathbf u_a\otimes\mathbf u_a,
\]
where \(\mathbf u_a := \tilde{\mathbf X}_a / \|\tilde{\mathbf X}_a\|\) are unit spatial directions (when \(\sin\Theta_a\neq 0\)).

From \(\cos^2\Theta_a + \sin^2\Theta_a = 1\) and \(\sum_a w_a = 1\) it follows that
\[
T_B + \operatorname{tr}\mathbf C_B = 1.
\]

Here
\begin{itemize}
  \item \(T_B\) encodes the temporal share of the total flow ``budget'';
  \item \(\mathbf C_B\) encodes both amount and shape of the spatially looped flow.
\end{itemize}


\subsection{Structural angle \texorpdfstring{\(\zeta\)}{ζ} from the second moment (without mass)}

We now define a purely structural ``budget angle'' \(\zeta\) by
\[
T_B = \cos^2\zeta,\qquad
\operatorname{tr}\mathbf C_B = \sin^2\zeta.
\]

This is not defined using any notion of physical mass or charge.  
It is determined solely by
\begin{itemize}
  \item the ensemble of streamlets \(\{\tilde{\boldsymbol\chi}_a, w_a\}\),
  \item the choice of the object's own temporal axis \(\hat e_\tau\),
  \item and the second moment data \((T_B, \mathbf C_B)\).
\end{itemize}

Thus \(\zeta\) is a geometric property of the proto–flow structure, not a function of the observed rest mass.

Interpretation:
\begin{itemize}
  \item \(\cos^2\zeta\) is the fraction of the unit flow magnitude that runs ``along time'' (along \(\hat e_\tau\));
  \item \(\sin^2\zeta\) is the fraction trapped in spatial loops and anisotropies encoded by \(\mathbf C_B\).
\end{itemize}


\subsection{Total proto–flow magnitude and temporal share}

We now introduce a convenient normalization of the total proto–flow:
\[
\tilde H := \Big\langle\|\tilde{\boldsymbol\chi}_a\|^2\Big\rangle^{1/2}
         = \left(\sum_a w_a\,1\right)^{1/2}
         = 1.
\]

So the total flow magnitude of the object is set to unity; all nontrivial physics is in how this unit is split between temporal and spatial directions.

The temporal share of the flow, in the object's rest configuration, is then
\[
\tilde X_0^2 := T_B = \cos^2\zeta,
\]
while the spatial share is \(\operatorname{tr}\mathbf C_B = \sin^2\zeta\).

After the standard SR calibration, where the physical flow is fixed as \(\boldsymbol\chi = c\,\widehat{\boldsymbol\chi}\) (so that \(\|\boldsymbol\chi\|=c\)), the temporal component of the physical flow becomes
\[
\chi_\tau^{(\text{phys})} = c\,\tilde X_0 = c\cos\zeta.
\]

In this sense, for the object itself the effective ``speed of light'' that controls the rate of its proper time is the temporal share \(c\cos\zeta\); the remaining \(c\sin\zeta\) is locked in spatial circulation.


\subsection{Phase parameter \texorpdfstring{\(\chi\)}{χ}, cyclic time \texorpdfstring{\(\tau\)}{τ}, and invariant flow \texorpdfstring{\(H\)}{H}}

Introduce the global phase parameter \(\chi\) and the local proper time \(\tau\) of the object.

By construction of the streamlet picture, each full cycle of the object's internal flow corresponds to a full phase \(2\pi\) in \(\chi\), while \(\tau\) parametrizes the local cycles of the object's internal clock.

We postulate that the rate at which the phase winds with respect to the local time is proportional to the temporal share of the flow:
\[
\frac{d\chi}{d\tau}
:= \dot\chi
 = k\,\tilde X_0^2
 = k\,\cos^2\zeta,
\]
where \(k\) is a geometric constant of the object, e.g.\ the ratio of the radii of two conjugate cycles \(k = R_1/R_2\) in the cyclic picture.

This is a key step:
\begin{itemize}
  \item we are not inserting the rest mass here;
  \item we are using only the structural angle \(\zeta\) extracted from the second moment of the streamlets.
\end{itemize}

Now define \(H(\chi)\) as the accumulated flow quantity along the phase:
\[
\frac{dH}{d\chi} := \tilde H.
\]

With \(\tilde H \equiv 1\), this reduces to
\[
\frac{dH}{d\chi} = 1
\quad\Rightarrow\quad
H(\chi) = \chi + \text{const}
\sim \chi.
\]

Differentiating with respect to the local time,
\[
\dot H := \frac{dH}{d\tau}
        = \frac{dH}{d\chi}\frac{d\chi}{d\tau}
        = 1\cdot \dot\chi
        = k\,\cos^2\zeta.
\]

Thus the instantaneous scalar flow rate in the proper frame of the object is
\[
\boxed{
\dot H = k\,\tilde X_0^2
       = k\,\cos^2\zeta.
}
\]

Its square gives the invariant quadratic form in the proper frame:
\[
dH^2 = \dot H^2\,d\tau^2
     = k^2\cos^4\zeta\,d\tau^2.
\]

This quantity is invariant under D-boosts: D-rotations of the observer do not change the internal \(\zeta\), nor the intrinsic cyclic geometry encapsulated in \(k\). They only change the observed decomposition of 4-velocity, not the internal flow rate \(\dot H\).


%==================== 6. Bilinear forms and static GEM ====================

\section{Bilinear forms and static gravito--electromagnetism}
\label{sec:bilinear-static-GEM}

In this section we show how the four quaternionic bilinear forms introduced in
\cref{sec:quat-algebra} naturally reproduce the static limits of Newtonian
gravity and Coulomb electrostatics when applied to metrically dressed body
quaternions.
The key point is that gravitation and electrostatics correspond to different
\emph{channels} of the same quaternionic object: the spatial part encodes
gravitational charge (mass), while the temporal part encodes electric charge.

\subsection{Dressed body quaternions in the static, isotropic case}
\label{subsec:dressed-static}

We consider isolated Newtonian bodies which are macroscopically at rest in a
common inertial frame and whose internal streamlet ensembles are isotropic in
space in that frame.
In this situation the spatial structure of each body can be collapsed to a
single unit imaginary direction, the \emph{free unit imaginary vector}
\(\uhat\) from \cref{sec:quat-algebra}, which is always aligned with the
radius vector connecting the bodies in a two-body configuration.

For each body \(B_i\) with gravitational mass \(m_i\) and electric charge
\(q_i\) we introduce a metrically dressed quaternion
\begin{equation}
  \widetilde{\mathsf q}_i
  := T_i\,\qh + S_i\,\uhat,
  \qquad
  T_i,S_i \in \mathbb{R},
  \qquad
  [T_i] = [S_i] = \mathrm{m},
  \label{eq:body-quat-static}
\end{equation}
with the identifications
\begin{equation}
  T_i = \lambda_E\,q_i,
  \qquad
  S_i = \lambda_G\,m_i.
  \label{eq:T-S-from-q-m}
\end{equation}
Here \(\lambda_E\) and \(\lambda_G\) are universal calibration constants with
units chosen so that \(\lambda_E q_i\) and \(\lambda_G m_i\) have dimensions
of length.
The scalar component \(T_i\) encodes the \emph{temporal} (electromagnetic)
channel, while the spatial component \(S_i\uhat\) encodes the
\emph{spatial} (gravitational) channel.

In this static, isotropic setting all three imaginary units \(\hat{\imath},
\hat{\jmath},\hat{k}\) collapse to the single direction \(\uhat\).
As a consequence:
\begin{itemize}[leftmargin=*]
  \item the scalar forms \(A\) and \(E\) remain nontrivial and will encode,
  respectively, interaction strengths and total energies;
  \item the vector form \(B\) is collinear with \(\uhat\) and plays the role
  of a unified ``radial field'';
  \item the vortical form \(C\) vanishes identically, reflecting the absence
  of static vorticity (no gravito--magnetic or magnetic components in the
  purely static configuration).
\end{itemize}

\subsection{Scalar bilinear form and Newton--Coulomb potentials}
\label{subsec:A-static-GEM}

Let two bodies \(B_1\) and \(B_2\) be described by dressed quaternions
\(\widetilde{\mathsf q}_1\) and \(\widetilde{\mathsf q}_2\) of the form
\eqref{eq:body-quat-static}.
The scalar bilinear form \(\color{red}{A}\) from \cref{sec:quat-algebra}
reduces in this two-channel case to
\begin{equation}
  A(\widetilde{\mathsf q}_1,\widetilde{\mathsf q}_2)
  = T_1 T_2 - S_1 S_2
  = \lambda_E^2\,q_1 q_2 - \lambda_G^2\,m_1 m_2.
  \label{eq:A-static-two-body}
\end{equation}
The temporal part \(T_1T_2\) corresponds to the electromagnetic channel;
the spatial part \(S_1S_2\) corresponds to the gravitational channel.
The opposite signs reflect the empirical fact that like electric charges
repel while like masses attract.

Let \(\bm r\) be the radius vector from \(B_1\) to \(B_2\) in the common rest
frame, with magnitude \(r = \|\bm r\|\).
We define the static interaction potential between the two bodies by
\begin{equation}
  V_{12}(r)
  := \frac{\kappa_{\mathrm{int}}}{r}\,
      A(\widetilde{\mathsf q}_1,\widetilde{\mathsf q}_2),
  \label{eq:V12-from-A-static}
\end{equation}
where \(\kappa_{\mathrm{int}}\) is a universal constant with units chosen so
that \([\kappa_{\mathrm{int}} A / r] = \mathrm{J}\).

Expanding \eqref{eq:V12-from-A-static} using
\eqref{eq:A-static-two-body} we obtain
\begin{equation}
  V_{12}(r)
  = \frac{\kappa_{\mathrm{int}}\lambda_E^2}{r}\,q_1 q_2
    - \frac{\kappa_{\mathrm{int}}\lambda_G^2}{r}\,m_1 m_2.
  \label{eq:V12-expanded-static}
\end{equation}
By choosing the dressing constants such that
\begin{equation}
  \kappa_{\mathrm{int}}\lambda_E^2 = \frac{1}{4\pi\epsilon_0},
  \qquad
  \kappa_{\mathrm{int}}\lambda_G^2 = G,
  \label{eq:calibration-G-eps0}
\end{equation}
we recover the familiar Newton and Coulomb potentials in a single expression:
\begin{equation}
  V_{12}(r)
  = \frac{1}{4\pi\epsilon_0}\,\frac{q_1 q_2}{r}
    - \frac{G\,m_1 m_2}{r}.
  \label{eq:V12-Newton-Coulomb-unified}
\end{equation}

Structurally, \eqref{eq:V12-Newton-Coulomb-unified} shows that:
\begin{itemize}[leftmargin=*]
  \item the static electromagnetic interaction arises from the
  \emph{temporal} channel \(T_i\) of the dressed quaternions;
  \item the static gravitational interaction arises from the
  \emph{spatial} channel \(S_i\) of the same quaternions;
  \item both are captured by the \emph{same} scalar bilinear form \(A\), once
  mass and charge are encoded into effective lengths via
  \eqref{eq:T-S-from-q-m}.
\end{itemize}
In this sense the Newton and Coulomb laws appear as two contributions of one
and the same quaternionic scalar interaction.

\subsection{Euclidean self-form and decomposition of total energy}
\label{subsec:E-static-energy}

The Euclidean self-form \(\color{blue}{E}\) of a single dressed body
quaternion \(\widetilde{\mathsf q}_i\) reads
\begin{equation}
  E(\widetilde{\mathsf q}_i,\widetilde{\mathsf q}_i)
  = T_i^2 + S_i^2
  = \lambda_E^2 q_i^2 + \lambda_G^2 m_i^2.
  \label{eq:E-static-single}
\end{equation}
As in \cref{subsec:single-body-E}, we associate to this a rest-energy scale
\begin{equation}
  E_{\mathrm{rest},i}
  = \kappa_{\mathrm{rest}}\,
    \sqrt{E(\widetilde{\mathsf q}_i,\widetilde{\mathsf q}_i)},
  \label{eq:E-rest-static}
\end{equation}
which reduces to \(m_i c^2\) in the purely gravitational limit after
calibration.

For a two-body system with total dressed quaternion
\begin{equation}
  \widetilde{\mathsf q}_{\mathrm{tot}}
  := \widetilde{\mathsf q}_1 + \widetilde{\mathsf q}_2,
\end{equation}
the Euclidean self-form expands as
\begin{equation}
  E(\widetilde{\mathsf q}_{\mathrm{tot}},
    \widetilde{\mathsf q}_{\mathrm{tot}})
  = E(\widetilde{\mathsf q}_1,\widetilde{\mathsf q}_1)
    + E(\widetilde{\mathsf q}_2,\widetilde{\mathsf q}_2)
    + 2\,A(\widetilde{\mathsf q}_1,\widetilde{\mathsf q}_2).
  \label{eq:E-total-two-body}
\end{equation}
Thus the cross term in the Euclidean norm is \emph{precisely} twice the
scalar bilinear form \(A\) that generates the interaction potential.
Up to a universal conversion factor this cross term can be identified with
the static interaction energy \eqref{eq:V12-from-A-static}, while the
diagonal terms represent the self-energies of the bodies.

This decomposition mirrors the role of the Hamiltonian in classical
mechanics:
\begin{equation}
  E_{\mathrm{tot}}
  = E_{\mathrm{rest},1} + E_{\mathrm{rest},2}
    + V_{12}(r),
\end{equation}
with \(V_{12}(r)\) directly inherited from \(A(\widetilde{\mathsf q}_1,
\widetilde{\mathsf q}_2)\) via \eqref{eq:V12-from-A-static}.
In the static GEM sector the Euclidean self-form \(E\) therefore plays the
role of a Hamiltonian channel (total energy), while the scalar form \(A\)
encodes the interaction part.

\subsection{Radial field from the vector form in the static regime}
\label{subsec:B-static-field}

Although the static configuration has no vortical component, the vector
bilinear form \(\color{green}{B}\) already carries useful information about
the direction and relative strength of the gravito--electromagnetic field.

For dressed quaternions of the form \eqref{eq:body-quat-static} we have
\begin{equation}
  \bm B(\widetilde{\mathsf q}_1,\widetilde{\mathsf q}_2)
  = B(\widetilde{\mathsf q}_1,\widetilde{\mathsf q}_2)
  = (T_1 S_2 + T_2 S_1)\,\uhat
  = \lambda_E\lambda_G\,(q_1 m_2 + q_2 m_1)\,\uhat.
  \label{eq:B-static-two-body}
\end{equation}
In the static, isotropic case we align \(\uhat\) with the radial unit vector
\(\widehat{\bm r} := \bm r / r\) connecting the bodies, so that
\(\bm B\parallel\widehat{\bm r}\).
The magnitude of \(\bm B\) combines temporal and spatial charges in a
symmetric way; it will play the role of a unified ``radial flux'' in the
dynamical GEM formulation, where relative motion and quaternionic
\(D\)-rotations create proper electric and magnetic fields.

In the strictly static limit considered here the vortical form \(\bm C\) is
identically zero, because all spatial directions collapse to \(\uhat\) and
no transverse circulation is possible:
\begin{equation}
  \bm C(\widetilde{\mathsf q}_1,\widetilde{\mathsf q}_2) = \bm 0.
\end{equation}
This matches the physical expectation that static mass and charge
distributions generate only scalar and radial vector fields, with no
magnetic-like components.

\subsection{Summary of the static GEM sector}
\label{subsec:static-GEM-summary}

To summarize, in the static, isotropic gravito--electromagnetic sector:
\begin{itemize}[leftmargin=*]
  \item Each Newtonian body is represented by a metrically dressed quaternion
  \(\widetilde{\mathsf q}_i = \lambda_E q_i\,\qh + \lambda_G m_i\,\uhat\),
  where the temporal channel encodes electric charge and the spatial channel
  encodes mass.
  \item The scalar bilinear form \(A(\widetilde{\mathsf q}_1,
  \widetilde{\mathsf q}_2)\) generates, after division by \(r\), the unified
  Newton--Coulomb potential \(V_{12}(r)\) in \eqref{eq:V12-Newton-Coulomb-unified}.
  \item The Euclidean self-form \(E\) of the sum
  \(\widetilde{\mathsf q}_1 + \widetilde{\mathsf q}_2\) splits into two
  self-energy terms and an interaction term proportional to
  \(A(\widetilde{\mathsf q}_1,\widetilde{\mathsf q}_2)\), and plays the role
  of a Hamiltonian channel.
  \item The vector form \(B\) reduces to a radial vector proportional to
  \(\widehat{\bm r}\) and combines mass and charge symmetrically, while the
  vortical form \(C\) vanishes, reflecting the absence of static
  gravito--magnetic and magnetic fields.
\end{itemize}

In the next section we will turn on relative motion and quaternionic
\(D\)-rotations of the dressed body quaternions.
This will lift the degeneracy of the vector forms: \(B\) and \(C\) will then
separate into gravito--electric and gravito--magnetic parts, providing a
quaternionic analogue of the Maxwell--Heaviside formulation of
gravito--electromagnetism.


%==================== 7. Quaternionic GEM field in R^3 ====================

\section{Quaternionic GEM field in the observable three--space}
\label{sec:quat-GEM-3space}

The previous section showed how the scalar and Euclidean bilinear forms
\(A\) and \(E\) acting on metrically dressed body quaternions reproduce the
static Newton and Coulomb potentials in a two--body configuration.
We now pass from this two--body picture to a field picture in the observable
three--space, and define a quaternion--valued gravito--electromagnetic (GEM)
field whose bilinear couplings to a test body generate forces and energies.

\subsection{Observable frame and 3+1 splitting}
\label{subsec:3plus1}

We fix an inertial ``laboratory'' frame with coordinate time \(t\) and
Euclidean three--space \(\mathbb R^3\) with orthonormal basis
\(\{\hat e_i\}_{i=1}^3\).
The observer's temporal axis \(\hat e_0\) defines a splitting of the
proto--space flow into temporal and spatial parts, as in
\cref{sec:flow-calibrations}, and all observable quantities are projected
onto this frame.

Each Newtonian body \(B_i\) is characterized by its worldline
\(\bm x_i(t)\in\mathbb R^3\), gravitational mass \(m_i\), charge \(q_i\) and a
metrically dressed quaternion \(\widetilde{\mathsf q}_i\) of the form
\eqref{eq:body-quat-static} in its own rest frame.
In the laboratory frame, where \(B_i\) may move with some three--velocity
\(\bm v_i(t)\), the corresponding dressed quaternion is obtained by a
quaternionic \(D\)--rotation,
\begin{equation}
  \widetilde{\mathsf q}_i^{\mathrm{lab}}(t)
  := D_i(t)\,\widetilde{\mathsf q}_i\,D_i(t),
  \label{eq:q-dressed-lab}
\end{equation}
where \(D_i(t)\) is the unimetrical rotor associated with the boost between
the rest frame of \(B_i\) and the laboratory frame
(see \cref{sec:D-rotations}).

In this section we concentrate on the quasi--static regime where the
velocities are small and retardation can be neglected.
The \(D\)--rotations will then mainly serve to generate the leading
gravito--magnetic and magnetic corrections via the vector bilinear forms.

\subsection{Local GEM field of a single static source}
\label{subsec:single-source-field}

We first consider a single static source \(B_s\) at position \(\bm x_s\) with
dressed quaternion
\(\widetilde{\mathsf q}_s = T_s \qh + S_s \uhat\) as in
\eqref{eq:body-quat-static}, where
\(\uhat\) is a free unit imaginary aligned with the radial direction from the
source.

For a point \(\bm x\in\mathbb R^3\) with radius vector
\(\bm r := \bm x - \bm x_s\) and \(r := \|\bm r\|\), we define the local
\emph{static GEM field quaternion} of the source by
\begin{equation}
  \mathsf F_s(\bm x)
  := \frac{\kappa_{\mathrm{f}}}{r^2}\,
      \bigl( T_s\,\qh + S_s\,\widehat{\bm r} \bigr),
  \qquad
  \widehat{\bm r} := \frac{\bm r}{r},
  \label{eq:F-single-source}
\end{equation}
where \(\kappa_{\mathrm{f}}\) is a universal field calibration constant.

The structure of \eqref{eq:F-single-source} mirrors that of the standard
Coulomb and Newton fields: the temporal component \(T_s/r^2\) produces a
radial electric--like field, while the spatial component \(S_s\widehat{\bm r}/r^2\)
produces a radial gravito--electric field.
The use of the free unit imaginary \(\widehat{\bm r}\) ensures that the field
is always aligned with the radius vector, independently of the internal
orientation of the source in proto--space.

A test body \(B_t\) at \(\bm x\) with dressed quaternion
\(\widetilde{\mathsf q}_t = T_t\qh + S_t\,\widehat{\bm r}\) then couples to
\(\mathsf F_s(\bm x)\) through the scalar and vector bilinear forms:
\begin{align}
  A\bigl(\mathsf F_s(\bm x), \widetilde{\mathsf q}_t\bigr)
  &= \frac{\kappa_{\mathrm{f}}}{r^2}
     \bigl( T_s T_t - S_s S_t \bigr),
  \label{eq:A-F-qtest-static}\\[0.5em]
  \bm B\bigl(\mathsf F_s(\bm x), \widetilde{\mathsf q}_t\bigr)
  &= \frac{\kappa_{\mathrm{f}}}{r^2}
     \bigl( T_s S_t + T_t S_s \bigr)\,\widehat{\bm r},
  \label{eq:B-F-qtest-static}\\[0.5em]
  \bm C\bigl(\mathsf F_s(\bm x), \widetilde{\mathsf q}_t\bigr)
  &= \bm 0,
\end{align}
where we used the one--dimensional spatial structure
(\(\uhat \parallel \widehat{\bm r}\)) to conclude that the vortical form
\(\bm C\) vanishes.

Up to universal calibration factors, the scalar channel
\(A(\mathsf F_s,\widetilde{\mathsf q}_t)\) reproduces the sum of the Newton
and Coulomb field energies at the position of the test body, while the vector
channel \(\bm B(\mathsf F_s,\widetilde{\mathsf q}_t)\) yields a radial
force--like quantity combining gravitational and electrostatic contributions.
The absence of a nontrivial \(\bm C\) in the static case reflects the absence
of static magnetic or gravito--magnetic fields.

\subsection{Superposition principle and total static field}
\label{subsec:superposition}

For a collection of static sources \(B_i\) at positions \(\bm x_i\) with
dressed quaternions \(\widetilde{\mathsf q}_i\), we define the total static
GEM field at \(\bm x\) by linear superposition,
\begin{equation}
  \mathsf F(\bm x)
  := \sum_i \mathsf F_i(\bm x),
  \qquad
  \mathsf F_i(\bm x)
  := \frac{\kappa_{\mathrm{f}}}{r_i^2}\,
      \bigl( T_i\,\qh + S_i\,\widehat{\bm r}_i \bigr),
  \label{eq:F-total-static}
\end{equation}
with
\(
  \bm r_i := \bm x - \bm x_i
\),
\(
  r_i := \|\bm r_i\|
\),
\(
  \widehat{\bm r}_i := \bm r_i / r_i
\).

A test body with dressed quaternion \(\widetilde{\mathsf q}_t\) then
experiences:
\begin{itemize}[leftmargin=*]
  \item a scalar field channel
  \begin{equation}
    A\bigl(\mathsf F(\bm x), \widetilde{\mathsf q}_t\bigr)
    = \sum_i A\bigl(\mathsf F_i(\bm x), \widetilde{\mathsf q}_t\bigr),
  \end{equation}
  which, after appropriate calibration, coincides with the usual sum of
  Newton and Coulomb field energies;
  \item a radial vector field channel
  \begin{equation}
    \bm B\bigl(\mathsf F(\bm x), \widetilde{\mathsf q}_t\bigr)
    = \sum_i \bm B\bigl(\mathsf F_i(\bm x), \widetilde{\mathsf q}_t\bigr),
  \end{equation}
  representing the static gravito--electric plus electrostatic force density
  at \(\bm x\).
\end{itemize}
In this way the quaternionic field \(\mathsf F(\bm x)\) collects in a single
object the information about the scalar potential and the radial field lines
of the combined gravito--electromagnetic configuration in the observable
three--space.

\subsection{Turning on motion: D--rotations and vortical channel}
\label{subsec:motion-vortical}

When the sources and/or the test body move with respect to the laboratory
frame, the corresponding dressed quaternions acquire nontrivial spatial
orientations through the \(D\)--rotations in
\eqref{eq:q-dressed-lab}.
As soon as two dressed quaternions with different spatial directions enter a
bilinear form, the vortical channel \(\bm C\) becomes nonzero.

To leading order in velocity, the static field expression
\eqref{eq:F-single-source} may still be used, but the dressed quaternions of
the moving sources are boosted before entering the bilinear combinations.
For a moving source \(B_s\) the field at \(\bm x\) is
\begin{equation}
  \mathsf F_s^{\mathrm{lab}}(\bm x,t)
  := \frac{\kappa_{\mathrm{f}}}{r^2}\,
      \bigl( T_s^{\mathrm{lab}}(t)\,\qh
            + \mathbf S_s^{\mathrm{lab}}(t,\bm x) \bigr),
\end{equation}
where the spatial part \(\mathbf S_s^{\mathrm{lab}}\) is no longer constrained
to be parallel to \(\widehat{\bm r}\).
The bilinear forms with a test body then produce:
\begin{align}
  \bm B\bigl(\mathsf F_s^{\mathrm{lab}}, \widetilde{\mathsf q}_t^{\mathrm{lab}}\bigr)
  &\sim \text{gravito--electric and electric fields},\\
  \bm C\bigl(\mathsf F_s^{\mathrm{lab}}, \widetilde{\mathsf q}_t^{\mathrm{lab}}\bigr)
  &\sim \text{gravito--magnetic and magnetic fields},
\end{align}
with the precise identification depending on the chosen calibration.

Dimensionally, the vortical channel \(\bm C\) differs from the radial channel
\(\bm B\) by a factor of the flow speed.
In the unimetrical calibration where the flow speed is fixed by the speed of
light \(\clight\), it is natural to define the \emph{physical} magnetic--like
field as
\begin{equation}
  \bm{\mathcal B}_{\mathrm{phys}}
  := \clight\,\bm C,
\end{equation}
so that \(\bm{\mathcal B}_{\mathrm{phys}}\) has the same units as the
gravito--electric and electric field obtained from \(\bm B\) and couples to
currents with the usual Lorentz--like structure.
In this sense the unimetrical picture explains the appearance of the factor
\(\clight\) in the relation between electric and magnetic fields as a
consequence of the different geometric nature of the radial and vortical
bilinear channels.

\subsection{Quaternionic GEM field as a 3D observable}
\label{subsec:GEM-3D-summary}

From the point of view of an observer living in the three--space \(\mathbb
R^3\), the quaternionic GEM field \(\mathsf F(\bm x,t)\) provides a compact
encoding of all static and quasi--static gravitational and electromagnetic
effects:
\begin{itemize}[leftmargin=*]
  \item its scalar bilinear coupling \(A(\mathsf F,\widetilde{\mathsf q}_t)\)
  yields potential energies;
  \item its vector bilinear coupling \(\bm B(\mathsf F,\widetilde{\mathsf q}_t)\)
  yields gravito--electric and electric field strengths;
  \item its vortical bilinear coupling \(\bm C(\mathsf F,\widetilde{\mathsf q}_t)\),
  rescaled by \(\clight\), yields gravito--magnetic and magnetic field
  strengths.
\end{itemize}
All three channels arise from the same underlying quaternionic structure and
the same dressed body quaternions; the distinction between gravitational and
electromagnetic phenomena is traced to the temporal vs spatial channels of
the dressing, not to different kinds of interaction at the algebraic level.

In the next section we will formulate dynamical equations for
\(\mathsf F(\bm x,t)\) and show how a Maxwell--Heaviside type system for the
GEM field emerges from unimetrical conservation laws for the streamlet flow.


%==================== 8. D-rotors and dynamics ====================

\section{D-rotors and relativistic dynamics of GEM}
\label{sec:D-rotors-GEM}

In the static setting of \cref{sec:bilinear-static-GEM,sec:quat-GEM-3space},
the gravito--electromagnetic (GEM) interaction is encoded entirely in the
scalar and vector bilinear forms \(A\) and \(\bm B\) on metrically dressed body
quaternions, with the vortical form \(\bm C\) vanishing.
Relativistic dynamics enters Unimetry through the \emph{D-rotors} --- special
unit quaternions implementing real--imaginary rotations between the temporal
and spatial channels of the flow.
In this section we recall their structure and show how they generate the
relativistic dynamics of the GEM field and its coupling to moving bodies.

\subsection{Unimetrical D-rotors as real--imaginary rotations}
\label{subsec:D-rotors-def}

Let \(\{\hat{h},\hat{\imath},\hat{\jmath},\hat{k}\}\) be the orthonormal quaternion basis as in
\cref{sec:quat-algebra}.
A \emph{D-rotation} is an orthogonal transformation of \(\mathbb H\) which
mixes the real unit \(\qh\) with a single imaginary unit \(\uhat\) and leaves
the orthogonal imaginary directions unchanged.
Geometrically, this is a rotation in the Euclidean plane spanned by the temporal axis \(\qh\) and the spatial direction \(\uhat\).



In matrix form, acting on the basis \((\qh, \uhat, \hat{\bm e}_\perp, \dots)\), the rotation by an angle \(\vartheta\) is:
\begin{equation}
  D(\vartheta)
  = \begin{pmatrix}
      \cos\vartheta & \sin\vartheta & 0 & 0 \\
      -\sin\vartheta & \cos\vartheta & 0 & 0 \\
      0 & 0 & 1 & 0 \\
      0 & 0 & 0 & 1
    \end{pmatrix}.
  \label{eq:D-matrix}
\end{equation}

In the quaternionic algebra, this transformation is implemented by the ``double-sided'' product with a unit quaternion \(\hat{\mathbf d}\).
Crucially, to generate a rotation of angle \(\vartheta\) in the \(\qh\)-\(\uhat\) plane, the generator \(\hat{\mathbf d}\) must carry the \emph{half-angle}:
\begin{equation}
  \mathbf{D}(\vartheta):\ \mathbf{q} \mapsto \hat{\mathbf d}\,\mathbf{q}\,\hat{\mathbf d},
  \qquad
  \hat{\mathbf d} = \cos\frac{\vartheta}{2}\,\qh + \sin\frac{\vartheta}{2}\,\uhat.
  \label{eq:D-rotor-conj}
\end{equation}
Direct calculation shows that \(\hat{\mathbf d}\,\qh\,\hat{\mathbf d} = \hat{\mathbf d}^2 = \cos\vartheta\,\qh + \sin\vartheta\,\uhat\), recovering the matrix action \eqref{eq:D-matrix}.

The D-rotation preserves the Euclidean self-form \(E\) (the total flow magnitude):
\begin{equation}
  E\bigl(\mathbf{D}(\mathbf{q}_1),\mathbf{D}(\mathbf{q}_2)\bigr)
  = E(\mathbf{q}_1,\mathbf{q}_2)
  \qquad\forall\,\mathbf{q}_1,\mathbf{q}_2\in\mathbb H,
  \label{eq:D-preserves-E}
\end{equation}
confirming it is an isometry of the Euclidean proto-space.
However, the scalar form \(A\) (the interaction channel) is \emph{not} preserved. Instead, it transforms as:
\begin{equation}
  A\bigl(\mathbf{D}(\mathbf{q}_1),\mathbf{D}(\mathbf{q}_2)\bigr)
  = A\bigl(\mathbf{q}_1,\mathbf{D}^2(\mathbf{q}_2)\bigr),
  \label{eq:D-acts-on-A}
\end{equation}
where \(\mathbf{D}^2\) corresponds to a rotation by \(2\vartheta\) in the active view.
This modification of the scalar channel under boosts is precisely what generates the \(\gamma\)-factors and time dilation effects in the unimetrical projection.

We parametrize the D-rotation by the kinematic angle \(\vartheta\) associated with the velocity \(v\) relative to the observer.
Using the unimetrical calibration \(\|\boldsymbol\chi\|=\clight\):
\begin{equation}
  \beta := \frac{v}{\clight} = \sin\vartheta,
  \qquad
  \gamma := \frac{1}{\cos\vartheta}.
\end{equation}
Note that \(\gamma\) appears here as a trigonometric secant of the Euclidean rotation angle, ensuring the formalism remains purely geometric.

\subsection{Boosting dressed body quaternions}
\label{subsec:D-on-bodies}

In \cref{sec:bilinear-static-GEM} a static Newtonian body \(B_i\) is represented by
\begin{equation}
  \widetilde{\mathsf q}_i
  = T_i\,\qh + S_i\,\uhat_i,
  \qquad
  T_i = \lambda_E q_i,
  \quad
  S_i = \lambda_G m_i.
  \label{eq:q-tilde-rest}
\end{equation}
In a laboratory frame where \(B_i\) moves with velocity \(\bm v_i\), the dressed quaternion becomes
\begin{equation}
  \widetilde{\mathsf q}_i^{\mathrm{lab}}
  := \mathbf{D}_i\,\widetilde{\mathsf q}_i\,\mathbf{D}_i,
  \qquad
  \mathbf{D}_i = \mathbf{D}(\vartheta_i,\widehat{\bm v}_i),
  \label{eq:q-tilde-lab}
\end{equation}
where \(\widehat{\bm v}_i\) provides the axis of the D-rotation.
Physically:
\begin{itemize}[leftmargin=*]
  \item The \emph{intrinsic} invariants (rest mass, charge) are protected by the invariance of the Euclidean form \(E\).
  \item The \emph{projected} components change: the temporal part \(T_i\) feeds into the spatial vector channel (creating currents), and the spatial part \(S_i\) feeds into the temporal channel.
\end{itemize}

\subsection{Field quaternions under D-rotors}
\label{subsec:D-on-field}

The GEM field quaternion \(\mathsf F(\bm x,t)\) transforms under a change of frame via conjugation by the observer's D-rotor \(\mathbf{D}_{\text{obs}}\):
\begin{equation}
  \mathsf F'(\bm x',t')
  = \mathbf{D}_{\text{obs}}\,\mathsf F(\bm x,t)\,\mathbf{D}_{\text{obs}}.
  \label{eq:F-transform}
\end{equation}
Writing \(\mathsf F = \Phi\,\qh + \mathbf{\mathcal E} + \mathbf{\mathcal C}\) (where \(\mathbf{\mathcal E}\) is symmetric/radial and \(\mathbf{\mathcal C}\) is vortical), this transformation mixes \(\mathbf{\mathcal E}\) and \(\mathbf{\mathcal C}\) exactly as Lorentz boosts mix electric and magnetic fields. A pure radial field in one frame appears as a superposition of radial and vortical fields in a boosted frame.

\subsection{Relativistic GEM force via bilinear couplings}
\label{subsec:relativistic-GEM-force}

For a test body \(B_t\) moving with rotor \(\mathbf{D}_t\), the effective dressed quaternion is \(\widetilde{\mathsf q}_t^{\mathrm{lab}} = \mathbf{D}_t \widetilde{\mathsf q}_t \mathbf{D}_t\). The force channels are defined by the bilinear forms with the local field \(\mathsf F\):

\begin{align}
  \mathcal P
  &:= A\bigl(\mathsf F, \widetilde{\mathsf q}_t^{\mathrm{lab}}\bigr)
  \quad \text{(Power channel)},
  \label{eq:power-channel}\\[0.4em]
  \bm{\mathcal F}_B
  &:= \bm B\bigl(\mathsf F, \widetilde{\mathsf q}_t^{\mathrm{lab}}\bigr)
  \quad \text{(Radial/Symm. force)},
  \label{eq:force-B-channel}\\[0.4em]
  \bm{\mathcal F}_C
  &:= \bm C\bigl(\mathsf F, \widetilde{\mathsf q}_t^{\mathrm{lab}}\bigr)
  \quad \text{(Vortical force)}.
  \label{eq:force-C-channel}
\end{align}

The vortical channel \(\bm{\mathcal F}_C\) is identically zero for static bodies but activates for moving ones. Dimensional analysis suggests the identification of the physical magnetic-like force density as:
\begin{equation}
  \bm{\mathcal F}_{\mathrm{mag}}
  := \clight\,\bm{\mathcal F}_C.
\end{equation}
In the low-velocity limit, \(\bm{\mathcal F}_C\) reproduces the structure \(q (\bm v \times \bm B)\), confirming that the magnetic force is the manifestation of the antisymmetry of the quaternion product (the \(\bm C\)-form) acting on boosted flow components.

\subsection{D-rotor fields and Unimetric equivalence principle}
\label{subsec:D-field-equivalence}

Gravity can be essentially described as a field of D-rotors \(\mathbf{D}(\bm x)\) acting on the local time direction. Free fall corresponds to a trajectory that follows the ``straight'' flow in proto-space, meaning the body's internal rotor is parallel to the local background rotor field. The equivalence principle arises because a spatially varying D-rotor field (gravity) is locally indistinguishable from the time-varying D-rotor of an accelerated observer.

\section{D-rotors and relativistic dynamics of GEM}
\label{sec:D-rotors-GEM}

In the static setting of \cref{sec:bilinear-static-GEM,sec:quat-GEM-3space}
the gravito--electromagnetic (GEM) interaction is encoded entirely in the
scalar and vector bilinear forms \(A\) and \(B\) on metrically dressed body
quaternions, with the vortical form \(C\) vanishing.
Relativistic dynamics enters Unimetry through the \emph{D-rotors} --- special
unit quaternions implementing real--imaginary rotations between the temporal
and spatial channels of the flow.
In this section we recall their structure and show how they generate the
relativistic dynamics of the GEM field and its coupling to moving bodies.

\subsection{Unimetrical D-rotors as real--imaginary rotations}
\label{subsec:D-rotors-def}

Let \(\{\hat{h},\hat{\imath},\hat{\jmath},\hat{k}\}\) be the orthonormal quaternion basis as in
\cref{sec:quat-algebra}.
A \emph{D-rotation} is an orthogonal transformation of \(\mathbb H\) which
mixes the real unit \(\qh\) with a single imaginary unit \(\uhat\) and leaves
the orthogonal imaginary directions unchanged.
In matrix form its action on the basis can be written as
\begin{equation}
  D(\psi)
  = \begin{pmatrix}
      \cos\psi & \sin\psi & 0 & 0 \\
      -\sin\psi & \cos\psi & 0 & 0 \\
      0 & 0 & 1 & 0 \\
      0 & 0 & 0 & 1
    \end{pmatrix},
  \qquad
  \det D(\psi) = 1,
  \label{eq:D-matrix}
\end{equation}
where the first two rows/columns correspond to \(\qh\) and \(\uhat\), and the
remaining ones to the transverse imaginary basis vectors.

Equivalently, one can represent the same transformation by a unit quaternion
\(\hat{\mathbf d}\) acting by conjugation:
\begin{equation}
  \mathbf{D}:\ \mathbf{q} \mapsto \hat{\mathbf d}\,\mathbf{q}\,\hat{\mathbf d},
  \qquad
  \|\hat{\mathbf d}\| = 1.
  \label{eq:D-rotor-conj}
\end{equation}
In this representation, \(\hat{\mathbf d}\) implements a rotation in the
\((\qh,\uhat)\)-plane, and the equality between \eqref{eq:D-matrix} and
\eqref{eq:D-rotor-conj} can be checked explicitly by applying
\(\mathbf{D}\) to the basis vectors.

The crucial property of the D-rotation is that it preserves the Euclidean
self-form \(E\):
\begin{equation}
  E\bigl(\mathbf{D}(\mathbf{q}_1),\mathbf{D}(\mathbf{q}_2)\bigr)
  = E(\mathbf{q}_1,\mathbf{q}_2)
  \qquad\forall\,\mathbf{q}_1,\mathbf{q}_2\in\mathbb H,
  \label{eq:D-preserves-E}
\end{equation}
hence it is an orthogonal transformation in the sense of the Euclidean
metric.
In contrast, the scalar form \(A\) is not preserved; one finds
\begin{equation}
  A\bigl(\mathbf{D}(\mathbf{q}_1),\mathbf{D}(\mathbf{q}_2)\bigr)
  = A\bigl(\mathbf{q}_1,\mathbf{D}^2(\mathbf{q}_2)\bigr)
  = A\bigl(\mathbf{D}^2(\mathbf{q}_1),\mathbf{q}_2\bigr),
  \label{eq:D-acts-on-A}
\end{equation}
so that the effect of a D-rotation on the scalar channel is equivalent to
applying the \emph{square} of the D-operator to one of the arguments.
This ``double'' action will be responsible for the familiar Lorentz
\(\gamma\)-factors in the kinematic sector.

As in the purely kinematic unimetrical formulation of special relativity,
we parametrize the D-rotation angle \(\psi\) by the kinematic angle
\(\vartheta\) associated with the velocity \(v\) of the body with respect to
a chosen observer.
In the calibration where the Euclidean flow speed equals the speed of light
(\(\|\boldsymbol\chi\|=\clight\)), the usual relations
\begin{equation}
  \beta := \frac{v}{\clight} = \sin\vartheta,
  \qquad
  \gamma := \frac{1}{\sqrt{1-\beta^2}} = \frac{1}{\cos\vartheta}
\end{equation}
allow us to identify the D-rotor angle with \(\vartheta\).
The D-rotor is purely dimensionless; the factor \(\clight\) enters only
through this calibration, not through the rotor itself.

\subsection{Boosting dressed body quaternions}
\label{subsec:D-on-bodies}

In \cref{sec:bilinear-static-GEM} each Newtonian body \(B_i\) was represented
in its rest frame by a metrically dressed quaternion
\begin{equation}
  \widetilde{\mathsf q}_i
  = T_i\,\qh + S_i\,\uhat_i,
  \qquad
  T_i = \lambda_E q_i,
  \quad
  S_i = \lambda_G m_i,
  \label{eq:q-tilde-rest}
\end{equation}
where \(\uhat_i\) is the free unit imaginary aligned with the radial
direction of interaction in the static configuration.
In a laboratory frame where \(B_i\) moves with velocity \(\bm v_i\), its
dressed quaternion becomes
\begin{equation}
  \widetilde{\mathsf q}_i^{\mathrm{lab}}
  := \mathbf{D}_i\,\widetilde{\mathsf q}_i\,\mathbf{D}_i,
  \qquad
  \mathbf{D}_i = \mathbf{D}(\vartheta_i,\widehat{\bm v}_i),
  \label{eq:q-tilde-lab}
\end{equation}
with \(\vartheta_i\) related to \(\bm v_i\) as above and
\(\widehat{\bm v}_i\) the direction of motion in the observer's three-space.

Because of \eqref{eq:D-preserves-E}, the Euclidean self-form --- and hence
the rest-energy scale \(\sqrt{E(\widetilde{\mathsf q}_i,\widetilde{\mathsf q}_i)}\)
--- is invariant under the boost:
\begin{equation}
  E\bigl(\widetilde{\mathsf q}_i^{\mathrm{lab}},
          \widetilde{\mathsf q}_i^{\mathrm{lab}}\bigr)
  = E(\widetilde{\mathsf q}_i,\widetilde{\mathsf q}_i),
\end{equation}
while the scalar form \(A(\cdot,\cdot)\) and the vector forms \(B\) and \(C\)
are modified according to \eqref{eq:D-acts-on-A} and the corresponding
vectorial identities.
Physically this means that:
\begin{itemize}[leftmargin=*]
  \item the internal structure of the body (encoded in the dressing) is
  unaffected by the boost;
  \item the decomposition of the flow between temporal and spatial channels
  changes, adjusting the effective temporal and spatial components seen by
  the laboratory observer;
  \item the cross-terms between bodies acquire \(\gamma\)- and
  \(\beta\)-dependent factors, reproducing relativistic time dilation and
  the emergence of magnetic- and gravito--magnetic contributions.
\end{itemize}

\subsection{Field quaternions under D-rotors}
\label{subsec:D-on-field}

Let \(\mathsf F(\bm x,t)\) be the quaternionic GEM field in the laboratory
frame as defined in \cref{sec:quat-GEM-3space}.
A change of inertial frame to an observer related by a D-rotation with rotor
\(\mathbf{D}\) acts on the field quaternion exactly as on any other
quaternionic quantity:
\begin{equation}
  \mathsf F'(\bm x',t')
  = \mathbf{D}\,\mathsf F(\bm x,t)\,\mathbf{D},
  \label{eq:F-transform}
\end{equation}
with \((\bm x',t')\) related to \((\bm x,t)\) by the same unimetrical
kinematics that underlie the SR reformulation.

Writing the field in the laboratory frame as
\begin{equation}
  \mathsf F = \Phi\,\qh + \mathbf{\mathcal E}\cdot\uhat
              + \mathbf{\mathcal C},
\end{equation}
where:
\begin{itemize}[leftmargin=*]
  \item \(\Phi\) is the scalar potential-like part;
  \item \(\mathbf{\mathcal E}\) is the radial gravito--electric plus electric
  field (associated mainly with the \(B\)-form);
  \item \(\mathbf{\mathcal C}\) is the vortical part (associated with the
  \(C\)-form),
\end{itemize}
the transformed field \(\mathsf F'\) will in general have mixtures of these
components, exactly as in the standard Lorentz transformation of electric
and magnetic fields.
In particular:
\begin{itemize}[leftmargin=*]
  \item a purely radial field in one frame acquires a vortical component in a
  boosted frame;
  \item the relative weights of the radial and vortical channels are
  governed by \(\beta\) and \(\gamma\), coming from the D-rotor angle
  \(\vartheta\).
\end{itemize}
The key advantage of the quaternionic formulation is that both
gravito--electric and electrostatic fields, as well as their magnetic
counterparts, transform via the \emph{same} conjugation rule
\eqref{eq:F-transform}.

\subsection{Relativistic GEM force via bilinear couplings}
\label{subsec:relativistic-GEM-force}

Consider now a test body \(B_t\) with dressed quaternion
\(\widetilde{\mathsf q}_t\) and kinematic D-rotor \(\mathbf{D}_t\)
corresponding to its motion in the laboratory frame.
The \emph{effective} dressed quaternion in the lab frame is
\(\widetilde{\mathsf q}_t^{\mathrm{lab}} = \mathbf{D}_t\,
\widetilde{\mathsf q}_t\,\mathbf{D}_t\).
The local field felt by the test body is \(\mathsf F(\bm x_t,t)\).
We define the GEM force channels by the bilinear couplings:
\begin{align}
  \mathcal P
  &:= A\bigl(\mathsf F(\bm x_t,t),
             \widetilde{\mathsf q}_t^{\mathrm{lab}}\bigr),
  \label{eq:power-channel}\\[0.4em]
  \bm{\mathcal F}_B
  &:= \bm B\bigl(\mathsf F(\bm x_t,t),
                 \widetilde{\mathsf q}_t^{\mathrm{lab}}\bigr),
  \label{eq:force-B-channel}\\[0.4em]
  \bm{\mathcal F}_C
  &:= \bm C\bigl(\mathsf F(\bm x_t,t),
                 \widetilde{\mathsf q}_t^{\mathrm{lab}}\bigr).
  \label{eq:force-C-channel}
\end{align}
Here:
\begin{itemize}[leftmargin=*]
  \item \(\mathcal P\) is a scalar channel which, after calibration, can be
  identified with the instantaneous power input to the test body;
  \item \(\bm{\mathcal F}_B\) is a radial force channel arising from the
  \(B\)-form and reproducing, in the nonrelativistic limit, the sum of the
  Newton and Coulomb forces;
  \item \(\bm{\mathcal F}_C\) is a vortical force channel arising from the
  \(C\)-form and reproducing magnetic- and gravito--magnetic forces.
\end{itemize}

Dimensionally, \(\bm{\mathcal F}_C\) differs from \(\bm{\mathcal F}_B\) by a
factor of the flow speed.
In the unimetrical calibration where the Euclidean flow speed is fixed at
\(\clight\), it is natural to define the physical magnetic-like force density
by
\begin{equation}
  \bm{\mathcal F}_{\mathrm{mag}}
  := \clight\,\bm{\mathcal F}_C.
\end{equation}
In the low-velocity limit (\(\beta\ll 1\)), the radial channel then yields
\begin{equation}
  \bm{\mathcal F}_B
  \;\longrightarrow\;
  -\,\frac{G\,m_s m_t}{r^2}\,\widehat{\bm r}
   + \frac{1}{4\pi\epsilon_0}\,\frac{q_s q_t}{r^2}\,\widehat{\bm r},
\end{equation}
while the vortical channel reduces to the familiar Lorentz-like form
\begin{equation}
  \bm{\mathcal F}_{\mathrm{mag}}
  \;\longrightarrow\;
  q_t\,\bm v_t \times \bm B
  \quad\text{(and its gravito--magnetic analogue)},
\end{equation}
with \(\bm B\) extracted from \(\bm C\) via the identification
\(\bm B_{\mathrm{phys}} = \clight\,\bm C\).

Thus the quaternionic bilinear couplings to the GEM field reproduce the
structure of the relativistic Lorentz force in both electromagnetic and
gravito--electromagnetic sectors, while keeping the underlying algebraic
machinery unified.

\subsection{D-rotor fields and Unimetric equivalence principle}
\label{subsec:D-field-equivalence}

Finally, we briefly comment on the role of spatially varying D-rotors.
In Unimetry, a purely gravitational field can be viewed as a field of
D-rotors \(\mathbf{D}(\bm x)\) acting on the temporal direction of the
observer and on the body quaternions.
In this picture:
\begin{itemize}[leftmargin=*]
  \item a uniform gravitational field corresponds to a linear variation of
  the D-rotor angle with position, mimicking a constant acceleration;
  \item free fall is represented by a trajectory along which the local
  D-rotor remains constant in the comoving frame --- the object follows the
  locally ``straight'' flow in proto-space;
  \item the equivalence between acceleration and gravity appears as the
  equivalence between a time-dependent D-rotor in a flat GEM field and a
  spatially varying D-rotor in a static field.
\end{itemize}
The GEM field quaternion \(\mathsf F(\bm x,t)\) and the D-rotor field
\(\mathbf{D}(\bm x,t)\) are thus two complementary descriptions of the same
underlying unimetrical dynamics: \(\mathsf F\) encodes the local
gravito--electromagnetic stresses in the observable three-space, while
\(\mathbf{D}\) encodes how the local notion of time and space is rotated with
respect to a global proto-space flow.

A full dynamical formulation would express the evolution of
\(\mathsf F(\bm x,t)\) and \(\mathbf{D}(\bm x,t)\) through unimetrical
conservation laws for streamlets, leading to a Maxwell--Heaviside--type
system for GEM in the three-space.
Here we have restricted ourselves to outlining how relativistic kinematics
enters the GEM sector via D-rotors and how the Lorentz structure of forces
emerges from the quaternionic bilinear forms.


%==================== 9. Calibration of E, B and C_phys ====================

\section{Quaternionic GEM field equations in three--space}
\label{sec:GEM-field-equations}

In the preceding sections we introduced metrically dressed body quaternions,
the bilinear forms \(A,B,C,E\), the static gravito--electromagnetic (GEM)
interaction, and the relativistic kinematics of bodies and fields via
D-rotors.
We now outline how a Maxwell--Heaviside--type system of field equations for
the quaternionic GEM field \(\mathbf F(\bm x,t)\) can be formulated in the
observable three--space, starting from conservation of streamlet flow.

Our goal here is structural rather than exhaustive: we emphasize how the
usual Maxwell and gravito--electromagnetic equations emerge as different
projections of a single quaternionic differential relation.

\subsection{Quaternionic differential operator and source current}
\label{subsec:quat-operator-current}

We work in a fixed laboratory frame with coordinate time \(t\) and Euclidean
three--space \(\mathbb R^3\) with orthonormal basis
\(\{\hat e_i\}_{i=1}^3\).
Following the spirit of \cref{sec:quat-algebra,sec:quat-GEM-3space}, we
introduce a quaternionic differential operator
\begin{equation}
  \mathcal D
  := \hat{h}\,\frac{\partial}{\partial t}
     + \sum_{i=1}^3 \hat e_i\,\frac{\partial}{\partial x^i},
  \label{eq:D-operator}
\end{equation}
which acts on quaternion-valued fields by left multiplication and ordinary
differentiation of components.

The total source current is represented by a quaternionic field
\(\mathbf J(\bm x,t)\) that combines mass and charge densities and their
currents:
\begin{equation}
  \mathbf J
  := \rho_E\,\hat{h}
     + \mathbf j_E
     + \rho_G\,\hat{\mathbf u}
     + \mathbf j_G,
  \label{eq:J-def}
\end{equation}
where
\begin{itemize}[leftmargin=*]
  \item \(\rho_E(\bm x,t)\) is the electric charge density;
  \item \(\mathbf j_E(\bm x,t)\) is the electric current density vector
        (expressed as a pure imaginary quaternion);
  \item \(\rho_G(\bm x,t)\) is the mass density (``gravitational charge'');
  \item \(\mathbf j_G(\bm x,t)\) is the mass current density (momentum
        density), again as a pure imaginary quaternion;
  \item \(\hat{\mathbf u}\) is the free unit imaginary introduced earlier,
        which in isotropic configurations aligns with radial directions of
        interaction.
\end{itemize}

The conservation of total streamlet flow in the laboratory frame is then
encoded in the quaternionic continuity equation
\begin{equation}
  A\bigl(\mathcal D, \mathbf J\bigr) = 0,
  \label{eq:quat-continuity}
\end{equation}
whose expansion yields the usual continuity equations for mass and charge:
\begin{align}
  \frac{\partial \rho_E}{\partial t}
  + \nabla\cdot\mathbf j_E &= 0,
  \label{eq:charge-continuity}\\[0.3em]
  \frac{\partial \rho_G}{\partial t}
  + \nabla\cdot\mathbf j_G &= 0.
  \label{eq:mass-continuity}
\end{align}
Here \(\nabla\) is the standard gradient in \(\mathbb R^3\).

\subsection{Quaternionic GEM field equation}
\label{subsec:quat-field-equation}

We now postulate that the quaternionic GEM field \(\mathbf F(\bm x,t)\)
satisfies a first-order differential equation driven by the source current:
\begin{equation}
  \mathcal D \circ \mathbf F = \mathbf J,
  \label{eq:quat-field-equation}
\end{equation}
where \(\circ\) denotes quaternion multiplication as in
\cref{sec:quat-algebra}.
This is the unimetrical analogue of the inhomogeneous Maxwell equations,
combining both gravito--electric and electro--magnetic sectors in a single
quaternionic relation.

A second equation of ``Bianchi type'' is obtained by demanding that the GEM
field derives from an underlying quaternionic potential \(\mathbf A\):
\begin{equation}
  \mathbf F = \mathcal D \circ \mathbf A,
  \label{eq:F-from-A}
\end{equation}
which implies the homogeneous constraint
\begin{equation}
  \mathcal D \circ \mathbf F = \mathcal D \circ\mathcal D \circ \mathbf A
  \equiv 0
  \qquad\Rightarrow\qquad
  \mathcal D \circ \mathbf F = \mathbf J
  \ \Longrightarrow\
  \mathcal D \circ \mathbf J = 0.
\end{equation}
Thus the continuity equation \eqref{eq:quat-continuity} is automatically
satisfied if \(\mathbf F\) is derived from a potential \(\mathbf A\).

The pair \eqref{eq:quat-field-equation}--\eqref{eq:F-from-A} is the
quaternionic GEM analogue of Maxwell's system \(dF = 0\),
\(d\!\star F = J\) in the language of differential forms.

\subsection{3+1 decomposition: gravito--electric and magnetic sectors}
\label{subsec:3plus1-GEM}

To connect \eqref{eq:quat-field-equation} with familiar field equations we
decompose \(\mathbf F\) into temporal, radial and vortical parts as in
\cref{sec:quat-GEM-3space}:
\begin{equation}
  \mathbf F
  = \Phi\,\hat{h}
    + \mathbf{\mathcal E}_E
    + \mathbf{\mathcal E}_G
    + \mathbf{\mathcal C},
  \label{eq:F-decomposition}
\end{equation}
where, schematically,
\begin{itemize}[leftmargin=*]
  \item \(\Phi\) is a scalar potential-like component;
  \item \(\mathbf{\mathcal E}_E\) is the electric field (temporal channel,
        sourced by \(\rho_E,\mathbf j_E\));
  \item \(\mathbf{\mathcal E}_G\) is the gravito--electric field (spatial
        channel, sourced by \(\rho_G,\mathbf j_G\));
  \item \(\mathbf{\mathcal C}\) is the vortical part associated with the
        vector bilinear form \(C\) and will give rise, after calibration, to
        magnetic and gravito--magnetic fields.
\end{itemize}

Inserting \eqref{eq:D-operator} and \eqref{eq:F-decomposition} into
\eqref{eq:quat-field-equation} and collecting the scalar and vector parts
yields, after projection on the temporal (electric) and spatial
(gravitational) channels, the following schematic system:
\begin{align}
  \nabla\cdot\mathbf{\mathcal E}_E
  &= \frac{\rho_E}{\epsilon_0},
  &
  \nabla\cdot\mathbf{\mathcal E}_G
  &= -\,4\pi G\,\rho_G,
  \label{eq:Gauss-E-G}
  \\[0.3em]
  \nabla\times\mathbf{\mathcal C}_E
    - \frac{1}{\clight^2}\,\frac{\partial \mathbf{\mathcal E}_E}{\partial t}
  &= \mu_0\,\mathbf j_E,
  &
  \nabla\times\mathbf{\mathcal C}_G
    - \frac{1}{\clight^2}\,\frac{\partial \mathbf{\mathcal E}_G}{\partial t}
  &= -\,\frac{4\pi G}{\clight^2}\,\mathbf j_G,
  \label{eq:Ampere-like}
\end{align}
where \(\mathbf{\mathcal C}_E\) and \(\mathbf{\mathcal C}_G\) are the
electric and gravito--magnetic parts of \(\mathbf{\mathcal C}\) selected by
projecting on the temporal and spatial dressing channels.

Similarly, the homogeneous constraint arising from
\eqref{eq:F-from-A} gives Bianchi-type identities
\begin{align}
  \nabla\cdot\mathbf{\mathcal C}_E &= 0,
  &
  \nabla\cdot\mathbf{\mathcal C}_G &= 0,
  \label{eq:no-monopoles}
  \\[0.3em]
  \nabla\times\mathbf{\mathcal E}_E
    + \frac{\partial \mathbf{\mathcal C}_E}{\partial t} &= 0,
  &
  \nabla\times\mathbf{\mathcal E}_G
    + \frac{\partial \mathbf{\mathcal C}_G}{\partial t} &= 0.
  \label{eq:Faraday-like}
\end{align}
In this way the standard Maxwell equations and their gravito--electromagnetic
analogue appear simply as different projections of the quaternionic field
equation \eqref{eq:quat-field-equation}.

The calibration constants \(\epsilon_0\), \(\mu_0\) and \(G\) are related to
the dressing constants \(\lambda_E\), \(\lambda_G\) and the global
unimetrical scale by the same relations as in
\cref{subsec:A-static-GEM,sec:quat-GEM-3space}; structurally they are
conversion factors between the dimensionless quaternionic forms and
laboratory units.

\subsection{Covariance under D-rotors and propagation}
\label{subsec:D-covariance-propagation}

Because both \(\mathcal D\) and \(\mathbf F\) transform by conjugation under
D-rotors, the quaternionic field equation \eqref{eq:quat-field-equation} is
covariant under unimetrical boosts:
\begin{equation}
  \mathbf{D}\,\bigl(\mathcal D \circ \mathbf F\bigr)\,\mathbf{D}
  = \mathcal D'\circ\mathbf F'
  = \mathbf J',
\end{equation}
where primed quantities are measured in the boosted frame.
This ensures that the projected 3+1 system
\eqref{eq:Gauss-E-G}--\eqref{eq:Faraday-like} reproduces the familiar Lorentz
transformation properties of electric, magnetic, gravito--electric and
gravito--magnetic fields.

Furthermore, in charge- and mass-free regions (\(\mathbf J=0\)) the field
equation reduces to
\begin{equation}
  \mathcal D \circ \mathbf F = 0,
\end{equation}
which, upon applying \(\mathcal D\) again and using the unimetrical
calibration of the flow speed, yields a wave equation for each component of
\(\mathbf F\) with propagation speed \(\clight\).
Thus the quaternionic GEM field naturally propagates as waves in the
observable three--space with the same universal speed that sets the scale in
the D-rotor kinematics.

\subsection{Summary and outlook}
\label{subsec:GEM-field-summary}

In this section we have:

\begin{itemize}[leftmargin=*]
  \item introduced a quaternionic differential operator \(\mathcal D\) and a
        source current \(\mathbf J\) combining mass and charge densities and
        currents;
  \item postulated a single quaternionic field equation
        \(\mathcal D\circ\mathbf F = \mathbf J\) together with a potential
        representation \(\mathbf F = \mathcal D\circ\mathbf A\);
  \item shown that the 3+1 decomposition of this equation reproduces, after
        appropriate calibration, the standard Maxwell equations for
        electromagnetism and their Newton--Heaviside analogue for gravity;
  \item observed that the whole system is covariant under D-rotors and that
        GEM disturbances propagate at the same universal speed as the
        unimetrical flow.
\end{itemize}

From the unimetrical perspective, these equations are not fundamental but
emerge as an effective description of the averaged dynamics of streamlets in
proto-space.
A more microscopic treatment would start from the 1-form of phase,
\(\alpha = d\Phi\), and derive the quaternionic GEM equations as effective
constraints on coarse-grained flow variables.
We leave this refinement for future work and focus here on the structural
unification already captured by the quaternionic GEM formulation.


%==================== 10. Hamiltonians and Lagrangians ====================

\section{Hamiltonians and Lagrangians of quaternionic GEM}
\label{sec:Ham-Lag-GEM}

The bilinear forms \(A,B,C,E\) provide a natural language for constructing
Hamiltonians and Lagrangians for both particles and fields.
Structurally, the picture is:

\begin{itemize}[leftmargin=*]
  \item the scalar form \(A\) plays the role of a Lorentzian (Minkowski-like)
        quadratic form and is suited for Lagrangians;
  \item the Euclidean self-form \(E\) plays the role of a Hamiltonian
        (energy) functional;
  \item for multi-body systems, the cross term
        \(A(\widetilde{\mathsf q}_1,\widetilde{\mathsf q}_2)\) in
        \(E(\widetilde{\mathsf q}_1+\widetilde{\mathsf q}_2,
             \widetilde{\mathsf q}_1+\widetilde{\mathsf q}_2)\) encodes the
        interaction energy;
  \item for fields, quadratic combinations of \(A,B,C\) over the GEM field
        quaternion \(\mathbf F(\bm x,t)\) yield a Lagrangian density whose
        3+1 decomposition reproduces the standard EM and linearized GEM
        Lagrangians.
\end{itemize}

\subsection{Free-body Hamiltonian}
\label{subsec:free-H}

For a single body \(B_i\) we use the metrically dressed quaternion
introduced in \cref{sec:bilinear-static-GEM}:
\begin{equation}
  \widetilde{\mathsf q}_i
  = T_i\,\hat{h} + S_i\,\hat{\mathbf u}_i,
  \qquad
  T_i = \lambda_E q_i,
  \quad
  S_i = \lambda_G m_i,
  \label{eq:q-tilde-free}
\end{equation}
with \(T_i,S_i\) having units of length and \(\hat{\mathbf u}_i\) the free
unit imaginary aligned with the effective spatial channel of the body in its
rest frame.

The Euclidean self-form of \(\widetilde{\mathsf q}_i\) is
\begin{equation}
  E(\widetilde{\mathsf q}_i,\widetilde{\mathsf q}_i)
  = T_i^2 + S_i^2,
  \label{eq:E-self-free}
\end{equation}
with the natural Euclidean norm
\begin{equation}
  \|\widetilde{\mathsf q}_i\|_{E}
  := \sqrt{E(\widetilde{\mathsf q}_i,\widetilde{\mathsf q}_i)}.
\end{equation}
Since \(\widetilde{\mathsf q}_i\) is metrically dressed, this norm has
dimension of length.
To obtain an energy we multiply by a universal stiffness scale with dimension
energy per length.
A convenient choice is to express this scale in terms of the Planck force
\(\clight^4/\GNewton\):
\begin{equation}
  H_{\rm free}^{(i)}
  := \kappa_{\rm H}\,
     \|\widetilde{\mathsf q}_i\|_{E},
  \qquad
  \kappa_{\rm H} \sim \frac{\clight^4}{\GNewton}\,\frac{1}{L_\ast},
  \label{eq:H-free-def}
\end{equation}
where \(L_\ast\) is a reference length scale (for example a Compton-like
length associated with the object).

In the unimetrical calibration used in
\cref{sec:flow-calibrations,sec:mass-from-streamlets}, the internal flow rate
\(\dot H\) and the volumetric coefficient \(\kappa\) are chosen such that in
the rest frame
\begin{equation}
  H_{\rm free}^{(i)} \equiv E_{\rm rest}^{(i)}
  = m_{0,i}\,\clight^2,
\end{equation}
with \(m_{0,i}\) defined by the streamlet structure of the body.
Thus, up to the universal conversion factor \(\kappa_{\rm H}\), the free-body
Hamiltonian is simply the Euclidean self-form of the metrically dressed
quaternion.

Formally, this rest energy defines a proper frequency of the global phase
\(\chi\):
\begin{equation}
  \omega_{0,i}
  := \frac{H_{\rm free}^{(i)}}{\hbar}
  \sim \frac{m_{0,i}\,\clight^2}{\hbar},
\end{equation}
which for an electron would correspond to the usual Compton frequency when
the dressing parameters \(\lambda_E,\lambda_G\) are chosen appropriately.
In this sense the unimetrical free-body Hamiltonian provides a bridge between
the internal geometric flow (through \(\widetilde{\mathsf q}_i\)) and the
proper-time oscillation familiar from relativistic quantum mechanics.

\subsection{Interaction Hamiltonian}
\label{subsec:interaction-H}

For a two-body system with dressed quaternions
\(\widetilde{\mathsf q}_1,\widetilde{\mathsf q}_2\), the Euclidean self-form
of the total dressed quaternion
\(\widetilde{\mathsf q}_{\rm tot}
 := \widetilde{\mathsf q}_1 + \widetilde{\mathsf q}_2\)
splits as
\begin{equation}
  E(\widetilde{\mathsf q}_{\rm tot},\widetilde{\mathsf q}_{\rm tot})
  = E(\widetilde{\mathsf q}_1,\widetilde{\mathsf q}_1)
    + E(\widetilde{\mathsf q}_2,\widetilde{\mathsf q}_2)
    + 2\,A(\widetilde{\mathsf q}_1,\widetilde{\mathsf q}_2),
  \label{eq:E-split-two-body}
\end{equation}
so the cross term is precisely the scalar bilinear form \(A\).

In the static regime, where the bodies are separated by a distance \(r\),
the interaction Hamiltonian can be defined as
\begin{equation}
  H_{\rm int}(r)
  := \kappa_{\rm int}\,\frac{A(\widetilde{\mathsf q}_1,
                               \widetilde{\mathsf q}_2)}{r},
  \label{eq:H-int-def}
\end{equation}
with \(\kappa_{\rm int}\) a universal constant.
Using the two-channel dressing \eqref{eq:q-tilde-free}, we obtain
\begin{equation}
  A(\widetilde{\mathsf q}_1,\widetilde{\mathsf q}_2)
  = T_1 T_2 - S_1 S_2
  = \lambda_E^2 q_1 q_2 - \lambda_G^2 m_1 m_2.
  \label{eq:A-split-q-m}
\end{equation}
The first term corresponds to the temporal (electromagnetic) channel,
the second to the spatial (gravitational) channel.

It is convenient to express \(\kappa_{\rm int}\) in terms of a fundamental
force scale \(\clight^4/\GNewton\):
\begin{equation}
  \kappa_{\rm int}
  = \frac{\clight^4}{\GNewton}\,\alpha_{\rm int},
\end{equation}
with \(\alpha_{\rm int}\) a dimensionless structural factor built from
\(\lambda_E,\lambda_G\).
Then
\begin{equation}
  H_{\rm int}(r)
  = \frac{\clight^4}{\GNewton}\,
    \frac{\alpha_{\rm int}}{r}\,
    \bigl(\lambda_E^2 q_1 q_2 - \lambda_G^2 m_1 m_2\bigr).
  \label{eq:H-int-expanded}
\end{equation}
By an appropriate choice of \(\alpha_{\rm int}\lambda_E^2\) and
\(\alpha_{\rm int}\lambda_G^2\) one recovers the familiar Newton and Coulomb
potentials:
\begin{equation}
  H_{\rm int}(r)
  = \frac{1}{4\pi\epsilon_0}\,\frac{q_1 q_2}{r}
    - \GNewton\,\frac{m_1 m_2}{r},
  \label{eq:H-int-Newt-Coulomb}
\end{equation}
as already discussed in \cref{subsec:A-static-GEM}.
The advantage of writing \eqref{eq:H-int-expanded} is that both gravitational
and electromagnetic couplings appear as projections of a single Planck-scale
stiffness \(\clight^4/\GNewton\) dressed by dimensionless geometric factors.

From the Hamiltonian point of view, the total energy of the two-body system
in the static limit is
\begin{equation}
  H_{\rm tot}
  = H_{\rm free}^{(1)} + H_{\rm free}^{(2)} + H_{\rm int}(r),
\end{equation}
with \(H_{\rm free}^{(i)}\) obtained from the self-form \(E\) as in
\cref{subsec:free-H}.
In the full unimetrical picture, boosts implemented by D-rotors add the
kinetic (relativistic) contribution by changing the decomposition of the flow
between temporal and spatial channels, while leaving the internal structure
of \(\widetilde{\mathsf q}_i\) intact.

\subsection{Strong--field gravitational sector and Schwarzschild limit}
\label{sec:strong-field-GEM}

In the weak--field limit the quaternionic GEM recovers the Newtonian
potential and the standard linearized metric,
\[
 g_{tt} \simeq -\clight^2\!\left(1 + \frac{2\Phi_{\rm G}}{\clight^2}\right),
 \qquad
 g_{ij} \simeq \delta_{ij}\!\left(1 - \frac{2\Phi_{\rm G}}{\clight^2}\right),
 \qquad
 \Phi_{\rm G}(\mathbf x) = -\frac{\GNewton M}{r},
\]
where \(\Phi_{\rm G}\) is extracted from the gravitational part of the
bilinear \(A(\widetilde{\mathsf q}_1,\widetilde{\mathsf q}_2)\) in the
interaction Hamiltonian.

To extend this picture to strong fields without leaving the Euclidean
proto--space, we introduce a \emph{gravitational angle} \(\zeta_{\rm G}\),
which measures the fraction of proto--flow locked into spatial directions
by the gravitational field. In the static case we define it by
\begin{equation}
  \sin^2\zeta_{\rm G}(\mathbf x)
  := -\,\frac{2\,\Phi_{\rm G}(\mathbf x)}{\clight^2},
  \qquad
  \Phi_{\rm G} < 0.
  \label{eq:zetaG-def}
\end{equation}
Thus the Newtonian potential is \emph{exactly}
\begin{equation}
  \Phi_{\rm G}(\mathbf x)
  = -\frac{\clight^2}{2}\,\sin^2\zeta_{\rm G}(\mathbf x),
  \label{eq:PhiG-via-zetaG}
\end{equation}
and the dimensionless quantity
\(
  1 - \cos 2\zeta_{\rm G}
  = 2\sin^2\zeta_{\rm G}
\)
plays the role of a geometric ``budget'' of spatialized flow.

\subsubsection{Spherically symmetric source and Schwarzschild sector}

For a static, spherically symmetric source of mass \(M\), the weak--field
limit of the interaction Hamiltonian,
\begin{equation}
  H_{\rm int}^{\rm grav}(r)
  = -\frac{\GNewton M m}{r}
  = \frac{\clight^4}{\GNewton}\,
    \frac{A_{\rm grav}(\widetilde{\mathsf q}_M,\widetilde{\mathsf q}_m)}{r},
\end{equation}
fixes the gravitational part of the bilinear \(A_{\rm grav}\) and gives
the usual Newtonian potential
\(
  \Phi_{\rm G}(r) = -\GNewton M/r.
\)
Substituting this into the angle definition \eqref{eq:zetaG-def} we obtain
\begin{equation}
  \sin^2\zeta_{\rm G}(r)
  = \frac{2\GNewton M}{r\clight^2}
  = \frac{r_{\rm s}}{r},
  \qquad
  r_{\rm s} := \frac{2\GNewton M}{\clight^2},
  \label{eq:zetaG-Schwarzschild}
\end{equation}
so that
\begin{equation}
  \cos^2\zeta_{\rm G}(r)
  = 1 - \sin^2\zeta_{\rm G}(r)
  = 1 - \frac{r_{\rm s}}{r}.
\end{equation}
Equivalently,
\begin{equation}
  1 - \cos 2\zeta_{\rm G}(r)
  = 2\sin^2\zeta_{\rm G}(r)
  = \frac{2r_{\rm s}}{r},
\end{equation}
which makes the structural similarity to the Schwarzschild factor explicit.

In Unimetry the observable metric components are Euclidean proto--components
rescaled by the temporal share of the flow. For the static spherical case
we can therefore write the effective line element as
\begin{equation}
  ds^2
  = -\,\clight^2\cos^2\zeta_{\rm G}(r)\,dt^2
    + \cos^{-2}\zeta_{\rm G}(r)\,dr^2
    + r^2(d\theta^2 + \sin^2\theta\,d\varphi^2).
  \label{eq:metric-zetaG}
\end{equation}
With \eqref{eq:zetaG-Schwarzschild} this becomes
\begin{equation}
  ds^2
  = -\clight^2\!\left(1 - \frac{r_{\rm s}}{r}\right) dt^2
    + \left(1 - \frac{r_{\rm s}}{r}\right)^{-1}\!dr^2
    + r^2(d\theta^2 + \sin^2\theta\,d\varphi^2),
\end{equation}
i.e. exactly the Schwarzschild metric in standard coordinates.

Thus, in the spherical sector the \emph{same} Newtonian potential
\(\Phi_{\rm G}(r) = -\GNewton M/r\), when encoded via the flow angle
\(\zeta_{\rm G}(r)\), reproduces both:
(i) the weak--field limit of quaternionic GEM, and
(ii) the full Schwarzschild strong--field geometry.

\subsubsection{Strong--field GEM equations}

In the linear GEM regime we work directly with \(\Phi_{\rm G}\) and the
gravitational ``electric'' field
\(
  \mathbf g = -\nabla\Phi_{\rm G}
\),
with sources determined by the scalar part of the bilinear
\(A(\widetilde{\mathsf q}_1,\widetilde{\mathsf q}_2)\).
The strong--field extension consists simply in \emph{not} expanding
\(\zeta_{\rm G}\) for small angles:

\begin{itemize}
  \item the scalar potential is still defined by
  \eqref{eq:PhiG-via-zetaG},
  \[
    \Phi_{\rm G}(\mathbf x)
    = -\frac{\clight^2}{2}\,\sin^2\zeta_{\rm G}(\mathbf x),
  \]
  with \(\zeta_{\rm G}\) sourced by the same quaternionic charge through
  the bilinear \(A\);
  \item the GEM fields follow as
  \(
    \mathbf g
    = -\nabla\Phi_{\rm G}
    = \frac{\clight^2}{2}\,\nabla\!\bigl(\sin^2\zeta_{\rm G}\bigr),
  \)
  and the vector/torsional fields (the \(B\) and \(C\) forms of the
  quaternionic product) remain linear in the sources;
  \item the electromagnetic sector stays strictly linear (Maxwell–like):
  no trigonometric dressing is introduced there; nonlinearities from
  extreme EM fields appear only via their contribution to the same
  gravitational angle \(\zeta_{\rm G}\) through the total energy content.
\end{itemize}

In this way the quaternionic GEM keeps its linear Maxwell structure in
both the gravitational and electromagnetic field equations at the level
of sources, while the \emph{geometry} seen by null and timelike probes
is upgraded from the linearized metric to the exact strong--field metric
\eqref{eq:metric-zetaG} through the single flow angle \(\zeta_{\rm G}\).


\subsection{Field Lagrangian in terms of \texorpdfstring{\(A,B,C\)}{A,B,C}}
\label{subsec:field-Lagrangian}

The quaternionic GEM field \(\mathbf F(\bm x,t)\) introduced in
\cref{sec:quat-GEM-3space} can be used to build a field Lagrangian in direct
analogy with the standard EM Lagrangian.
The basic idea is that:

\begin{itemize}[leftmargin=*]
  \item the scalar form \(A(\mathbf F,\mathbf F)\) plays the role of a
        Lorentzian invariant quadratic form (analogous to
        \(F_{\mu\nu}F^{\mu\nu}\));
  \item the norms of the vector forms \(B(\mathbf F,\mathbf F)\) and
        \(C(\mathbf F,\mathbf F)\) distinguish radial and vortical channels
        and allow us to control separately the electric/gravito--electric and
        magnetic/gravito--magnetic sectors;
  \item the Euclidean self-form \(E(\mathbf F,\mathbf F)\) acts as an
        energy density (Hamiltonian density) after appropriate calibration.
\end{itemize}

We introduce a field Lagrangian density of the schematic form
\begin{equation}
  \mathcal L_{\rm field}
  := \alpha_A\,A(\mathbf F,\mathbf F)
     + \alpha_B\,\bigl\|B(\mathbf F,\mathbf F)\bigr\|^2
     + \alpha_C\,\bigl\|C(\mathbf F,\mathbf F)\bigr\|^2,
  \label{eq:L-field-ABC}
\end{equation}
where \(\alpha_A,\alpha_B,\alpha_C\) are constants with dimensions chosen so
that \(\mathcal L_{\rm field}\) has units of energy density.
The full action is
\begin{equation}
  S_{\rm field}
  = \int \mathcal L_{\rm field}\,d^3x\,dt.
\end{equation}

To relate \eqref{eq:L-field-ABC} to the standard electromagnetic Lagrangian,
we decompose \(\mathbf F\) into its electric and magnetic components as in
\cref{subsec:3plus1-GEM}:
\begin{equation}
  \mathbf F
  = \Phi\,\hat{h}
    + \mathbf{\mathcal E}_E
    + \mathbf{\mathcal E}_G
    + \mathbf{\mathcal C},
\end{equation}
and then restrict to the purely electromagnetic sector by setting the
gravitational dressing to zero (\(\mathbf{\mathcal E}_G = 0\)) and
projecting onto the temporal channel.
With a suitable calibration one can choose the coefficients so that
\begin{equation}
  \mathcal L_{\rm field}^{\rm EM}
  := \left.\mathcal L_{\rm field}\right|_{\rm EM\,sector}
  = \frac{\epsilon_0}{2}\,\bigl(
      \|\mathbf E\|^2 - \clight^2\|\mathbf B\|^2
    \bigr),
  \label{eq:L-EM}
\end{equation}
where \(\mathbf E\) and \(\mathbf B\) are extracted from \(\mathbf F\) via
the appropriate projections of \(B(\mathbf F,\mathbf F)\) and
\(\clight\,C(\mathbf F,\mathbf F)\).

Similarly, in the purely gravitational (linearized) sector one can project
onto the spatial dressing channel and obtain a Lagrangian density of the
form
\begin{equation}
  \mathcal L_{\rm field}^{\rm GEM}
  \sim \frac{1}{8\pi\GNewton}\,\bigl(
        \|\mathbf{\mathcal E}_G\|^2
        - \clight^2\|\mathbf{\mathcal B}_G\|^2
      \bigr),
\end{equation}
where \(\mathbf{\mathcal E}_G\) and \(\mathbf{\mathcal B}_G\) are the
gravito--electric and gravito--magnetic fields obtained from \(\mathbf F\).
The precise coefficients again depend on the chosen calibration of the
dressing and the normalization of \(\mathbf F\), but structurally the form
is identical to the Maxwell Lagrangian, with \(\epsilon_0\) replaced by an
effective gravitational stiffness \(\sim 1/\GNewton\).

Finally, the total Lagrangian of a system of bodies interacting through the
GEM field is naturally written as
\begin{equation}
  L_{\rm total}
  = \sum_i L_{\rm free}^{(i)}
    + \sum_{i<j} H_{\rm int}^{(ij)}
    + \int \mathcal L_{\rm field}\,d^3x,
\end{equation}
where:
\begin{itemize}[leftmargin=*]
  \item \(L_{\rm free}^{(i)}\) is the free-body Lagrangian of the \(i\)-th
        body, which in the unimetrical calibration is generated by the scalar
        self-form \(A(\widetilde{\mathsf q}_i,\widetilde{\mathsf q}_i)\) and
        reproduces the usual relativistic free Lagrangian;
  \item \(H_{\rm int}^{(ij)}\) is the interaction Hamiltonian
        \eqref{eq:H-int-def} derived from the cross form
        \(A(\widetilde{\mathsf q}_i,\widetilde{\mathsf q}_j)\);
  \item \(\mathcal L_{\rm field}\) is the field Lagrangian density of the
        form \eqref{eq:L-field-ABC}.
\end{itemize}

In this way the same quaternionic bilinear forms \(A,B,C,E\) govern:

\begin{itemize}[leftmargin=*]
  \item the free dynamics of bodies (through \(A\) and \(E\));
  \item their mutual interactions (through cross terms in \(A\));
  \item the dynamics of the GEM field itself (through quadratic combinations
        of \(A,B,C\)).
\end{itemize}

This closes the algebraic circle: the Hamiltonian and Lagrangian structures
of the GEM sector emerge directly from the quaternionic decomposition of the
flow, without introducing separate ad hoc field tensors or potentials beyond
the unified quaternionic objects already present in Unimetry.


%==================== 11. Relation to Maxwell and GEM ====================

\section{Relation to Maxwell's equations and linearized GEM}

\subsection{Static equations and Poisson-type equations}
% TODO: Show how Poisson's equation for \phi_{\rm EM}
% and its gravitational analogue arise from the quaternionic field.

\subsection{Quasi-stationary regime and continuity equations}
% TODO: Show how, in the quasi-stationary regime,
% one recovers \nabla \cdot \bm E, \nabla \times \bm B,
% and continuity equations for charge and mass currents.

\subsection{Comparison with the classical GEM formalism}
% TODO: Discuss the limits L_E = 0 (pure gravity),
% L_G = 0 (pure EM), and correspondence with standard GEM formulations.

%==================== 12. Discussion and conclusions ====================

\section{Discussion and outlook}
\label{sec:discussion-outlook}

The quaternionic gravito--electromagnetic (GEM) formulation presented above
grew out of the unimetrical viewpoint, in which all observable structure is
traced back to a single underlying flow in a Euclidean proto--space.
In this closing section we summarize what is actually achieved by the present
construction, which parts remain explicitly provisional, and where it may be
useful to push the framework further.

\subsection{What is unified, and in what sense}
\label{subsec:what-is-unified}

At a purely algebraic level, the quaternionic GEM formulation brings together
several ingredients that are usually treated separately:
\begin{itemize}[leftmargin=*]
  \item The quaternion product is decomposed into four bilinear forms
        \(A,B,C,E\), which naturally separate into scalar, radial vector,
        vortical vector and Euclidean norm channels.
  \item Metrically dressed body quaternions encode mass and charge as
        effective lengths in temporal and spatial channels of the same
        quaternionic object, without introducing separate ``kinds'' of charge
        in the algebra.
  \item Static Newton and Coulomb interactions are recovered as different
        contributions of a single scalar form \(A\), once mass and charge
        are encoded by the dressing and the familiar constants
        \(G,\,\epsilon_0\) are interpreted as calibration factors between
        quaternionic and laboratory units.
  \item The Euclidean self-form \(E\) of dressed quaternions plays the role
        of a Hamiltonian channel: for a two-body system it splits into two
        self-energy terms and an interaction term proportional to \(A\),
        providing a compact energy bookkeeping scheme.
  \item The vector forms \(B\) and \(C\) supply, respectively, the radial and
        vortical channels that become, after 3+1 decomposition, the
        gravito--electric/electric and gravito--magnetic/magnetic fields.
  \item D-rotors act as fully geometric implementations of relativistic
        boosts, preserving \(E\) and reshuffling the contributions of
        \(A,B,C\) in a way that reproduces the Lorentz structure of forces
        and the mixing of electric and magnetic fields.
  \item A single quaternionic field equation
        \(\mathcal D\circ\mathbf F = \mathbf J\) generates, after
        projection, both Maxwell's equations and their Newton--Heaviside
        analogue for gravity.
\end{itemize}

In this sense the unification achieved here is structural: gravity and
electromagnetism are not merged into one new interaction, but they are
realized as different channels of the same quaternionic objects and the same
underlying bilinear forms.
The distinction between the ``gravitational'' and ``electromagnetic''
sectors is attributed to the temporal versus spatial dressing of the body
quaternions, and to how the free unit imaginary vector \(\hat{\mathbf u}\)
is deployed, not to different algebraic rules.

From the unimetrical viewpoint, this is a natural extension: special
relativity and GEM are no longer separate layers (Minkowski metric plus
Maxwell plus Newton), but different aspects of a single Euclidean
phase--space kinematics with D-rotors.

\subsection{Relation to existing frameworks}
\label{subsec:relation-existing}

On the electromagnetic side, the quaternionic GEM construction is closely
related to familiar biquaternion and Clifford-algebra formulations of
Maxwell's theory, where electric and magnetic fields are combined into a
single multicomponent object and the Maxwell equations take a compact
first-order form.
The difference here is that:
\begin{itemize}[leftmargin=*]
  \item we keep the underlying metric Euclidean and interpret the Minkowski
        structure as an emergent consequence of the unimetrical flow;
  \item we include mass and gravity on the same footing as charge and
        electromagnetism, via the spatial channel of the dressing and the
        gravitational constants in the calibration;
  \item D-rotors are treated as geometric rotations of the real and imaginary
        parts rather than as abstract Lorentz transformations on a separate
        spacetime manifold.
\end{itemize}

On the gravitational side, the connection to general relativity is more
indirect.
The present GEM system corresponds to a linearized or Newton--Heaviside
regime on a fixed background: it captures gravito--electric and
gravito--magnetic fields generated by slowly moving masses, but it does not
yet reproduce the full non-linear Einstein--Hilbert dynamics of the metric.
Instead, gravitational effects are encoded in:
\begin{itemize}[leftmargin=*]
  \item the spatial dressing of body quaternions (how much proto-flow is
        locked into spatial loops);
  \item the D-rotor field \(\mathbf{D}(\bm x,t)\) that rotates local time and
        space directions in response to mass distributions;
  \item the gravito--electric and gravito--magnetic parts of the quaternionic
        field \(\mathbf F\) in three-space.
\end{itemize}
A genuine replacement for Einstein's equations would require promoting the
D-rotor field (or its derivative invariants) to the primary gravitational
variable and deriving its dynamics from unimetrical flow conservation at the
streamlet level.
This step is not carried out in the present text.

\subsection{Limitations and open conceptual questions}
\label{subsec:limitations}

Several limitations of the present formulation are worth stating explicitly.

\paragraph{Linear regime and weak fields.}
The GEM equations derived from the quaternionic field equation are of
Maxwell--Heaviside type and are therefore intrinsically linear in the fields.
Non-linear gravitational phenomena --- strong-field effects, black holes,
cosmological solutions --- lie outside this regime.
The unimetrical picture suggests that such effects should be encoded in
non-linear constraints on the D-rotor field and the streamlet ensemble
statistics, but this has not been made explicit.

\paragraph{Backreaction and self-energy.}
The use of dressed body quaternions and bilinear forms provides a compact
description of interaction energies, but it sidesteps the delicate question
of self-energy and radiation reaction.
In particular, the feedback of the GEM field \(\mathbf F\) on the internal
streamlet structure of a body (and hence on its mass functional) is not
modelled.
A consistent unimetrical treatment would have to specify how \(\mathbf C_B\)
and the structural angle \(\zeta_B\) evolve under emission and absorption of
GEM radiation.

\paragraph{Geometry of the vacuum and ``rion'' medium.}
At several points the construction tacitly assumes the existence of a
background flow or condensate (the ``rion'' medium) whose effective
stiffness sets the values of the gravitational and electromagnetic
constants.
While the unified dressing suggests that \(G\) and \(\epsilon_0\) should be
related to different projections of the same vacuum quaternion, no explicit
relation is derived here.
Understanding this link would require a microscopic model of the vacuum as
a streamlet ensemble and an analysis of how its second moments determine
the effective GEM calibration constants.

\paragraph{Global hyperbolicity and causal structure.}
The unimetrical framework treats time as a geometric direction emerging from
the phase flow, and the Minkowski-like interval arises from the Euclidean
norm of the flow vector and the cyclic interpretation of the temporal axis.
However, the global causal structure --- horizons, singularities, and
cosmological boundary conditions --- is not addressed.
It remains to be seen how far the emergent Minkowski picture can be pushed
before explicit non-Euclidean geometry becomes unavoidable.

\subsection{Possible observational and theoretical tests}
\label{subsec:tests}

Although the present work is largely structural, it hints at several
directions where concrete predictions or consistency checks might be
extracted.

\paragraph{Relations between coupling constants.}
The unified dressing and the appearance of a single vacuum scale in both
the gravitational and electromagnetic sectors suggest that dimensionless
combinations of \(G\), \(\epsilon_0\), \(\clight\) and possibly \(\hbar\)
could be interpreted as ratios of geometric invariants of the underlying
streamlet ensemble.
If the same structural parameter \(k\) (or the same class of invariants
built from \(\mathbf C_B\)) controls both mass and charge channels, this may
lead to constraints on how these constants can vary in cosmological
settings driven by changes in the background flow.

\paragraph{GEM in strong-field but slow-motion regimes.}
In settings where velocities are small but gravitational fields are strong
(e.g.\ near compact objects, in certain binary systems), the Maxwell--Heaviside
picture is often used as an effective approximation.
The quaternionic GEM formulation provides an alternative parametrization of
such regimes.
Comparing its predictions for gravito--magnetic effects (precession,
frame-dragging, waveforms) with those of standard post-Newtonian GR could
reveal whether the unimetrical dressing introduces any subtle, testable
differences.

\paragraph{Structure of massive bodies from streamlet ensembles.}
On the microscopic side, the identification of rest mass with functionals of
the spatial second moment \(\mathbf C_B\) suggests that different classes of
objects (elementary particles, composite nuclei, macroscopic bodies) may
correspond to distinct families of streamlet configurations.
Even simple toy models of such ensembles could be used to check whether
reasonable mass spectra and scaling relations can be reproduced without
fine-tuning.

\subsection{Future directions}
\label{subsec:future-directions}

Several natural extensions of the present work suggest themselves.

\paragraph{From GEM to full Unimetry.}
Here we treated GEM as a sector of Unimetry, focusing on how the existing
Newton--Coulomb--Maxwell phenomenology fits into the unimetrical phase-space
picture.
A more ambitious project is to invert the logic: to start from the
unimetrical postulates about phase 1-forms, streamlet ensembles and D-rotors
and derive GEM, special relativity and parts of general relativity as
emergent effective descriptions.
This would require, in particular, a systematic treatment of the mass
functional \(m_0(\zeta_B,\mathbf C_B)\) and the dynamics of \(\mathbf C_B\)
under interactions.

\paragraph{Non-linear D-rotor dynamics and effective Einstein equations.}
If the gravitational field is fundamentally a field of D-rotors
\(\mathbf{D}(\bm x,t)\), then one can attempt to derive effective Einstein-like
equations by constructing curvature-like invariants from spatial
gradients and time derivatives of \(\mathbf{D}\).
Such invariants would replace the Riemann tensor as descriptors of how local
time and space are rotated relative to the global proto-space flow.
The challenge is to connect these invariants back to observable quantities
in three-space and to the stress-energy content encoded in dressed
quaternions and \(\mathbf J\).

\paragraph{Quantization and relation to quantum theory.}
The quaternionic structure and the role of phase in Unimetry suggest
uneasy but intriguing parallels with quantum mechanics.
Whether the streamlet picture can provide a realist underpinning for
quantum amplitudes, or whether the quaternionic GEM fields can be related to
spinor or gauge-field descriptions at the quantum level, remains an open
question.
Any such attempt would have to respect the well-tested structure of quantum
field theory while offering a genuinely new geometric interpretation.

\medskip

In summary, the quaternionic GEM formulation proposed here is a step towards
expressing gravity and electromagnetism in a language entirely adapted to
the unimetrical phase-space picture: Euclidean at the fundamental level, but
capable of reproducing relativistic kinematics and field dynamics in the
observable three-space.
It is neither a complete theory of gravity nor a replacement for quantum
field theory, but a structural bridge between familiar phenomenology and a
different way of thinking about flow, time and interaction.
Further work will show whether this bridge can carry quantitative weight or
remains primarily a conceptual tool for organizing known physics.


%==================== Appendices ====================

\appendix

\section{Quaternion algebra and matrix representation (details)}
% TODO: Provide more detailed algebraic and matrix derivations.

\section{Extended notation table}
% TODO: Full table of symbols used in the paper.

\section{Dimensional analysis and numerical estimates}
% TODO: Work out explicit examples for the electron, proton, etc.

\section{D-rotors, Lorentz transformations and Wigner--Thomas rotation}
% TODO: Show in detail how the D-rotor formalism reproduces Lorentz boosts
% and the Wigner--Thomas rotation.

\end{document}
