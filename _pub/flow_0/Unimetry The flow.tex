\documentclass[11pt]{article}

\usepackage[a4paper,margin=1in]{geometry}
\usepackage{microtype}
\usepackage{amsmath,amssymb,amsfonts}
\usepackage{hyperref}
\usepackage{float}
\usepackage{placeins}
\usepackage{amsmath}
\usepackage{booktabs} 
\usepackage{array,booktabs,tabularx}
\usepackage{tikz}
\usepackage{indentfirst}
\usepackage{amsthm}

\theoremstyle{definition}
\newtheorem{definition}{Definition}[section]

\theoremstyle{plain}
\newtheorem{theorem}{Theorem}[section]

\theoremstyle{remark}
\newtheorem*{proofsketch}{Proof sketch} % если хотите отдельное окружение

\usetikzlibrary{arrows.meta,calc,angles,quotes}

\newcolumntype{L}{>{\raggedright\arraybackslash}X}
\newcommand{\sech}{\operatorname{sech}}


\numberwithin{equation}{section}

\title{Unimetry: A Phase-Space Reformulation of Special Relativity}
\author{Timur Abizgeldin\\ \small Independent researcher, Austria\\ \small \texttt{timurabizgeldin@gmail.com}}
\date{\today}


% --- Unimetry macros ---
\usepackage{tikz}
\usetikzlibrary{arrows.meta,positioning,calc}
\providecommand{\bi}{\mathbf{i}}
\providecommand{\bj}{\mathbf{j}}
\providecommand{\bk}{\mathbf{k}}
\providecommand{\uhat}{\hat{\mathbf u}}
\providecommand{\rotor}[2]{\cos\frac{#2}{2} + #1\,\sin\frac{#2}{2}}

\begin{document}
\maketitle
\begin{abstract}
We propose a compact reformulation of special relativity in which spacetime measures (time and length) are treated as phase velocities, defined as directional derivatives of a single underlying parameter, the flow $\boldsymbol{\chi}\in\mathbb{H}$. In this framework, the observable Minkowski interval emerges as a conserved quantity under a change of parameter from the phase coordinate $\chi$ to the observer’s proper time $\tau$. Familiar relativistic effects, such as time dilation, the Lorentz factor, the Doppler shift, and relativistic velocity composition, arise as elementary projections and rotations within a Euclidean phase plane. Hyperbolic features of Lorentz kinematics reappear after a reparametrization of time, yielding the classical relations without altering empirical content. We provide closed-form derivations for the longitudinal and transverse Doppler effects, prove a lemma equating the total flow speed to the conserved Minkowski norm, and outline connections to gauge phases, rapidity, and a cosmological time gauge. Composition of non-collinear boosts (quaternionic $d$-rotations in $\mathbb{H}$) yields a Wigner rotation; in the continuous limit, this gives Thomas precession. Both effects emerge as purely kinematical consequences of the quaternionic phase formalism. The approach is advantageous for applications – especially in modeling non-collinear acceleration of particle beams – where it replaces matrix diagonalizations with algebraic rotor compositions and improves numerical stability.
\end{abstract}

\paragraph{Keywords:} special relativity; phase; rapidity; Doppler shift; Lorentz factor; Wigner rotation; Thomas precession; phase parametrization.

\paragraph{MSC (2020):} 83A05; 70A05. % (Consider also PACS: 03.30.+p; 04.20.Cv.)

%1 ====================================================================
\subsection*{The tilt mechanism: changing frames as angle subtraction on the phase circle}

\paragraph{Setup (recall).}
In observer $A$’s split, let
\[
\widehat{\mathbf F}=\frac{\boldsymbol\chi}{\|\boldsymbol\chi\|}
=\cos\vartheta_{|A}\,\mathbf e_t^A+\sin\vartheta_{|A}\,\mathbf u_{|A},
\quad \mathbf u_{|A}\in S^A,\ \|\widehat{\mathbf F}\|=1.
\]
Here $\vartheta_{|A}$ is the \emph{observer–object angle}. It is \emph{not} your global phase angle $\Theta$
(except in the co–moving case when the time directions coincide).

\paragraph{Tilt operation (co–planar case).}
A \emph{frame tilt by angle} $\alpha$ in the 2–plane spanned by $\{\mathbf e_t^A,\mathbf u_{|A}\}$ replaces $A$’s basis by
\[
\mathbf e_t^{A'}=\cos\alpha\,\mathbf e_t^{A}+\sin\alpha\,\mathbf u_{|A},
\qquad
\mathbf u_{|A'}=\cos\alpha\,\mathbf u_{|A}-\sin\alpha\,\mathbf e_t^{A}.
\]
Decomposing $\widehat{\mathbf F}$ in the new basis gives
\[
\cos\vartheta_{|A'}=\widehat{\mathbf F}\!\cdot\!\mathbf e_t^{A'}=\cos(\vartheta_{|A}-\alpha),
\quad
\sin\vartheta_{|A'}=\widehat{\mathbf F}\!\cdot\!\mathbf u_{|A'}=\sin(\vartheta_{|A}-\alpha).
\]
\emph{Mechanism in one line:}
\[
\boxed{\ \vartheta_{|A'}=\vartheta_{|A}-\alpha\ } \qquad\text{(tilt = angle subtraction on the phase circle).}
\]

\paragraph{Minkowski interval from the tilt mechanism.}
Let $dt_{A'}$ be $A'$’s lab time, $d\mathbf x_{|A'}\in S^{A'}$ the measured spatial shift, and $d\tau$ the object’s proper time.
By definition in each frame,
\[
\frac{d\tau}{dt_{A'}}=\cos\vartheta_{|A'},\qquad
\frac{\|d\mathbf x_{|A'}\|}{c\,dt_{A'}}=\sin\vartheta_{|A'}.
\]
Using $\cos^2+\sin^2=1$,
\[
\boxed{\; c^2\,d\tau^2=c^2\,dt_{A'}^2-\|d\mathbf x_{|A'}\|^2 \;}
\]
holds automatically. Thus the \emph{tilt} changes what the observer reads (\(\vartheta\), $dt$, $d\mathbf x$), yet the
Minkowski form stays invariant because the object’s $d\tau$ is the same and the pair \((\cos\vartheta,\sin\vartheta)\) simply rotates.

\paragraph{Velocity and $\gamma$ update (dictionary).}
In any frame $X$,
\[
\beta_{|X}=\sin\vartheta_{|X},\qquad \gamma_{|X}=\sec\vartheta_{|X},\qquad \frac{d\tau}{dt_X}=\cos\vartheta_{|X}.
\]
Under a tilt by $\alpha$ (co–planar case),
\[
\beta_{|A'}=\sin(\vartheta_{|A}-\alpha),\quad
\gamma_{|A'}=\sec(\vartheta_{|A}-\alpha).
\]
Introduce half–angles $t=\tan(\vartheta_{|A}/2)$ and $s=\tan(\alpha/2)$.
Then
\[
\tan\frac{\vartheta_{|A'}}{2}=\frac{t-s}{1+ts}.
\]
Reading back to $\beta$ yields the Einstein collinear velocity addition in circular clothes:
\[
\boxed{\ \beta_{|A'}=\frac{\beta_{|A}-\beta_{\rm tilt}}{1-\beta_{|A}\,\beta_{\rm tilt}}\ },\quad
\beta_{\rm tilt}:=\sin\alpha.
\]
Equivalently, in “rapidity” $\eta$ (with $\tanh\eta=\beta$) the tilt is just \(\eta\mapsto \eta-\eta_{\rm tilt}\),
using the gudermann relation $\tan(\vartheta/2)=\tanh(\eta/2)$.

\paragraph{Two independent tilts (non–collinear) and the Wigner rotation.}
Tilting by $\alpha_1$ along $\mathbf u_1$ and then by $\alpha_2$ along $\mathbf u_2$ gives a net
\emph{tilt} plus a \emph{spatial rotation} about $\hat{\mathbf n}\parallel \mathbf u_2\times \mathbf u_1$
(the Wigner--Thomas rotation).
Let $t_i=\tan(\alpha_i/2)$ and let $\chi$ be the angle between $\mathbf u_1$ and $\mathbf u_2$.
The rotation angle satisfies the half–angle identity
\[
\boxed{\ \tan\frac{\psi_W}{2}=\frac{\sin\chi\;t_1 t_2}{1+\cos\chi\;t_1 t_2}\ },
\]
while the net tilt in the plane spanned by the resultant spatial direction obeys the same
half–angle addition rule as above. (All of this follows from rotating the phase pair
\(\{\mathbf e_t,\mathbf u\}\) and then re-projecting.)

\paragraph{Relation to the global phase angle $\Theta$.}
Your $\Theta$ is defined from the object’s \emph{self} split (its own “time” direction built from $\boldsymbol\chi_\perp$),
whereas $\vartheta_{|A}$ is the object’s angle \emph{in the observer’s split}.
They coincide only when the observer’s time axis aligns with the object’s self time axis.
The tilt mechanism above is exactly “how to move between splits.”

\paragraph{Optical readout (if the medium changes).}
If time is read by EM propagation in a medium with $n(\omega)=\sec\zeta(\omega)$,
use $dt_{\rm ph}=n\,dt$; then
\[
\frac{d\tau}{dt_{\rm ph}}=\cos\vartheta_{|X}\,\cos\zeta(\omega),
\]
so \emph{tilt} controls kinematics (\(\vartheta\)), while \emph{refractivity} rescales the clock (\(\zeta\)).
They are distinct mechanisms and compose multiplicatively.



\end{document}
