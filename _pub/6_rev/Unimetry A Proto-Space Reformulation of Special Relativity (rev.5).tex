\documentclass[11pt]{article}

\usepackage[a4paper,margin=1in]{geometry}
\usepackage{microtype}
\usepackage{comment}
\usepackage{amsmath,amssymb,amsfonts,amsthm}
\usepackage{hyperref}
\usepackage{float}
\usepackage{placeins}
\usepackage{booktabs} 
\usepackage{array,booktabs,tabularx}
\usepackage{tikz}
\usepackage{physics}
\usepackage{tensor}
\usepackage{bm}
\usepackage{geometry}
\geometry{margin=1in}

% --- tables, floats, barriers ---
\usepackage{booktabs}     % for \toprule \midrule \bottomrule
\usepackage{array,tabularx}
\newcolumntype{L}{>{\raggedright\arraybackslash}X}
\usepackage{placeins}     % for \FloatBarrier
\usetikzlibrary{arrows.meta, calc, angles, quotes, decorations.pathreplacing}

% --- TikZ libraries used in figures ---
\usetikzlibrary{calc,arrows.meta,positioning,decorations.pathreplacing}

% --- optional but useful cross-refs ---
\usepackage[nameinlink]{cleveref}

% --- notation helpers ---
\newcommand{\bell}{\boldsymbol\ell} % spatial projection vector
\newcommand{\clight}{c}             % speed of light macro
\newcommand{\Espace}{\mathcal{E}}
% \newcommand{\dd}{\mathrm{d}}
% \newcommand{\sech}{\operatorname{sech}}
\newcommand{\const}{\mathrm{const}}
\newcommand{\Vol}{\operatorname{Vol}}
\newcommand{\Hopf}{\mathcal H}
\DeclareMathOperator{\Ad}{Ad}
\newcommand{\Cl}{\mathcal{C}\ell}
\newcommand{\Spin}{Spin}


\usepackage{indentfirst}
\usetikzlibrary{arrows.meta,calc,angles,quotes}

% ===== PREAMBLE ADD-ON (put in the preamble) ==========================
% A lightweight, breakable tech-box environment
\usepackage{enumitem}
\usepackage[most]{tcolorbox} % loads needed TikZ libs internally
\tcbset{
  colback=gray!3, colframe=black!50, boxrule=0.35pt, arc=2mm,
  left=6pt,right=6pt,top=6pt,bottom=6pt, enhanced, breakable,
  fonttitle=\bfseries
}
\newtcolorbox{techbox}[2][]{title={#2},#1}
% ======================================================================


\theoremstyle{plain}
\newtheorem{theorem}{Theorem}[section]
\newtheorem{lemma}[theorem]{Lemma}
\newtheorem{proposition}[theorem]{Proposition}
\newtheorem{corollary}[theorem]{Corollary}

\theoremstyle{remark}
\newtheorem{remark}[theorem]{Remark}

\theoremstyle{definition}
\newtheorem{definition}[theorem]{Definition}


\numberwithin{equation}{section}

\title{Unimetry: Proto-Space Reformulation of Special Relativity}
\author{Timur Abizgeldin\\ \small Independent researcher, Austria\\ \small \texttt{timurabizgeldin@gmail.com}}
\date{\today}


% --- Unimetry macros ---
\usepackage{tikz}
\usetikzlibrary{arrows.meta,positioning,calc}
\providecommand{\bi}{\mathbf{i}}
\providecommand{\bj}{\mathbf{j}}
\providecommand{\bk}{\mathbf{k}}
\providecommand{\uhat}{\hat{\mathbf u}}
\providecommand{\rotor}[2]{\cos\frac{#2}{2} + #1\,\sin\frac{#2}{2}}

\begin{document}
\maketitle
\begin{abstract}
We present a coordinate--free reformulation of inertial special--relativistic kinematics in a four--dimensional Euclidean proto--space $(\Espace,\delta)$.  An observer is specified by a unit axis $N$, with time 1--form $\alpha=\delta(N,\cdot)$ and spatial projector $h=\delta-\alpha\otimes\alpha$.  The Lorentzian interval is then the induced quadratic form $g=2\,\alpha\otimes\alpha-\delta$, and proper time arises operationally via $c^2 d\tau^2=g(dX,dX)$ as a reparametrization along worldlines.  Introducing a $\delta$--calibrated proto--parameter $\chi$ yields a constant--norm proto--velocity and identifies the instantaneous kinematic state space with the calibrated sphere $S^3_{c}\subset T_p\Espace$, with the physically admissible sector given by $g>0$ and a null boundary at a finite Euclidean tilt.  We distinguish the circular tilt $\xi$ from the conic ratio coordinate $\vartheta$ and relate both to rapidity $\eta$, providing a compact Euclidean description of boosts.  Within $\Cl_{4,0}$, observer axes and kinematics admit rotor and quaternionic representations that streamline projection--based derivations of standard SR optical effects.
\end{abstract}



\paragraph{Keywords:} special relativity; phase; rapidity; Doppler shift; aberration; Lorentz factor; Wigner-Thomas rotation; phase parametrization.

\paragraph{MSC (2020):} 83A05; 70A05. % (Consider also PACS: 03.30.+p; 04.20.Cv.)

% \tableofcontents

% ===============================
\section{Introduction}
\label{sec:Intro}
% ===============================

\subsection{Philosophical foundations}

We refer to the guiding viewpoint behind the present reformulation as Unimetry.
At the level of primitives, the given concept treats the Euclidean proto--space $(\Espace,\delta)$ as the
primary kinematical arena (an ``empty'' geometric receptacle in the classical absolutist/relational
discussion), while time is introduced operationally rather than postulated as a fundamental
coordinate or substance: it is an observer-indexed parameter used to calibrate and compare the rates
of physical processes relative to a chosen unit axis $N$ (equivalently, its $1$--form
$\alpha=\delta(N,\cdot)$). In this sense, Unimetry sits in a broad historical and modern tradition in
which time is understood as derivative of change and/or of operational comparison: from Aristotle's
definition of time as ``a number of change with respect to the before and after'' \cite{AristotlePhysics,Coope2005TimeAristotle},
through Mach's insistence that temporal structure is abstracted from the systematic relations among
changes \cite{MachMechanics1893}, to contemporary relational and informational viewpoints where dynamics
is formulated in terms of correlations among partial observables (Rovelli), thermodynamical state
(thermal time; Connes--Rovelli), or quantum clocks (Page--Wootters) 
\cite{Rovelli2002PartialObs,ConnesRovelli1994ThermalTime,PageWootters1983,Barbour2009NatureOfTime}.
For background on the absolutist/relational vocabulary used above, see \cite{HoeferHuggettReadSEP2024}.

\paragraph{Etymology (``uni--metria'').}
Etymologically, the name Unimetry is intended literally as \emph{uni--metria}:
``one metry,'' i.e.\ a single principle of measurement/metrication.
The prefix \emph{uni-} (from \emph{unus}, one) emphasizes unity --- one underlying calibration scheme --- while
\emph{-metria} (from \emph{metron}, measure) is meant to evoke both (i) measurement in the operational
sense and (ii) metric in the geometric sense. In the present paper this double meaning is realized
concretely: the observer datum $N$ determines a consistent calibration and a $1{+}3$ split on $(\Espace,\delta)$,
and the usual Lorentzian interval structure is recovered as an induced (observer-dependent) bilinear form
built from $(\delta,N)$, rather than assumed as a background spacetime metric.


\subsection{Motivation}

Special relativity (SR) is usually presented as geometry in a Lorentzian
spacetime.  While conceptually economical, that viewpoint can hide two practical
aspects that matter operationally:

\begin{itemize}
\item[(i)] An observer does not measure an arbitrary curve parameter; physical
clocks implement a specific phase (proper--time) parametrization along
worldlines.

\item[(ii)] Many elementary relativistic effects (time dilation, the light cone,
Doppler shift, aberration) can be traced to a single observer splitting: how a
total displacement budget decomposes into a component along an observer time
axis and a component orthogonal to it.
\end{itemize}

The present paper develops a Euclidean proto--space formulation that isolates
this splitting as the primary geometric datum.  We work on a four--dimensional
Euclidean manifold $(\Espace,\delta)$ and introduce a "time" axis $N$
(a unit vector field, $\delta(N,N)=1$).    For an inertial observer, this axis is identified 
with the normalized $\delta$--calibrated proto--velocity direction of that observer, 
$N=\tilde X_{\mathrm{obs}}/\clight$ (Definition~\ref{def:chi-cal}).  The associated time 1--form
$\alpha:=N^\flat=\delta(N,\cdot)$ and spatial projector
$h:=\delta-\alpha\otimes\alpha$ encode the operational notions of ``time
reading'' and ``spatial distance'' for that observer.  The Lorentzian interval
then appears as the induced quadratic form
\[
   g=\alpha\otimes\alpha-h = 2\,\alpha\otimes\alpha-\delta,
\]
and proper time emerges as the operational reparametrization
$c^2 d\tau^2=g(dX,dX)$ along worldlines.

A second motivation is structural and computational.  The Euclidean proto--space
admits the strictly Euclidean Clifford algebra $\Cl(4,0)$, so that orthogonal
changes of observer gauge can be encoded by rotors acting by sandwiching.  After
fixing a fiducial unit vector $e_0$ (used to define a stable quaternionic
coordinate subalgebra), an observer axis $N$ may be represented as a rotor image
$N=q e_0 \widetilde q$, while relative states between an observer axis and an
object axis can be packaged as even elements built from the product of two unit
vectors.  In this language, calibrated proto--velocities form a compact state
space: at each point $p\in\Espace$, the kinematically admissible calibrated
proto--velocity states lie on the Euclidean sphere
$S^3_{\clight}\subset T_p\Espace$, and causal classification with respect to $g$
selects a physical sector of that sphere.  The null boundary is reached at a
finite Euclidean tilt to the time axis, corresponding to an equal split of the
proto--space budget between the longitudinal and transverse components.

\paragraph{Axiomatic shift (primitives and ontology).}
The present reformulation does not modify the empirical content of inertial SR.
It changes the choice of primitives and the direction of explanation.
Instead of taking a Lorentzian spacetime metric as fundamental, we start from a
Euclidean proto--space $(\Espace,\delta)$ (and its Euclidean geometric algebra
$\Cl(4,0)$) and treat ``time'' and ``space'' as observer--induced notions.
The minimal observer input is a choice of a chosen unit time axis $N$ (equivalently,
its time 1--form $\alpha=\delta(N,\cdot)$), which induces a $1+3$ split and
defines simultaneity as the $\delta$--orthogonal hyperplanes
$\alpha(\Delta X)=0$.  The Lorentzian bilinear form used to encode causal
classification and intervals is then constructed from this observer data by the
pointwise projector correspondence of Reddy--Sharma--Sivaramakrishnan,
\[
   g \;=\; 2\,\alpha\otimes\alpha \;-\; \delta,
\]
rather than postulated as a background structure.  In this sense, Minkowski
geometry enters as an operationally induced structure on top of the Euclidean
proto--space.

\paragraph{Status of SR postulates in the proto--space picture.}
With this choice of primitives, several standard SR statements move from the
axiomatic level to the level of derived consequences.  Proper time is not taken
as an a priori coordinate but is defined along worldlines by the constructed
interval, $c^2 d\tau^2=g(dX,dX)$.  A convenient unobservable proto--parameter
$\chi$ calibrates the total Euclidean budget through
$\delta(dX/d\chi,dX/d\chi)=c^2$, while time dilation and the usual $\gamma$--factor
emerge from the reparametrization $\chi\mapsto\tau$ (equivalently, from the
projection of the calibrated proto--velocity onto the observer axis).
The speed--of--light barrier corresponds to the null condition $g(dX,dX)=0$ and
is realized as a finite boundary in the Euclidean tilt geometry of the calibrated
sphere.  Inertial frame changes are implemented as changes of the observer axis,
encoded by rotors in $\Cl(4,0)$ after fixing the fiducial axis $e_0$.  Textbook
velocity composition, including transverse $1/\gamma$ suppression and
Wigner--Thomas rotation, arises once one consistently works with proper--time
normalized axes rather than with raw $\chi$--calibrated directions on $S^3_{\clight}$.

\subsection{Relation to previous work}

Attempts to express Lorentzian kinematics in Euclidean terms have a long history.
Early geometric constructions already appear in Karapetoff \cite{Karapetoff1912},
where relativistic transformations are visualized by Euclidean angle geometry.
More recent works study various embeddings and correspondences between Euclidean
and Lorentzian structures, including \cite{Brands2021,Akintsov2020}.

A distinct line of literature aims at an explicitly Euclidean reformulation of
special relativity.  Euclidean SR can be obtained by changes of variables in
which Lorentz transformations are represented as rotations in a Euclidean space
(e.g.\ Gersten \cite{Gersten2003ESR}).  Other proposals postulate an absolute
Euclidean background and reinterpret relativistic observables in terms of proper
time (e.g.\ Montanus \cite{Montanus1991AbsoluteE,Montanus2001ProperTime}), or
develop related ``four--dimensional optics'' frameworks in which $\tau$ plays a
central operational role (e.g.\ Almeida \cite{Almeida2001AltMinkowski,Almeida2002Kcalc}).
For broader context on Euclidean viewpoints beyond SR, see also Atkinson
\cite{Atkinson1963EuclidGR}.

Our construction differs in emphasis.  We do not identify the Euclidean norm
with the Lorentz interval by a coordinate trick, and we do not postulate an
absolute Euclidean time.  Instead, we start from an observer splitting encoded
by a chosen unit time axis $N$ and its time 1--form $\alpha$, and interpret the
Lorentzian interval as the operational quadratic form naturally associated with
this splitting.  Proper time arises as a reparametrization along worldlines,
rather than being built in as a primary coordinate.  This standpoint is aligned
with relational viewpoints in generally covariant physics, where ``time'' is
implemented by a chosen clock observable (cf.\ Rovelli's partial observables
perspective \cite{Rovelli2002,Rovelli1991Time}).

A rigorous pointwise correspondence between a Riemannian metric and a Lorentzian
one was established by Reddy, Sharma and Sivaramakrishnan \cite{Reddy2009}.
Given a Riemannian manifold $(M,h)$ and a unit vector field $U$, they define a
Lorentzian metric by $g=h-2\,U^\flat\!\otimes U^\flat$.  We adopt the sign--flipped
variant adapted to the particle--physics convention $(+---)$,
\[
   g \;=\; 2\,N^\flat\!\otimes N^\flat \;-\; \delta,
\]
and then make explicit how SR kinematics follows from phase reparametrization
and Euclidean tilt geometry relative to $\alpha$.

\paragraph{Inertial specialization (SR kinematics).}
Although the construction applies to any smooth unit field $N$ on $(\Espace,\delta)$,
the present paper focuses on inertial SR kinematics and therefore restricts to
the case where $N$ is parallel with respect to the flat Euclidean connection:
\begin{equation}\label{eq:inertial-N}
   \nabla^{\delta} N = 0 .
\end{equation}
Equivalently, in global Cartesian coordinates on $\Espace\simeq\mathbb{R}^4$, the
components $N^A$ are constant and the induced form
$g=2\,N^\flat\!\otimes N^\flat-\delta$ is constant (globally Minkowskian).
Relativistic effects in the present framework arise from the operational
reparametrization and tilt geometry, rather than from curvature.

\paragraph{Observer dependence and Lorentz invariance.}
The axis field $N$ is a geometric datum (no dynamics is postulated for it in this
paper), but it is not an absolute frame in the physical sense.  In the inertial
sector, choosing a constant $N$ is choosing an inertial observer.  Any other
constant unit axis $N'$ is related to $N$ by a Euclidean isometry $R\in O(4)$,
$N'=R_*N$, and the induced form transforms tensorially, $g'=R^*g$.

\emph{Convention.} We distinguish (i) Euclidean gauge rotations of the calibrated pair
$(\delta,N)$, implemented by $R\in O(4)$ and sending $N\mapsto N':=R_*N$ (hence $g\mapsto g':=R^*g$),
from (ii) Lorentz transformations $\Lambda\in O(g)$ acting on vectors once a particular induced metric $g$ is fixed.

Physical statements are formulated in $g$--covariant (relational) quantities, so
the theory does not privilege a particular representative $N$.  Note that the
quadratic form $g$ is invariant under $N\mapsto -N$, whereas the oriented time
1--form $\alpha=N^\flat$ changes sign; causal classification is insensitive to
this sign, while a choice of ``future'' requires a consistent orientation.

\paragraph{Geometric algebra viewpoint (role of $\Cl(4,0)$).}
Algebraically, our presentation is close in spirit to the geometric--algebra
approach to relativity, in which boosts and rotations are handled by rotors (see,
e.g., Hestenes' Space--Time Algebra \cite{Hestenes1966,Hestenes2003} and the
discussion of spacetime algebra versus ``imaginary time'' in \cite{GullLasenbyDoran1993}).
The difference is that we work throughout in the strictly Euclidean Clifford
algebra $\Cl(4,0)$: the observer split is encoded by the choice of $N$ (hence
$\alpha$) and is observer--dependent, while Lorentzian boost kinematics is
recovered by rotor transport of properly normalized axes and directions.

\subsection{Contributions}
\label{subsec:contributions}

The main contributions of the present paper are:
\begin{enumerate}[label=(C\arabic*),leftmargin=*,itemsep=2pt,topsep=2pt]
  \item Observer splitting and operational origin of the Lorentz interval.
  Starting from a Euclidean proto--space $(\Espace,\delta)$ and a unit axis $N$,
  we define the time 1--form $\alpha:=N^\flat=\delta(N,\cdot)$ and the spatial
  projector $h=\delta-\alpha\otimes\alpha$, and construct the induced bilinear form
  $g=\alpha\otimes\alpha-h=2\,\alpha\otimes\alpha-\delta$.
  Proper time is then defined operationally by $c^2 d\tau^2=g(dX,dX)$.

  \item Euclidean Clifford algebra implementation of observer axes.
  We formulate the construction in $\Cl(4,0)$, represent observer axes as rotor
  images $N=q e_0\widetilde q$, and make explicit the associated gauge
  nonuniqueness and the $N\mapsto -N$ sign issue at the level of time orientation.

  \item Reparameterization equivalence on the calibrated sphere.
  Proper--time normalization in the induced Lorentzian form is shown to be
  equivalent to a constant--speed Euclidean proto--velocity constrained to the
  calibrated sphere $S^3_{\clight}\subset T\Espace$.

  \item Geometric characterization of the physical sector.
  The timelike condition $g(X,X)>0$ selects an open subset of $S^3_{\clight}$
  (two polar caps), with the null boundary reached at the finite Euclidean tilt
  $\xi=\pi/4$.

  \item Circular versus conic/hyperbolic parametrizations of tilt.
  We distinguish the circular angle $\xi$ (normalized by $\|X\|_\delta$) from the
  conic ratio coordinate $\vartheta$ (normalized by $N\cdot X$), and relate both
  to the additive rapidity $\eta$ via $\tanh\eta=\tan\vartheta$.

  \item Projection--based optics and Euclidean GA derivations.
  Frequency and direction of null rays are obtained as $g$--projections, yielding
  compact derivations of Doppler shift and aberration; boosts are encoded within
  the Euclidean rotor framework tied to the $N$--split.
\end{enumerate}

\subsection{Outline}

The paper is organized as follows:
\begin{itemize}
  \item Section \ref{sec:LMC} introduces the Euclidean proto--space $(\Espace,\delta)$,
  formulates $\Cl(4,0)$ on $T\Espace$, and defines observer data ($e_0$ as a fiducial
  axis and $N$ as a time axis).  The induced Lorentzian form
  $g=2\,N^\flat\!\otimes N^\flat-\delta$ is constructed pointwise.

  \item Section \ref{sec:LMP} develops basic properties of the induced Lorentzian
  structure: the orthogonal decomposition relative to $N$, norm identities,
  causal classification, and the geometry of the null cone in the $\xi$--tilt picture.

  \item Section \ref{sec:PRT} formulates the operational clock viewpoint: we introduce
  a $\delta$--calibrated proto--parameter $\chi$, define coordinate time and proper
  time via $\alpha$, and make explicit the phase reparametrization $\chi\mapsto\tau$.

  \item Section \ref{sec:Hyper} develops the tilt geometry and distinguishes circular
  ($\xi$) and conic/hyperbolic ($\vartheta,\eta$) parametrizations, including the
  identity $\tanh\eta=\tan\vartheta$ and the $45^\circ$ null boundary.

  \item Section \ref{sec:velocity-invariants-S3} reformulates 4--velocity normalization
  as a constraint on calibrated proto--velocities, mapping admissible states to the
  sphere $S^3_{\clight}\subset T\Espace$ and identifying the physical sector.

  \item Section \ref{sec:why-protospace} illustrates the computational payoff:
  optical effects are derived as $g$--projections and boosts are reformulated using
  Euclidean geometric algebra.
\end{itemize}

% ===============================
\section{Lorentzian metric construction}
\label{sec:LMC}
% ===============================

\subsection{Euclidean proto-space $(\Espace,\delta)$}
\label{subsec:epspace}
We work on a four--dimensional Euclidean manifold $(\Espace,\delta)$ equipped
with the flat metric
\[
   \delta_{AB}=\mathrm{diag}(1,1,1,1).
\]
Indices are raised and lowered with $\delta$:
\[
   X_A := \delta_{AB}X^B,\qquad
   X^A := \delta^{AB}X_B,
\]
and we use the $\delta$--inner product notation
\[
   X\cdot Y := \delta(X,Y)=\delta_{AB}X^A Y^B.
\]

\begin{remark}[Index conventions: $\delta$ vs.\ $g$]
\label{rem:index-conventions}
Throughout, $\delta$ is treated as the background Euclidean metric on
$\Espace$ and is used for index gymnastics unless explicitly stated otherwise.
The Lorentzian tensor $g$ constructed in \S\ref{subsec:RSS} is regarded
as a derived bilinear form on $T\Espace$ used to define interval--type scalars
such as $g(X,X)$, rather than as the default device for raising/lowering.

In particular, we distinguish the $\delta$--raised components
\[
   g^{AB}_{(\delta)} := \delta^{AC}\delta^{BD} g_{CD}
\]
from the inverse metric $(g^{-1})^{AB}$ defined by
$(g^{-1})^{AC} g_{CB} = \delta^{A}{}_{B}$.
For the special form $g_{AB}=2N_A N_B-\delta_{AB}$ with $\delta(N,N)=1$, one
indeed has $(g^{-1})^{AB}=g^{AB}_{(\delta)}=2N^{A}N^{B}-\delta^{AB}$, but the
two notions remain conceptually distinct.
\end{remark}

% ------------------------------------------------------------
\subsection{Clifford algebra $\Cl(4,0)$ on $\Espace$}
\label{subsec:cl40}
At each point $p\in\Espace$ we equip the tangent space $T_p\Espace\simeq\mathbb R^4$
with its geometric (Clifford) algebra $\Cl(4,0)$ generated by vectors and the
Euclidean metric $\delta$.  Concretely, choose an oriented $\delta$--orthonormal
basis $\{e_0,e_1,e_2,e_3\}$ of $T_p\Espace$ with
\[
   e_a^2=+1,\qquad e_a e_b=-e_b e_a\quad(a\neq b).
\]
The geometric product of vectors $a,b\in T_p\Espace$ decomposes as
\[
   ab \;=\; a\cdot b \;+\; a\wedge b,
\]
where $a\cdot b=\delta(a,b)$ is the scalar (inner) product and $a\wedge b$ is the
bivector (outer) product.  We use standard grade projections
$\langle\,\cdot\,\rangle_k$ ($k=0,\dots,4$) and the reversion (reverse)
$\widetilde{(\cdot)}$, defined on basis blades by reversing the order of basis
vectors (hence $\widetilde{ab}= \widetilde b\,\widetilde a$ and
$\widetilde{a}=a$ for vectors).

\paragraph{Rotors.}
A rotor is an even multivector $r\in\Cl^+(4,0)$ satisfying
\[
   r\widetilde r = 1.
\]
Rotors act on vectors by the sandwich map $v\mapsto r v \widetilde r$ and
realize orthogonal transformations ($\Spin(4)\to SO(4)$ is the usual double cover).

\paragraph{Quaternionic coordinates (observer gauge).}
Fixing a unit vector $e_0$ selects the even subspace
\[
   \mathbb H_{e_0}:=\mathrm{span}\{1,\ e_{10},\ e_{20},\ e_{30}\}\subset\Cl^+(4,0),
   \qquad e_{i0}:=e_i e_0,
\]
which is isomorphic to the quaternion algebra once the orientation is fixed.
This provides a convenient coordinate language for even elements (notably the
unit 3--sphere $S^3$ of unit quaternions), but it does not introduce additional
physics by itself.

% ------------------------------------------------------------
\subsection{Observer axis, inertial SR sector, and the role of $\nabla^\delta$}
\label{subsec:observer-axis}
To speak about ``time'' and ``space'' within the Euclidean proto--space, one must
choose a reference unit direction.  In the special--relativistic (inertial) sector
studied in this paper we encode the observer by a constant unit vector field
$e_0$ on the region of interest:
\[
   \delta(e_0,e_0)=1,\qquad \nabla^\delta e_0=0.
\]
Here $\nabla^\delta$ denotes the Levi--Civita connection of $\delta$; in Cartesian
coordinates on $\Espace\simeq\mathbb R^4$ one has $\Gamma(\delta)=0$ and
$\nabla^\delta$ reduces to the ordinary derivative, so $\nabla^\delta e_0=0$
simply means that the components of $e_0$ are constant.

\begin{remark}[Physical axis $N$ versus fiducial axis $e_0$]
\label{rem:N-vs-e0}
The Lorentzian form constructed below depends on a unit direction $N$ (the observer's
time axis).  For a single inertial observer one may, and often will,
identify $N\equiv e_0$ without loss of generality.
We nevertheless keep the notation $e_0$ available as a fixed fiducial axis when
working in $\Cl(4,0)$: it provides a stable quaternionic coordinate subalgebra
$\mathbb H_{e_0}$ even in later extensions where a physical field $N$ may vary.
In the present SR sector, no observable depends on this bookkeeping choice.
\end{remark}

% ------------------------------------------------------------
\subsection{Two axes give a relative quaternion and a rotor}
\label{subsec:relative-quat-rotor}
Let $u,v\in T_p\Espace$ be two unit vectors, $\delta(u,u)=\delta(v,v)=1$.
Think of $u$ as the observer time axis (typically $u=N_p$, and in the inertial
SR sector $u=e_0$), and $v$ as the unit proto--velocity direction of an object.
(Operationally, later we will identify $v=\tilde X/\clight$ for a calibrated
proto--velocity $\tilde X=dX/d\chi$.)

\paragraph{Relative quaternion from two vectors.}
Define the associated even element
\begin{equation}\label{eq:rel-quat-def}
   q(v,u)\;:=\; v u \;\in\;\Cl^+(4,0).
\end{equation}
It is automatically unit:
\[
   q(v,u)\,\widetilde{q(v,u)}=(vu)(uv)=v(u^2)v=v^2=1.
\]
Moreover,
\[
   q(v,u)=\langle q\rangle_0+\langle q\rangle_2=(v\!\cdot\!u)\;+\;(v\!\wedge\!u).
\]
Let $\xi\in[0,\pi]$ be the Euclidean angle between $u$ and $v$, defined by
\begin{equation}\label{eq:xi-def-uv}
   \cos\xi:=u\!\cdot\!v.
\end{equation}
Then $\|v\wedge u\|_\delta=\sin\xi$ and, for $\xi\notin\{0,\pi\}$,
\begin{equation}\label{eq:q-circular-xi}
   q(v,u)=\cos\xi+\sin\xi\,\widehat B,
   \qquad
   \widehat B:=\frac{v\wedge u}{\|v\wedge u\|_\delta},
   \qquad
   \widehat B^2=-1.
\end{equation}
If $u=e_0$, then $q(v,e_0)=v e_0\in \mathbb H_{e_0}$ and may be treated literally
as a unit quaternion in the basis $\{1,e_{10},e_{20},e_{30}\}$.

\paragraph{A canonical rotor mapping $u$ to $v$.}
While $q(v,u)=vu$ compactly encodes the relative state, the rotor that acts
by conjugation and maps $u$ to $v$ is the normalized ``half--angle'' element
\begin{equation}\label{eq:min-rotor-uv}
   r(v,u)\;:=\;\frac{1+v u}{\sqrt{2(1+u\!\cdot\!v)}}
   \;=\;\cos\!\Bigl(\frac{\xi}{2}\Bigr)+\sin\!\Bigl(\frac{\xi}{2}\Bigr)\,\widehat B,
   \qquad (\xi\neq\pi),
\end{equation}
which satisfies
\begin{equation}\label{eq:r-maps-u-to-v}
   r(v,u)\,u\,\widetilde{r(v,u)} \;=\; v.
\end{equation}

\begin{remark}[Gauge nonuniqueness (stabilizer)]
\label{rem:stabilizer-uv}
The rotor implementing $u\mapsto v$ is not unique: if $s$ is any rotor fixing $u$
(i.e.\ $s u\widetilde s=u$), then $r' = r s$ yields the same image
$r' u \widetilde{r'}=v$.  Equivalently, $r$ is defined up to right multiplication
by the stabilizer $\mathrm{Stab}(u)\cong\Spin(3)\cong SU(2)$.
For example, when $u=e_0$, $\mathrm{Stab}(e_0)$ consists of the purely spatial
rotations in $e_0^{\perp_\delta}$; composing $r$ with such a spatial rotor changes
the spatial gauge but does not change the image of the time axis.
\end{remark}

% ------------------------------------------------------------
\subsection{Time axis $N$ and the induced Lorentzian form}
\label{subsec:RSS}
Let $N$ be a smooth unit vector field on an open set $U\subset\Espace$:
\begin{equation}\label{eq:N-unit}
   \delta(N,N)=1 \qquad\text{on }U.
\end{equation}
In the inertial SR sector we assume
\begin{equation}\label{eq:N-parallel}
   \nabla^\delta N = 0 \qquad\text{on }U,
\end{equation}
so that $N$ is constant (in Cartesian coordinates).  For a single inertial observer
we may set $N\equiv e_0$.

\begin{remark}[Observer axis as normalized proto--velocity]
\label{rem:N-as-protovel}
The choice of $N$ is not an external gauge datum: it encodes the state of motion of
an inertial observer in proto--space.  If $X_{\mathrm{obs}}:I\to U$ is the observer
worldline parametrized by the $\delta$--calibrated proto--parameter $\chi$
(Definition~\ref{def:chi-cal}), then the calibrated proto--velocity
$\tilde X_{\mathrm{obs}}:=dX_{\mathrm{obs}}/d\chi$ satisfies
$\delta(\tilde X_{\mathrm{obs}},\tilde X_{\mathrm{obs}})=\clight^2$, and we set
\[
   N := \frac{1}{\clight}\,\tilde X_{\mathrm{obs}}.
\]
In the inertial SR sector, $\nabla^\delta N=0$ expresses that this proto--velocity
direction is constant; different inertial observers correspond to $SO(4)$ rotations
transporting $N$ (implemented by rotors in $\Spin(4)$).
\end{remark}

\paragraph{Time 1--form and spatial projector.}
Let
\[
   \alpha := N^\flat := \delta(N,\cdot)
   \qquad (\text{so } \alpha_A=\delta_{AB}N^B=N_A),
\]
and define the spatial projector
\begin{equation}\label{eq:h-def}
   h := \delta - \alpha\otimes\alpha
   \qquad\text{(i.e.\ } h_{AB}=\delta_{AB}-N_A N_B\text{)}.
\end{equation}
Then $h$ has rank $3$, satisfies $h(\cdot,N)=0$, and projects onto $N^{\perp_\delta}$.

\paragraph{Reddy--Sharma--Sivaramakrishnan Lorentzization.}
Define a symmetric $(0,2)$--tensor field $g$ on $U$ by
\begin{equation}\label{eq:g-def}
   g \;:=\; 2\,\alpha\otimes\alpha \;-\;\delta,
   \qquad\text{i.e.}\qquad
   g_{AB}=2N_A N_B-\delta_{AB}.
\end{equation}
In an adapted $\delta$--orthonormal basis $\{N,e_1,e_2,e_3\}$ this has signature
$(+---)$.  The definition is insensitive to the sign flip $N\mapsto -N$ because
$N\otimes N$ is unchanged.

\begin{remark}[Relative quaternion viewpoint (observer--object)]
\label{rem:VN-quaternion}
Given an object proto--velocity direction $v$ (unit) and an observer axis $u=N_p$,
the even element $q(v,u)=vu$ encodes the circular angle $\xi$ via
$\langle q\rangle_0=\cos\xi$ and $\|\langle q\rangle_2\|_\delta=\sin\xi$.
This is the basic algebraic device used later to package the longitudinal/transverse
proto--velocity budget split relative to $N$ into a single unit quaternion--like
element.
\end{remark}

\begin{remark}[Coherence as axis alignment]
\label{rem:coherence-axis-alignment}
In the present kinematical (SR) sector, it is convenient to name the special
case where two inertial systems share the same time axis.
We call two observers (or an observer and an object) coherent on an overlap
region if their time axes coincide there, $N_1=N_2$ (with the same orientation).
Equivalently, their relative angle is $\xi=0$ and the corresponding even element
$q(N_2,N_1)=N_2N_1$ reduces to the scalar $+1$.
In this coherent case, the kinematic frequency shift and aberration effects
associated with a nontrivial tilt are absent.
\end{remark}

% ===============================
\section{Lorentzian metric properties}
\label{sec:LMP}
% ===============================

Throughout this section, $p\in\Espace$ is arbitrary and all statements are
understood pointwise at $p$.  We write $T_p\Espace$ for the tangent space,
endowed with the Euclidean inner product $\delta$ and the associated Clifford
algebra $\Cl(4,0)$ from \S\ref{subsec:cl40}.  The time axis at $p$ is a
$\delta$--unit vector $N\in T_p\Espace$, $\delta(N,N)=1$, with associated
Lorentzian bilinear form
\[
   g(X,Y)=2\,(N\!\cdot\!X)\,(N\!\cdot\!Y)-\delta(X,Y)
   \qquad\text{(cf.\ \eqref{eq:g-def}).}
\]
When convenient, we also fix a fiducial axis $e_0$ and a rotor $q\in\Cl^+(4,0)$
such that
\[
   N=q\,e_0\,\widetilde q,
\]
in which case $q$ may be regarded as a time--axis rotor (an ``observer gauge'')
relative to the bookkeeping axis $e_0$.

% ------------------------------------------------------------
\subsection{Orthogonal decomposition of tangent vectors}
\label{subsec:orth-decomp}

For any $X\in T_p\Espace$ we define the $\delta$--longitudinal and
$\delta$--transverse components relative to $N$ by
\begin{equation}\label{eq:decomp}
   X_\parallel := (N\cdot X)\,N,\qquad
   X_\perp := h(X)=X-(N\cdot X)\,N,
\end{equation}
where $h=\delta-N^\flat\!\otimes N^\flat$ is the $\delta$--orthogonal projector
onto $N^{\perp_\delta}$ (cf.\ \eqref{eq:h-def}).

\begin{lemma}\label{lem:decomp-unique}
For every $X\in T_p\Espace$,
\[
   X=X_\parallel+X_\perp,
\]
where $X_\parallel\in\mathrm{span}\{N\}$ and $X_\perp\in N^{\perp_\delta}$.
The decomposition is unique.
\end{lemma}

\begin{proof}
Since $h$ is a projector with $\ker(h)=\mathrm{span}\{N\}$ and
$\mathrm{Im}(h)=N^{\perp_\delta}$, the splitting is the standard direct sum
decomposition associated with complementary subspaces.
\end{proof}

\begin{remark}[Clifford reflection viewpoint]
\label{rem:clifford-reflection}
In $\Cl(4,0)$, the sandwich map by a unit vector implements an orthogonal
reflection.  Define, for $Y\in T_p\Espace$,
\[
   Y^\star := -\,N\,Y\,N .
\]
Then $Y^\star$ is the $\delta$--reflection of $Y$ across the hyperplane
$N^{\perp_\delta}$ and satisfies
\[
   Y^\star = Y_\parallel - Y_\perp.
\]
Consequently, the Lorentzian bilinear form can be written as a Euclidean pairing
with a reflected argument,
\begin{equation}\label{eq:g-as-reflection}
   g(X,Y) \;=\; \delta\bigl(X,\,Y^\star\bigr)
   \;=\; \delta\bigl(X,\,Y_\parallel-Y_\perp\bigr),
\end{equation}
which is the Clifford--algebraic content of the RSS ``Lorentzization''.
\end{remark}

% ------------------------------------------------------------
\subsection{Circular angle $\xi$ and norm identities}
\label{subsec:xi-norms}

The decomposition \eqref{eq:decomp} naturally defines a circular Euclidean
angle between $X$ and the time axis $N$.

\begin{definition}[Circular angle $\xi$ relative to $N$]
\label{def:xi}
Let $X\in T_p\Espace$ be nonzero.  Define $\xi\in[0,\pi]$ by
\begin{equation}\label{eq:xi-def}
   \cos\xi := \frac{N\cdot X}{\|X\|_\delta},
   \qquad
   \sin\xi := \frac{\|X_\perp\|_\delta}{\|X\|_\delta},
\end{equation}
where $\|X\|_\delta:=\sqrt{\delta(X,X)}$.
Equivalently, $\xi$ is the Euclidean angle between $X$ and $N$.
\end{definition}

\begin{remark}[Why $\xi$ is distinguished from $\vartheta$]
\label{rem:xi-vs-vartheta}
The angle $\xi$ is the circular (trigonometric) Euclidean angle defined by
\eqref{eq:xi-def}.  Later we introduce a tangential parameter $\vartheta$
adapted to velocity ratios (e.g.\ $\beta=\tan\vartheta$ and rapidity).  Keeping
$\xi$ for the circular parametrization avoids conflating these two choices.
\end{remark}

\begin{proposition}\label{prop:g-norm-split}
For any $X\in T_p\Espace$,
\begin{equation}\label{eq:g-norm-split}
   g(X,X)=(N\cdot X)^2-\delta(X_\perp,X_\perp).
\end{equation}
Equivalently, in terms of the circular angle $\xi$,
\begin{equation}\label{eq:g-cos2xi}
   g(X,X)=\|X\|_\delta^2\,\cos(2\xi).
\end{equation}
\end{proposition}

\begin{proof}
Insert \eqref{eq:decomp} into $g(X,X)$ and use:
$g(N,N)=1$, $g(N,X_\perp)=0$ (since $X_\perp\in N^{\perp_\delta}$), and
$g(X_\perp,X_\perp)=-\delta(X_\perp,X_\perp)$.
This yields \eqref{eq:g-norm-split}.  Using
$(N\cdot X)=\|X\|_\delta\cos\xi$ and $\|X_\perp\|_\delta=\|X\|_\delta\sin\xi$
from \eqref{eq:xi-def} gives \eqref{eq:g-cos2xi}.
\end{proof}

\begin{corollary}\label{cor:causal-class}
A vector $X\neq 0$ satisfies:
\begin{itemize}
  \item $g(X,X)>0$ iff $\cos(2\xi)>0$ iff $\xi\in[0,\tfrac{\pi}{4})\cup(\tfrac{3\pi}{4},\pi]$,
  \item $g(X,X)=0$ iff $\cos(2\xi)=0$ iff $\xi\in\{\tfrac{\pi}{4},\tfrac{3\pi}{4}\}$,
  \item $g(X,X)<0$ iff $\cos(2\xi)<0$ iff $\xi\in(\tfrac{\pi}{4},\tfrac{3\pi}{4})$.
\end{itemize}
Equivalently, in split form:
\[
  g(X,X)\gtrless 0
  \quad\Longleftrightarrow\quad
  (N\cdot X)^2 \gtrless \delta(X_\perp,X_\perp).
\]
\end{corollary}

Define the three disjoint subsets of $T_p\Espace$:
\[
  \mathcal T_p := \{X\in T_p\Espace:\ g(X,X)>0\},\qquad
  \mathcal P_p := \{X\in T_p\Espace:\ g(X,X)=0\},\qquad
  \mathcal S_p := \{X\in T_p\Espace:\ g(X,X)<0\}.
\]
We also single out the future time cone (relative to $N$):
\begin{equation}\label{eq:Tp-plus}
  \mathcal T_p^{+} := \{X\in \mathcal T_p:\ N\cdot X>0\}.
\end{equation}

% ------------------------------------------------------------
\subsection{Geometry of the null cone and the $45^\circ$ Euclidean tilt}
\label{subsec:nullcone}

\begin{proposition}\label{prop:nullcone}
The set of $g$--null vectors at $p$ is the quadratic cone
\[
   \mathcal{C}_p
   = \{\,X\in T_p\Espace:\ \delta(X_\perp,X_\perp)=(N\cdot X)^2\,\}.
\]
Under the decomposition $T_p\Espace=\mathrm{span}\{N\}\oplus N^{\perp_\delta}$,
it is a double cone given by
\[
   N\cdot X = \pm \|X_\perp\|_\delta.
\]
\end{proposition}

\begin{proof}
Immediate from Corollary~\ref{cor:causal-class} (or directly from
\eqref{eq:g-norm-split}).
\end{proof}

\begin{remark}[Why ``light'' corresponds to a $45^\circ$ Euclidean tilt]
\label{rem:why-45deg}
Operationally, ``light propagation'' refers to a causal relation between an
emission event and a detection event.  During the signal's flight the observer
advances along the time axis $N$, hence the proto--space separation between
emission and detection is not purely spatial but decomposes as
\[
   X \;=\; X_{\parallel} + X_{\perp},
   \qquad
   X_{\parallel} := (N\!\cdot\!X)\,N,
   \qquad
   X_{\perp} := X-(N\!\cdot\!X)\,N .
\]
The defining kinematic content of ``light'' is that the induced Lorentzian
interval vanishes,
\[
   g(X,X)=0.
\]
Using \eqref{eq:g-norm-split}, this is equivalent to equality of the longitudinal
and transverse $\delta$--magnitudes,
\begin{equation}\label{eq:null-45deg}
   \|X_{\parallel}\|_{\delta} \;=\; \|X_{\perp}\|_{\delta}.
\end{equation}
In terms of the circular Euclidean angle $\xi$ between $X$ and $N$
(Definition~\ref{def:xi}), the condition \eqref{eq:null-45deg} reads
$\cos\xi=\sin\xi$, hence
\[
   \xi=\frac{\pi}{4}\quad\text{or}\quad \xi=\frac{3\pi}{4}.
\]
Thus the $g$--null directions are precisely those at a $45^\circ$ Euclidean tilt
to the time axis: the proto--space displacement budget is split equally between
$N$ and its orthogonal complement.
\end{remark}

% ------------------------------------------------------------
\subsection{Proto--velocity viewpoint and the observer split}
\label{subsec:protovelocity-viewpoint}

While the results above hold for arbitrary tangent vectors, the proto--space
formulation is most transparent when applied to a calibrated proto--velocity.
Let $X(\chi)$ be a regular curve and define its proto--velocity by
\[
   \tilde X := \frac{dX}{d\chi}\in T_p\Espace.
\]
If $\chi$ is calibrated (cf.\ Definition~\ref{def:chi-cal}), then
\begin{equation}\label{eq:protovelocity-cal}
   \delta(\tilde X,\tilde X)=\clight^2.
\end{equation}
Thus, at each point, the set of admissible calibrated proto--velocities is the
Euclidean 3--sphere $S^3_{\clight}(p)\subset T_p\Espace$.

Relative to the time axis $N$, write the observer split
\[
   \tilde X = \tilde H\,N + \tilde X_\perp,
   \qquad
   \tilde H := N\!\cdot\!\tilde X,
   \qquad
   \tilde X_\perp := h(\tilde X)\in N^{\perp_\delta},
\]
and set $\tilde L:=\|\tilde X_\perp\|_\delta$.
Then \eqref{eq:protovelocity-cal} implies the proto--velocity budget identity
\[
   \tilde H^{\,2}+\tilde L^{\,2}=\clight^2,
\]
while the induced Lorentzian rate is
\begin{equation}\label{eq:g-on-protovelocity}
   g(\tilde X,\tilde X)=\tilde H^{\,2}-\tilde L^{\,2}
   \;=\;\clight^2\cos(2\xi),
\end{equation}
where $\xi$ is the circular tilt angle of $\tilde X$ relative to $N$.

% ------------------------------------------------------------
\subsection{Spatial rotations preserving $\delta$ and $N$}
\label{subsec:spatial-rotations}

Let $\mathrm{Aut}(\delta,N)$ denote the stabilizer of $N$ in the Euclidean
orthogonal group:
\[
   \mathrm{Aut}(\delta,N)
   := \{\,L:T_p\Espace\to T_p\Espace\text{ linear}:\ \delta(LX,LY)=\delta(X,Y),\ LN=N\,\}.
\]
In an adapted $\delta$--orthonormal basis $\{e'_0=N,e'_1,e'_2,e'_3\}$ one has
\[
   L=\mathrm{diag}(1,R),\qquad R\in O(3),
\]
so $\mathrm{Aut}(\delta,N)\cong O(3)$ and contains no boost--like maps mixing
$N$ with $N^{\perp_\delta}$.

\begin{remark}[Rotor description of $\mathrm{Aut}(\delta,N)$]
\label{rem:aut-rotors}
In the Clifford algebra, the connected component of $\mathrm{Aut}(\delta,N)$ is
realized by spatial rotors $s\in\Cl^+(4,0)$ satisfying
\[
   s\widetilde s=1,\qquad s N \widetilde s = N.
\]
Equivalently, $s=\exp(-\tfrac{\phi}{2}B)$ with a bivector generator
$B\in\Lambda^2(N^{\perp_\delta})=\mathrm{span}\{e'_{12},e'_{23},e'_{31}\}$.
This is the same stabilizer mechanism as in Remark~\ref{rem:stabilizer-uv}:
the time axis is held fixed while the spatial triad is rotated.  Such stabilizer
freedom is precisely the source of rotor nonuniqueness when one encodes only the
axis $N$ (and not a full tetrad).
\end{remark}

\begin{lemma}\label{lem:aut-preserves-g}
Every $L\in\mathrm{Aut}(\delta,N)$ preserves $g$:
\[
   g(LX,LY)=g(X,Y)\qquad \text{for all } X,Y\in T_p\Espace.
\]
\end{lemma}

\begin{proof}
Since $LN=N$ and $L$ is $\delta$--orthogonal,
\[
   g(LX,LY)
   =2(N\cdot LX)(N\cdot LY)-\delta(LX,LY)
   =2(N\cdot X)(N\cdot Y)-\delta(X,Y)
   =g(X,Y).
\]
\end{proof}

Thus $\mathrm{Aut}(\delta,N)$ is a spatial subgroup of $\mathrm{O}(g)$: it
preserves $g$ and fixes $N$, but generates only Euclidean rotations on
$N^{\perp_\delta}$.

\begin{remark}[Time--axis rotor gauge, revisited]
\label{rem:timeaxis-rotor-gauge-LMP}
If $N=q e_0\widetilde q$ for a rotor $q$, then $q$ is defined only up to right
multiplication by a stabilizer rotor $s$ that fixes $e_0$
(cf.\ Remark~\ref{rem:stabilizer-uv}), because
$(qs)e_0\widetilde{(qs)}=q e_0\widetilde q$.
Pointwise, this means that specifying the axis $N$ does not specify a unique
orientation of the spatial complement: the latter may be rotated by an element
of $\Spin(3)$ without affecting $N$, hence without affecting $g_N$.
\end{remark}
% ===============================
\section{Phase reparametrization and the operational origin of the Lorentz interval}
\label{sec:PRT}
% ===============================

This section makes explicit the logical bridge between the Euclidean proto--metric
$\delta$ on $\Espace$ and the Lorentzian form $g$ constructed in \S\ref{sec:LMC}.
The construction starts from an oriented observer time axis $N$
(and, if desired, a rotor gauge $q$ such that $N=q e_0 \widetilde q$), and then
defines operational time and space measurements as $\delta$--projections relative
to $N$.  The key point is operational: an observer does not have direct access
to an arbitrary curve parameter.  Instead, clocks and rulers implement specific
projection rules; the Lorentzian interval is the quadratic form that governs the
proper--time phase induced by those rules.

% ------------------------------------------------------------
\subsection{A $\delta$--calibrated proto--parameter $\chi$ and proto--velocity}
\label{subsec:chi-affine}

Let $X:I\to\Espace$ be a $C^1$ curve.  For a parameter $\lambda$ on $I$ we write
\[
   X'(\lambda):=\frac{dX}{d\lambda}\in T_{X(\lambda)}\Espace.
\]
Because $(\Espace,\delta)$ is Euclidean, one may always reparametrize $X$ by a
scaled $\delta$--arc length.  We single out the following calibration, which fixes
the total proto--space budget along the curve.

\begin{definition}[$\delta$--calibrated (proto--affine) parameter]
\label{def:chi-cal}
A parameter $\chi$ along $X$ is called $\delta$--calibrated (or
proto--affine) if the associated proto--velocity
\[
   \tilde X:=\frac{dX}{d\chi}
\]
has constant $\delta$--norm equal to $\clight$, i.e.
\begin{equation}\label{eq:chi-cal}
   \delta(\tilde X,\tilde X)=\clight^2.
\end{equation}
\end{definition}

\begin{remark}[Interpretation of $\chi$]
\label{rem:chi-interpretation}
In flat Euclidean geometry, \eqref{eq:chi-cal} is simply a scaled arc--length
parametrization:
\[
   d\chi = \frac{1}{\clight}\,\|dX\|_\delta.
\]
The parameter $\chi$ is an auxiliary calibration: it fixes a convenient
reference parameter in $\Espace$ relative to which the total Euclidean expenditure
$\|\tilde X\|_\delta=\clight$ is constant.  By itself, $\chi$ is not an operational
time variable; operational time is introduced via the time axis $N$ below.
\end{remark}

% ------------------------------------------------------------
\subsection{A time axis $N$: time 1--form and spatial projector}
\label{subsec:alpha-timeform}

Throughout, $N$ denotes the $\delta$--unit time-axis field of \S\ref{sec:LMC},
\[
   \delta(N,N)=1.
\]
When working in $\Cl(4,0)$ one may (optionally) encode $N$ by a rotor $q$ relative
to a fixed fiducial axis $e_0$ via
\[
   N=q\,e_0\,\widetilde q,
\]
but the operational content depends only on $N$ (and its chosen orientation), not
on a particular representative $q$.

The $\delta$--dual 1--form (the time form) of $N$ is
\begin{equation}\label{eq:alpha-def}
   \alpha := N^\flat := \delta(N,\cdot),
   \qquad \text{i.e.}\qquad
   \alpha_A = \delta_{AB}N^B = N_A.
\end{equation}
The induced spatial projector is
\begin{equation}\label{eq:h-def-PRT}
   h := \delta - \alpha\otimes\alpha,
   \qquad\text{i.e.}\qquad
   h_{AB}=\delta_{AB}-N_A N_B,
\end{equation}
so that $h(\,\cdot\,,N)=0$ and $\mathrm{Im}(h)=N^{\perp_\delta}$.

For any $V\in T_p\Espace$ one has the orthogonal $\delta$--split
\begin{equation}\label{eq:delta-split}
   \delta(V,V) = \alpha(V)^2 + h(V,V),
\end{equation}
i.e.\ ``total proto--budget = longitudinal$^2$ + transverse$^2$'' relative to $N$.

\begin{remark}[Sign of $N$ and what is (in)sensitive to it]
\label{rem:sign-of-N}
The RSS Lorentzization $g=2\,N^\flat\!\otimes N^\flat-\delta$ depends on $N$ only
through $N\otimes N$ and is therefore invariant under the sign flip $N\mapsto -N$.
Hence the light cone and causal classification determined by $g$ are insensitive to
the orientation of $N$.

However, the time 1--form $\alpha=N^\flat$ changes sign under $N\mapsto -N$, and so do
the projected time increments defined below.  Thus, to speak about an oriented
``future'' direction and about a monotone time coordinate, one must choose an
orientation of $N$ (a future--pointing branch) and keep it fixed on the region
under consideration.
\end{remark}

\begin{remark}[Rotor gauge does not affect $\alpha$, $h$, or $g$]
If $N=q e_0\widetilde q$, then replacing $q$ by $q s$ with any stabilizer rotor $s$
satisfying $s e_0\widetilde s=e_0$ leaves $N$ unchanged, hence also leaves $\alpha$,
$h$, and the induced $g$ unchanged.  Operational quantities depend on $N$ (and its
chosen orientation), not on a particular representative $q$.
\end{remark}

% ------------------------------------------------------------
\subsection{Coordinate time $t$ and spatial distance $\ell$ as projections}
\label{subsec:t-vs-tau}

Given an oriented time axis $N$, the time form $\alpha$ defines the observer--adapted
coordinate time $t$ along a worldline by the operational projection rule
\begin{equation}\label{eq:dt-def}
   dt := \frac{1}{\clight}\,\alpha(dX)
   \qquad\Longleftrightarrow\qquad
   \frac{dt}{d\lambda} = \frac{1}{\clight}\,\alpha\bigl(X'(\lambda)\bigr).
\end{equation}
\begin{remark}[Comoving inertial gauge]
If $X$ is the worldline of the observer that defines the axis $N$
(Remark~\ref{rem:N-as-protovel}), so that $dX=\clight N\,d\chi$, then
\eqref{eq:dt-def} yields $dt=d\chi$ along $X$.  In addition,
$g(\clight N,\clight N)=\clight^2$ implies $d\tau=d\chi$ along the same worldline.
Thus, in the comoving inertial gauge the three parameters $t$, $\chi$, and $\tau$
agree up to additive constants.
\end{remark}

In general, \eqref{eq:dt-def} defines $t$ only along the given curve; a global time
function $t$ on an open set requires an integrability condition on $\alpha$
(e.g.\ $\alpha$ closed/exact, equivalently $N$ hypersurface--orthogonal).

Likewise, the induced spatial line element along the curve is defined by
\begin{equation}\label{eq:dl-def}
   d\ell^2 := h(dX,dX)
   \qquad\Longleftrightarrow\qquad
   \Bigl(\frac{d\ell}{d\lambda}\Bigr)^2 = h\bigl(X'(\lambda),X'(\lambda)\bigr).
\end{equation}

\begin{proposition}[Operational form of the Lorentz interval]
\label{prop:operational-interval}
For every curve $X$ and every parameter $\lambda$ one has the identity
\begin{equation}\label{eq:ds-operational}
   g(dX,dX)=\clight^2\,dt^2 - d\ell^2,
\end{equation}
where $dt$ and $d\ell$ are given by \eqref{eq:dt-def}--\eqref{eq:dl-def}.
\end{proposition}

\begin{proof}
By the RSS definition $g=\alpha\otimes\alpha-h$ (equivalently $g=2\alpha\otimes\alpha-\delta$),
for any $V$ we have $g(V,V)=\alpha(V)^2-h(V,V)$.  Apply this to $V=dX$ and substitute
\eqref{eq:dt-def} and \eqref{eq:dl-def}.
\end{proof}

% ------------------------------------------------------------
\subsection{Proper time $\tau$ and the $\chi\mapsto\tau$ phase reparametrization}
\label{subsec:budget-equivalence}

We define the proper time $\tau$ along a $g$--timelike curve as the $g$--arc length
parameter.

\begin{definition}[Proper time]
\label{def:proper-time}
Along a $g$--timelike curve (i.e.\ $g(dX,dX)>0$) the proper time is defined by
\begin{equation}\label{eq:tau-def}
   \clight^2\,d\tau^2 := g(dX,dX).
\end{equation}
Equivalently, combining \eqref{eq:tau-def} with \eqref{eq:ds-operational} yields
\begin{equation}\label{eq:tau-from-tl}
   d\tau^2 = dt^2 - \frac{1}{\clight^2}\,d\ell^2.
\end{equation}
\end{definition}

Assume henceforth that $\chi$ is $\delta$--calibrated in the sense of Definition~\ref{def:chi-cal},
and write $\tilde X=dX/d\chi$.  Define the longitudinal and transverse rates (per unit $\chi$) by
\begin{equation}\label{eq:SL-def}
   S(\chi) := \alpha(\tilde X)=N\!\cdot\!\tilde X,
   \qquad
   L(\chi)^2 := h(\tilde X,\tilde X)=\delta(\tilde X_\perp,\tilde X_\perp).
\end{equation}
Then \eqref{eq:delta-split} becomes the exact budget identity
\begin{equation}\label{eq:budget-identity}
   \underbrace{\delta(\tilde X,\tilde X)}_{\clight^2}
   \;=\;
   \underbrace{S^2}_{\text{longitudinal (time-axis) rate}}
   \;+\;
   \underbrace{L^2}_{\text{transverse (spatial) rate}}.
\end{equation}

\begin{theorem}[Phase reparametrization $\chi\mapsto\tau$]
\label{thm:chi-to-tau}
Let $X(\chi)$ be $\delta$--calibrated and $g$--timelike along the curve, i.e.
$g(\tilde X,\tilde X)>0$.  Then
\begin{equation}\label{eq:tau-rate}
   \frac{d\tau}{d\chi}
   \;=\;
   \frac{1}{\clight}\,\sqrt{g(\tilde X,\tilde X)}
   \;=\;
   \frac{1}{\clight}\,\sqrt{S^2 - L^2}.
\end{equation}
Equivalently, reparametrizing the same geometric curve by $\tau$
(i.e.\ setting $\dot X:=dX/d\tau$) yields the unit--speed condition
\begin{equation}\label{eq:unit-speed-proper}
   g(\dot X,\dot X)=\clight^2.
\end{equation}
\end{theorem}

\begin{proof}
By Definition~\ref{def:proper-time}, $\clight^2(d\tau/d\chi)^2=g(\tilde X,\tilde X)$,
giving \eqref{eq:tau-rate}.  Then $\dot X=(d\chi/d\tau)\tilde X$ implies
\[
   g(\dot X,\dot X)=\Bigl(\frac{d\chi}{d\tau}\Bigr)^2 g(\tilde X,\tilde X)
   =\frac{\clight^2}{g(\tilde X,\tilde X)}\,g(\tilde X,\tilde X)=\clight^2,
\]
which is \eqref{eq:unit-speed-proper}.
\end{proof}

\begin{remark}[Roles of $\chi$, $t$, and $\tau$]
\label{rem:roles-chi-t-tau}
The three parameters used in this paper have distinct status:
\begin{itemize}
\item $\chi$ is an auxiliary $\delta$--calibration fixing the total proto--space budget
      $\|\tilde X\|_\delta=\clight$;
\item $t$ is the observer--adapted coordinate time obtained by projecting $dX$ onto the
      time axis $N$ via \eqref{eq:dt-def};
\item $\tau$ is the proper time of an ideal comoving clock, defined invariantly as the
      $g$--arc length by \eqref{eq:tau-def}.
\end{itemize}
Operationally, the observer has access to $t$ (relative to its chosen time axis $N$)
and to $\tau$ (along its own worldline); $\chi$ is a convenient proto--space gauge used
to expose the fixed Euclidean budget underlying the reparametrization equivalence.
\end{remark}

\begin{remark}[Circular tilt $\xi$ and the explicit $\chi$--$\tau$ factor]
\label{rem:xi-chi-tau-factor}
Under the $\chi$--calibration \eqref{eq:chi-cal}, the proto--velocity $\tilde X$ lies on
the sphere $S^3_{\clight}(p)\subset T_p\Espace$.  Let $\xi$ denote the circular Euclidean
angle between $\tilde X$ and $N$,
\[
   \cos\xi=\frac{N\cdot \tilde X}{\|\tilde X\|_\delta},\qquad
   \sin\xi=\frac{\|\tilde X_\perp\|_\delta}{\|\tilde X\|_\delta}.
\]
Then $S=\clight\cos\xi$ and $L=\clight\sin\xi$, so
\[
   g(\tilde X,\tilde X)=S^2-L^2=\clight^2\cos(2\xi),
\]
and Theorem~\ref{thm:chi-to-tau} becomes
\[
   \frac{d\tau}{d\chi}=\sqrt{\cos(2\xi)}.
\]
Thus the proper--time rate is controlled by the double--angle factor $\cos(2\xi)$
already at the circular level; the tangential (hyperbolic) parametrization introduced later
is a different coordinate choice adapted to velocity composition, not a different source of
the $\cos(2\,\cdot)$ structure.
\end{remark}

\begin{remark}[Why $g$ is observed rather than $\delta$]
\label{rem:g-observed}
The proto--metric $\delta$ measures the total Euclidean budget with respect to the auxiliary
calibration $\chi$.  Physical clocks, however, realize the proper--time parametrization $\tau$
defined by \eqref{eq:tau-def}, hence by the quadratic form $g$.  In this operational sense the
observer ``lives by its own phase'': the time variable implemented by ideal clocks is $\tau$,
and the interval controlling it is $g$, not $\delta$.
\end{remark}


% ===============================
\section{Tilt geometry: circular ($\xi$) versus conic/hyperbolic ($\vartheta,\eta$) parametrizations}
\label{sec:Hyper}
% ===============================

Throughout this section, $p\in\Espace$ is fixed and all statements are understood
pointwise at $p$.  We work with the $\delta$--orthogonal splitting relative to
the time axis $N$ (cf.\ \S\ref{sec:LMP}):
\[
   X = X_\parallel + X_\perp,
   \qquad
   X_\parallel := (N\!\cdot\!X)\,N,
   \qquad
   X_\perp := X-(N\!\cdot\!X)\,N.
\]
Recall that the induced Lorentzian form is $g=2N^\flat\!\otimes N^\flat-\delta$,
so that
\[
   g(X,X)=(N\!\cdot\!X)^2-\delta(X_\perp,X_\perp)
   \qquad\text{and}\qquad
   \delta(X,X)=(N\!\cdot\!X)^2+\delta(X_\perp,X_\perp).
\]

% ------------------------------------------------------------
\subsection{Circular parametrization: the Euclidean tilt angle $\xi$}
\label{subsec:circular-xi}

For any nonzero $X\in T_p\Espace$ define its circular (Euclidean) tilt angle
$\xi\in[0,\pi]$ relative to the axis $N$ by
\begin{equation}\label{eq:xi-def-Hyper}
   \cos\xi := \frac{N\cdot X}{\|X\|_\delta},
   \qquad
   \sin\xi := \frac{\|X_\perp\|_\delta}{\|X\|_\delta},
   \qquad \|X\|_\delta:=\sqrt{\delta(X,X)}.
\end{equation}
Whenever $X_\perp\neq 0$, define the transverse unit direction
\[
   E := \frac{X_\perp}{\|X_\perp\|_\delta}\in N^{\perp_\delta},
   \qquad \delta(E,E)=1,\qquad N\cdot E=0.
\]
Then $X$ admits the explicit circular decomposition
\begin{equation}\label{eq:X-circular-decomp}
   X \;=\; \|X\|_\delta\bigl(\cos\xi\,N+\sin\xi\,E\bigr).
\end{equation}

\begin{lemma}[Circular budget identities]\label{lem:circular-budget}
For any nonzero $X\in T_p\Espace$,
\[
   \|X\|_\delta^2=(N\cdot X)^2+\delta(X_\perp,X_\perp),
   \qquad
   \cos^2\xi+\sin^2\xi=1.
\]
\end{lemma}

\begin{proof}
Immediate from $\delta(X_\parallel,X_\perp)=0$ and Definition~\eqref{eq:xi-def-Hyper}.
\end{proof}

\begin{proposition}[Lorentzian norm in circular form]\label{prop:g-circular}
For any nonzero $X\in T_p\Espace$,
\begin{equation}\label{eq:g-cos2xi-Hyper}
   g(X,X)
   =\|X\|_\delta^2(\cos^2\xi-\sin^2\xi)
   =\|X\|_\delta^2\cos(2\xi).
\end{equation}
\end{proposition}

\begin{proof}
Substitute $(N\!\cdot\!X)=\|X\|_\delta\cos\xi$ and $\|X_\perp\|_\delta=\|X\|_\delta\sin\xi$
into $g(X,X)=(N\!\cdot\!X)^2-\|X_\perp\|_\delta^2$.
\end{proof}

\begin{remark}[Timelike domain in circular coordinates]
\label{rem:domain-xi}
In the future timelike cone $\mathcal T_p^+=\{X:\ g(X,X)>0,\ N\cdot X>0\}$ one has
$\cos(2\xi)>0$ and $\cos\xi>0$, hence
\[
   X\in\mathcal T_p^+
   \quad\Longleftrightarrow\quad
   \xi\in\Bigl[0,\frac{\pi}{4}\Bigr).
\]
The boundary $\xi=\pi/4$ corresponds to $g(X,X)=0$ (null directions).
\end{remark}

% ------------------------------------------------------------
\subsection{Conic (tangential) parametrization: the ratio angle $\vartheta$}
\label{subsec:conic-vartheta}

While $\xi$ is defined by normalizing with the total Euclidean magnitude
$\|X\|_\delta$, the physically relevant kinematic ratio on the future timelike
cone is the transverse--to--longitudinal budget ratio
\[
   \beta := \frac{\|X_\perp\|_\delta}{N\cdot X},
   \qquad (X\in\mathcal T_p^+).
\]
This suggests a conic (tangential) coordinate $\vartheta$ defined by
\begin{equation}\label{eq:vartheta-def}
   \tan\vartheta := \frac{\|X_\perp\|_\delta}{N\cdot X},
   \qquad (X\in\mathcal T_p^+),
\end{equation}
so that $\beta=\tan\vartheta\in[0,1)$ on $\mathcal T_p^+$.
Null vectors satisfy $\tan\vartheta=1$ (equivalently $\vartheta=\pi/4$), and
$g$--spacelike vectors satisfy $\tan\vartheta>1$.

\begin{remark}[Relation between $\xi$ and $\vartheta$]
\label{rem:xi-vartheta-relation}
On the future sector, $\cos\xi>0$, hence the ratio definition gives
\[
   \tan\vartheta=\frac{\|X_\perp\|_\delta}{N\cdot X}
   =\frac{\|X\|_\delta\sin\xi}{\|X\|_\delta\cos\xi}
   =\tan\xi,
\]
so numerically $\vartheta=\xi$ on $\mathcal T_p^+$.  The distinction is
conceptual: $\xi$ is a circular normalization by $\|X\|_\delta$, whereas
$\vartheta$ is a conic normalization by the longitudinal component $N\cdot X$.
This is exactly the choice that makes $\beta=\tan\vartheta$ the natural speed--like
parameter saturating at $\beta\to 1$ in the null limit.
\end{remark}

\begin{proposition}[Lorentzian norm in conic form]\label{prop:g-conic}
For $X\in\mathcal T_p^+$,
\begin{equation}\label{eq:g-conic}
   g(X,X)
   =(N\!\cdot\!X)^2\Bigl(1-\tan^2\vartheta\Bigr)
   =(N\!\cdot\!X)^2(1-\beta^2)
   =\delta(X,X)\,\frac{1-\beta^2}{1+\beta^2},
   \qquad \beta=\tan\vartheta.
\end{equation}
Equivalently,
\begin{equation}\label{eq:cos2xi-via-beta}
   \cos(2\xi)=\frac{1-\beta^2}{1+\beta^2}.
\end{equation}
\end{proposition}

\begin{proof}
From $g(X,X)=(N\!\cdot\!X)^2-\|X_\perp\|_\delta^2$ and $\|X_\perp\|_\delta=(N\!\cdot\!X)\tan\vartheta$
one gets $g(X,X)=(N\!\cdot\!X)^2(1-\tan^2\vartheta)=(N\!\cdot\!X)^2(1-\beta^2)$.
Moreover,
\[
   \delta(X,X)=(N\!\cdot\!X)^2+\|X_\perp\|_\delta^2=(N\!\cdot\!X)^2(1+\beta^2),
\]
hence $g(X,X)=\delta(X,X)\,(1-\beta^2)/(1+\beta^2)$.
Using $\beta=\tan\xi$ and $\cos(2\xi)=(1-\tan^2\xi)/(1+\tan^2\xi)$ yields \eqref{eq:cos2xi-via-beta}.
\end{proof}

% ------------------------------------------------------------
\subsection{Hyperbolic parameter (rapidity) $\eta$}
\label{subsec:eta-def}

A genuine group parameter for boosts is the rapidity $\eta$.
For $X\in\mathcal T_p^+$ define $\eta\ge 0$ by
\begin{equation}\label{eq:def-eta}
   \tanh\eta := \tan\vartheta = \beta.
\end{equation}
Equivalently, $\eta$ can be defined invariantly by the pair of relations
\begin{equation}\label{eq:eta-invariant}
   \cosh\eta := \frac{N\cdot X}{\sqrt{g(X,X)}},
   \qquad
   \sinh\eta := \frac{\|X_\perp\|_\delta}{\sqrt{g(X,X)}},
   \qquad (X\in\mathcal T_p^+),
\end{equation}
which immediately implies $\tanh\eta=\|X_\perp\|_\delta/(N\cdot X)=\tan\vartheta$.

\begin{proposition}[Hyperbolic decomposition of timelike vectors]\label{prop:hyperbolic-decomp}
For $X\in\mathcal T_p^+$ and $E=X_\perp/\|X_\perp\|_\delta$ one has
\begin{equation}\label{eq:X-hyperbolic-decomp}
   X \;=\; \sqrt{g(X,X)}\bigl(\cosh\eta\,N+\sinh\eta\,E\bigr),
\end{equation}
and the identities \eqref{eq:eta-invariant} hold.
\end{proposition}

\begin{proof}
Write $X=(N\!\cdot\!X)N+\|X_\perp\|_\delta E$ and factor out $\sqrt{g(X,X)}$ using
\eqref{eq:eta-invariant}.
\end{proof}

\begin{lemma}\label{lem:cosh-sinh-vartheta}
For $X\in\mathcal T_p^+$,
\[
   \cosh\eta=\frac{1}{\sqrt{1-\tan^2\vartheta}}
            =\frac{\cos\vartheta}{\sqrt{\cos(2\vartheta)}},
   \qquad
   \sinh\eta=\frac{\tan\vartheta}{\sqrt{1-\tan^2\vartheta}}
            =\frac{\sin\vartheta}{\sqrt{\cos(2\vartheta)}}.
\]
\end{lemma}

\begin{proof}
From $\tanh\eta=\tan\vartheta$ we have
\[
   \cosh^2\eta=\frac{1}{1-\tanh^2\eta}=\frac{1}{1-\tan^2\vartheta}.
\]
Taking the positive square root (since $\eta\ge 0$ and $\vartheta\in[0,\pi/4)$)
gives $\cosh\eta$.  Then $\sinh\eta=\tanh\eta\,\cosh\eta=\tan\vartheta\,\cosh\eta$.
Finally, $\cos(2\vartheta)=\cos^2\vartheta-\sin^2\vartheta=\cos^2\vartheta(1-\tan^2\vartheta)$
yields the alternative expressions.
\end{proof}

% ------------------------------------------------------------
\subsection{Differential relation between $\eta$ and the conic angle $\vartheta$}
\label{subsec:deta}

\begin{proposition}\label{prop:deta}
For $X\in\mathcal T_p^+$, the parameters $\eta$ and $\vartheta$ satisfy
\[
   \frac{d\eta}{d\vartheta}=\frac{1}{\cos(2\vartheta)}.
\]
\end{proposition}

\begin{proof}
Differentiate $\tanh\eta=\tan\vartheta$:
\[
   \sech^2\eta\, d\eta = \sec^2\vartheta\, d\vartheta.
\]
Using $\sech^2\eta=1-\tanh^2\eta=1-\tan^2\vartheta=\cos(2\vartheta)/\cos^2\vartheta$
gives
\[
   \frac{d\eta}{d\vartheta}
   =\frac{\sec^2\vartheta}{\sech^2\eta}
   =\frac{1/\cos^2\vartheta}{\cos(2\vartheta)/\cos^2\vartheta}
   =\frac{1}{\cos(2\vartheta)}.
\]
\end{proof}

% ------------------------------------------------------------
\subsection{Boost subgroup and additivity of the hyperbolic parameter}
\label{subsec:boost-additivity}

Let $\mathrm{O}(g)$ denote the Lorentz group of $(T_p\Espace,g)$:
\[
   \mathrm{O}(g)
   := \{\,\Lambda:T_p\Espace\to T_p\Espace\ \text{linear}:\ g(\Lambda X,\Lambda Y)=g(X,Y)\,\}.
\]
Fix a $\delta$--unit transverse direction $E\in N^{\perp_\delta}$, $\delta(E,E)=1$.
The boost in the $2$--plane $\mathrm{span}\{N,E\}$ with rapidity $\eta$
is the unique $\Lambda(\eta)\in\mathrm{O}(g)$ acting as a hyperbolic rotation on
$\mathrm{span}\{N,E\}$ and as the identity on its $g$--orthogonal complement:
\[
   \Lambda(\eta)N = (\cosh\eta)\,N + (\sinh\eta)\,E,\qquad
   \Lambda(\eta)E = (\sinh\eta)\,N + (\cosh\eta)\,E,
\]
\[
   \Lambda(\eta)X=X
   \quad \text{for } X\perp_g \mathrm{span}\{N,E\}.
\]
Such boosts preserve $g$ but, in general, do not preserve $\delta$ and do not fix $N$.

\begin{theorem}[Additivity of rapidity]\label{thm:additivity}
For boosts $\Lambda(\eta_1)$ and $\Lambda(\eta_2)$ in the same $(N,E)$--plane,
their composition is a boost with parameter $\eta_1+\eta_2$:
\[
   \Lambda(\eta_1)\circ \Lambda(\eta_2)=\Lambda(\eta_1+\eta_2).
\]
\end{theorem}

\begin{proof}
On $\mathrm{span}\{N,E\}$ the boosts are represented (in the basis $\{N,E\}$) by
\[
\begin{pmatrix}
\cosh\eta & \sinh\eta\\
\sinh\eta & \cosh\eta
\end{pmatrix},
\]
whose multiplication adds rapidities.  On the $g$--orthogonal complement the
action is the identity, hence the statement holds on all of $T_p\Espace$.
\end{proof}

% ------------------------------------------------------------
\subsection{Comparison with classical angle conventions}
\label{subsec:compare-angles}

The geometric Euclidean tilt is captured by the circular angle $\xi$
(cf.\ \eqref{eq:xi-def-Hyper}), but different reformulations choose different
dimensionless parameters derived from it.

A common choice is the sine--based parameter
\[
   \beta_{\mathrm{sin}}:=\sin\xi=\frac{\|X_\perp\|_\delta}{\|X\|_\delta},
\]
whereas in the present work the kinematically natural parameter is the
ratio (conic) parameter
\[
   \beta:=\frac{\|X_\perp\|_\delta}{N\cdot X}=\tan\vartheta=\tan\xi,
   \qquad (X\in\mathcal T_p^+),
\]
which satisfies $\beta\in[0,1)$ on the future timelike cone and reaches the null
limit at $\beta\to 1$.

\begin{remark}[Photon limit and the $45^\circ$ Euclidean tilt]
\label{rem:photon-limit-45deg}
The null cone is characterized by $g(X,X)=0$, equivalently $\cos(2\xi)=0$, hence
the lightlike limit corresponds to $\xi\to\pi/4$ in the Euclidean picture.
In this limit one has
\[
   \beta=\tan\xi \to 1,
   \qquad
   \beta_{\mathrm{sin}}=\sin\xi \to \frac{1}{\sqrt{2}}.
\]
Thus a light ray is reached at a finite Euclidean tilt of $45^\circ$ relative
to $N$ (not at $90^\circ$).  The ratio parameter $\beta=\tan\xi$ is therefore
better adapted to the speed--of--light barrier than the sine parameter.
\end{remark}

\begin{remark}[Nonadditivity of $\xi$ and $\vartheta$]
\label{rem:nonadditivity-xi-vartheta}
Neither the circular angle $\xi$ nor the conic coordinate $\vartheta$ is a group
parameter for boosts.  Even in the collinear case, where rapidities add,
$\eta_{12}=\eta_1+\eta_2$, the corresponding tilt coordinates do not add:
\[
   \xi_{12}\neq \xi_1+\xi_2,
   \qquad
   \vartheta_{12}\neq \vartheta_1+\vartheta_2.
\]
Indeed, the collinear velocity--composition law
\[
   \beta_{12}=\frac{\beta_1+\beta_2}{1+\beta_1\beta_2}
\]
translates into
\[
   \tan\vartheta_{12}=\frac{\tan\vartheta_1+\tan\vartheta_2}{1+\tan\vartheta_1\tan\vartheta_2},
\]
showing explicitly that $\vartheta$ is a nonlinear reparameterization of the
additive rapidity.
\end{remark}

\begin{remark}[Why we keep both parametrizations]
\label{rem:why-both}
The circular angle $\xi$ is geometrically immediate and makes the double--angle
structure $g(X,X)=\|X\|_\delta^2\cos(2\xi)$ transparent.  The conic/hyperbolic
parametrization $(\vartheta,\eta)$ is operationally adapted to velocity ratios and
to the boost subgroup: $\beta=\tan\vartheta$ saturates at $\beta\to 1$ in the null
limit and $\eta$ is additive under boosts.  The two descriptions therefore
complement each other.
\end{remark}

% ===============================
\section{Proto--velocity invariants and the emergence of $S^3$ in the Euclidean proto-space}
\label{sec:velocity-invariants-S3}
% ===============================

This section formalizes a central equivalence of the proto--space approach.
Once an observer time axis $N$ (hence the Lorentzian form $g$) is fixed,
the standard SR normalization of the 4--velocity in proper time,
\[
   g(\dot X,\dot X)=\clight^2,
\]
is equivalent (up to reparameterization) to a constant Euclidean proto--velocity budget
in proto--space,
\[
   \delta(\widetilde X,\widetilde X)=\clight^2,
\]
with respect to a calibrated proto--parameter $\chi$.
The latter condition forces the instantaneous proto--velocity states to lie on the Euclidean sphere
$S^3_{\clight}\subset T_p\Espace$.

% ------------------------------------------------------------
\subsection{Worldlines, proto--parameters, and the proto--velocity}
\label{subsec:protovelocity}

Let $X:I\to\Espace$ be a smooth regular worldline (geometric curve).
A proto--parameter along $X$ is any smooth parameter $\chi$ with
nowhere--vanishing derivative.  The associated proto--velocity is
\begin{equation}\label{eq:protovelocity-def}
   \widetilde{X} \;:=\; \frac{dX}{d\chi}\ \in T_{X(\chi)}\Espace.
\end{equation}

\begin{definition}[$\delta$--calibrated proto--parameter]
\label{def:calibrated-chi}
A proto--parameter $\chi$ is called $\delta$--calibrated (with scale
$\clight$) if
\begin{equation}\label{eq:calibrated-speed}
   \delta(\widetilde{X},\widetilde{X}) \;=\; \clight^2
   \qquad\text{along }X.
\end{equation}
\end{definition}

\begin{remark}[Existence and gauge nature]
\label{rem:chi-existence}
For any regular curve $X$ with $dX\neq 0$, a $\delta$--calibrated parameter
always exists: one may define $\chi$ (up to an additive constant) by
\[
   d\chi := \frac{1}{\clight}\,\|dX\|_\delta.
\]
Fixing an orientation ($d\chi>0$ along the chosen direction of traversal),
the calibrated proto--parameter is unique up to translation.
This calibration is auxiliary: it fixes the total Euclidean proto--velocity budget
in $\Espace$ but is not, by itself, the observer's operational time variable.
\end{remark}

In words: in a $\delta$--calibrated proto--parameter, the proto--velocity $\widetilde X$
has fixed Euclidean norm.  This is the proto--space counterpart of the
standard SR statement that the 4--velocity has fixed Minkowski norm in proper
time.

% ------------------------------------------------------------
\subsection{Observer splitting and the interval--rate identity}
\label{subsec:interval-rate}

Fix a $\delta$--unit time-axis field $N$ as in \S\ref{sec:LMC}--\S\ref{sec:LMP}
(with a chosen time orientation).  Pointwise along $X$, decompose the proto--velocity
into $\delta$--longitudinal and $\delta$--transverse parts relative to $N$:
\begin{equation}\label{eq:protovelocity-decomp}
   \widetilde{X}
   \;=\;
   S\,N + \widetilde{X}_\perp,
   \qquad
   S := N\cdot \widetilde{X},
   \qquad
   \widetilde{X}_\perp := h(\widetilde{X})\in N^{\perp_\delta}.
\end{equation}
Let $L:=\|\widetilde{X}_\perp\|_\delta$ and, when $L\neq 0$,
$E:=\widetilde{X}_\perp/L\in N^{\perp_\delta}$ so that $\widetilde{X}=S\,N+L\,E$
with $\delta(E,E)=1$.

\begin{lemma}[Euclidean and Lorentzian norms of the proto--velocity]
\label{lem:euclid-vs-lorentz}
Along $X$ one has
\begin{equation}\label{eq:norm-identities-protovelocity}
   \delta(\widetilde{X},\widetilde{X}) = S^{2}+L^{2},
   \qquad
   g(\widetilde{X},\widetilde{X}) = S^{2}-L^{2}.
\end{equation}
\end{lemma}
\begin{proof}
Since $\widetilde{X}_\perp\in N^{\perp_\delta}$, one has $\delta(N,\widetilde{X}_\perp)=0$,
hence $\delta(\widetilde X,\widetilde X)=S^2+\delta(\widetilde X_\perp,\widetilde X_\perp)=S^2+L^2$.
For $g$, use $g(N,N)=1$, $g(N,\widetilde X_\perp)=0$, and
$g(\widetilde X_\perp,\widetilde X_\perp)=-\delta(\widetilde X_\perp,\widetilde X_\perp)=-L^2$.
\end{proof}

Motivated by the phase/clock viewpoint, define the interval rate with respect
to $\chi$ by
\begin{equation}\label{eq:interval-rate-def}
   \widetilde{s} \;:=\; \sqrt{g(\widetilde{X},\widetilde{X})}
   \;=\; \sqrt{S^{2}-L^{2}}
   \qquad (g\text{--timelike case}).
\end{equation}
Equivalently, along a timelike segment one may write
$g(dX,dX)=\widetilde{s}^{\,2}\,d\chi^{2}$.

\begin{remark}[Circular angle $\xi$ on calibrated protovelocities]
\label{rem:xi-on-calibrated-protovelocity}
If $\chi$ is $\delta$--calibrated, $\delta(\widetilde X,\widetilde X)=\clight^2$, then
\eqref{eq:norm-identities-protovelocity} implies $S^2+L^2=\clight^2$ and one may set
\[
   \cos\xi:=\frac{S}{\clight},\qquad \sin\xi:=\frac{L}{\clight}.
\]
Then
\[
   g(\widetilde X,\widetilde X)=\clight^2(\cos^2\xi-\sin^2\xi)=\clight^2\cos(2\xi),
   \qquad
   \widetilde{s}=\clight\sqrt{\cos(2\xi)}.
\]
This is the same double--angle structure that governs the phase rate $d\tau/d\chi$
in \S\ref{sec:PRT}.
\end{remark}

% ------------------------------------------------------------
\subsection{Equivalence: proper--time normalization $\Longleftrightarrow$ calibrated proto--velocity}
\label{subsec:equivalence-local-time}

Let $\tau$ denote the proper time along a $g$--timelike worldline $X$, i.e.\ a
parameter such that $\dot X:=dX/d\tau$ satisfies
\begin{equation}\label{eq:proper-time-norm}
   g(\dot{X},\dot{X}) = \clight^2.
\end{equation}
Equivalently, $g(dX,dX)=\clight^2\,d\tau^2$ along $X$.

\begin{remark}[Units and normalized 4--velocity]\label{rem:U-normalization}
The proper--time tangent $\dot X$ has the physical dimension of a speed and is
normalized by \eqref{eq:proper-time-norm}.  It is often convenient to introduce
the dimensionless unit 4--velocity
\[
   U \;:=\; \frac{1}{\clight}\,\dot{X},
   \qquad\text{so that}\qquad g(U,U)=1.
\]
Whenever $U$ arises from a worldline it is understood as the normalized
proper--time tangent.
\end{remark}

\begin{theorem}[Reparameterization equivalence]
\label{thm:equiv}
Let $X$ be a regular $g$--timelike worldline (geometric curve).  Then the
following statements are equivalent up to reparameterization:
\begin{itemize}
\item[(A)] $X$ is parameterized by proper time $\tau$ so that
      $g(\dot{X},\dot{X})=\clight^2$.
\item[(B)] $X$ is parameterized by a $\delta$--calibrated proto--parameter $\chi$
      so that $\delta(\widetilde{X},\widetilde{X})=\clight^2$.
\end{itemize}
Moreover, when both parameters are used on the same curve, they satisfy
\begin{equation}\label{eq:tau-chi-relation}
   \frac{d\tau}{d\chi}
   \;=\;
   \frac{\sqrt{g(\widetilde{X},\widetilde{X})}}{\clight}
   \;=\;
   \frac{\widetilde{s}}{\clight},
   \qquad
   \frac{d\chi}{d\tau}
   \;=\;
   \frac{\|\dot{X}\|_\delta}{\clight}.
\end{equation}
\end{theorem}

\begin{proof}
Assume (A).  Define $\chi$ (up to an additive constant) by
\[
   \frac{d\chi}{d\tau} := \frac{\|\dot{X}\|_\delta}{\clight},
\]
which is smooth and positive since $\dot X\neq 0$.  Then
$\widetilde X=dX/d\chi=(d\tau/d\chi)\dot X$, so
\[
   \delta(\widetilde X,\widetilde X)
   =\Bigl(\frac{d\tau}{d\chi}\Bigr)^2\delta(\dot X,\dot X)
   =\frac{\clight^2}{\|\dot X\|_\delta^2}\,\|\dot X\|_\delta^2
   =\clight^2,
\]
which is (B).

Conversely, assume (B).  Define $\tau$ (up to an additive constant) by
\[
   \frac{d\tau}{d\chi} := \frac{\sqrt{g(\widetilde{X},\widetilde{X})}}{\clight},
\]
which is well--defined and positive for timelike $\widetilde X$ since
$g(\widetilde X,\widetilde X)>0$.  Then $\dot X=dX/d\tau=(d\chi/d\tau)\widetilde X$, hence
\[
   g(\dot X,\dot X)
   =\Bigl(\frac{d\chi}{d\tau}\Bigr)^2 g(\widetilde X,\widetilde X)
   =\frac{\clight^2}{g(\widetilde X,\widetilde X)}\,g(\widetilde X,\widetilde X)
   =\clight^2,
\]
which is (A).  The relations \eqref{eq:tau-chi-relation} are exactly the two
defining ODEs.
\end{proof}

\paragraph{Operational meaning.}
Statement (A) is the standard SR normalization of the 4--velocity in proper time.
Statement (B) is the corresponding calibration of the proto--space proto--velocity.
Theorem~\ref{thm:equiv} shows that these two ``invariants'' are equivalent and
amount to a change of parameter: constant norm in one metric corresponds to
constant norm in the other once the operational projection structure ($N$, hence $g$)
has been fixed.

% ------------------------------------------------------------
\subsection{The $S^3$ of admissible proto--velocity states}
\label{subsec:S3-of-velocities}

Fix $p\in\Espace$.  The set of all $\delta$--calibrated protovelocities at $p$ is
the Euclidean 3--sphere of radius $\clight$ inside $T_p\Espace$:
\begin{equation}\label{eq:S3-sphere}
   S^3_{\clight}(p)
   \;:=\;
   \bigl\{\,V\in T_p\Espace:\ \delta(V,V)=\clight^2\,\bigr\}
   \;\cong\;
   S^3.
\end{equation}
Thus, once the calibration \eqref{eq:calibrated-speed} is imposed, every
instantaneous proto--kinematic state is a point on $S^3_{\clight}(p)$.

Relative to the time axis $N_p$, each $V\in S^3_{\clight}(p)$ admits the
decomposition
\begin{equation}\label{eq:V-on-S3-decomp}
   V=\clight\bigl(\cos\xi\,N+\sin\xi\,E\bigr),
   \qquad
   \xi\in[0,\pi],\quad
   E\in N^{\perp_\delta},\ \delta(E,E)=1,
\end{equation}
where $\xi$ is the circular tilt angle and $E$ is the transverse unit direction.
When $\sin\xi=0$ (the poles), the direction $E$ is irrelevant and the state
reduces to $V=\pm\clight\,N_p$.

\begin{remark}[Quaternionic encoding of $S^3$ (optional viewpoint)]
\label{rem:S3-as-unit-quaternions}
Fix a reference unit vector $e_0$ and the associated quaternionic subalgebra
$\mathbb H_{e_0}\subset\Cl^+(4,0)$ generated by $I_i:=e_i e_0$ ($i=1,2,3$),
so that $I_i^2=-1$ and $I_i I_j=-I_j I_i$ for $i\neq j$.
Then any vector $V=V^0 e_0+V^1 e_1+V^2 e_2+V^3 e_3\in T_p\Espace$ can be encoded as
\[
   Q(V):=V^0 + V^1 I_1 + V^2 I_2 + V^3 I_3\in\mathbb H_{e_0},
\]
and one has $\delta(V,V)=|Q(V)|^2$ (quaternionic norm).  Hence the calibrated
sphere $S^3_{\clight}(p)$ corresponds to the set of quaternions of norm $\clight$,
and the unit sphere corresponds to unit quaternions.  This identification
depends on the chosen reference axis (a gauge choice); the underlying manifold
structure is canonical.
\end{remark}

\begin{remark}[The physical sector of $S^3$ and time orientation]
\label{rem:physical-sector-S3}
While the calibrated proto--velocity states form the full sphere $S^3_{\clight}(p)$,
the requirement that a worldline be timelike with respect to $g$,
\[
   g(V,V)>0,
\]
restricts admissible states to
\[
   |S|>|L|
   \qquad\text{where}\qquad
   S:=N\cdot V,\quad L:=\|h(V)\|_\delta,
\]
i.e.\ to the union of two polar caps on $S^3_{\clight}(p)$ centered at
$\pm \clight N$.  The boundary $|S|=|L|$ is the null locus.

Because $g$ is insensitive to $N\mapsto -N$ but the oriented time projection
$\alpha=N^\flat$ is not, an observer fixes a future branch by imposing
$S=N\cdot V>0$.  This selects a single polar cap (the future timelike sector).
Along any sequence approaching the null boundary within that cap one has
$d\tau/d\chi=\sqrt{g(V,V)}/\clight\to 0$, consistently with
\eqref{eq:tau-chi-relation}.
\end{remark}


% ===============================
\section{Why the Euclidean proto--space viewpoint}
\label{sec:why-protospace}
% ===============================

The induced Lorentzian form $g$ constructed from $(\delta,N)$ equips each tangent
space $(T_p\Espace,g)$ with the standard causal classification and light cone.
The additional advantage of the Euclidean proto--space viewpoint is that it keeps,
in a single geometric package, both
(i) directional data in the Euclidean spatial complement
$\mathrm{Im}(h_p)=N_p^{\perp_\delta}$ and
(ii) clock calibration data encoded by $g$--projections onto oriented time axes.
In particular, ``light'' does not require an extra postulate: a null direction is
a $g$--null ray, and choosing a convenient $\delta$--normalization picks a canonical
cross--section of that ray on the calibrated Euclidean sphere.

The material in this section provides the operational dictionary needed for later
frequency and angle computations.  It uses only the observer splitting determined
by a chosen oriented time axis $N$.


% ------------------------------------------------------------
\subsection{$\xi$--formulation of inertial composition}
\label{subsec:xi-only-composition}
% ------------------------------------------------------------

Fix an inertial observer axis $e_0$ (constant on the region of interest),
$\delta(e_0,e_0)=1$, $\nabla^\delta e_0=0$, and set $N\equiv e_0$ in the SR sector.
Let $g=2\,e_0^\flat\!\otimes e_0^\flat-\delta$ be the induced Lorentzian form.
All statements below are pointwise at $p\in\Espace$.
Unless stated otherwise, we work on the future timelike sector $\mathcal T_p^{+}$,
so that $0\le\xi<\tfrac{\pi}{4}$, $\tilde H>0$, and $\cos(2\xi)>0$.

\paragraph{Calibrated proto--velocities and circular angle data.}
A $\chi$--calibrated proto--velocity state is a vector $\tilde X\in T_p\Espace$
satisfying $\delta(\tilde X,\tilde X)=\clight^2$.  Relative to $e_0$ write
\[
  \tilde X
  =\tilde H\,e_0+\tilde L\,E,
  \qquad
  E\in e_0^{\perp_\delta},\ \delta(E,E)=1,
  \qquad
  \tilde H^2+\tilde L^2=\clight^2,
\]
and define the circular tilt $\xi\in[0,\pi]$ as in Definition~\ref{def:xi}.
Since $\delta(\tilde X,\tilde X)=\clight^2$, Eq.~\eqref{eq:xi-def} yields
\begin{equation}\label{eq:xi-HL}
  \cos\xi=\frac{\tilde H}{\clight},
  \qquad
  \sin\xi=\frac{\tilde L}{\clight}.
\end{equation}

\paragraph{Operational time and the $\chi\!\to\!\tau$ rescaling.}
Along a timelike worldline, proper time satisfies
\begin{equation}\label{eq:tau-chi-xi}
  \frac{d\tau}{d\chi}=\sqrt{\cos(2\xi)},
  \qquad
  \cos(2\xi)=\frac{g(\tilde X,\tilde X)}{\delta(\tilde X,\tilde X)}.
\end{equation}
Hence the proper--time tangent and the dimensionless unit $g$--velocity are
\begin{equation}\label{eq:U-from-tildeX-xi}
  \dot X=\frac{dX}{d\tau}=\frac{1}{\sqrt{\cos(2\xi)}}\,\tilde X,
  \qquad
  U:=\frac{1}{\clight}\dot X=\frac{1}{\clight\sqrt{\cos(2\xi)}}\,\tilde X,
\end{equation}
so that $g(U,U)=1$.

\paragraph{Observer split in $\xi$--variables.}
Define the observer Lorentz factor and the spatial speed vector by
\[
  \gamma:=e_0\cdot_g U,
  \qquad
  \boldsymbol\beta:=\frac{1}{\gamma}\,h_g(U),
\]
where $h_g$ denotes the $g$--orthogonal projector onto $e_0^{\perp_g}$.
From \eqref{eq:U-from-tildeX-xi} and \eqref{eq:xi-HL} one obtains purely in terms of $\xi$
\begin{equation}\label{eq:gamma-beta-from-xi-only}
  \gamma
  =\frac{\cos\xi}{\sqrt{\cos(2\xi)}},
  \qquad
  \gamma\,\|\boldsymbol\beta\|
  =\frac{\sin\xi}{\sqrt{\cos(2\xi)}},
  \qquad
  \|\boldsymbol\beta\|
  =\tan\xi.
\end{equation}
Thus the proper--time normalization $g(U,U)=1$ is exactly the $\chi$--calibrated
state rescaled by $(\cos 2\xi)^{-1/2}$.


% ------------------------------------------------------------
\subsection{Light rays: canonical frequency and direction from the $N$--split}
\label{subsec:light-as-null}

Fix $p\in\Espace$ and an oriented time axis $N_p$ (future branch chosen).
A light ray direction at $p$ is a projective class $[K]$ of nonzero vectors
$K\in T_p\Espace$ satisfying
\[
   g(K,K)=0,\qquad N\cdot K>0,
\]
where $K\sim \lambda K$ for $\lambda>0$ represents the same ray.

A convenient representative is obtained by encoding the scale of $K$ as the
frequency measured by $N$.

\begin{definition}[Proto--frequency of a null ray (observer $N$)]
\label{def:omegaN}
Let $K\neq 0$ be future--directed and $g$--null.  The frequency measured by $N$ is
\begin{equation}\label{eq:omega-def-proj}
   \omega := g(K,N).
\end{equation}
\end{definition}

\begin{lemma}[Euclidean and Lorentzian projections coincide on $N$]
\label{lem:gKN-equals-dot}
For any $K\in T_p\Espace$ one has
\begin{equation}\label{eq:gKN-equals-dot}
   g(K,N)=N\cdot K.
\end{equation}
In particular, a future direction satisfies $\omega=N\cdot K>0$.
\end{lemma}

\begin{proof}
Using $g=2\,N^\flat\!\otimes N^\flat-\delta$ and $\delta(N,N)=1$,
\[
   g(K,N)=2(N\cdot K)(N\cdot N)-\delta(K,N)=2(N\cdot K)-(N\cdot K)=N\cdot K.
\]
\end{proof}

Decompose $K$ into longitudinal and transverse parts relative to $N$:
\[
   K = (N\cdot K)\,N + K_\perp,
   \qquad K_\perp\in N^{\perp_\delta}.
\]
The null condition fixes the transverse magnitude.

\begin{lemma}[Canonical null decomposition]
\label{lem:null-canonical}
Every future--directed null vector $K\neq 0$ admits a unique decomposition
\begin{equation}\label{eq:nullK-param}
   K = \omega\,(N+E),
\end{equation}
where $\omega=g(K,N)>0$ and $E\in N^{\perp_\delta}$ is uniquely determined by
\[
   \delta(E,E)=1,\qquad N\cdot E=0.
\]
\end{lemma}

\begin{proof}
Write $K=\omega N+K_\perp$ with $\omega=N\cdot K>0$ and $K_\perp\in N^{\perp_\delta}$.
Then
\[
   0=g(K,K)=g(\omega N,\omega N)+2g(\omega N,K_\perp)+g(K_\perp,K_\perp)
          =\omega^2-\delta(K_\perp,K_\perp),
\]
so $\|K_\perp\|_\delta=\omega$.  Set $E:=K_\perp/\omega$ to obtain \eqref{eq:nullK-param}.
Uniqueness follows from uniqueness of the $N$--split and the positivity of $\omega$.
\end{proof}

\begin{remark}[Null rays as a calibrated slice of the null cone]
\label{rem:null-slice-S3}
The scale in \eqref{eq:nullK-param} is encoded by $\omega$; correspondingly,
\[
   \delta(K,K)=\delta(\omega(N+E),\omega(N+E))=2\omega^2.
\]
If one prefers a $\delta$--calibrated representative on the Euclidean sphere,
define
\[
   \widehat K \;:=\; \frac{\clight}{\sqrt{2}\,\omega}\,K
   \;=\; \frac{\clight}{\sqrt{2}}\,(N+E).
\]
Then $\delta(\widehat K,\widehat K)=\clight^2$ and $g(\widehat K,\widehat K)=0$.
Thus the set of (future) null directions corresponds to the $S^2$ cross--section
\[
   S^3_{\clight}(p)\cap\{V:\ g(V,V)=0,\ N\cdot V>0\}
   =\Bigl\{\frac{\clight}{\sqrt{2}}(N+E):\ E\in S^2\subset N^{\perp_\delta}\Bigr\}.
\]
In the circular tilt language of \S\ref{sec:PRT}--\S\ref{sec:Hyper}, this is the
boundary $\xi=\pi/4$ of the future timelike cap on $S^3_{\clight}(p)$.
\end{remark}

% ------------------------------------------------------------
\subsection{Observers as unit timelike states; frequency as a $g$--projection}
\label{subsec:observers-as-U}

A local observer at $p$ is represented by a future unit timelike vector
(dimensionless)
\[
   U\in T_p\Espace,\qquad g(U,U)=1,\qquad g(U,N)>0.
\]
All measurable scalars are obtained by taking $g$--contractions.

\begin{definition}[Frequency measured by an observer]
\label{def:omegaU}
For a null ray represented by $K\neq 0$, the frequency measured by the observer
$U$ is
\begin{equation}\label{eq:omegaU-def}
   \omega_U := g(U,K).
\end{equation}
\end{definition}

\begin{remark}[Ray vector vs.\ wave covector]
\label{rem:K-vs-k}
We represent a light ray by a future--directed null vector $K\in T_p\Espace$.
Its metric dual 1--form
\[
   k \;:=\; g(K,\cdot)\ \in T_p^*\Espace
\]
is the standard wave covector of geometric optics.  With this notation,
$\omega_U=k(U)=g(U,K)$.
\end{remark}

% ------------------------------------------------------------
\subsection{Doppler shift as a one--line contraction}
\label{subsec:doppler-3lines}

Let $E_v\in N^{\perp_\delta}$ be a $\delta$--unit direction, $\delta(E_v,E_v)=1$.
Let $U$ be obtained from $N$ by a boost of rapidity $\eta\ge 0$ in the $(N,E_v)$--plane:
\begin{equation}\label{eq:U-boost}
   U := (\cosh\eta)\,N + (\sinh\eta)\,E_v.
\end{equation}
Let the null ray be written in canonical form \eqref{eq:nullK-param},
\[
   K=\omega\,(N+E),\qquad \delta(E,E)=1,\ \ E\in N^{\perp_\delta}.
\]
Using bilinearity and $g(N,N)=1$, $g(N,E)=0$, $g(E_v,N)=0$, $g(E_v,E)=-\delta(E_v,E)$,
we obtain
\begin{align}
   \omega_U
   &= g(U,K) \notag\\
   &= \omega\,g\bigl((\cosh\eta)N+(\sinh\eta)E_v,\ N+E\bigr)\notag\\
   &= \omega\Bigl(\cosh\eta-\sinh\eta\,\delta(E_v,E)\Bigr).
   \label{eq:doppler-raw}
\end{align}

Introduce the standard parameters
\[
   \beta := \tanh\eta,\qquad
   \gamma := \cosh\eta,
\]
and define the Euclidean angle $\psi\in[0,\pi]$ between the velocity axis $E_v$
and the ray direction $E$ inside the observer space $N^{\perp_\delta}$ by
\begin{equation}\label{eq:psi-def}
   \cos\psi := \delta(E_v,E).
\end{equation}
Then \eqref{eq:doppler-raw} becomes the standard relativistic Doppler law
\begin{equation}\label{eq:doppler-angle}
   \frac{\omega_U}{\omega}
   = \gamma\bigl(1-\beta\cos\psi\bigr).
\end{equation}

\paragraph{Interpretation.}
In the proto--space picture, \eqref{eq:doppler-angle} is literally a projection:
the measured frequency is the scalar $g$--projection of a null direction $K$
onto an observer state $U$.

% ------------------------------------------------------------
\subsection{Aberration as projection plus normalization}
\label{subsec:aberration}

The direction of the ray measured by $U$ is the normalized $U$--spatial part of
$K$ (in the $g$--orthogonal complement of $U$).  Define the $g$--spatial component
\begin{equation}\label{eq:KU-spatial}
   K_{\perp U} := K - (g(U,K))\,U.
\end{equation}
Then $g(U,K_{\perp U})=0$, hence $K_{\perp U}\in U^{\perp_g}$, and
\[
   g(K_{\perp U},K_{\perp U}) = -\,\omega_U^2
   \qquad(\text{since } g(K,K)=0,\ g(U,U)=1).
\]
Thus a unit spatial direction of the ray in the $U$--frame can be taken as
\begin{equation}\label{eq:EU-def}
   E_U := \frac{1}{\omega_U}\,K_{\perp U}
   = \frac{1}{g(U,K)}\Bigl(K-(g(U,K))U\Bigr),
   \qquad
   g(E_U,E_U)=-1,\ \ g(U,E_U)=0.
\end{equation}

Specialize to the same kinematics as in \S\ref{subsec:doppler-3lines}.
Let $\psi'$ denote the angle between the ray direction in the $U$--rest space and
the $U$--spatial image of the boost axis.  A direct computation yields the standard
aberration law
\begin{equation}\label{eq:aberration}
   \cos\psi'
   = \frac{\cos\psi-\beta}{1-\beta\cos\psi},
\end{equation}
with $\cos\psi=\delta(E_v,E)$ as in \eqref{eq:psi-def}.

\paragraph{Interpretation.}
Doppler and aberration are the same operation in two steps:
\begin{itemize}
\item Doppler: take the $g$--projection $g(U,K)$ (a scalar).
\item Aberration: subtract the time component $(g(U,K))U$ and normalize the
      remaining $U$--spatial part.
\end{itemize}
Both effects are thus immediate consequences of the null cone geometry together
with the observer--dependent splitting induced by $U$.

% ------------------------------------------------------------
\paragraph{A $\xi$--rotor that maps $e_0$ to the unit axis $U(\xi,E)$.}
Given circular data $(\xi,E)$ with $E\in e_0^{\perp_\delta}$, $\delta(E,E)=1$, define
\begin{equation}\label{eq:U-xiE}
  U(\xi,E):=\frac{1}{\sqrt{\cos(2\xi)}}\bigl(\cos\xi\,e_0+\sin\xi\,E\bigr),
\end{equation}
and define the associated (minimal) boost rotor by the standard axis--to--axis formula
\begin{equation}\label{eq:R-xiE}
  R(\xi,E):=
  \frac{1+U(\xi,E)\,e_0}{\sqrt{2\bigl(1+e_0\cdot_g U(\xi,E)\bigr)}}
  \quad\in\Spin^+(1,3),
\end{equation}
\noindent\emph{Remark.} Here $\Spin^+(1,3)$ is understood in the Clifford algebra
associated with the induced Lorentzian form $g$; i.e., the geometric product in
this subsection is the one determined by $g$ (not by $\delta$).
so that
\[
  R(\xi,E)\,e_0\,\widetilde{R(\xi,E)}=U(\xi,E),
  \qquad
  R(\xi,E)\widetilde{R(\xi,E)}=1.
\]
No rapidity parameter is used: $R$ is defined directly from $\xi$ via the rescaled axis
$U(\xi,E)$.

\begin{remark}[Why the rescaling is essential]
If one replaces $U(\xi,E)$ in \eqref{eq:R-xiE} by the unrescaled proto--velocity direction
$\tilde X/\clight=\cos\xi\,e_0+\sin\xi\,E$ (which lives on the calibrated sphere $S^3$),
the resulting sandwich map is a Euclidean rotation in the $(e_0,E)$--plane and does
not reproduce SR velocity composition.  The factor $(\cos 2\xi)^{-1/2}$ in \eqref{eq:U-xiE}
is precisely the operational correction that converts $\chi$--calibrated geometry into
proper--time (hence Lorentz) normalization.
\end{remark}

% ------------------------------------------------------------
\begin{proposition}[Two--step inertial composition in $\xi$--variables]
\label{prop:xi-only-two-step}
Let observer $0$ be encoded by $e_0$.
Assume object $1$ is given in the observer algebra by circular data $(\xi_1,E_1)$, where
$E_1\in e_0^{\perp_\delta}$ and $\delta(E_1,E_1)=1$.  Define its unit axis and rotor
\[
  U_1:=U(\xi_1,E_1),
  \qquad
  R_{10}:=R(\xi_1,E_1).
\]
Assume object $2$ is given in the comoving algebra of object $1$ by circular data
$(\xi_2',E_2')$, where $E_2'\in (U_1)^{\perp_\delta}$ is unit in the spatial complement
of $U_1$ (the ``space'' of object $1$).  Transport this direction into the observer algebra by
\begin{equation}\label{eq:transport-E2}
  E_2:=R_{10}\,E_2'\,\widetilde{R}_{10}.
\end{equation}
Define the unit axis of object $2$ relative to $U_1$ by the same $\xi$--recipe:
\begin{equation}\label{eq:U2-relative-U1}
  U_2:=\frac{1}{\sqrt{\cos(2\xi_2')}}\bigl(\cos\xi_2'\,U_1+\sin\xi_2'\,E_2\bigr).
\end{equation}
Then the corresponding rotor $R_{21}:=R(U_2,U_1)$ and the composite rotor
\[
  R_{20}:=R_{21}\,R_{10}
\]
map the observer axis to $U_2$:
\[
  R_{20}\,e_0\,\widetilde{R}_{20}=U_2.
\]
Moreover, the 3--velocity of object $2$ measured by observer $0$ obtained from
the observer split of $U_2$ coincides with the textbook SR velocity addition rule.
\end{proposition}

\begin{proof}[Proof sketch in $\xi$--language]
The only nontrivial step is operational normalization.  Each circular datum $(\xi,E)$
defines a proto--velocity direction $\cos\xi\,(\text{time})+\sin\xi\,(\text{space})$ on
the calibrated sphere, but physical composition of inertial frames acts on $g$--unit timelike
axes.  The proper--time rescaling \eqref{eq:U-from-tildeX-xi} converts the circular datum into
the $g$--unit axis $U(\xi,E)$, and thereby fixes the correct $\gamma$--factor
$\gamma=\cos\xi/\sqrt{\cos(2\xi)}$ as in \eqref{eq:gamma-beta-from-xi-only}.  Once $U_1$ is so
normalized, the transport \eqref{eq:transport-E2} is exactly the change of spatial basis from
the comoving algebra of object $1$ into the observer algebra by rotor conjugation.  Finally,
the definition \eqref{eq:U2-relative-U1} constructs the second axis from the same circular rule,
but now in the $1+3$ split of $U_1$.  Projecting the resulting $U_2$ onto the observer split
recovers the standard longitudinal and transverse SR composition laws, with the characteristic
$1/\gamma_1$ suppression of transverse components.
\end{proof}

% ------------------------------------------------------------
\paragraph{Worked example (orthogonal data; no rapidity).}
Let object $1$ have speed $\beta_1=0.6$ along $x$ in the observer algebra, hence
$\xi_1=\arctan(0.6)$ and $E_1=e_x$.
Let object $2$ have speed $\beta_2'=0.6$ along $y'$ in the comoving algebra of object $1$,
hence $\xi_2'=\arctan(0.6)$ and $E_2'=e_{y'}$.

From $\beta=\tan\xi$ one has $\cos(2\xi)=\frac{1-\beta^2}{1+\beta^2}$, and therefore
\[
  \gamma_1=\frac{\cos\xi_1}{\sqrt{\cos(2\xi_1)}}=\frac{1}{\sqrt{1-\beta_1^2}}=1.25.
\]
Applying Proposition~\ref{prop:xi-only-two-step} and projecting $U_2$ onto the observer split yields
\[
  \beta_x^{(20)}=\beta_1=0.6,
  \qquad
  \beta_y^{(20)}=\frac{\beta_2'}{\gamma_1}=\frac{0.6}{1.25}=0.48,
\]
which coincides exactly with the standard SR result for orthogonal velocity composition.

\paragraph{Wigner--Thomas rotation (optional; no rapidity).}
Let $R_{20}^{\min}:=R(U_2,e_0)$ denote the minimal rotor mapping $e_0$ directly to $U_2$.
Then
\[
  W:=\widetilde{R}_{20}^{\min}\,R_{21}\,R_{10}
\]
fixes $e_0$ and hence belongs to the observer's spatial stabilizer $\Spin(3)$; this is the
Wigner--Thomas rotation induced by the noncollinear two--step composition.


% ------------------------------------------------------------
\subsection{Geometric Algebra viewpoint: circular versus hyperbolic rotors}
\label{subsec:GA}

The Euclidean proto--space admits the Clifford algebra
$\Cl(T_p\Espace,\delta)\cong\Cl_{4,0}$.  For vectors $u,v\in T_p\Espace$ the
geometric product is defined by
\[
   uv = u\cdot v + u\wedge v,
\]
where $u\cdot v=\delta(u,v)$ is the Euclidean inner product.
The induced Lorentzian form $g$ can be encoded by the metric extensor
$G:=2\,N\otimes N-\mathrm{Id}$, satisfying
\[
   g(u,v)=\delta(Gu,v),
\]
which makes explicit how the $g$--geometry is obtained from Euclidean data once
$N$ is fixed.

\paragraph{Observer split inside $\Cl_{4,0}$.}
Multiplying a vector $X$ by $N$ decomposes it into a scalar (longitudinal) part
and a bivector (transverse) part relative to $N$:
\begin{equation}\label{eq:GA-split}
   XN = \underbrace{X\cdot N}_{\text{scalar}} \;+\; \underbrace{X\wedge N}_{\text{bivector}}.
\end{equation}
The bivectors of the form $E\wedge N$ (with $E\in N^{\perp_\delta}$) constitute the
observer's Euclidean spatial algebra.  The even subalgebra $\Cl^+_{4,0}$ is
isomorphic to the quaternion algebra; this is the algebraic origin of the
``unit quaternion'' representation of calibrated velocity states on $S^3$.

\paragraph{Circular rotor in Euclidean signature.}
Let $E\in N^{\perp_\delta}$ be $\delta$--unit.  In $\Cl_{4,0}$ one has
\[
   N^2=1,\qquad E^2=1,
\]
and the bivector $\mathbf{I}:=NE$ satisfies
\[
   \mathbf{I}^2=(NE)(NE)=-\,N^2E^2=-1.
\]
Hence $\exp(-\mathbf{I}\,\xi/2)$ generates an ordinary circular rotation in
the $(N,E)$--plane.  This is the natural algebraic encoding of the circular tilt
coordinate $\xi$ used in \S\ref{sec:PRT} and \S\ref{sec:Hyper}.

\paragraph{Hyperbolic rotor in Lorentzian signature.}
In the induced Lorentzian geometry, $N$ remains unit timelike while a spatial
unit direction $\mathbf{e}$ satisfies $\mathbf{e}^2\overset{g}{=}-1$.
Accordingly, the boost bivector $\mathbf{K}:=N\mathbf{e}$ satisfies
\[
   \mathbf{K}^2=(N\mathbf{e})(N\mathbf{e})=-\,N^2\mathbf{e}^2=+1,
\]
so $\exp(-\mathbf{K}\,\eta/2)$ generates a hyperbolic rotor (a boost) with
rapidity $\eta$.  Applying such a rotor to $N$ yields the standard boosted observer
state
\[
   U=\cosh\eta\,N+\sinh\eta\,\mathbf{e},
\]
and $\eta$ is additive under collinear boost composition.
In the conic parametrization of \S\ref{sec:Hyper} one has $\tanh\eta=\beta=\tan\vartheta$.

\begin{remark}[What ``emerges'' from the proto--space viewpoint]
No analytic continuation is invoked: the distinction between circular and
hyperbolic behavior is controlled by the sign of the square of the generator
bivector ($-1$ versus $+1$), which in turn is fixed by whether the spatial basis
squares to $+1$ (Euclidean $\delta$) or to $-1$ (Lorentzian $g$).  The proto--space
framework makes this dependence explicit by separating the Euclidean algebra
$\Cl_{4,0}$ from the induced Lorentzian structure determined by $N$.
\end{remark}

% ------------------------------------------------------------
\subsection{Scope and limitations}
\label{subsec:scope-limitations}

The present work is purely kinematical.  We do not introduce dynamical equations
for the field $N$, nor do we model localized sources, defects, or global topology.
Such extensions require additional structure (e.g.\ a field theory for $N$ and/or
a source model) and are deferred to separate work.

% ===============================
\section{Discussion}
\label{sec:discussion}
% ===============================

The primary contribution of this paper is a coordinate--free geometric
reformulation of special--relativistic kinematics in a Euclidean phase space
$(\Espace,\delta)$ equipped with an induced Lorentzian metric
$g=2\,\alpha\!\otimes\!\alpha-\delta$ generated by a unit field $N$
(or $\alpha:=\delta(N,\cdot)$).
A natural question is why it is useful to represent SR in a strictly Euclidean
$4$D setting without introducing a distinguished time coordinate at the outset.

First, the Euclidean proto--space separates two logically different ingredients
that are often conflated in the standard Minkowski presentation: the kinematic
state of motion and the operational choice of clock.
In our framework, $\delta$ provides a positive--definite reference geometry for
proto--velocities and for the calibrated budget constraint
$\delta(\tilde X,\tilde X)=c^2$, while the Lorentzian interval arises only after
an observer chooses a time axis $N$ and hence a time 1--form $\alpha$.
Proper time is then obtained by reparametrization along worldlines,
$c^2 d\tau^2=g(dX,dX)$.
This makes explicit that SR kinematics can be organized around a choice of axis
(and a clock parametrization) rather than around a fundamental time coordinate
built into the background manifold.

Second, the calibrated Euclidean picture provides a compact and geometrically
transparent state space.
At each point, the set of calibrated proto--velocities is the sphere
$S^3_c\subset T_p\Espace$, and the timelike and null conditions with respect to
$g$ select, respectively, an open sector (two polar caps) and its boundary.
In this representation the speed--of--light barrier is realized as a finite
geometric boundary on $S^3_c$ (a $45^\circ$ Euclidean tilt relative to $N$),
so the causal classification and its limiting behavior are read off directly
from Euclidean angle geometry and the identity $g(X,X)=\|X\|_\delta^2\cos(2\xi)$.

Third, working in $\Cl(4,0)$ keeps all kinematical transformations within a
strictly Euclidean rotor calculus.
Once a reference axis is fixed, relative states and frame changes can be encoded
by even elements (quaternionic coordinates) and rotors acting by sandwiching,
while the Lorentzian bilinear form is implemented by the RSS reflection/projection
relative to $N$.
This gives a uniform algebraic mechanism for transporting directions and
proto--velocities between observer gauges, and it isolates the genuinely
operational step: the reparametrization from the calibrated proto--parameter
to proper time.

Finally, although the present paper is restricted to the inertial SR sector
(where $N$ is taken constant), the formulation clarifies what must change beyond
it: departures from inertial kinematics correspond to variations of the time-axis field
time axis and/or its associated 1--form.
In this sense the Euclidean proto--space is not introduced as an alternative
time coordinate, but as a neutral geometric carrier in which the observer's time
structure is a field--like datum that can later be allowed to vary without
changing the basic kinematic bookkeeping.

% ===============================
\subsection{Outlook: towards fields via a tilted field vector}
\label{subsec:outlook-fields}
% ===============================

The special--relativistic sector treated in this paper assumes an inertial
(inertially constant) "time" axis represented by a field vector $N$ with
$\nabla^\delta N=0$, so that the induced Lorentzian form
$g_N=2\,N^\flat\!\otimes N^\flat-\delta$ is globally Minkowskian.
A natural extension is to allow $N$ to vary smoothly, while keeping the same
RSS ``Lorentzization'' rule for $g_N$.  In this view, what is usually called a
``gravitational field'' is encoded kinematically by the spatial variation of
the time direction, rather than by introducing a Lorentzian metric as
a primitive object.

To illustrate the mechanism in the simplest setting, consider a static,
spherically symmetric configuration on the Euclidean proto--space
$(\Espace,\delta)$ and write the flat background in spherical coordinates as
\[
  \delta = c^2\,dt^2 + dr^2 + r^2 d\Omega^2,
  \qquad
  d\Omega^2=d\theta^2+\sin^2\theta\,d\phi^2 .
\]
Let $e_0$ denote the constant fiducial time axis and $e_r$ the outward unit
radial vector with respect to $\delta$.  Introduce a radial tilt field
$\alpha=\alpha(r)$ and set
\begin{equation}\label{eq:N-tilt-radial}
   N(r) \;:=\; \cos\alpha(r)\,e_0 \;+\; \sin\alpha(r)\,e_r,
   \qquad
   \delta(N,N)=1 .
\end{equation}
The induced Lorentzian form
$g_N=2\,N^\flat\!\otimes N^\flat-\delta$ then acquires, in the $(t,r)$ block, the
components
\begin{equation}\label{eq:g-components-tilt}
   g_{tt}=\cos\!\bigl(2\alpha(r)\bigr),\qquad
   g_{tr}=\sin\!\bigl(2\alpha(r)\bigr),\qquad
   g_{rr}=-\cos\!\bigl(2\alpha(r)\bigr),
\end{equation}
while $g_{\theta\theta}=-r^2$ and $g_{\phi\phi}=-r^2\sin^2\theta$.  Hence
\begin{equation}\label{eq:lineelement-tilt-offdiag}
   ds^2
   \;=\;
   \cos(2\alpha)\,c^2dt^2
   \;+\;
   2\sin(2\alpha)\,c\,dt\,dr
   \;-\;
   \cos(2\alpha)\,dr^2
   \;-\;
   r^2 d\Omega^2 .
\end{equation}
The off--diagonal term is removed by the shifted time coordinate
\begin{equation}\label{eq:tschift}
   c\,dt_S \;:=\; c\,dt \;+\; \tan\!\bigl(2\alpha(r)\bigr)\,dr,
\end{equation}
which yields the diagonal form
\begin{equation}\label{eq:diag-form-A}
   ds^2
   \;=\;
   A(r)\,c^2 dt_S^2
   \;-\;
   \frac{1}{A(r)}\,dr^2
   \;-\;
   r^2 d\Omega^2,
   \qquad
   A(r):=\cos\!\bigl(2\alpha(r)\bigr).
\end{equation}
Thus a radial variation of the time-axis field produces the characteristic
reciprocal pair of lapse--type factors $A$ and $A^{-1}$ by a purely geometric
identity.  Choosing, for $r>r_s$, the profile
\begin{equation}\label{eq:A-schwarzschild-choice}
   A(r) \;=\; 1-\frac{r_s}{r},
   \qquad
   r_s:=\frac{2GM}{c^2},
\end{equation}
recovers the Schwarzschild form (in the $(+---)$ convention) as a special case,
with the ``tilt angle'' determined by
\begin{equation}\label{eq:alpha-from-A}
   \cos\!\bigl(2\alpha(r)\bigr)=1-\frac{r_s}{r}.
\end{equation}
In this parametrization the horizon corresponds to $A\to 0^+$, i.e. to the
finite Euclidean tilt $2\alpha\to\pi/2$ (so $\alpha\to\pi/4$), which is the same
distinguished $45^\circ$ value that appears in the null limit of the SR tilt
geometry.

The point of the present paragraph is not to derive general relativity within
the kinematical scope of this paper, but to clarify why the Euclidean
proto--space formulation is structurally useful: the same data $(\delta,N)$ that
encode inertial SR kinematics extend to a field setting by relaxing the
constancy of $N$.  The remaining dynamical question is then shifted from
postulating a Lorentzian metric to specifying an evolution law for the tilt
field (or for $N$), for instance via a variational principle on $(\Espace,\delta)$
or a coarse--grained source model.  Developing such field equations and their
relation to the Einstein vacuum equation is left for future work.


\bibliographystyle{plain}
\begin{thebibliography}{9}

\bibitem{HoeferHuggettReadSEP2024}
C.\ Hoefer, N.\ Huggett, J.\ Read,
Absolute and Relational Space and Motion: Classical Theories,
\emph{The Stanford Encyclopedia of Philosophy} (Fall 2024 Edition),
E.\,N.\ Zalta, U.\ Nodelman (eds.),
Metaphysics Research Lab, Stanford University.
\url{https://plato.stanford.edu/archives/fall2024/entries/spacetime-theories-classical/}

\bibitem{AristotlePhysics}
Aristotle,
\emph{Physics}, Book IV, Chs.\ 10--14 (esp.\ IV.11, 219b1--2).

\bibitem{Coope2005TimeAristotle}
U.\ Coope,
\emph{Time for Aristotle: Physics IV.10--14},
Oxford University Press, 2005.

\bibitem{MachMechanics1893}
E.\ Mach,
\emph{The Science of Mechanics: A Critical and Historical Exposition of Its Principles},
Open Court, Chicago, 1893 (English transl.\ by T.\,J.\ McCormack).

\bibitem{Rovelli2002PartialObs}
C.\ Rovelli,
Partial observables,
\emph{Phys.\ Rev.\ D} \textbf{65} (2002), 124013.
doi:10.1103/PhysRevD.65.124013.
arXiv:gr-qc/0110035.

\bibitem{ConnesRovelli1994ThermalTime}
A.\ Connes, C.\ Rovelli,
Von Neumann Algebra Automorphisms and Time--Thermodynamics Relation in General Covariant Quantum Theories,
arXiv:gr-qc/9406019 (1994).

\bibitem{PageWootters1983}
D.\,N.\ Page, W.\,K.\ Wootters,
Evolution without evolution: Dynamics described by stationary observables,
\emph{Phys.\ Rev.\ D} \textbf{27} (1983), 2885.
doi:10.1103/PhysRevD.27.2885.

\bibitem{Barbour2009NatureOfTime}
J.\ Barbour,
The Nature of Time,
arXiv:0903.3489 [gr-qc] (2009).

% Optional (if you want an explicit secondary pointer to the Aristotle discussion):
\bibitem{Falcon2006NDPR}
A.\ Falcon,
Review of U.\ Coope, \emph{Time for Aristotle: Physics IV.10--14},
\emph{Notre Dame Philosophical Reviews} (2006).
\url{https://ndpr.nd.edu/reviews/time-for-aristotle/}


\bibitem{Karapetoff1912}
V. Karapetoff, \emph{The Electric Circuit}, McGraw--Hill, 1912.

\bibitem{Brands2021}
H. Brands, Euclidean structure and Lorentz transformations,
\emph{Foundations of Physics}, \textbf{51} (2021), 1--22.

\bibitem{Akintsov2020}
A. Akintsov et al., On Euclidean embeddings of Minkowski space,
\emph{Journal of Geometry and Physics}, \textbf{153} (2020), 103655.

\bibitem{Reddy2009}
V.\,V. Reddy, R. Sharma, S. Sivaramakrishnan,
Construction of Lorentzian metrics from Riemannian metrics,
\emph{Differential Geometry and its Applications},
\textbf{27} (2009), no.\,6, 767--773.

\bibitem{Hestenes1966}
D.\,Hestenes,
\emph{Space-Time Algebra},
Gordon and Breach, New York, 1966.

\bibitem{Hestenes2003}
D.\,Hestenes,
``Spacetime physics with geometric algebra'',
\emph{American Journal of Physics},
\textbf{71} (2003), no.\,6, 691--714.
doi:10.1119/1.1571836.


\bibitem{Rovelli2002}
C.\,Rovelli,
Partial observables,
\emph{Physical Review D},
\textbf{65} (2002), 124013.
doi:10.1103/PhysRevD.65.124013;
arXiv:gr-qc/0110035.

\bibitem{Rovelli1991Time}
C.\ Rovelli,
Time in quantum gravity: An hypothesis,
\emph{Phys.\ Rev.\ D} \textbf{43} (1991) 442--456.
doi:10.1103/PhysRevD.43.442.

\bibitem{Gersten2003ESR}
A.\ Gersten,
Euclidean special relativity,
\emph{Foundations of Physics} \textbf{33} (2003), no.\,8, 1237--1251.
doi:10.1023/A:1025631125442.

\bibitem{Montanus1991AbsoluteE}
J.\,M.\,C.\ Montanus,
Special relativity in an absolute Euclidean space--time,
\emph{Physics Essays} \textbf{4} (1991), no.\,3, 350--356.

\bibitem{Montanus2001ProperTime}
J.\,M.\,C.\ Montanus,
Proper--time formulation of relativistic dynamics,
\emph{Foundations of Physics} \textbf{31} (2001), no.\,9, 1357--1400.
doi:10.1023/A:1012274211780.

\bibitem{Almeida2001AltMinkowski}
J.\,B.\ Almeida,
An alternative to Minkowski space--time,
arXiv:gr-qc/0104029 (2001).

\bibitem{Almeida2002Kcalc}
J.\,B.\ Almeida,
K-calculus in 4-dimensional optics,
arXiv:physics/0201002 (2002).

\bibitem{Atkinson1963EuclidGR}
R.\,d'E.\ Atkinson,
General relativity in Euclidean terms,
\emph{Proc.\ Roy.\ Soc.\ A} \textbf{272} (1963), 60--78.
doi:10.1098/rspa.1963.0042.

\bibitem{GullLasenbyDoran1993}
S.\ Gull, A.\ Lasenby, C.\ Doran,
Imaginary numbers are not real---the geometric algebra of spacetime,
\emph{Foundations of Physics} \textbf{23} (1993), no.\,9, 1175--1201.
doi:10.1007/BF01883676.

\end{thebibliography}

\end{document}