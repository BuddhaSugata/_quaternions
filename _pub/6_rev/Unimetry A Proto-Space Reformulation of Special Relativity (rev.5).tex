\documentclass[11pt]{article}

\usepackage[a4paper,margin=1in]{geometry}
\usepackage{microtype}
\usepackage{comment}
\usepackage{amsmath,amssymb,amsfonts,amsthm}
\usepackage{hyperref}
\usepackage{float}
\usepackage{placeins}
\usepackage{booktabs} 
\usepackage{array,booktabs,tabularx}
\usepackage{tikz}
\usepackage{physics}
\usepackage{tensor}
\usepackage{bm}
\usepackage{geometry}
\geometry{margin=1in}

% --- tables, floats, barriers ---
\usepackage{booktabs}     % for \toprule \midrule \bottomrule
\usepackage{array,tabularx}
\newcolumntype{L}{>{\raggedright\arraybackslash}X}
\usepackage{placeins}     % for \FloatBarrier
\usetikzlibrary{arrows.meta, calc, angles, quotes, decorations.pathreplacing}

% --- TikZ libraries used in figures ---
\usetikzlibrary{calc,arrows.meta,positioning,decorations.pathreplacing}

% --- optional but useful cross-refs ---
\usepackage[nameinlink]{cleveref}

% --- notation helpers ---
\newcommand{\bell}{\boldsymbol\ell} % spatial projection vector
\newcommand{\clight}{c}             % speed of light macro
\newcommand{\Espace}{\mathcal{E}}
% \newcommand{\dd}{\mathrm{d}}
% \newcommand{\sech}{\operatorname{sech}}
\newcommand{\const}{\mathrm{const}}
\newcommand{\Vol}{\operatorname{Vol}}
\newcommand{\Hopf}{\mathcal H}
\DeclareMathOperator{\Ad}{Ad}


\usepackage{indentfirst}
\usetikzlibrary{arrows.meta,calc,angles,quotes}

% ===== PREAMBLE ADD-ON (put in the preamble) ==========================
% A lightweight, breakable tech-box environment
\usepackage{enumitem}
\usepackage[most]{tcolorbox} % loads needed TikZ libs internally
\tcbset{
  colback=gray!3, colframe=black!50, boxrule=0.35pt, arc=2mm,
  left=6pt,right=6pt,top=6pt,bottom=6pt, enhanced, breakable,
  fonttitle=\bfseries
}
\newtcolorbox{techbox}[2][]{title={#2},#1}
% ======================================================================


\theoremstyle{plain}
\newtheorem{theorem}{Theorem}[section]
\newtheorem{lemma}[theorem]{Lemma}
\newtheorem{proposition}[theorem]{Proposition}
\newtheorem{corollary}[theorem]{Corollary}

\theoremstyle{remark}
\newtheorem{remark}[theorem]{Remark}

\theoremstyle{definition}
\newtheorem{definition}[theorem]{Definition}


\numberwithin{equation}{section}

\title{Unimetry: Proto-Space Reformulation of Special Relativity}
\author{Timur Abizgeldin\\ \small Independent researcher, Austria\\ \small \texttt{timurabizgeldin@gmail.com}}
\date{\today}


% --- Unimetry macros ---
\usepackage{tikz}
\usetikzlibrary{arrows.meta,positioning,calc}
\providecommand{\bi}{\mathbf{i}}
\providecommand{\bj}{\mathbf{j}}
\providecommand{\bk}{\mathbf{k}}
\providecommand{\uhat}{\hat{\mathbf u}}
\providecommand{\rotor}[2]{\cos\frac{#2}{2} + #1\,\sin\frac{#2}{2}}

\begin{document}
\maketitle
\begin{abstract}
We present a geometric unification of special relativity and field theory based on a four-dimensional Euclidean proto-space $(\Espace, \delta)$ equipped with a distinguished unit vector field $N$. By defining the metric $g_{AB} = 2N_A N_B - \delta_{AB}$, we recover the standard Minkowski signature and causal structure. A key feature of this formalism is the identity $\tanh \eta = \tan \vartheta$, which compactifies the infinite hyperbolic rapidity range into a finite Euclidean tilt domain $\vartheta \in [0, \pi/4)$. We demonstrate that relativistic proper-time normalization is equivalent to a constant Euclidean flow on a sphere $S^3_{\clight} \subset T\Espace$.

Furthermore, we extend this geometric framework to the algebraic and topological domains. Using the Euclidean Clifford algebra $\mathcal{C}\ell_{4,0}$, we show that Lorentzian boosts emerge naturally from Euclidean rotations via a change in the bivector basis. Finally, we propose that matter sources can be modeled as topological defects of the field $N$ governed by the Hopf fibration $S^3 \to S^2$. This perspective interprets particle charge as a topological invariant and suggests that quantum uncertainty arises from the dimensional reduction of flow fibers in the proto-space.
\end{abstract}


\paragraph{Keywords:} special relativity; phase; rapidity; Doppler shift; aberration; Lorentz factor; phase parametrization.

\paragraph{MSC (2020):} 83A05; 70A05. % (Consider also PACS: 03.30.+p; 04.20.Cv.)

\tableofcontents

% ===============================
\section{Introduction}
\label{sec:Intro}
% ===============================

\subsection{Motivation}
The transition from the definite signature of Euclidean geometry to the indefinite signature of Lorentzian spacetime has traditionally been treated as a fundamental postulate of physics. However, the rigidity and well-behaved spectral properties of elliptic operators suggest that the Euclidean signature may be more fundamental. This paper proposes \emph{Unimetry}—a framework where the physical Lorentzian structure is not axiomatic but derived from a ``proto-space'': a four-dimensional Euclidean manifold $(\Espace,\delta)$ permeated by a distinguished flow field $N$.

Our primary motivation is to unify the kinematic rigidity of special relativity with the topological richness of field theories. By treating the speed of light not merely as a limit but as a geometric saturation point of the Euclidean tilt angle, we obtain a phase-space formulation where relativistic effects (time dilation, Doppler shift) are essentially projection artifacts. Moreover, this Euclidean vantage point allows us to import powerful tools from topology and geometric algebra, offering new insights into the nature of boosts, charges, and the origin of quantum indeterminacy.

\subsection{Relation to previous work}

Attempts to reinterpret Lorentzian geometry through Euclidean structures have a long history. Early geometric perspectives go back to the work of Karapetoff \cite{Karapetoff1912}, who proposed Euclidean angle constructions for visualizing relativistic transformations. More recent studies by Brands \cite{Brands2021}, Akintsov et al. \cite{Akintsov2020} investigate embeddings and correspondences between the two geometries.

A rigorous pointwise correspondence was established by Reddy, Sharma and Sivaramakrishnan \cite{Reddy2009}. Given a Riemannian manifold $(M,h)$ and a unit vector field~$U$, they introduce a Lorentzian metric via $g = h - 2\, U^\flat \!\otimes U^\flat$. We adopt a sign-flipped variant of this projector method to generate the $(+---)$ signature used in particle physics.

Algebraically, our work draws inspiration from the \emph{Space-Time Algebra} (STA) pioneered by Hestenes \cite{Hestenes1966, Hestenes2003}. While STA traditionally operates within a Lorentzian Clifford algebra, we demonstrate that the physics can be fully described within a strictly Euclidean algebra $\mathcal{C}\ell_{4,0}$, where the "space-time split" is dynamic and observer-dependent. Additionally, the interpretation of particles as geometric flows resonates with Hestenes' \emph{Zitterbewegung} model \cite{Hestenes2008}, which we reframe here in terms of topological fibrations.

\subsection{Outline}
The paper is organized as follows:
\begin{itemize}
    \item \textbf{Section \ref{sec:LMC}} constructs the Lorentzian metric $g$ from the Euclidean background $(\delta, N)$ and establishes the causal signature.
    \item \textbf{Section \ref{sec:LMP}} explores the properties of the induced metric, including the orthogonal decomposition of tangent vectors and the geometry of the null cone.
    \item \textbf{Section \ref{sec:Hyper}} derives the central kinematic identity $\tanh\eta = \tan\vartheta$ and shows that the speed of light corresponds to a $45^\circ$ Euclidean tilt.
    \item \textbf{Section \ref{sec:flow-invariants-S3}} reformulates 4-velocity normalization as a constraint on the Euclidean flow, mapping admissible states to the sphere $S^3$.
    \item \textbf{Section \ref{sec:why-protospace}} demonstrates the explanatory power of the framework. We derive optical effects (Doppler, aberration) as geometric projections, reformulate boosts using Euclidean Geometric Algebra, and conclude with a topological outlook where sources are modeled as defects characterized by the Hopf invariant.
\end{itemize}

% ===============================
\section{Lorentzian metric construction}
\label{sec:LMC}
% ===============================

\subsection{Euclidean proto-space}
\label{subsec:epspace}
We work on a four--dimensional Euclidean manifold $(\Espace,\delta)$,
equipped with the flat metric
\[
   \delta_{AB} = \mathrm{diag}(1,1,1,1).
\]
Throughout, indices are raised and lowered with $\delta$:
\[
   X_A := \delta_{AB}X^B,\qquad
   X^A := \delta^{AB}X_B,
\]
and we use the $\delta$--inner product notation
\[
   X\cdot Y := \delta(X,Y) = \delta_{AB}X^A Y^B.
\]

\begin{remark}[Index conventions: $\delta$ vs.\ $g$]
\label{rem:index-conventions}
Throughout, $\delta$ is treated as the \emph{background} Euclidean metric on
$\Espace$, and we use $\delta$ to raise and lower abstract indices unless
explicitly stated otherwise.
The Lorentzian tensor $g_{AB}$ introduced in \S\ref{subsec:LMD} is regarded
primarily as a derived bilinear form on $T\Espace$ (used to define
interval--type scalars such as $g(X,X)$), and not as the default device for
index gymnastics.

In particular, we distinguish the $\delta$--raised components
\[
   g^{AB}_{(\delta)} := \delta^{AC}\delta^{BD} g_{CD}
\]
from the inverse metric $(g^{-1})^{AB}$ defined by
$(g^{-1})^{AC} g_{CB} = \delta^{A}{}_{B}$.
For the special form $g_{AB}=2N_A N_B-\delta_{AB}$ with $\delta(N,N)=1$, one
indeed has $(g^{-1})^{AB}=g^{AB}_{(\delta)}=2N^{A}N^{B}-\delta^{AB}$, but the
two notions remain conceptually distinct.
\end{remark}

\subsection{Distinguished unit vector field}
\label{subsec:distN}
Let $N$ be a smooth vector field on $\Espace$ satisfying the unit condition
\[
   \delta(N,N)=1.
\]
In particular, $N$ is nowhere vanishing and defines at each point a
distinguished $\delta$--unit direction.

\begin{remark}[Fiducial observer field]
\label{rem:N-reference-observer}
In the unimetrical interpretation, the distinguished unit field $N$ plays the role of a
fiducial inertial observer field (a ``laboratory'' or vacuum rest frame).
It selects, at each $p\in\Espace$, the canonical $\delta$--orthogonal splitting
\[
   T_p\Espace = \mathbb R\,N_p \ \oplus\ N_p^{\perp_\delta},
\]
which we use consistently to interpret longitudinal components as temporal and
transverse components as spatial.
When light rays are represented by null tangents $K$ (see \S\ref{subsec:light-as-null}),
the coefficient $\omega$ in the parametrization $K=\omega(N+E)$ is precisely the angular
frequency measured by these $N$--observers, i.e.\ $\omega=g(N,K)$; Doppler shifts for other
observers are encoded by $\omega_U=g(U,K)$ (Definition~\ref{def:omegaU}).
\end{remark}

We introduce the $\delta$--orthogonal projector onto the complement of $N$:
\begin{equation}\label{eq:h-def}
   h_{AB} := \delta_{AB} - N_A N_B.
\end{equation}
Then $h$ has rank $3$ and satisfies
\[
   h_{AB}N^B=0,\qquad
   h_{A}{}^{C}h_{CB}=h_{AB}.
\]
We write $\mathrm{Im}(h_p)=N_p^{\perp_\delta}\subset T_p\Espace$ for the
$\delta$--orthogonal complement at $p$.

\subsection{Lorentzian metric definition}
\label{subsec:LMD}
Define a symmetric $(0,2)$--tensor field $g$ on $\Espace$ by
\begin{equation}\label{eq:g-def}
   g_{AB} := 2N_A N_B - \delta_{AB}.
\end{equation}
Equivalently, using \eqref{eq:h-def},
\begin{equation}\label{eq:g-def-proj}
   g_{AB} = N_A N_B - h_{AB}.
\end{equation}

\begin{itemize}
  \item[(i)] $N$ is $g$--unit:
  \[
     g(N,N)=2(\delta(N,N))^2-\delta(N,N)=2\cdot 1-1=1.
  \]

  \item[(ii)] $N$ is $g$--orthogonal to $N^{\perp_\delta}$:
  if $X\in N^{\perp_\delta}$, i.e. $N\cdot X=0$, then
  \[
     g(N,X)=2(N\cdot N)(N\cdot X)-\delta(N,X)=0.
  \]

  \item[(iii)] On $N^{\perp_\delta}$ one has $g=-\delta$:
  if $X,Y\in N^{\perp_\delta}$, then
  \[
     g(X,Y)=2(N\cdot X)(N\cdot Y)-\delta(X,Y)=-\delta(X,Y).
  \]
\end{itemize}

Hence, at each $p\in\Espace$,
\[
   T_p\Espace=\mathrm{span}\{N_p\}\oplus N_p^{\perp_\delta},
\]
and in an adapted $\delta$--orthonormal basis $\{e_0=N,e_1,e_2,e_3\}$ the
bilinear form $g_p$ has the Minkowski block form
\[
   g_p = (+1)\oplus(-1)\oplus(-1)\oplus(-1).
\]
In particular, $g$ has Lorentzian signature $(+---)$.


% ===============================
\section{Lorentzian metric properties}
\label{sec:LMP}
% ===============================

Throughout this section, $p\in\Espace$ is arbitrary and all statements are
understood pointwise at $p$.

\subsection{Orthogonal decomposition of tangent vectors}
\label{subsec:orth-decomp}

For any $X\in T_p\Espace$ we define the $\delta$--longitudinal and
$\delta$--transverse components relative to $N$ by
\begin{equation}\label{eq:decomp}
   X_\parallel := (N\cdot X)\,N,\qquad
   X_\perp := h(X)=X-(N\cdot X)\,N.
\end{equation}

\begin{lemma}\label{lem:decomp-unique}
For every $X\in T_p\Espace$,
\[
   X=X_\parallel+X_\perp,
\]
where $X_\parallel\in\mathrm{span}\{N_p\}$ and $X_\perp\in N_p^{\perp_\delta}$.
The decomposition is unique.
\end{lemma}

\begin{proof}
Since $h_p$ is a projector with $\ker(h_p)=\mathrm{span}\{N_p\}$ and
$\mathrm{Im}(h_p)=N_p^{\perp_\delta}$, the splitting is the standard
direct sum decomposition associated with complementary subspaces.
\end{proof}

\subsection{Norm identities and classification of vectors}
\label{subsec:norms}

\begin{proposition}\label{prop:g-norm-split}
For any $X\in T_p\Espace$,
\[
   g(X,X)=(N\cdot X)^2-\delta(X_\perp,X_\perp).
\]
\end{proposition}

\begin{proof}
Insert \eqref{eq:decomp} into $g(X,X)$ and use:
$g(N,N)=1$, $g(N,X_\perp)=0$ (since $X_\perp\in N^{\perp_\delta}$), and
$g(X_\perp,X_\perp)=-\delta(X_\perp,X_\perp)$ from (iii).
\end{proof}

\begin{corollary}\label{cor:causal-class}
A vector $X$ satisfies:
\begin{itemize}
  \item $g(X,X)>0$ iff $(N\cdot X)^2>\delta(X_\perp,X_\perp)$,
  \item $g(X,X)=0$ iff $(N\cdot X)^2=\delta(X_\perp,X_\perp)$,
  \item $g(X,X)<0$ iff $(N\cdot X)^2<\delta(X_\perp,X_\perp)$.
\end{itemize}
\end{corollary}

Define the three disjoint subsets of $T_p\Espace$:
\[
  \mathcal T_p := \{X\in T_p\Espace:\ g(X,X)>0\},\qquad
  \mathcal P_p := \{X\in T_p\Espace:\ g(X,X)=0\},\qquad
  \mathcal S_p := \{X\in T_p\Espace:\ g(X,X)<0\}.
\]
We also single out the \emph{future} time cone (relative to $N$):
\begin{equation}\label{eq:Tp-plus}
  \mathcal T_p^{+} := \{X\in \mathcal T_p:\ N\cdot X>0\}.
\end{equation}

\subsection{Geometry of the null cone}
\label{subsec:nullcone}

\begin{proposition}\label{prop:nullcone}
The set of $g$--null vectors at $p$ is the quadratic cone
\[
   \mathcal{C}_p
   = \{\,X\in T_p\Espace:\ \delta(X_\perp,X_\perp)=(N\cdot X)^2\,\}.
\]
Under the decomposition $T_p\Espace=\mathrm{span}\{N_p\}\oplus N_p^{\perp_\delta}$,
it is a double cone given by
\[
   N\cdot X = \pm \|X_\perp\|_\delta.
\]
\end{proposition}

\begin{proof}
Immediate from Corollary~\ref{cor:causal-class}.
\end{proof}

\subsection{Spatial rotations preserving $\delta$ and $N$}
\label{subsec:spatial-rotations}

Let $\mathrm{Aut}(\delta,N)$ denote the stabilizer of $N$ in the Euclidean
orthogonal group:
\[
   \mathrm{Aut}(\delta,N)
   := \{\,L:T_p\Espace\to T_p\Espace \text{ linear}:\ \delta(LX,LY)=\delta(X,Y),\ LN=N\,\}.
\]
In an adapted $\delta$--orthonormal basis $\{e_0=N,e_1,e_2,e_3\}$ one has
\[
   L=\mathrm{diag}(1,R),\qquad R\in O(3),
\]
so $\mathrm{Aut}(\delta,N)\cong O(3)$ and contains no boost--like maps mixing
$N$ with $N^{\perp_\delta}$.

\begin{lemma}\label{lem:aut-preserves-g}
Every $L\in\mathrm{Aut}(\delta,N)$ preserves $g$:
\[
   g(LX,LY)=g(X,Y)\qquad \text{for all } X,Y\in T_p\Espace.
\]
\end{lemma}

\begin{proof}
Since $LN=N$ and $L$ is $\delta$--orthogonal,
\[
   g(LX,LY)
   =2(N\cdot LX)(N\cdot LY)-\delta(LX,LY)
   =2(N\cdot X)(N\cdot Y)-\delta(X,Y)
   =g(X,Y).
\]
\end{proof}

Thus $\mathrm{Aut}(\delta,N)$ is a spatial subgroup of $\mathrm{O}(g)$: it
preserves $g$ and fixes $N$, but generates only Euclidean rotations on
$N^{\perp_\delta}$.


% ===============================
\section{Tilt angle geometry and hyperbolic parametrization}
\label{sec:Hyper}
% ===============================

\subsection{Euclidean tilt angle}
\label{subsec:tilt}

For any nonzero $X\in T_p\Espace$ define its Euclidean tilt angle
$\vartheta\in[0,\pi]$ by
\[
   \cos\vartheta := \frac{N\cdot X}{\|X\|_\delta},
   \qquad
   \sin\vartheta := \frac{\|X_\perp\|_\delta}{\|X\|_\delta},
\]
where $X_\perp$ is defined by \eqref{eq:decomp}.  Whenever $X_\perp\neq 0$,
define the $\delta$--unit transverse direction
\[
   E := \frac{X_\perp}{\|X_\perp\|_\delta}\in N_p^{\perp_\delta}.
\]
Then
\begin{equation}\label{eq:X-tilt-decomp}
   X = \|X\|_\delta\cos\vartheta\,N + \|X\|_\delta\sin\vartheta\,E.
\end{equation}

\begin{lemma}\label{lem:tilt-id}
For any nonzero $X\in T_p\Espace$,
\[
   \|X\|_\delta^2=(N\cdot X)^2+\delta(X_\perp,X_\perp),
   \qquad
   \cos^2\vartheta+\sin^2\vartheta=1.
\]
\end{lemma}

\begin{proof}
Immediate from $\delta(X_\parallel,X_\perp)=0$.
\end{proof}

\subsection{Lorentzian norm expressed via $\vartheta$}
\label{subsec:g-via-tilt}

From Proposition~\ref{prop:g-norm-split},
\[
   g(X,X)=(N\cdot X)^2-\delta(X_\perp,X_\perp)
         =\|X\|_\delta^2(\cos^2\vartheta-\sin^2\vartheta)
         =\|X\|_\delta^2\cos(2\vartheta).
\]

\begin{proposition}\label{prop:g-tilt}
For any nonzero $X\in T_p\Espace$,
\[
   g(X,X)=\|X\|_\delta^2\cos(2\vartheta).
\]
\end{proposition}

\subsection{Domain of hyperbolic parametrization}
\label{subsec:domain-eta}

A real hyperbolic parameter is naturally attached to vectors in the
future time cone $\mathcal T_p^+$ defined in \eqref{eq:Tp-plus}.
For $X\in\mathcal T_p^+$ one has
\[
   g(X,X)>0
   \quad\Longleftrightarrow\quad
   \cos(2\vartheta)>0
   \quad\Longleftrightarrow\quad
   \vartheta\in\Bigl[0,\frac{\pi}{4}\Bigr),
\]
and moreover $N\cdot X>0$ implies $\cos\vartheta>0$, so
\[
   \beta := \tan\vartheta \in [0,1).
\]
Null vectors satisfy $\tan\vartheta=1$ (equivalently $\vartheta=\pi/4$),
while $g$--negative vectors have $\tan\vartheta>1$.

\subsection{Hyperbolic parameter (rapidity)}
\label{subsec:eta-def}

For $X\in\mathcal T_p^+$ define $\eta\ge 0$ by
\begin{equation}\label{eq:def-eta}
   \tanh\eta := \tan\vartheta.
\end{equation}
Equivalently, one may define $\eta$ invariantly by
\begin{equation}\label{eq:eta-invariant}
   \cosh\eta := \frac{N\cdot X}{\sqrt{g(X,X)}},\qquad
   \sinh\eta := \frac{\|X_\perp\|_\delta}{\sqrt{g(X,X)}},
   \qquad (X\in\mathcal T_p^+),
\end{equation}
which immediately implies $\tanh\eta=\|X_\perp\|_\delta/(N\cdot X)=\tan\vartheta$.

\begin{lemma}\label{lem:cosh-sinh-tilt}
For $X\in\mathcal T_p^+$,
\[
   \cosh\eta=\frac{\cos\vartheta}{\sqrt{\cos(2\vartheta)}},
   \qquad
   \sinh\eta=\frac{\sin\vartheta}{\sqrt{\cos(2\vartheta)}}.
\]
\end{lemma}

\begin{proof}
From \eqref{eq:def-eta}, $\tanh\eta=\tan\vartheta$ gives
\[
   \cosh^2\eta=\frac{1}{1-\tanh^2\eta}
             =\frac{1}{1-\tan^2\vartheta}
             =\frac{\cos^2\vartheta}{\cos^2\vartheta-\sin^2\vartheta}
             =\frac{\cos^2\vartheta}{\cos(2\vartheta)}.
\]
Taking the positive square root (since $\eta\ge 0$ and $\vartheta\in[0,\pi/4)$)
yields the expression for $\cosh\eta$, and multiplying by $\tanh\eta=\tan\vartheta$
yields $\sinh\eta$.
\end{proof}

\subsection{Differential relation between $\eta$ and $\vartheta$}
\label{subsec:deta}

\begin{proposition}\label{prop:deta}
For $X\in\mathcal T_p^+$,
\[
   \frac{d\eta}{d\vartheta}=\frac{1}{\cos(2\vartheta)}.
\]
\end{proposition}

\begin{proof}
Differentiate $\tanh\eta=\tan\vartheta$:
\[
   \sech^2\eta\, d\eta = \sec^2\vartheta\, d\vartheta.
\]
Using $\sech^2\eta=1-\tanh^2\eta=1-\tan^2\vartheta=\dfrac{\cos(2\vartheta)}{\cos^2\vartheta}$,
we obtain
\[
   \frac{d\eta}{d\vartheta}
   =\frac{\sec^2\vartheta}{\sech^2\eta}
   =\frac{1/\cos^2\vartheta}{\cos(2\vartheta)/\cos^2\vartheta}
   =\frac{1}{\cos(2\vartheta)}.
\]
\end{proof}

\subsection{Boost subgroup and additivity of the hyperbolic parameter}
\label{subsec:boost-additivity}

Let $\mathrm{O}(g)$ denote the Lorentz group of $(T_p\Espace,g)$:
\[
   \mathrm{O}(g)
   := \{\,\Lambda:T_p\Espace\to T_p\Espace\ \text{linear}:\ g(\Lambda X,\Lambda Y)=g(X,Y)\,\}.
\]
Fix a $\delta$--unit transverse direction $E\in N_p^{\perp_\delta}$ with $\delta(E,E)=1$.
The \emph{boost} in the $2$--plane $\mathrm{span}\{N,E\}$ with parameter $\eta$
is the unique $\Lambda(\eta)\in\mathrm{O}(g)$ acting as a hyperbolic rotation
on $\mathrm{span}\{N,E\}$ and as the identity on its $g$--orthogonal complement:
\[
   \Lambda(\eta)N = (\cosh\eta)\,N + (\sinh\eta)\,E,\qquad
   \Lambda(\eta)E = (\sinh\eta)\,N + (\cosh\eta)\,E,
\]
\[
   \Lambda(\eta)X=X
   \quad \text{for } X\perp_g \mathrm{span}\{N,E\}.
\]
Such boosts preserve $g$ but, in general, do not preserve $\delta$ and do not fix $N$.

\begin{theorem}[Additivity]\label{thm:additivity}
For boosts $\Lambda(\eta_1)$ and $\Lambda(\eta_2)$ in the same $(N,E)$--plane,
their composition is a boost with parameter $\eta_1+\eta_2$:
\[
   \Lambda(\eta_1)\circ \Lambda(\eta_2)=\Lambda(\eta_1+\eta_2).
\]
\end{theorem}

\begin{proof}
On $\mathrm{span}\{N,E\}$ the boosts are represented (in the basis $\{N,E\}$)
by the matrices
\[
\begin{pmatrix}
\cosh\eta & \sinh\eta\\
\sinh\eta & \cosh\eta
\end{pmatrix},
\]
whose multiplication adds rapidities. On the $g$--orthogonal complement the
action is the identity, hence the statement holds on all of $T_p\Espace$.
\end{proof}

\subsection{Comparison with classical angle conventions}
\label{subsec:compare-angles}

The Euclidean angle $\vartheta$ above coincides with the geometric tilt angle
used in classical constructions (e.g.\ Karapetoff) and in later reformulations;
the difference lies in which trigonometric function is taken as the primary
dimensionless parameter.

A common choice is
\[
   \beta_{\mathrm{std}}:=\sin\vartheta=\frac{\|X_\perp\|_\delta}{\|X\|_\delta},
\]
whereas in the present work we use the tangent--based parameter
\[
   \beta_{\mathrm{phase}}:=\tan\vartheta=\frac{\|X_\perp\|_\delta}{N\cdot X}.
\]
On $\mathcal T_p^+$ one has $\beta_{\mathrm{phase}}\in[0,1)$, and the rapidity
$\eta$ is introduced directly by \eqref{eq:def-eta}.

\begin{remark}[Photon limit and the $45^\circ$ Euclidean tilt]
\label{rem:photon-limit-45deg}
The null cone is characterized by $g(X,X)=0$, equivalently $\cos(2\vartheta)=0$,
so that the lightlike limit corresponds to
\(
\vartheta\to \pi/4
\)
in the Euclidean picture.
In this limit one has
\[
   \beta_{\mathrm{phase}}=\tan\vartheta \to 1,
   \qquad
   \beta_{\mathrm{std}}=\sin\vartheta \to \frac{1}{\sqrt{2}}.
\]
\textbf{Thus a light ray is reached at a finite Euclidean tilt of $45^\circ$ relative
to $N$ (not at $90^\circ$).}
This is precisely why the tangent parameterization is better adapted to the
projector identity $g(X,X)=\|X\|_\delta^2\cos(2\vartheta)$: it saturates at the
speed--of--light barrier as $\vartheta\to\pi/4$, while the sine parameter does not.
\end{remark}

This choice is adapted to
the orthogonal splitting \eqref{eq:decomp} and to the identity of
Proposition~\ref{prop:g-tilt},
\[
   g(X,X)=\|X\|_\delta^2\cos(2\vartheta),
\]
so that the domain of the hyperbolic parametrization is exactly the future
$g$--time cone $\mathcal T_p^+$, without additional postulates.



% ===============================
\section{Flow invariants and the emergence of $S^3$ in the Euclidean proto-space}
\label{sec:flow-invariants-S3}
% ===============================

This section makes precise a key equivalence of the proto--space approach:
invariance of the Minkowski interval in an observer's local time
is equivalent to constancy of the full proto--space flow vector
with respect to a calibrated proto--parameter.


\subsection{Worldlines, proto--parameter, and the full flow vector}
\label{subsec:full-flow-vector}

Let $X:I\to\Espace$ be a smooth timelike worldline.
A \emph{proto--parameter} $\chi$ along $X$ is any smooth parameter with
nowhere--vanishing derivative.  We define the associated \emph{full flow vector}
(the proto--space tangent) by
\begin{equation}\label{eq:flow-vector-tildeX}
   \tilde{X} \;:=\; \frac{dX}{d\chi}\ \in T_{X(\chi)}\Espace.
\end{equation}

 \begin{definition}[Calibrated proto--parameter]
 \label{def:calibrated-chi}
 A proto--parameter $\chi$ is called \emph{calibrated} (with scale $\clight$) if
 \begin{equation}\label{eq:calibrated-speed}
    \delta(\tilde{X},\tilde{X}) \;=\; \clight^2
    \qquad\text{along }X.
 \end{equation}
 \end{definition}

\begin{remark}[Existence and gauge nature]
\label{rem:chi-existence}
For any timelike worldline $X$, a calibrated proto--parameter always exists.
Indeed, if $\tau$ denotes proper time, one may define $\chi$ by
$d\chi/d\tau=\|\dot X\|_\delta/\clight$.
Fixing the time orientation ($d\chi/d\tau>0$), the resulting calibrated
parameter is unique up to an additive constant.
\end{remark}

In words: in a calibrated proto--parameter, the full flow vector $\tilde{X}$
has a fixed Euclidean norm in the proto--space.  This is the proto--space
counterpart of the standard SR statement that the 4--velocity has fixed
Minkowski norm in proper time.

\subsection{Observer splitting and the interval--rate identity}
\label{subsec:interval-rate}

Fix the distinguished unit field $N$ (hence $g$ and $h$) as in
\S\ref{sec:LMC}--\S\ref{sec:LMP}.  Pointwise along $X$, decompose $\tilde{X}$
into $\delta$--longitudinal and $\delta$--transverse parts relative to $N$:
\begin{equation}\label{eq:tildeX-decomp}
   \tilde{X}
   \;=\;
   \tilde{H}\,N + \tilde{X}_\perp,
   \qquad
   \tilde{H} := N\cdot \tilde{X},
   \qquad
   \tilde{X}_\perp := h(\tilde{X})\in \mathrm{Im}(h).
\end{equation}
Let $\tilde{L}:=\|\tilde{X}_\perp\|_\delta$ and, when $\tilde{L}\neq 0$,
$E:=\tilde{X}_\perp/\tilde{L}\in\mathrm{Im}(h)$ so that
$\tilde{X}=\tilde{H}N+\tilde{L}E$ with $\delta(E,E)=1$.

\begin{lemma}[Euclidean and Lorentzian norms of the flow]
\label{lem:euclid-vs-lorentz}
Along $X$ one has the identities
\begin{equation}\label{eq:norm-identities-flow}
   \delta(\tilde{X},\tilde{X}) = \tilde{H}^{\,2}+\tilde{L}^{\,2},
   \qquad
   g(\tilde{X},\tilde{X}) = \tilde{H}^{\,2}-\tilde{L}^{\,2}.
\end{equation}
\end{lemma}
\begin{proof}
Since $\tilde{X}_\perp\in N^{\perp_\delta}$, we have $\delta(N,\tilde{X}_\perp)=0$.
Hence
$\delta(\tilde{X},\tilde{X})=\tilde{H}^2+\delta(\tilde{X}_\perp,\tilde{X}_\perp)
=\tilde{H}^2+\tilde{L}^2$.
For $g$, use $g(N,N)=1$, $g(N,\tilde{X}_\perp)=0$, and
$g(\tilde{X}_\perp,\tilde{X}_\perp)=-\delta(\tilde{X}_\perp,\tilde{X}_\perp)
=-\tilde{L}^2$.
\end{proof}

Motivated by the phase--formalism viewpoint, we introduce the \emph{interval rate}
with respect to $\chi$:
\begin{equation}\label{eq:S-tilde-def}
   \tilde{S}^{\,2}
   \;:=\;
   g(\tilde{X},\tilde{X})
   \;=\;
   \tilde{H}^{\,2}-\tilde{L}^{\,2}.
\end{equation}
This is the precise proto--space analogue of writing
$ds^2 = g(dX,dX) = \tilde{S}^{\,2}\,d\chi^2$ and viewing $\tilde{S}$ as a
``Minkowski projection'' of the full Euclidean flow.

\subsection{Equivalence: invariant interval in local time $\Longleftrightarrow$
calibrated full flow}
\label{subsec:equivalence-local-time}

Let $\tau$ denote the \emph{local time} (proper time) along the timelike curve
$X$, i.e.\ a parameter such that the tangent $\dot{X}:=dX/d\tau$ satisfies
\begin{equation}\label{eq:proper-time-norm}
   g(\dot{X},\dot{X}) = \clight^2.
\end{equation}
Equivalently, $ds^2 = g(dX,dX) = \clight^2\,d\tau^2$ along $X$.

\begin{theorem}[Reparameterization equivalence]
\label{thm:equiv}
For any timelike worldline $X$, the following statements are equivalent:
\begin{itemize}
\item[(A)] $X$ is parameterized by local time $\tau$ so that
$g(\dot{X},\dot{X})=\clight^2$.
\item[(B)] $X$ is parameterized by a calibrated proto--parameter $\chi$ so that
$\delta(\tilde{X},\tilde{X})=\clight^2$.
\end{itemize}
Moreover, when both parameters are used on the same curve, they are related by
\begin{equation}\label{eq:tau-chi-relation}
   \frac{d\tau}{d\chi}
   \;=\;
   \frac{\sqrt{g(\tilde{X},\tilde{X})}}{\clight}
   \;=\;
   \frac{\tilde{S}}{\clight},
   \qquad
   \frac{d\chi}{d\tau}
   \;=\;
   \frac{\|\dot{X}\|_\delta}{\clight}.
\end{equation}
\end{theorem}

\begin{proof}
Assume (A).  Define $\chi$ (up to an additive constant) by the ODE
\[
   \frac{d\chi}{d\tau} := \frac{\|\dot{X}\|_\delta}{\clight},
\]
which is smooth and positive since $\dot{X}\neq 0$.
Then $\tilde{X}=dX/d\chi = (d\tau/d\chi)\dot{X}$, so
\[
   \delta(\tilde{X},\tilde{X})
   = \Bigl(\frac{d\tau}{d\chi}\Bigr)^2\delta(\dot{X},\dot{X})
   = \frac{\clight^2}{\|\dot{X}\|_\delta^2}\,\|\dot{X}\|_\delta^2
   = \clight^2,
\]
which is (B).

Conversely, assume (B).  Define $\tau$ (up to an additive constant) by
\[
   \frac{d\tau}{d\chi} := \frac{\sqrt{g(\tilde{X},\tilde{X})}}{\clight},
\]
which is well--defined and positive for timelike $\tilde{X}$ since
$g(\tilde{X},\tilde{X})>0$.
Then $\dot{X}=dX/d\tau=(d\chi/d\tau)\tilde{X}$, hence
\[
   g(\dot{X},\dot{X})
   = \Bigl(\frac{d\chi}{d\tau}\Bigr)^2 g(\tilde{X},\tilde{X})
   = \frac{\clight^2}{g(\tilde{X},\tilde{X})}\,g(\tilde{X},\tilde{X})
   = \clight^2,
\]
which is (A).  The relations \eqref{eq:tau-chi-relation} are exactly the two
defining ODEs.
\end{proof}

\paragraph{Operational meaning.}
Statement (A) is the standard SR normalization of the 4--velocity in proper
time, $g(\dot X,\dot X)=\clight^2$.
Statement (B) is the corresponding calibration of the proto--parameter,
$\delta(\tilde X,\tilde X)=\clight^2$.
Theorem~\ref{thm:equiv} shows these are equivalent and amount to a change of
parameter (``norm vs projection''), rather than an additional dynamical
assumption.


\subsection{The $S^3$ of admissible flow states}
\label{subsec:S3-of-flows}

Fix $p\in\Espace$.  The set of all calibrated flow vectors at $p$ is the
Euclidean 3--sphere of radius $\clight$ inside $T_p\Espace$:
\begin{equation}\label{eq:S3-sphere}
   S^3_{\clight}(p)
   \;:=\;
   \bigl\{\,V\in T_p\Espace:\ \delta(V,V)=\clight^2\,\bigr\}
   \;\cong\;
   S^3.
\end{equation}
Thus, once the calibration \eqref{eq:calibrated-speed} is imposed, every
instantaneous \emph{kinematic state} in the proto--space is a point of an
$S^3$ in the tangent space.

Relative to a chosen time direction $N_p$, each $V\in S^3_{\clight}(p)$
admits the decomposition
\[
   V=\tilde{H}\,N+\tilde{L}\,E,
   \qquad
   \tilde{H}^2+\tilde{L}^2=\clight^2,
   \qquad
   E\in \mathrm{Im}(h_p),\ \delta(E,E)=1,
\]
so that the observable ``spatial direction'' is carried by $E\in S^2\subset
\mathrm{Im}(h_p)$ while the pair $(\tilde{H},\tilde{L})$ lies on a circle
$\tilde{H}^2+\tilde{L}^2=\clight^2$.
When $\tilde{L}=0$, the direction $E$ is irrelevant (any unit choice in
$\mathrm{Im}(h_p)$ yields the same vector), and the state reduces to
$V=\pm\clight\,N_p$.
This is precisely the geometric reason why $S^3$ is the natural state
manifold for calibrated flows in the Euclidean proto--space: it is the locus of
constant full flow magnitude, whereas Lorentzian interval effects arise from
the $g$--projection \eqref{eq:S-tilde-def} and the reparameterization
\eqref{eq:tau-chi-relation}.

\begin{remark}[The physical sector of $S^3$]
\label{rem:physical-sector-S3}
While the calibrated flow states form the full sphere
\[
  S^3_{\clight}(p)=\{\,\tilde{X}\in T_p\Espace:\ \delta(\tilde{X},\tilde{X})=\clight^2\,\},
\]
the requirement that a worldline be timelike with respect to $g$,
\[
   g(\tilde{X},\tilde{X})>0,
\]
restricts the physically realizable states to the open subset
\[
   |\tilde{H}|>|\tilde{L}|,
   \qquad
   \tilde{H}:=N\cdot \tilde{X},\ \ \tilde{L}:=\|h(\tilde{X})\|_\delta.
\]
Geometrically, this is the union of two polar caps on $S^3_{\clight}(p)$
centered at $\pm \clight N$.
The boundary $|\tilde{H}|=|\tilde{L}|$ is the null locus (lightlike curves);
equivalently, $g(\tilde{X},\tilde{X})=0$ implies $d\tau/d\chi=0$ in
\eqref{eq:tau-chi-relation}.
\end{remark}

% ===============================
\section{Why the Euclidean proto-space viewpoint}
\label{sec:why-protospace}
% ===============================

The Lorentzian metric $g$ constructed from $(\delta,N)$ endows each tangent
space $(T_p\Espace,g)$ with the usual causal structure.  The additional
advantage of the Euclidean proto--space viewpoint is that it keeps, in the
same object, both (i) the observable \emph{spatial} direction data encoded in
$\mathrm{Im}(h_p)=N_p^{\perp_\delta}$ and (ii) the \emph{calibration} data
(frequencies, time rates) encoded by $g$--projections onto time directions.
In this sense, ``light'' is not an extra entity but a geometric slice of the
null cone by the Euclidean layers $\Sigma_\chi\simeq S^3$ introduced later.

\subsection{Light rays: Derivation of frequency and direction}
\label{subsec:light-as-null}

Fix $p\in\Espace$. A light ray is represented by a nonzero null vector $K\in T_p\Espace$, satisfying $g(K,K)=0$ and future-directed ($N\cdot K > 0$).
Rather than postulating a parametrization, we derive it from the geometry of the $N$-split.

\begin{definition}[Proto-frequency]
The frequency of the ray $K$ measured by the distinguished observer $N$ is defined as the $g$-projection:
\begin{equation}\label{eq:omega-def-proj}
   \omega := g(K,N).
\end{equation}
\end{definition}

Note that due to the structure of the metric \eqref{eq:g-def}, this coincides with the Euclidean projection:
\[
   g(K,N) = 2(N\cdot K)(N\cdot N) - \delta(K,N) = 2(N\cdot K) - (N\cdot K) = N\cdot K.
\]
Thus, $\omega = N\cdot K > 0$.

Now, decompose $K$ into longitudinal and transverse parts relative to $N$:
\[
   K = \omega N + K_\perp, \qquad K_\perp \in N^{\perp_\delta}.
\]
The null condition $g(K,K)=0$ implies:
\[
   0 = g(K,K) = \omega^2 g(N,N) + g(K_\perp, K_\perp) = \omega^2 - \|K_\perp\|_\delta^2.
\]
Hence $\|K_\perp\|_\delta = \omega$. We can therefore write $K_\perp = \omega E$, where $E$ is a unique $\delta$-unit spatial direction ($E \in N^{\perp_\delta}, \|E\|_\delta=1$).

\begin{lemma}[Canonical form]
Every future-directed null vector $K$ admits the unique decomposition:
\begin{equation}\label{eq:nullK-param}
   K = \omega\,(N+E),
\end{equation}
where $\omega = g(K,N)$ is the frequency and $E$ is the observable propagation direction.
\end{lemma}


\subsection{Observers as unit timelike states; measured frequency as a projection}
\label{subsec:observers-as-U}

In the proto--space formalism, a (local) observer is represented by a unit
future timelike vector
\[
   U\in T_p\Espace,\qquad g(U,U)=1,\qquad g(U,N)>0.
\]
All measurable scalars are obtained by taking $g$--contractions.

\begin{definition}[Frequency measured by an observer]
\label{def:omegaU}
For a null ray $K\neq 0$, the frequency measured by the observer $U$ is
\begin{equation}\label{eq:omegaU-def}
   \omega_U := g(U,K).
\end{equation}
\end{definition}

This is the standard invariant definition used in geometric optics in SR.
In the present framework, however, \eqref{eq:omegaU-def} will become an
explicit function of Euclidean angles in $\mathrm{Im}(h_p)$.

\subsection{Doppler shift}
\label{subsec:doppler-3lines}

Let $E_v\in\mathrm{Im}(h_p)$ be a $\delta$--unit direction,
\[
   \delta(E_v,E_v)=1,
\]
and let $U$ be the observer obtained from $N$ by a boost of rapidity $\eta\ge 0$
in the $(N,E_v)$--plane:
\begin{equation}\label{eq:U-boost}
   U := (\cosh\eta)\,N + (\sinh\eta)\,E_v.
\end{equation}
Let the null ray be given by \eqref{eq:nullK-param},
\[
   K=\omega\,(N+E),\qquad \delta(E,E)=1,\ \ E\in\mathrm{Im}(h_p).
\]
Then, using bilinearity and the basic identities
$g(N,N)=1$, $g(N,E)=0$, $g(E_v,N)=0$, and $g(E_v,E)=-\delta(E_v,E)$, we obtain
\begin{align}
   \omega_U
   &= g(U,K) \notag\\
   &= \omega\,g\bigl((\cosh\eta)N+(\sinh\eta)E_v,\ N+E\bigr)\notag\\
   &= \omega\Bigl(\cosh\eta-\sinh\eta\,\delta(E_v,E)\Bigr).
   \label{eq:doppler-raw}
\end{align}

Introduce
\[
   \beta := \tanh\eta,\qquad
   \gamma := \cosh\eta,
\]
and define the Euclidean angle $\psi\in[0,\pi]$ between the velocity axis $E_v$
and the ray direction $E$ in $\mathrm{Im}(h_p)$ by
\begin{equation}\label{eq:psi-def}
   \cos\psi := \delta(E_v,E).
\end{equation}
Then \eqref{eq:doppler-raw} becomes the standard relativistic Doppler law:
\begin{equation}\label{eq:doppler-angle}
   \frac{\omega_U}{\omega}
   = \gamma\bigl(1-\beta\cos\psi\bigr).
\end{equation}

\paragraph{Interpretation.}
Equation \eqref{eq:doppler-angle} is obtained here as a direct function of an
ordinary Euclidean angle on the $\delta$--unit sphere inside
$\mathrm{Im}(h_p)$ (the ``$S^3$ layers'' $\Sigma_\chi\simeq S^3$ will be
introduced later).  In this sense the Doppler shift is simply ``projection
geometry'' in the proto--space: the measured frequency is the $g$--projection
of a null direction onto an observer state.

\subsection{Aberration as projection plus normalization}
\label{subsec:aberration}

The direction of the ray measured by $U$ is the normalized spatial part of $K$
in the $g$--orthogonal complement of $U$.  Define the $g$--spatial component
of $K$ relative to $U$ by
\begin{equation}\label{eq:KU-spatial}
   K_{\perp U} := K - (g(U,K))\,U.
\end{equation}
Then $g(U,K_{\perp U})=0$, so $K_{\perp U}\in U^{\perp_g}$, and its $g$--norm is
\[
   g(K_{\perp U},K_{\perp U}) = -\,\omega_U^2
   \qquad(\text{since } g(K,K)=0,\ g(U,U)=1).
\]
Hence the \emph{unit} spatial direction of the ray in the $U$--frame may be
taken as
\begin{equation}\label{eq:EU-def}
   E_U := \frac{1}{\omega_U}\,K_{\perp U}
   = \frac{1}{g(U,K)}\Bigl(K-(g(U,K))U\Bigr),
   \qquad
   g(E_U,E_U)=-1,\ \ g(U,E_U)=0.
\end{equation}

To extract the usual aberration formula, specialize to the same kinematics as
in \S\ref{subsec:doppler-3lines}, i.e.\ $U$ as in \eqref{eq:U-boost} and
$K=\omega(N+E)$.
Let $\psi'$ denote the angle between the ray and the velocity axis in the
\emph{$U$--rest space}.  Equivalently, $\cos\psi'$ is the (Euclidean) cosine of
the angle between the spatial ray direction measured by $U$ and the axis of
motion, which is encoded invariantly by the contraction of $E_U$ with the
boost axis.

A direct computation (substituting \eqref{eq:U-boost} and $K=\omega(N+E)$ into
\eqref{eq:EU-def} and taking the component along the $U$--spatial image of
$E_v$) yields the standard aberration law:
\begin{equation}\label{eq:aberration}
   \cos\psi'
   = \frac{\cos\psi-\beta}{1-\beta\cos\psi},
\end{equation}
with $\cos\psi=\delta(E_v,E)$ as in \eqref{eq:psi-def}.

\paragraph{Interpretation.}
Aberration and Doppler are the same operation in two steps:
\begin{itemize}
\item Doppler: take the $g$--projection $g(U,K)$ (a scalar).
\item Aberration: subtract the time component $(g(U,K))U$ and normalize the
      remaining $U$--spatial part.
\end{itemize}
In the proto--space picture, both effects are thus immediate consequences of
the null cone geometry together with the observer--dependent splitting induced
by $U$.

\subsection{Geometric Algebra perspective: The Euclidean origin of Lorenz rotation}
\label{subsec:GA}

The Euclidean proto-space framework admits a natural algebraic structure provided by the Clifford algebra $\mathcal{C}\ell(\Espace, \delta)$. Let the geometric product of vectors $u, v \in T_p\Espace$ be defined by the fundamental relation
\[
   uv = \delta(u,v) + u\wedge v.
\]
Unlike the standard Spacetime Algebra which postulates a Lorentzian signature, here the algebra is strictly Euclidean ($\mathcal{C}\ell_{4,0}$). The distinguished field $N$ serves as the generator of the \emph{space-time split}.

\paragraph{Observable algebra.}
Multiplication of any vector $X$ by the distinguished element $N$ decomposes it into a scalar part (time) and a bivector part (space) relative to $N$:
\begin{equation}\label{eq:GA-split}
   XN = \underbrace{X\cdot N}_{\text{scalar}} + \underbrace{X\wedge N}_{\text{bivector}}.
\end{equation}
Identifying the bivectors containing $N$ with the spatial vectors of the observer, the algebra of observables corresponds to the even subalgebra $\mathcal{C}\ell^+_{4,0}$, which is isomorphic to the Pauli algebra (and the quaternions).

The kinematic results of Section~\ref{sec:flow-invariants-S3} find their most natural expression in the language of Geometric Algebra $\mathcal{C}\ell(\Espace, \delta)$, applying the principles of Space-Time Algebra pioneered by Hestenes~\cite{Hestenes1966, Hestenes2003}. Specifically, we utilize the observer-dependent ``space-time split'' technique. This formalism clarifies that the transition from a Euclidean rotation to a Lorentzian boost is not an ad-hoc analytic continuation, but a direct consequence of the metric signature changing the algebraic properties of the basis vectors.

\paragraph{Euclidean rotation.}
In the Euclidean proto-space $(\Espace, \delta)$, both the distinguished vector $N$ and any transverse spatial direction $E$ square to $+1$:
\[
    N^2 = 1, \qquad E^2 = 1.
\]
The plane spanned by them is described by the bivector $\mathbf{I} = N E$. Due to the Euclidean signature, this generator squares to $-1$:
\[
    \mathbf{I}^2 = (NE)(NE) = -N^2 E^2 = -(1)(1) = -1.
\]
Consequently, the exponential of this bivector generates standard trigonometric rotations. The kinematic relationship between two flow vectors $\tilde{X}_1, \tilde{X}_2 \in S^3_{\clight}$ is governed by the rotor $R = e^{-\mathbf{I}\vartheta/2}$:
\[
    \tilde{X}_2 = R \tilde{X}_1 \tilde{R}^{-1} = \tilde{X}_1 \cos\vartheta + \tilde{X}_{1\perp} \sin\vartheta.
\]

\paragraph{Lorentzian boost via vector substitution.}
The construction of the Lorentzian metric $g$ effectively replaces the Euclidean spatial vector $E$ with a physical spatial vector $\mathbf{e}$ which, while parallel to $E$, squares to $-1$ in the metric $g$:
\[
    \mathbf{e}^2 \overset{g}{=} -1, \qquad \text{while } N^2 \overset{g}{=} 1.
\]
This change in the basis vector fundamentally alters the bivector describing the time-space plane. The new generator (the boost bivector) $\mathbf{K} = N \mathbf{e}$ satisfies:
\[
    \mathbf{K}^2 = (N\mathbf{e})(N\mathbf{e}) = -N^2 \mathbf{e}^2 = -(1)(-1) = +1.
\]
Because the generator now squares to $+1$, its exponential series produces hyperbolic functions instead of trigonometric ones. Using the rapidity $\eta$ (related to tilt by $\tanh\eta = \tan\vartheta$), the physical transformation becomes a boost $L = e^{-\mathbf{K}\eta/2}$:
\[
    L = \cosh\frac{\eta}{2} - \mathbf{K}\sinh\frac{\eta}{2}.
\]
Applying this to the rest vector $N$ yields the 4-velocity $U$:
\[
    U = L N L^{-1} = N \cosh\eta + \mathbf{e} \sinh\eta.
\]
Thus, the phenomenon of "Lorentzian" kinematics emerges algebraically simply because the spatial part of the basis becomes "imaginary" (squares to $-1$) relative to the time direction, flipping the sign of the bivector square and transitioning the geometry from circular to hyperbolic.

\subsection{Topological defects and the Hopf structure of sources}
\label{subsec:topology-hopf}

The construction so far has assumed the vector field $N$ to be smooth and non-vanishing everywhere. A natural completion of the proto-space geometry arises by considering the singularities of this field, which allows for the geometric modeling of localized sources (particles).

\paragraph{Sources as singularities.}
Let $\Sigma \subset \Espace$ be the singular set where the direction of $N$ is undefined. In the simplest case of a point-like defect $p$, the boundary of a small neighborhood around $p$ has the topology of a 3-sphere, $\partial \mathcal{U}_p \cong S^3$.
Unlike the kinematic velocity sphere $S^3_{\clight}$ in tangent space (discussed in \S\ref{subsec:S3-of-flows}), this is a \emph{spatial} $S^3$ embedded in the manifold, enclosing the defect.

\paragraph{The Hopf map and charge quantization.}
On this bounding sphere $S^3$, the unit field $N$ defines a mapping to the sphere of directions $S^2$. This geometric configuration naturally invokes the \emph{Hopf fibration} structure:
\[
    S^1 \hookrightarrow S^3 \xrightarrow{\pi} S^2.
\]
In this picture, the fibers ($S^1$) correspond to the integral curves of the proto-field $N$. If these field lines are topologically non-trivial (linked), the defect possesses a conserved topological charge defined by the Hopf invariant (linking number) $\gamma \in \pi_3(S^2) \cong \mathbb{Z}$.

This implies that matter sources need not be introduced as external entities but can be interpreted as topological defects (e.g., Hopfions) of the underlying Euclidean vacuum field. The discrete nature of the Hopf invariant offers a purely geometric origin for the quantization of physical charges.

\paragraph{Fibers as hidden variables and quantum uncertainty.}
This topological picture offers a compelling geometric interpretation of quantum uncertainty. If an elementary object is identified not with a singular point, but with the entire fiber $S^1$ (a closed flow line of $N$), then its representation as a point in the observable base space $S^2$ is merely a projection.
The position along the fiber corresponds to the unobservable global phase of the quantum state ($U(1)$ symmetry). This geometric view resonates with the \emph{Zitterbewegung} interpretation of quantum mechanics proposed by Hestenes~\cite{Hestenes2008}, where the electron's spin and phase are intrinsic kinematic features of a light-like circulation, rather than abstract quantum numbers.
Consequently, the object is fundamentally delocalized in the proto-space. An observer measuring kinematic variables in the quotient space interacts with the projection, while the cyclic degree of freedom along the fiber remains indeterminate. This suggests that quantum indeterminacy may arise from the dimensional reduction of a deterministic, but topologically extended, flow in the Euclidean proto-space.


\bibliographystyle{plain}
\begin{thebibliography}{9}
\bibitem{Karapetoff1912}
V. Karapetoff, \emph{The Electric Circuit}, McGraw--Hill, 1912.

\bibitem{Brands2021}
H. Brands, Euclidean structure and Lorentz transformations,
\emph{Foundations of Physics}, \textbf{51} (2021), 1--22.

\bibitem{Akintsov2020}
A. Akintsov et al., On Euclidean embeddings of Minkowski space,
\emph{Journal of Geometry and Physics}, \textbf{153} (2020), 103655.

\bibitem{Reddy2009}
V.\,V. Reddy, R. Sharma, S. Sivaramakrishnan,
Construction of Lorentzian metrics from Riemannian metrics,
\emph{Differential Geometry and its Applications},
\textbf{27} (2009), no.\,6, 767--773.

\bibitem{Hestenes1966}
D.\,Hestenes,
\emph{Space-Time Algebra},
Gordon and Breach, New York, 1966.

\bibitem{Hestenes2003}
D.\,Hestenes,
``Spacetime physics with geometric algebra'',
\emph{American Journal of Physics},
\textbf{71} (2003), no.\,6, 691--714.
doi:10.1119/1.1571836.

\bibitem{Hestenes2008}
D.\,Hestenes,
``Zitterbewegung in quantum mechanics'',
\emph{Foundations of Physics},
\textbf{38} (2008), no.\,10, 829--847.
doi:10.1007/s10701-008-9228-1.

\end{thebibliography}

\end{document}
