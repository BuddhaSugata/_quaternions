
\documentclass[11pt]{article}
\usepackage[utf8]{inputenc}
\usepackage[T2A]{fontenc}
\usepackage[russian,english]{babel}
\usepackage{amsmath,amssymb}
\usepackage{bm}

\title{Краткий вывод: представление гравитационно--электромагнитного поля\\
через билинейные формы кватернионов и D-повороты в униметрии}
\author{}
\date{}

\begin{document}
\maketitle

\section{Одевание тел в метрические кватернионы}

Для тела $i$ с массой $m_i$ и зарядом $Q_i$ вводится \emph{метрически одетый}
кватернион потока
\begin{equation}
  \widetilde{\mathsf q}_i
  = L_{E,i}\,\hat h + L_{G,i}\,\widehat{\bm n}_i,
\end{equation}
где обе компоненты имеют размерность длины (метры):
\begin{equation}
  L_{E,i}
  = \sqrt{\frac{G}{4\pi\varepsilon_0 c^4}}\;Q_i,
  \qquad
  L_{G,i}
  = \frac{G}{c^2}\,m_i.
\end{equation}
Здесь $\hat h$ --- темпоральный (электромагнитный) канал, а
$\widehat{\bm n}_i$ --- направление гравитационного потока тела
в наблюдаемом трёхмерном пространстве (или его проекции).

Таким образом, вся размерностная физика ($G,\varepsilon_0,c$) упакована
во внутренние длины $L_E,L_G$, а ориентации потоков задают чистую геометрию.

\section{Билинейные формы $A,B,C$}

Для двух кватернионов
\begin{equation}
  q_1 = T_1\hat h + \bm S_1,
  \qquad
  q_2 = T_2\hat h + \bm S_2
\end{equation}
естественным образом выделяются три билинейные формы:
\begin{align}
  A(q_1,q_2) &= T_1T_2 - \bm S_1\cdot\bm S_2,
  \label{eq:A-def}\\[0.4em]
  B(q_1,q_2) &= T_1\bm S_2 + T_2\bm S_1,
  \label{eq:B-def}\\[0.4em]
  C(q_1,q_2) &= \bm S_1\times\bm S_2.
  \label{eq:C-def}
\end{align}

Интерпретация:
\begin{itemize}
  \item $A$ --- скалярный (минковский-подобный) инвариант. В нашем контексте
        он задаёт \emph{скалярный потенциал} взаимодействия
        гравитационно--электромагнитного поля (ГЭМ).
  \item $B$ --- симметричная векторная форма; описывает \emph{токовые каналы}:
        гравитационные и электрические потоки при относительном движении тел.
  \item $C$ --- антисимметричная векторная форма; вихревой канал. Для EM-сектора
        она даёт магнитную составляющую, а в чисто статической гравитации
        $C\equiv 0$.
\end{itemize}

\section{Статический ГЭМ: Кулон + Ньютон из $A$-формы}

Рассмотрим два тела $1$ и $2$ с одетыми кватернионами
$\widetilde{\mathsf q}_1,\widetilde{\mathsf q}_2$.
Статическая скалярная форма
\begin{equation}
  A_0 := A(\widetilde{\mathsf q}_1,\widetilde{\mathsf q}_2)
       = L_{E,1}L_{E,2} - L_{G,1}L_{G,2}
\end{equation}
даёт при подстановке определений
\begin{equation}
  A_0
  = \frac{G}{4\pi\varepsilon_0 c^4}Q_1Q_2
    - \frac{G^2}{c^4}m_1m_2.
\end{equation}

Потенциальная энергия взаимодействия на расстоянии $r$ берётся в виде
\begin{equation}
  U(r)
  = \frac{c^4}{G}\,\frac{A_0}{r}.
\end{equation}
Тогда
\begin{equation}
  U(r)
  = \frac{1}{4\pi\varepsilon_0}\frac{Q_1Q_2}{r}
    - G\frac{m_1m_2}{r},
\end{equation}
то есть \emph{строго} стандартный Кулоновский плюс ньютоновский потенциал.
Таким образом, одна скалярная форма $A$ при одной общей калибровке
$\tfrac{c^4}{G}$ полностью воспроизводит статический ГЭМ.

\section{Расстояние и поле в трёхмерном пространстве}

Расстояние до источника $j$ в точке $\bm x$ представляется чисто мнимым
кватернионом
\begin{equation}
  R_j(\bm x) = \bm r_j = \bm x - \bm x_j,
  \qquad
  R_j^{-1} = -\frac{\widehat{\bm r}_j}{r_j},
\end{equation}
где $\widehat{\bm r}_j$ --- единичный вектор, $r_j = \|\bm r_j\|$.

Тогда \emph{безразмерный} кватернионный потенциал ГЭМ-поля в точке $\bm x$:
\begin{equation}
  \mathcal U_{\rm GEM}(\bm x)
  := \sum_j R_j(\bm x)^{-1}\,\widetilde{\mathsf q}_j,
\end{equation}
а ``силовой'' объект (с размерностью $1/{\rm m}$):
\begin{equation}
  \mathcal F_{\rm GEM}(\bm x)
  := \sum_j R_j(\bm x)^{-1}\,\widetilde{\mathsf q}_j\,R_j(\bm x)^{-1}.
\end{equation}

Электростатическое поле $\bm E$ и гравитационный потенциал $U_G$ получаются
из скалярного канала (формы $A$) этих кватернионов при внешней калибровке
через $\varepsilon_0$ и $G$ соответственно.

\section{Динамика: D-поворот как буст потока}

D-ротор вводится как чисто геометрическое преобразование в фазовом
пространстве:
\begin{equation}
  d(\zeta) = \cos\frac{\zeta}{2} + \widehat{\bm u}\,\sin\frac{\zeta}{2},
\end{equation}
где $\zeta$ --- безразмерный угол, а $\widehat{\bm u}$ --- направление
в наблюдаемом $3$-пространстве.

Он действует на кватернион потока по правилу
\begin{equation}
  \widetilde{\mathsf q}' = d\,\widetilde{\mathsf q}\,d.
\end{equation}
В плоскости $(\hat h,\widehat{\bm u})$ это обычный поворот:
\begin{equation}
  T'   = T\cos\zeta - S_u\sin\zeta,
  \qquad
  S_u' = T\sin\zeta + S_u\cos\zeta,
\end{equation}
где $S_u$ --- компонента вектора $\bm S$ вдоль $\widehat{\bm u}$.

Связь угла $\zeta$ с безразмерной скоростью $\beta = v/c$ задаётся \emph{на
уровне физической интерпретации} (например, $\beta = \sin\zeta$).
При этом D-поворот остаётся чисто безразмерным оператором; световой масштаб
$c$ входит только во внешнюю калибровку потоков.

\section{Поведение форм $A,B,C$ при движении}

\subsection*{Чистая гравитация}

Для чисто гравитационного канала ($L_{E,i}=0$) имеем
$\widetilde{\mathsf q}_i = L_{G,i}\widehat{\bm n}_i$.
В статике при коллинеарных потоках $\widehat{\bm n}_1 \parallel
\widehat{\bm n}_2$:
\begin{equation}
  A = -L_{G,1}L_{G,2},\qquad
  B = 0,\qquad
  C = 0.
\end{equation}
Это соответствует чистому ньютоновскому потенциалу без токов и вихрей.

При радиальном D-повороте (скорость вдоль линии соединения):

\begin{itemize}
  \item $A$ уменьшается по $\cos\vartheta$:
        часть потока переходит в темпоральный канал (релятивистская
        кинематика);
  \item появляется продольный $B$ --- гравитационный ``ток'' массы;
  \item $C$ остаётся нулевой: гравитационных вихрей в этом режиме нет.
\end{itemize}

\subsection*{Заряд + масса, поперечное движение (магнетизм)}

Если заряд движется поперёк радиус-вектора, D-поворот в плоскости
$(\hat h,\widehat{\bm u}\perp\widehat{\bm n})$ даёт:

\begin{itemize}
  \item модификацию $A$ (смешение вкладов EM и GR в результате буста);
  \item ненулевой $B$ --- токовые каналы;
  \item ненулевую вихревую форму
  \begin{equation}
    C \;\propto\; L_E\,L_{G,{\rm t}}\,\sin\vartheta\,
      (\widehat{\bm u}\times\widehat{\bm n}),
  \end{equation}
  геометрически совпадающую с $\bm v\times\widehat{\bm r}$.

  При учёте $2$-сферической симметрии гравитационного потока массивных
  стабильных тестовых тел вклад через $L_{G,{\rm t}}$ усреднённо исчезает,
  и вихревой EM-канал становится чисто зарядовым.
\end{itemize}

Таким образом:
\begin{itemize}
  \item форма $B$ описывает токи (гравитационные и электрические);
  \item форма $C$ задаёт вихревой EM-канал (магнитное поле), причём в чистой
        статике и в чистой гравитации $C=0$, как и ожидается.
\end{itemize}

\section{Калибровка $\bm E$ и $\bm B$, роль $\varepsilon_0,\mu_0,c^3$}

Из скалярного канала (A-формы) получаем электростатическое поле:
\begin{equation}
  \bm E(\bm r)
  = \frac{1}{4\pi\varepsilon_0}\,\frac{Q}{r^2}\,\widehat{\bm r}.
\end{equation}
Вихревая часть (C-канал) геометрически даёт (после учёта расстояния и
безразмерной скорости $\beta=v/c$)
\begin{equation}
  C_{\rm geom}
  \;\propto\;
  \frac{Q}{r^2}\,\beta\,(\widehat{\bm v}\times\widehat{\bm r}),
  \qquad \beta = \frac{v}{c}.
\end{equation}

Вводим калибровку магнитного поля через тот же масштаб $k_E$:
\begin{equation}
  \bm B := \frac{1}{4\pi\varepsilon_0 c}\,C_{\rm geom}.
\end{equation}
Тогда
\begin{equation}
  \bm B(\bm r)
  = \frac{1}{4\pi\varepsilon_0 c^2}
    \frac{Q\,\bm v\times\widehat{\bm r}}{r^2}
  = \frac{\mu_0}{4\pi}
    \frac{Q\,\bm v\times\widehat{\bm r}}{r^2},
\end{equation}
так как $\mu_0\varepsilon_0 c^2 = 1$.

Одновременно выполняется стандартное соотношение
\begin{equation}
  \bm B = \frac{1}{c^2}\,\bm v\times\bm E,
\end{equation}
что согласуется с силой Лоренца
$\bm F = q(\bm E + \bm v\times\bm B)$ и размерностной ролью $\mu_0$:
\begin{equation}
  \frac{\mu_0}{c}
  = \frac{1}{\varepsilon_0 c^3}
\end{equation}
естественно интерпретируется как ``магнитная жёсткость вакуума'' на единицу
объёмного фазового потока (масштаб $c^3$).

\section{Карта соответствий}

Сводно:
\begin{itemize}
  \item Одетоe тело $\widetilde{\mathsf q}_i$:
    \begin{itemize}
      \item $L_{E,i}$ соответствует заряду $Q_i$,
      \item $L_{G,i}$ соответствует массе $m_i$,
      \item $\widehat{\bm n}_i$ задаёт направление гравитационного потока.
    \end{itemize}
  \item Формы:
    \begin{itemize}
      \item $A(q_1,q_2)$ --- скалярный GEM-потенциал (Кулон + Ньютон);
      \item $B(q_1,q_2)$ --- токовые каналы (гравитационные и электрические);
      \item $C(q_1,q_2)$ --- вихревой EM-канал (магнитное поле); в чистой
            статике и для чисто гравитационного взаимодействия $C=0$.
    \end{itemize}
  \item D-поворот $q' = d q d$:
    \begin{itemize}
      \item чисто геометрический поворот в фазовом пространстве;
      \item реализует релятивистские бусты через перераспределение потока
            между темпоральной и пространственной частью;
      \item связь угла с $\beta=v/c$ задаётся на уровне интерпретации,
            константа $c$ входит только во внешнюю калибровку.
    \end{itemize}
  \item Статика:
    \begin{itemize}
      \item $A$ даёт ровно
        $U(r)=k_e Q_1Q_2/r - Gm_1m_2/r$ при $k_e=1/(4\pi\varepsilon_0)$;
      \item EM-поле $\bm E$ и гравитационный потенциал следуют из того же
            инварианта $A$.
    \end{itemize}
  \item Динамика:
    \begin{itemize}
      \item $B$ и $C$ возникают при D-поворотах (относительном движении),
            описывая токи и магнитные вихри;
      \item калибровка через $\varepsilon_0,\mu_0,c$ даёт стандартные
            максвелловские выражения для $\bm E$ и $\bm B$ без введения
            новых фундаментальных констант.
    \end{itemize}
\end{itemize}

Таким образом, гравитация и электромагнетизм объединяются в униметрии
в едином кватернионном представлении: статически --- через одну скалярную
форму $A$, динамически --- через D-повороты, порождающие токовые ($B$)
и вихревые ($C$) каналы, согласованные с релятивистской кинематикой
и стандартными константами $G,\varepsilon_0,\mu_0,c$.
\end{document}
