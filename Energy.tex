% !TEX TS-program = pdflatex
% Unimetry — Energy in a Phase‑Space
\documentclass[11pt,a4paper]{article}

% ====== Packages ======
\usepackage[T1]{fontenc}
\usepackage[utf8]{inputenc}
\usepackage{lmodern}
\usepackage{microtype}
\usepackage{geometry}
\geometry{margin=1in}
\usepackage{amsmath,amssymb,mathtools}
\usepackage{bm}
\usepackage{physics}
\usepackage{siunitx}
\usepackage{hyperref}
\usepackage[capitalise]{cleveref}
\usepackage{graphicx}
\usepackage{xcolor}
\hypersetup{colorlinks=true, linkcolor=blue, citecolor=blue, urlcolor=blue}

% ====== Macros & Notation ======
\newcommand{\tH}{\tilde{H}}
\newcommand{\tS}{\tilde{S}}
\newcommand{\tL}{\tilde{L}}
\newcommand{\dd}{\mathrm{d}}
\newcommand{\E}{\mathcal{E}}
\newcommand{\be}{\beta}
\newcommand{\ga}{\gamma}
\DeclareMathOperator{\sech}{sech}

\title{Unimetry: Energy in a Phase\textendash Space}
\author{Timur Abizgeldin}
\date{\today}

\begin{document}
\maketitle

\begin{abstract}
We propose a phase\textendash space formulation (\emph{Unimetry}) where mass is a volume\textendash normalized structural coefficient of a flow, and the standard relativistic energy\textendash momentum relation emerges from the geometry of two orthogonal components of the flow. We develop a compact dictionary to SR, motivate the cubic scaling behind rest mass, and derive a scalar phase\textendash space energy density whose observed energy is obtained by kinematic projection. Consequences and empirical anchors (binding energy, heat, stress/pressure) are discussed.
\end{abstract}

\paragraph{Keywords:} phase\textendash space, relativistic energy, structural mass, volumetric normalization, emergent time.

\section{Introduction}
Why does relativistic energy take the form it does, and how can ``mass'' be read off from internal structure rather than postulated? Unimetry treats an object as a flow in phase\textendash space with modulus $\tH$ split into an internal (temporal) and external (spatial) part. This viewpoint suggests a cubic (volumetric) normalization for rest mass and recovers standard SR kinematics as a rotation in the $(\tS,\tL)$\textendash plane.

\paragraph{Contributions.}
(i) We model rest mass as a volume\textendash normalized structural coefficient $\kappa\propto k^3$ of a cyclic normalization $k$; (ii) we derive $E=\gamma m_0 c^2$ and $E^2=m_0^2c^4+p^2c^2$ directly from the phase geometry; (iii) we clarify why Euclidean quadratic invariants built from a cubic scale yield sixth\textendash power laws; (iv) we identify a scalar phase\textendash space energy density $e=\kappa\dot H^3$ and relate it to empirical effects (binding, heat, stress/pressure).

\paragraph{Roadmap.}
Sec.~2 fixes notation; Sec.~3 introduces structural mass; Sec.~4 derives relativistic energy; Sec.~5 justifies volume normalization and the ``$\times3$ rule''; Sec.~6 generalizes energy formulas, where we \emph{show} (Proposition~\#) that $e$ defines a phase\textendash space invariant; Sec.~7 discusses verification paths.

\subsection*{Postulates (informal)}
\begin{enumerate}
  \item (Phase flow) Each physical object is represented by a flow with modulus $\tH$ and orthogonal components $(\tS,\tL)$ such that $\tH^2=\tS^2+\tL^2$.
  \item (Kinematic angle) Define $\be=\sin\theta=\tL/\tH$ and $\ga=\sec\theta=1/\sqrt{1-\be^2}$.
  \item (Gauge) The speed of light $c$ is identified with the local\textendash time phase speed: $c\equiv\dot H$.
  \item (Cyclic time) Local time arises as a cyclic action with frequency $\nu=\dot\chi=k\,\tH$, where $k=R_1/R_2$ is a normalization factor of the cycle radii.
\end{enumerate}

\section{Preliminaries and Notation}
We employ tilded symbols for proto\textendash space quantities and dots for local\textendash time derivatives. The basic geometric decomposition reads
\begin{equation}
  \tH^2 = \tS^2 + \tL^2, \qquad \tS=\tH\cos\theta,\quad \tL=\tH\sin\theta,\quad \be=\sin\theta,\quad \ga=\sec\theta.
  \label{eq:decomp}
\end{equation}
The cyclic\textendash time normalization gives
\begin{equation}
  \nu=\frac{\dd\chi}{\dd\tau}=\dot\chi = k\,\tH, \qquad k=\frac{R_1}{R_2}.
  \label{eq:nu}
\end{equation}

\subsection*{Why does $\nu = k\,\tH$? Three equivalent derivations}
\paragraph{(A) Flux continuity on the two circles.} View the internal dynamics as a steady flow on a two-torus $\mathbb T^2=S^1_{\chi}\times S^1_{\tau}$ with circumferences $C_{\chi}=2\pi R_1$ and $C_{\tau}=2\pi R_2$. One tick corresponds to transporting the arc length $\Delta\chi=C_{\chi}$ while advancing the time circle by $\Delta\tau=C_{\tau}/\tH$ (since the speed along the $\tau$-fiber is $\tH$ by the gauge $c\equiv \dot H$). Equating the steady fluxes through the two fundamental cycles,
\[
J_{\chi}=\tH\,C_{\chi},\qquad J_{\tau}=\nu\,C_{\tau},
\]
forces $\nu \equiv d\chi/d\tau = (C_{\chi}/C_{\tau})\,\tH = (R_1/R_2)\,\tH$.

\paragraph{(B) Circle-group homomorphism (winding).} The only smooth homomorphisms of the circle group $S^1$ are rotations with degree $k$; in angle variables $\theta_{\tau}=k\,\theta_{\chi}$ (mod $2\pi$). Passing to arc-length coordinates $\chi=R_1\theta_{\chi}$ and $\zeta=R_2\theta_{\tau}$ and differentiating with respect to $\tau$ gives $\dot\zeta=R_2\dot\theta_{\tau}=k(R_2/R_1)\dot\chi$. Identifying $\dot\zeta=\tH$ (the speed along the time-fiber under the gauge) yields $\dot\chi= k\,\tH$, i.e. $\nu=k\,\tH$.

\paragraph{(C) Dimensional and symmetry argument.} A frequency must be built from the available scalars $\tH$ (a speed) and the two radii $R_1,R_2$. Rotational invariance rules out vectorial combinations; scale invariance on each circle restricts the dependence to their ratio. Thus the unique invariant with dimensions of $\mathrm{s^{-1}}$ is $\nu \propto \tH (R_1/R_2)$. Fixing the proportionality by the rest calibration leads to $\nu=k\,\tH$ with $k=R_1/R_2$.

\paragraph{Remark.} Alternative bookkeeping that treats $k$ as carrying inverse-length units and $\tH$ as a speed is equivalent after absorbing constants into $\kappa$; all physical relations (e.g., $E=\ga m_0 c^2$) are unchanged.

\section{Mass as a Structural Coefficient}
We define a structural (volumetric) coefficient $\kappa$ and \emph{rest mass}
\begin{equation}
  \kappa(k) = \kappa_*\!\left(\frac{k}{k_*}\right)^{\!3},
  \qquad m_0(k) := \kappa(k)\,c,
  \qquad E_0(k) := \kappa(k)\,c^3 = m_0(k)c^2.
  \label{eq:struct}
\end{equation}
The cubic dependence reflects a volumetric Jacobian of the internal phase normalization. Small variations obey
\begin{equation}
  \frac{\Delta m_0}{m_0}=\frac{\Delta E_0}{E_0}=3\,\frac{\Delta k}{k}.
  \label{eq:small}
\end{equation}
A \emph{simple} (structureless) flow (photon) has $\tS=0$, hence $m_0=0$, while $E\propto c^3$ via its own scale factor $\kappa_\gamma$.

\section{Relativistic Energy from Phase Geometry}
With \cref{eq:decomp} and the gauge $c\equiv\dot H$, pure boosts are rotations in the $(\tS,\tL)$ plane that leave $\tH$ and $k$ invariant. Calibrating energy by rest we obtain
\begin{equation}
  E(\theta,k)=\frac{E_0(k)}{\cos\theta}=\ga\,m_0(k)c^2,\qquad p=\frac{E}{c}\sin\theta=\ga m_0(k) v,\quad v=c\sin\theta.
  \label{eq:energy}
\end{equation}
Immediately,
\begin{equation}
  E^2 = m_0^2 c^4 + p^2 c^2,
  \label{eq:emrelation}
\end{equation}
with the usual low\textendash velocity expansion $E=m_0c^2+\tfrac12 m_0 v^2+O(v^4/c^2)$.

\subsection*{Anisotropic inertia (geometry)}
For a boost along $x$ the transverse flows remain unchanged ($\tL_y,\tL_z$ invariant), yielding the geometric form of longitudinal and transverse inertial responses:
\begin{equation}
  m_{\parallel}=\frac{\dd p_x}{\dd v_x}=\ga^3 m_0,\qquad m_{\perp}=\frac{\dd p_y}{\dd v_y}=\ga m_0.
  \label{eq:anis}
\end{equation}

\section{Justification of Volume\textendash Normalized Mass}
\subsection{Composites and Jensen inequality}
For a composite where $k$ varies internally,
\begin{equation}
  m_0 \propto \langle k^3 \rangle \ge (\langle k \rangle)^3,
\end{equation}
so inhomogeneities (internal stresses/pressures) increase $m_0$ at fixed average normalization.

\subsection{Empirical anchors}
\begin{itemize}
  \item \textbf{Mass defect:} negative binding lowers $k$ and $m_0$, consistent with nuclear data.
  \item \textbf{Heat/fields/rotation:} added internal energy raises $m_0$ by $\Delta E/c^2$, i.e. $\Delta k/k=\tfrac13\,\Delta E/E_0$.
  \item \textbf{Gravitational redshift of clocks:} $\Delta\nu/\nu\simeq\Delta\Phi/c^2$ implies $\Delta k/k\simeq\Delta\Phi/c^2$ for the normalization factor.
  \item \textbf{Stress\textendash energy link:} isotropic radiation with $p=\rho/3$ contributes via $(\rho+3p)$, mirroring the ``cubic'' internal degrees of freedom.
\end{itemize}

\section{Generalized Energy in Phase\textendash Space}
At the kinematic level a convenient ``mixed'' representation is
\begin{equation}
  E = \ga\,\kappa\,\dot H^{2}\,\tH,\qquad (\text{with } c\equiv\dot H),
  \label{eq:mixed}
\end{equation}
which collapses to $E=\ga\,\kappa c^3$ under dynamic renormalization $\tH\to\dot H$. For a photon (simple flow) in vacuum: $m_0=0$, $E=\kappa_\gamma c^3$, $p=E/c$.

\subsection{Quadratic invariants and the sixth\textendash power law}
Let $e := \kappa\,\dot H^{3}$ denote the local intensive energy scale of a single flow. Any Euclidean quadratic invariant built from a field with this scaling (e.g., self\textendash energy bilinears, $L^2$ norms in the phase domain, quadratic action densities) takes the form
\begin{equation}
  \mathcal I_2 \;=\; \int e^2\,\dd V_{\chi} \;\propto\; \int \kappa^{2}\,\dot H^{6}\,\dd V_{\chi} .
\end{equation}
Thus a quadratic invariant maps the cubic phase\textendash speed scaling into a sixth\textendash power law. More generally, $m$\textendash linear invariants scale as $\dot H^{3m}$.

Equivalently, in the rest\textendash normalization $\kappa\propto k^3$ one has $e_0\propto k^3$ and any quadratic invariant in the varying normalization $k$ scales as $k^6$. This is the precise sense in which a Euclidean quadratic norm preserves an invariant built from a cubic structural coefficient.


\subsection{Energy as a phase\textendash space invariant}
\paragraph{Proposition (phase\textendash space energy).}
Define the \emph{phase\textendash space energy density} by
\begin{equation}
  e(\chi) := \kappa(\chi)\,\dot H^{3},\qquad (c\equiv \dot H\ 	ext{const}).
\end{equation}
\textbf{Boost invariance.} Pure boosts are rotations in $(\tS,\tL)$ that leave $\tH$, $\dot H$ and $\kappa$ unchanged; hence $e$ is invariant. The integral
\begin{equation}
  E_{\chi} := \int_{\Sigma_{\chi}} e\,\dd V_{\chi} \equiv m_0 c^2
\end{equation}
— the energy measured in the intrinsic phase frame — is a scalar independent of the state of motion. The observed (laboratory) energy and momentum are projections
\begin{equation}
  E = \ga E_{\chi},\qquad p = \frac{E}{c}\sin\theta = \ga m_0 v,
\end{equation}
so that
\begin{equation}
  E^2 - (pc)^2 = E_{\chi}^{2} = (m_0 c^2)^2,
\end{equation}
which makes the usual SR invariant explicit as the square of the phase energy.

\textbf{Reparameterization invariance.} Under a local reparametrization $\chi\mapsto \chi'(\chi)$ with Jacobian $J=\dd\chi'/\dd\chi$, the structural density transforms as a 3\textendash density $\kappa' = \kappa/J^{3}$ while $\dd V_{\chi'} = J^{3}\dd V_{\chi}$, so that $e\,\dd V_{\chi}$ and $E_{\chi}$ are invariant.

\textbf{Dynamics vs kinematics.} Changes in internal structure (massogenesis) modify $\kappa$ and thus $e$ physically; the invariance statements above refer to kinematic transformations (boosts and phase reparametrizations), not to dynamics that pump energy into or out of the system.

\subsection{Continuity and Noether\textendash like view (sketch)}
Treat $\kappa(k)$ as a density on internal phase: local conservation takes the form
\begin{equation}
  \partial_\tau (\kappa c^3) + \nabla_{\chi}\!\cdot (\kappa c^2\,\bm{J}) = 0,
\end{equation}
where $\bm{J}$ is a phase\textendash space current; energy emerges as the charge of $\tau$\textendash translations.

\subsection{Gravitational field as a rotor-of-rotors}
\label{sec:grav-rotor-of-rotors}

In this subsection we temporarily switch off the charge channel and keep only the spatial (mass–gravitational) part of the dressed body quaternion. The aim is to formulate a unimetric analogue of the Einsteinian ``equivalence of acceleration and gravity'' in the language of D–rotations of the flow.

\subsubsection*{Assumptions and notation}

We work in the Euclidean proto-space $(\mathcal E,\langle\cdot,\cdot\rangle)$ introduced in \S\ref{sec:proto-space}, with the following assumptions:
\begin{enumerate}
  \item The fundamental object is the normalized flow direction
  \[
    \widehat{\boldsymbol\chi}(x)\in S^3\subset\mathbb H,\qquad
    \|\widehat{\boldsymbol\chi}(x)\|=1,
  \]
  and the physical flow vector is $\boldsymbol\chi := \clight\,\widehat{\boldsymbol\chi}$.
  \item Each massive body $B$ is represented by a (dimensionless) unit body quaternion $q_B(x)\in S^3$, whose temporal and spatial channels encode, respectively, the temporal coefficient $T_B(x)$ and the spatial shape tensor $\mathbf C_B(x)$ as described in \S\ref{sec:objects}.
  \item In the dressed-quaternion picture of GEM, the gravitational channel is carried by the spatial part of
  \begin{equation}
    \label{eq:dressed-q-mass-only}
    \mathsf q_m
    := \frac{m}{\sqrt{4\pi\varepsilon_G}}\,\widehat{\mathbf n},
    \qquad
    \varepsilon_G := \frac{1}{4\pi G},
  \end{equation}
  where $\widehat{\mathbf n}$ is a unit spatial direction of the flow and the charge channel (the temporal part proportional to $Q$) is set to zero for the purely gravitational sector.
\end{enumerate}

\subsubsection*{D--rotation as kinematic relative motion}

We briefly recall the D--rotation used in the SR reformulation in \S\ref{sec:D-rotation}.

Let $\widehat{\mathbf u}$ be a unit 3--vector in the observed spatial 3--manifold and let $\vartheta\in(0,\pi/2)$ be the kinematic angle between the body's flow and the observer's time axis. The corresponding D--rotor is the unit quaternion
\begin{equation}
  \label{eq:D-rotor-def}
  d(\widehat{\mathbf u},\vartheta)
  := \cos\frac{\vartheta}{2}
    + \widehat{\mathbf u}\,\sin\frac{\vartheta}{2},
  \qquad \|d\|=1.
\end{equation}
For any quaternion $q = T\,\hat h + \mathbf S$ the D--rotation acts as
\begin{equation}
  q' = d\,q\,d,
\end{equation}
which implements a Lorentz boost in the $(\hat h,\widehat{\mathbf u})$–plane. The kinematic relations are
\begin{equation}
  \beta := \frac{\|\mathbf v\|}{\clight} = \sin\vartheta,
  \qquad
  \gamma = \frac{1}{\sqrt{1-\beta^2}} = \sec\vartheta,
\end{equation}
and for the temporal coefficient $T_B$ of a single-stream body one has
\begin{equation}
  T_B = \cos^2\vartheta,
  \qquad
  \frac{d\tau}{dt} = \cos\vartheta,
\end{equation}
so that the D--rotation reproduces the standard SR time dilation.

In particular, for a body whose rest-time axis is represented by the pure temporal quaternion $\hat h$, its flow quaternion in the laboratory frame is
\begin{equation}
  \label{eq:q-from-D}
  q_B(\tau) = D(\tau)\,\hat h\,D(\tau),
\end{equation}
where $D(\tau)$ is the D--rotor determined by the body's instantaneous relative velocity with respect to the laboratory frame at proper time~$\tau$.

\subsubsection*{Gravitational field as a rotor acting on D--rotors}

Einstein's equivalence principle states that, locally, a homogeneous gravitational field is indistinguishable from uniform acceleration. In the unimetric language we adopt the following formulation:

\begin{quote}
  \emph{The gravitational field is represented by a field of rotors that acts on the D--rotors describing kinematic motion.}
\end{quote}

More precisely, let $D(x)$ be the D--rotor that maps the local rest frame of a freely falling observer into the laboratory frame at point $x\in\mathcal E$. We define the \emph{gravitational rotor connection} as the quaternion--valued 1--form
\begin{equation}
  \label{eq:grav-connection}
  \mathcal A_a(x)
  := 2\,(\partial_a D(x))\,D(x)^{-1},
\end{equation}
where $a$ is an index in the proto-space (or, equivalently, in the induced 4--manifold) and $D(x)^{-1}=\overline{D(x)}$ since $D(x)$ is unit.

Under an infinitesimal displacement $dx^a$ one has
\begin{equation}
  D(x+dx)
  \approx \Big(1 + \tfrac12 \mathcal A_a(x)\,dx^a\Big)D(x),
\end{equation}
so that $\mathcal A_a$ plays the role of a ``rotor-of-rotors'': it rotates the D--rotor that encodes the local kinematic state.

Along a timelike worldline $\gamma:\tau\mapsto x(\tau)$ the total change of the D--rotor is governed by the \emph{gravitational angular velocity} quaternion
\begin{equation}
  \label{eq:Omega-g-def}
  \Omega_g(\tau)
  := \mathcal A_a(x(\tau))\,\frac{dx^a}{d\tau},
\end{equation}
so that
\begin{equation}
  \frac{dD}{d\tau}
  = \frac{1}{2}\,\Omega_g(\tau)\,D(\tau)
\end{equation}
in pure free fall. In the absence of gravitational field we have $\mathcal A_a\equiv 0$ and $D(\tau)$ is constant (uniform inertial motion).

The equivalence principle is then expressed as the statement that in a sufficiently small neighbourhood of any point $x_0$ one can choose a freely falling frame $I(x)$ (another rotor field) such that
\begin{equation}
  D_{\rm loc}(x) := I(x)^{-1} D(x)
\end{equation}
satisfies
\begin{equation}
  \left.\frac{dD_{\rm loc}}{d\tau}\right|_{x_0} = 0,
\end{equation}
i.e.\ the gravitational contribution to $\Omega_g$ is locally gauged away by an appropriate choice of $I(x)$, leaving pure SR kinematics.

The curvature of the gravitational rotor connection is given by the quaternion--valued 2--form
\begin{equation}
  \label{eq:grav-curv}
  \mathcal F_{ab}
  := \partial_a\mathcal A_b - \partial_b\mathcal A_a
     + \frac{1}{2}\big[\mathcal A_a,\mathcal A_b\big],
\end{equation}
which encodes tidal effects and frame precession, in direct analogy with the Riemann tensor in the metric formulation and with field-strength tensors in gauge theory.

\subsubsection*{Example: electron in a weak gravitational field}

Consider an electron with rest mass $m_e$ and charge $Q_e=-e$. In the GEM picture its dressed quaternion is
\begin{equation}
  \mathsf q_e
  = \frac{Q_e}{\sqrt{4\pi\varepsilon_0}}\,\hat h
    + \frac{m_e}{\sqrt{4\pi\varepsilon_G}}\,\widehat{\mathbf n}_e,
  \qquad \varepsilon_G = \frac{1}{4\pi G}.
\end{equation}
For the purely gravitational sector we ``switch off'' the charge interaction by setting $Q_e=0$ in the interaction bilinear, but the electron still carries its internal temporal channel in the rest frame through the intrinsic angle $\zeta_e$; in particular, its Compton frequency
\begin{equation}
  \omega_{0,e} = \frac{m_e \clight^2}{\hbar}
\end{equation}
sets the rest-frame phase flow of the corresponding streamlet.

Let the electron move in a weak, static gravitational field described by a rotor connection $\mathcal A_a(x)$ that is purely ``vertical'' in some adapted chart (e.g.\ $\mathcal A_0\approx 0$, $\mathcal A_z\neq 0$). Along the electron worldline $x(\tau)$ the D--rotor $D(\tau)$ evolves according to
\begin{equation}
  \frac{dD}{d\tau}
  = \frac{1}{2}\,\Omega_g(\tau)\,D(\tau),
  \qquad
  \Omega_g(\tau) = \mathcal A_a(x(\tau))\,\frac{dx^a}{d\tau}.
\end{equation}
The effective clock angle $\zeta(\tau)$ entering the temporal coefficient $T_B(\tau)=\cos^2\zeta(\tau)$ and the time-dilation relation
\begin{equation}
  \frac{d\tau}{dt} = \cos\zeta(\tau)
\end{equation}
is thus influenced by the gravitational rotor field via the evolution of $D(\tau)$. The electron's observed phase frequency in the laboratory frame is
\begin{equation}
  \omega_e^{\rm (lab)}(\tau)
  = \omega_{0,e}\,\frac{d\tau}{dt}
  = \omega_{0,e}\,\cos\zeta(\tau),
\end{equation}
so that the usual gravitational and kinematic time dilations are both realised as consequences of the rotation of the D--rotor under the action of $\Omega_g(\tau)$.

\subsubsection*{Example: neutral massive body as a pure gravitational flow}

Consider now a neutral macroscopic body of mass $m$ with no net charge, $Q=0$. Its dressed quaternion is purely spatial,
\begin{equation}
  \mathsf q_m
  = \frac{m}{\sqrt{4\pi\varepsilon_G}}\,\widehat{\mathbf n}_m,
\end{equation}
and the GEM interaction bilinear reduces to the Newtonian term
\begin{equation}
  \mathcal I(\mathsf q_{m,1},\mathsf q_{m,2})
  = -\,G\,m_1 m_2.
\end{equation}

Let the body be initially at rest in the laboratory frame, so that $D(\tau_0)=1$ and $q_B(\tau_0)=\hat h$. When the body is released in a gravitational field characterised by a rotor connection $\mathcal A_a(x)$, its D--rotor obeys
\begin{equation}
  \label{eq:free-fall-D}
  \frac{dD}{d\tau}
  = \frac{1}{2}\,\Omega_g(\tau)\,D(\tau),
\end{equation}
with $\Omega_g$ given by \eqref{eq:Omega-g-def}. The spatial projection of the flow quaternion
\begin{equation}
  q_B(\tau) = D(\tau)\,\hat h\,D(\tau)
  = T_B(\tau)\,\hat h + \mathbf S_B(\tau)
\end{equation}
acquires a non-zero part $\mathbf S_B(\tau)$, which in the non-relativistic limit corresponds to the usual 3--velocity $\mathbf v(\tau)$ induced by gravitational acceleration.

At the same time, the temporal coefficient $T_B(\tau)=\cos^2\zeta(\tau)$ decreases from its rest value, encoding the gravitational time dilation of the body's proper time. In this way, the entire effect of gravity on a neutral massive body is captured by the rotor field $\mathcal A_a(x)$ acting on the D--rotor $D(\tau)$, i.e.\ by a field of rotors acting on rotors, fully in the spirit of the equivalence principle.


\section{Verification and Predictions}
\begin{enumerate}
  \item High\textendash $Q$ cavity: trapped field energy and pressure ($T^{ii}$) increase weight by $(E+\text{pressure term})/c^2$.
  \item Flywheel test: compare mass at rest vs spinning, including elastic stress contribution; prediction from \cref{eq:anis,eq:struct}.
  \item Nonuniform heating: at fixed $\Delta E$, inhomogeneous $k(\bm{x})$ gives slightly larger $\langle k^3\rangle$ than uniform heating.
\end{enumerate}

\section{Discussion and Outlook}
We summarized how relativistic energy is recovered from a phase\textendash geometric decomposition with mass as a volume\textendash normalized structural coefficient. Open directions include: a full Lagrangian on phase\textendash space, coupling to curvature (mapping to $T^{\mu\nu}$ in GR), and quantum extensions where $k$ becomes an operator linked to cyclic spectra.

\paragraph{Acknowledgments} ———

\appendix
\section{Dimensional Analysis and Units}
With $[E]=\mathrm{J}$ and $[\dot H]=\mathrm{m\,s^{-1}}$, one has $[\kappa]=[E]/[\dot H]^3=\mathrm{J\,s^3\,m^{-3}}$. Using $m_0=\kappa c$ yields $[\kappa]=\mathrm{kg\,s\,m^{-1}}$.

\paragraph{Cyclic-time normalization.} In $\nu= k\,\tH$, if $k$ is taken dimensionless (e.g., $k=R_1/R_2$), we treat $\tH$ here as an \emph{effective frequency scale} inherited from the normalization map; equivalently, if $\tH$ is regarded as a speed, then $k$ carries units of inverse length so that $\nu$ has units of $\mathrm{s^{-1}}$. Both conventions are equivalent after absorbing constants into $\kappa$ and do not affect $E=\gamma m_0 c^2$.

\section{Derivation details for \cref{eq:energy}} 
Using $E_0=m_0 c^2$ and the rotation in $(\tS,\tL)$ with invariant $\tH$, the energy scales as $1/\cos\theta=\ga$, while $p=(E/c)\sin\theta$; eliminating $\theta$ gives \cref{eq:emrelation}.

\section{Dictionary to standard SR variables}
$\tS\leftrightarrow$ internal (proper\textendash time) projection; $\tL\leftrightarrow$ spatial projection; $\theta$ is the boost rapidity angle via $\tan\theta = v/\sqrt{c^2-v^2}$; $k$ encodes internal normalization of the cyclic time.

\end{document}
