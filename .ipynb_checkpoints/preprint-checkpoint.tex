\documentclass[11pt]{article}

\usepackage[a4paper,margin=1in]{geometry}
\usepackage{amsmath,amssymb,amsfonts,amsthm}
\usepackage{bm}
\usepackage{hyperref}

\numberwithin{equation}{section}

\title{Unimetry: A Phase-Space Reformulation of Special Relativity}
\author{Timur Abizgeldin\\ \small Independent researcher, Austria\\ \small \texttt{<replace-with-your-email>}}
\date{\today}

\begin{document}
\maketitle

\begin{abstract}
We propose a compact reformulation of special relativity in which spacetime units (time and length) are treated as \emph{phase velocities}---directional derivatives of a single underlying parameter, the phase $\vec{\chi}\in\mathbb{C}$. The observable Minkowski interval emerges as a conserved quantity under a change of parameter from the hidden phase coordinate $\chi$ to the observer's proper time $\tau$. In this \emph{unimetry} formalism, familiar relativistic effects---time dilation, Lorentz factor, Doppler shift, and relativistic velocity composition---arise as elementary projections and rotations in a Euclidean phase plane. Hyperbolic features of Lorentz kinematics reappear after a reparametrization of time, yielding the standard relations without altering empirical content. We provide closed-form derivations of the longitudinal/transverse Doppler factors, identify a simple lemma equating the total phase speed to the conserved Minkowski norm, and outline connections to gauge phases, rapidity, and a cosmological time gauge.
\end{abstract}

\paragraph{Keywords:} special relativity; phase; rapidity; Doppler shift; Lorentz factor; alternative formalism.

\paragraph{MSC/PhCS:} 83A05; 83-10; 70A05.

\section{Introduction}
We usually take time and space as primitive. The \emph{phase formalism} introduced here suggests a different viewpoint: time and space are \emph{derived projections} of a single parameter $\vec{\chi}\in\mathbb{C}$ (``phase''). In this picture, relativistic effects such as time dilation and the Doppler shift are geometric consequences of phase-flow rotations.

The proposal does not modify physics; it reorganizes familiar relations in a simpler language. In spirit it is akin to Lagrangian/Hamiltonian re-descriptions of classical mechanics: same empirical content, different coordinates. Throughout, Greek $\theta$ will denote the \emph{external} rotation angle associated with relative motion, while $\zeta$ denotes an \emph{internal} angle associated with the object's intrinsic state (mass/density heuristic).

\paragraph{Notation.} Tildes, dots and primes indicate derivatives with respect to the phase parameter, proper time, and spatial arclength:
\[
\tilde{X}:=\frac{dX}{d\chi},\qquad \dot{X}:=\frac{dX}{d\tau},\qquad X':=\frac{dX}{dl}.
\]
We use $\mathtt{c}$ for the speed of light and $\beta:=V/\mathtt{c}$, $\gamma:=1/\sqrt{1-\beta^2}$, rapidity $\tanh\eta=\beta$.

\section{Time and space as phase derivatives}
Let $\vec{\chi}\in\mathbb{C}$ be a variable whose change generates observable time-space effects. We treat the time and space units as directional derivatives (phase velocities) along the real and imaginary directions of a complex basis $(\hat{h},\mathbf{l})$:
\begin{equation}
\hat{h}\,dx_0=\frac{\partial\vec{\chi}}{\partial\chi_h}\frac{d\chi_h}{d\chi}\,d\chi
=\tilde{H}\,d\chi,\qquad
\mathbf{l}\,dx_l=\frac{\partial\vec{\chi}}{\partial\chi_l}\frac{d\chi_l}{d\chi}\,d\chi
=\tilde{L}\,d\chi,\quad l=1,2,3.
\label{eq:11}
\end{equation}
Introduce the phase speed of the SR interval $ds=\tilde{S}\,d\chi$. The interval conservation takes the form
\begin{equation}
\tilde{S}^2=\frac{ds^2}{d\chi^2}
=\frac{g_{ij}\,dx^i dx^j}{d\chi^2}
=\tilde{H}^2-\tilde{L}^2,
\label{eq:12}
\end{equation}
equivalently
\begin{equation}
\tilde{H}^2=\tilde{S}^2+\tilde{L}^2.
\label{eq:13}
\end{equation}
Writing $\tilde{S}=\tilde{H}\cos\theta$ and $\tilde{L}=\tilde{H}\sin\theta$,
\begin{equation}
\tilde{S}=\tilde{H}\cos\theta,\qquad \tilde{L}=\tilde{H}\sin\theta,
\label{eq:14}
\end{equation}
where $\theta$ is the angle of the phase speed relative to the real axis. Algebraically, \eqref{eq:13} is a Euclidean decomposition of a single speed into orthogonal projections; physically, we will see that under reparametrization the \emph{projection} $\tilde{S}$, not the Euclidean norm $\tilde{H}$, is the conserved Minkowski quantity.

\section{Phase space (\textit{kh\={o}ra})}
Let the phase vector space (``\emph{kh\={o}ra}'', after Plato) be $\mathbb{C}$ with orthonormal basis $(\hat{h},\mathbf{l})$. For a phase vector $\vec{\chi}=R\,e^{\theta\mathbf{l}}$ with $\theta\in[-\pi,\pi]$,
\begin{equation}
\tilde{H}=R,\qquad \tilde{S}=R\cos\theta,\qquad \tilde{L}=R\,\mathbf{l}\sin\theta.
\label{eq:21}
\end{equation}
Choosing coordinates where the projectors onto $(\hat{h},\mathbf{l})$ are unit, \eqref{eq:11} simplifies to
\begin{equation}
\hat{h}\,dx_0=\frac{d\chi_h}{d\chi}\,d\chi=\tilde{H}\,d\chi,\qquad
\mathbf{l}\,dx_l=\frac{d\chi_l}{d\chi}\,d\chi=\tilde{L}\,d\chi.
\label{eq:22}
\end{equation}
The map from phase to observables is an integral transform:
\begin{equation}
x^i(\chi)=x^i(\chi_0)+\int_{\chi_0}^{\chi}\tilde{X}^i(u)\,du,\qquad i=0,1,2,3,
\label{eq:23}
\end{equation}
where $\tilde{X}^i$ are projections of $d\vec{\chi}/d\chi$ onto $(\hat{h},\mathbf{l})$ and $x^i(\chi_0)$ fix initial conditions.

\section{Objects}
A \emph{fundamental particle} is an \emph{elementary object} with nonzero phase $\vec{\chi}\neq0$. Composite \emph{objects} are phase configurations; to represent them \emph{in phase space} one may require additional dimensions, except for the \emph{photon}, whose phase is always aligned with the imaginary axis:
\begin{equation}
\mathbf{p}=\frac{d\vec{\chi}}{d\chi_l}=p\,\mathbf{l}\in\Im.
\label{eq:31}
\end{equation}
Non-photonic phenomena are associated with nonzero real projection and nonzero mass. A complex object can be identified with an \emph{event} or worldline; the photon corresponds to a null-interval point encoding information about the event.

Any object's phase can be rotated to the \emph{zero} (purely real) direction,
\begin{equation}
\vec{\chi}_0=R\in\Re.
\label{eq:32}
\end{equation}
An object $A$ moving with speed $V$ relative to a rest observer has
\begin{equation}
\vec{\chi}_A=R\,e^{\theta_A\mathbf{l}},\qquad
\sin\theta_A=\frac{V}{\mathtt{c}}\equiv\beta.
\label{eq:33}
\end{equation}

\subsection{Space as a symmetric phase pair}
From \eqref{eq:14}, a naive zero-angle limit would remove the imaginary projection, contradicting observability. We enforce a nonvanishing spatial projection by pairing opposite-phase tilts:
\begin{equation}
\vec{\chi}^{\pm}=R\,e^{\pm\zeta\,\mathbf{l}},\qquad
\vec{\chi}_l:=\frac{\vec{\chi}^+-\vec{\chi}^-}{2}=R\,\mathbf{l}\sin\zeta,
\label{eq:311}
\end{equation}
where $\zeta$ is an \emph{internal angle} (intrinsic to the object; heuristically linked to mass/density). The local decomposition is
\begin{equation}
\vec{\chi}_0=\vec{\chi}_\tau+\vec{\chi}_l
=R\cos\zeta+R\,\mathbf{l}\sin\zeta,
\label{eq:312}
\end{equation}
with unit bases (normalized by $R$): $\hat{h}=\cos\zeta$, $\hat{\mathbf{l}}=\sin\zeta$.

\subsection{Absolute, local, and observed time}
Define \emph{absolute} time $t=t(\tilde{H})$ at the zero phase direction; it is the fastest clock and useful for normalization between different phase speeds. Along the local real direction,
\begin{equation}
dx_0=\frac{d}{d\chi}\Re(\vec{\chi})\,d\chi
=\frac{\vec{\chi}^++\vec{\chi}^-}{2}\,d\chi
=\cos\zeta\,d\chi
=:d\tau.
\label{eq:321}
\end{equation}
Here $d\chi_0:=\cos\zeta\,d\chi$ is the projection onto the local real axis; in Sec.~\ref{sec:norm} we calibrate $d\tau=(1/\nu_0)\,d\chi_0$. The observed proper time of $A$ relative to the rest observer is
\begin{equation}
\tilde{H}_A=\Re\!\left(\frac{d\vec{\chi}_A}{d\vec{\chi}_0}\right)
=\cos\theta_A
=\sqrt{1-\sin^2\theta_A}
=\sqrt{1-\frac{V^2}{\mathtt{c}^2}}
=\frac{1}{\gamma}.
\label{eq:322}
\end{equation}

\subsection{Normalization}\label{sec:norm}
Let local time be parameterized by \emph{phase}; introduce a reference frequency $\nu_0$ and set
\begin{equation}
d\tau=\frac{1}{\nu_0}\,d\chi_0.
\label{eq:331}
\end{equation}
By the chain rule,
\begin{equation}
dx_0=\tilde{H}\,d\chi
=\frac{dx_0}{d\chi_0}\frac{d\chi_0}{d\tau}\,d\tau
=\tilde{H}\,\dot{\chi}\,d\tau
=: \dot{H}\,d\tau,
\label{eq:332}
\end{equation}
where $\nu:=d\chi/d\tau$, $\dot{\chi}:=\nu/\nu_0$, and $\dot{H}:=\tilde{H}\,\dot{\chi}$. Choosing the calibration $\dot{H}\equiv \mathtt{c}$ gives $dx_0=\mathtt{c}\,d\tau$. Similarly for space,
\begin{equation}
dx_l=\tilde{L}\,d\chi
=\frac{dx_l}{d\chi_0}\frac{d\chi_0}{dl}\,dl
=\tilde{L}\,\chi'\,dl
=:L'\,dl,\qquad \chi':=\frac{d\chi}{dl}.
\label{eq:333}
\end{equation}
From $dx_0=dx_l$ for light one gets
\begin{equation}
\mathtt{c}=\tilde{L}'\,\frac{dl}{d\tau},
\label{eq:334}
\end{equation}
hence with temporal calibration to $\mathtt{c}$ the spatial scale becomes unit: $\tilde{L}'=1$.

\subsection{Light and $\mathtt{c}$ as a calibration constant}
From the normalized forms,
\begin{equation}
\frac{\mathtt{c}}{\dot{\chi}}\,d\chi=\frac{1}{\chi'}\,d\chi
\quad\Rightarrow\quad
\mathtt{c}=\frac{\dot{\chi}}{\chi'}=\frac{dl}{d\tau},
\label{eq:341}
\end{equation}
i.e.\ $\mathtt{c}$ is a \emph{calibration constant} tying temporal and spatial measures, independent of local phase variation. Equation \eqref{eq:341} also reads
\begin{equation}
\mathtt{c}=\left(\frac{d\chi}{d\tau}\right)\!\left[\frac{dl}{d\chi}\right]\sim (\nu)\,[\lambda],
\label{eq:343}
\end{equation}
matching frequency and wavelength of a photon, with $\chi$ as its phase. For a lightlike trajectory,
\begin{equation}
ds^2=\mathtt{c}^2\!\left(\frac{d\chi^2}{\dot{\chi}^2}-\frac{d\chi^2}{\dot{\chi}^2}\right)=0.
\label{eq:344}
\end{equation}
At unit frequency, $\tau=\chi$: the photon's ``proper time'' is its phase, and the length of its phase-speed vector equals its wavelength, $\tilde{H}_p=\lambda$. Finally, the kinematic slope in phase coordinates is
\begin{equation}
\frac{dx_l}{dx_0}
=\frac{\tilde{L}\,d\chi}{\tilde{H}\,d\chi}
=\sin\theta
=\frac{V}{\mathtt{c}}
\equiv \beta,
\label{eq:345}
\end{equation}
so $\theta=\pi/2$ implies $V=\mathtt{c}$.

\subsection{Lorentz factor via reparametrization}
A change of direction of the phase speed transforms
\begin{equation}
\tilde{H}^2=\tilde{S}^2+\tilde{L}^2 \;\longmapsto\;
\dot{H}^2=\dot{S}^2+\dot{L}^2.
\label{eq:351}
\end{equation}
\textbf{Lemma (parameter-change identity).} The transition $\tilde{H}\to\dot{S}$ is the manifestation of evolving phase speed under the parameter change $\chi\mapsto \tau(\chi)$, with local Jacobian
\begin{equation}
\frac{d\tau}{d\chi}=\cos\zeta(\chi)\cos\theta(\chi)
\quad\Rightarrow\quad
\mathcal{J}(\zeta,\theta):=\frac{d\chi}{d\tau}=\frac{1}{\cos\zeta\,\cos\theta}.
\label{eq:353}
\end{equation}
Then
\begin{equation}
\dot{H}=\tilde{H}\,\mathcal{J},\qquad \dot{L}=\tilde{L}\,\mathcal{J}.
\label{eq:354}
\end{equation}
In differential form,
\begin{equation}
d\ln\dot{H}=d\ln\mathcal{J}=\tan\zeta\,d\zeta+\tan\theta\,d\theta.
\label{eq:355}
\end{equation}
For a \emph{pure boost} ($d\zeta=0$) one has $d\dot{H}=\dot{H}\tan\theta\,d\theta$. Absorbing a constant $\cos\zeta$ into the calibration (set $\zeta=0$ henceforth), we obtain
\begin{equation}
\tilde{H}^2=\dot{H}^2-\dot{L}^2=\sec^2\theta\,(\tilde{H}^2-\tilde{L}^2)=\gamma^2(\tilde{H}^2-\tilde{L}^2).
\label{eq:356}
\end{equation}
\textbf{Corollary.} In phase space the Euclidean norm $\tilde{H}$ is conserved; in observed time the Minkowski norm $\dot{S}$ is conserved; they are identical as quantities:
\begin{equation}
\boxed{\ \tilde{H}=\dot{S}\ }.
\label{eq:357}
\end{equation}

\subsection{Rapidity and the phase angle}
By definition,
\begin{equation}
\beta=\frac{V}{\mathtt{c}}=\sin\theta,\qquad \tanh\eta=\beta,\qquad
d\eta=\frac{d\beta}{1-\beta^2}.
\label{eq:361}
\end{equation}
With $d\beta=\cos\theta\,d\theta$ and $1-\beta^2=\cos^2\theta$,
\begin{equation}
d\eta=\sec\theta\,d\theta,\qquad
\eta(\theta)=\int\sec\theta\,d\theta
=\ln|\sec\theta+\tan\theta|
=\tfrac12\ln\frac{|1+\sin\theta|}{|1-\sin\theta|}.
\label{eq:364}
\end{equation}
Fixing $\eta(0)=0$,
\begin{equation}
e^{\eta(\theta)}=\sqrt{\frac{1+\sin\theta}{1-\sin\theta}},\qquad
\gamma=\frac{1}{\sqrt{1-\beta^2}}=\sec\theta=\cosh\eta.
\label{eq:365}
\end{equation}

\paragraph{Remark (groups).} Observables satisfy $\beta=\sin\theta=\tanh\eta$ and $\gamma=\sec\theta=\cosh\eta$. Thus Euclidean rotations in the phase circle (\,$U(1)$ with angle $\theta$\,) reproduce the numerical factors of hyperbolic boosts in $SO^+(1,1)$ (rapidity $\eta$) \emph{after} reparametrizing time. We do not claim an isomorphism $U(1)\cong SO(1,1)$; only the equality of observable combinations under the change of parameter.

\subsection{Velocity addition}
Rapidity is additive:
\begin{equation}
\eta_{12}=\eta_1+\eta_2,\qquad
\beta_{12}=\tanh(\eta_1+\eta_2)=\frac{\beta_1+\beta_2}{1+\beta_1\beta_2}.
\label{eq:372}
\end{equation}
Equivalently,
\begin{equation}
\gamma_{12}=\gamma_1\gamma_2(1+\beta_1\beta_2),\qquad
\gamma_{12}\beta_{12}=\gamma_1\gamma_2(\beta_1+\beta_2).
\label{eq:374}
\end{equation}

\subsection{Doppler shift}
Define the observed frequency as the phase growth rate in the observer's proper time:
\begin{equation}
\nu:=\frac{d\chi}{d\tau}.
\label{eq:381}
\end{equation}
For two successive wavefronts the phase increment is identical, hence
\begin{equation}
\frac{\nu_{\mathrm{obs}}}{\nu_{\mathrm{src}}}
=\frac{d\chi/d\tau_{\mathrm{obs}}}{d\chi/d\tau_{\mathrm{src}}}
=\frac{d\tau_{\mathrm{src}}}{d\tau_{\mathrm{obs}}}.
\label{eq:382}
\end{equation}
Longitudinal case: during $\gamma\,d\tau_{\mathrm{src}}$ in the observer frame the source displaces by $\pm V\,\gamma\,d\tau_{\mathrm{src}}$ (``$+$'' receding, ``$-$'' approaching). Then
\begin{equation}
d\tau_{\mathrm{obs}}=\gamma\,d\tau_{\mathrm{src}}(1\pm\beta),\qquad
\Rightarrow\quad
\boxed{\ \frac{\nu_{\mathrm{obs}}}{\nu_{\mathrm{src}}}=\frac{1}{\gamma(1\pm\beta)}\ }.
\label{eq:384}
\end{equation}
Equivalent forms (with $\beta=\sin\theta$, $\gamma=\sec\theta$ and rapidity $\eta$):
\begin{equation}
\frac{\nu_{\mathrm{obs}}}{\nu_{\mathrm{src}}}
=\sqrt{\frac{1\mp\beta}{1\pm\beta}}
=\sec\theta\,(1\mp\sin\theta)
=e^{\mp\eta}.
\label{eq:385}
\end{equation}
Transverse Doppler ($\varphi=90^\circ$ in observer frame):
\begin{equation}
\frac{\nu_{\mathrm{obs}}}{\nu_{\mathrm{src}}}=\frac{1}{\gamma}=\cos\theta.
\label{eq:389}
\end{equation}
General angle $\varphi$ (in observer frame):
\begin{equation}
\boxed{\ \frac{\nu_{\mathrm{obs}}}{\nu_{\mathrm{src}}}=\gamma\,(1-\beta\cos\varphi)\ }.
\label{eq:3810}
\end{equation}
Wavelength ratios are inverse to frequency ratios.

\section{Discussion: links to known structures}
\paragraph{Gauge phases.} A global shift $\chi\mapsto\chi+\chi_0$ is unobservable. Allowing local reparametrizations $\chi\mapsto\chi+\alpha(x)$ induces a connection when comparing phases at different points. On wavefunctions $\psi\sim e^{i\chi}$ this is the familiar $U(1)$ gauge freedom $\psi\to e^{i\alpha(x)}\psi$ with $D_\mu=\partial_\mu-iA_\mu$ as the \emph{geometric} phase-transport operator.

\paragraph{Mass and the internal angle.} With the decomposition by $\zeta$, mass heuristically correlates with an irreducible real projection: massless objects have $\zeta=\pm\pi/2$ (no proper time; photon subspace), while massive objects have $|\zeta|<\pi/2$ (proper time exists). In the present paper we set $\zeta=0$ in boost kinematics by calibration; a detailed mass-generation mechanism is left for future work.

\paragraph{Cosmological gauge.} A natural global calibration of ``absolute'' time is the comoving frame with vanishing CMB dipole. This fixes a cosmological time $t$ (FLRW) as a gauge, without affecting local Lorentz invariance; Doppler factors are then operationally referenced to that frame.

\section{Conclusion}
In unimetry, time and space are integrals of phase velocities; the Minkowski interval appears as a conserved quantity under parameter change. The core relations of SR---$\gamma$, rapidity, velocity addition, and Doppler factors---follow from elementary phase-plane geometry with a single rotation angle $\theta$, while hyperbolic structure re-emerges upon reparametrizing time. The formalism is empirically equivalent to standard SR but can clarify causality and composition by treating all effects as projections of a single flow.

\paragraph{Outlook.} Future directions include (i) a more explicit group-theoretic embedding, (ii) a rigorous treatment of the internal angle $\zeta$ and its relation to mass, and (iii) exploration of curved metrics as spatially varying Jacobians $\mathcal{J}(x)$ in the phase-to-observable map.

\bibliographystyle{plain}
\begin{thebibliography}{9}

\bibitem{Einstein1905}
A.~Einstein.
\newblock Zur Elektrodynamik bewegter K\"{o}rper.
\newblock {\em Annalen der Physik}, 17:891--921, 1905.
(English translation: On the electrodynamics of moving bodies.)

\bibitem{Rindler}
W.~Rindler.
\newblock {\em Relativity: Special, General, and Cosmological}.
\newblock Oxford University Press, 2nd ed., 2006.

\bibitem{TaylorWheeler}
E.~F. Taylor and J.~A. Wheeler.
\newblock {\em Spacetime Physics}.
\newblock W. H. Freeman, 2nd ed., 1992.

\end{thebibliography}

\end{document}
