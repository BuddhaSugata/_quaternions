\documentclass[11pt,a4paper]{article}

% ====== Packages (adapt / sync with the rev.2 paper as needed) ======
\usepackage[utf8]{inputenc}
\usepackage[T1]{fontenc}
\usepackage[english]{babel}

\usepackage{lmodern}
\usepackage{microtype}

\usepackage{geometry}
\geometry{margin=2.5cm}

\usepackage{amsmath,amssymb,amsfonts,mathtools}
\usepackage{bm}
\usepackage{physics}
\usepackage{siunitx}
\usepackage{tensor}
\usepackage{csquotes}

\usepackage{graphicx}
\usepackage{xcolor}
\usepackage{tikz}
\usepackage{tikz-cd}

\usepackage{hyperref}
\hypersetup{
  colorlinks=true,
  linkcolor=blue!40!black,
  citecolor=blue!40!black,
  urlcolor=blue!60!black
}

\usepackage[nameinlink,capitalize]{cleveref}

\usepackage{enumitem}

% ====== Common macros (adapt from rev.2 as needed) ======
% TODO: copy or sync macros from the rev.2 unimetry paper here.
\newcommand{\clight}{c}
\newcommand{\GNewton}{G}
\newcommand{\epszero}{\varepsilon_0}
\newcommand{\muzero}{\mu_0}

\newcommand{\R}{\mathbb{R}}
\newcommand{\Espace}{\mathcal{E}}

\newcommand{\qh}{\hat h}  % temporal unit
\newcommand{\qvec}[1]{\bm{#1}}

\title{Unimetry: A Quaternionic Gravito--Electromagnetic Formulation}
\author{Timur Abizgeldin%
  \thanks{Email: \texttt{foo@example.org}. TODO: update contact/affiliation.}}
\date{\today}

\begin{document}

\maketitle

\begin{abstract}
% TODO: Briefly describe the goal of the paper and the main idea:
% quaternionic representation of flows in unimetry, metrically dressed body quaternions,
% bilinear forms A,B,C, D-rotors, static limit (Newton + Coulomb),
% the role of C_{\text{phys}} = \clight\,\mathbf B and of \GNewton, \epszero, \muzero, \clight.
\end{abstract}

\tableofcontents

%==================== 1. Introduction ====================

\section{Introduction}
\section{Introduction}

\subsection{Context and motivation}

Unimetry is a phase--space reformulation of special relativity (SR) in which the 
fundamental objects are not worldlines in a pre-given Minkowski space, but 
stationary flows in an underlying Euclidean proto-space \(\Espace\).
A dimensionless scalar phase potential \(\Phi:\Espace\to\mathbb{R}\) and its 
gradient define a normalized flow direction, while the familiar Minkowski metric 
and relativistic kinematics emerge as effective descriptions of how this flow is 
seen by observers tied to local ``rest frames'' of massive objects.
In this sense, unimetry treats space--time not as a primitive arena, but as a 
derived structure obtained from the orientation of a flow in phase space.

The present paper extends this viewpoint to the problem of unifying gravity and 
electromagnetism (EM).
At the classical level, gravito--electromagnetic (GEM) analogies are well known:
in the weak-field, slow-motion limit of general relativity, the Einstein equations 
can be written in a Maxwell-like form, and moving masses generate a 
``gravitomagnetic'' field.
Similarly, quaternions and related formalisms have long been used to compactly 
encode rotations and the Maxwell equations.
What unimetry adds to this landscape is a concrete physical interpretation:
a single quaternionic object encodes both the temporal and spatial parts of a 
flow, and bilinear forms of such objects naturally split into scalar, symmetric 
vector, and axial (vorticity-like) channels.

Our goal here is to exploit this structure to construct a quaternionic GEM 
formulation in which gravitational and electromagnetic interactions arise from 
the \emph{same} bilinear machinery acting on appropriately dressed ``body 
quaternions''.
In particular, we show that Newton and Coulomb potentials appear as two branches 
of a single scalar form, while the magnetic and gravitomagnetic sectors are 
associated with a vortical bilinear form whose physical calibration reveals a 
natural role for the constants \(\epszero\), \(\muzero\), \(\GNewton\) and 
\(\clight\).
The resulting picture is entirely Euclidean at the level of the proto-space, yet 
reproduces relativistic kinematics and GEM fields in the observable three-space.

\subsection{Relation to the base unimetry paper}

This work is a direct sequel to the base unimetry paper, 
\emph{``Unimetry: A Phase-Space Reformulation of Special Relativity''}
(henceforth ``Paper~I'').
In Paper~I the phase potential \(\Phi\), the phase 1-form \(\alpha = d\Phi\), the 
normalized flow vector \(\widehat{\boldsymbol\chi}\), and the calibration 
\(\boldsymbol\chi = \clight\,\widehat{\boldsymbol\chi}\) are introduced in 
detail, together with the derivation of the Minkowski interval and standard SR 
effects from a purely Euclidean proto-space.
The unimetrical D-rotation, which encodes Lorentz boosts as Euclidean rotations 
in a suitable plane of the proto-space, is also defined there.

The present paper assumes familiarity with the conceptual framework of Paper~I, 
but is written to be as self-contained as reasonably possible.
We briefly recall the key definitions of the phase proto-space, the flow vector, 
and the two calibrations of the flow that lead to kinematic and energetic 
interpretations.
All proofs and derivations that are essential for the GEM construction are 
reproduced or adapted here; more detailed discussions of SR and cosmological 
applications remain in Paper~I and are only referenced when needed.

\subsection{Main results}

The main technical contributions of this paper can be summarized as follows.

\begin{itemize}[leftmargin=*]
  \item We introduce \emph{metrically dressed body quaternions}
  \(\widetilde{\mathsf q}_i = L_{E,i}\,\qh + L_{G,i}\,\widehat{\bm n}_i\),
  whose components have the dimension of length.
  The ``electric'' and ``gravitational'' lengths
  \[
    L_{E,i} = \sqrt{\frac{\GNewton}{4\pi \epszero \clight^4}}\,Q_i,
    \qquad
    L_{G,i} = \frac{\GNewton}{\clight^2}\,m_i
  \]
  encode the charge \(Q_i\) and mass \(m_i\) of the body in a unified geometric 
  fashion.
  The unit vector \(\widehat{\bm n}_i\) represents the spatial flow direction 
  associated with the body.
  
  \item We show that the scalar bilinear form
  \[
    A(\widetilde{\mathsf q}_1,\widetilde{\mathsf q}_2)
    = L_{E,1}L_{E,2} - \bm S_1\cdot\bm S_2
  \]
  (with \(\bm S_i = L_{G,i}\widehat{\bm n}_i\)) yields, after a single global 
  calibration by \(\clight^4/\GNewton\) and a geometric \(1/r\) factor, the 
  combined Newton--Coulomb potential:
  \[
    U(r) = \frac{\clight^4}{\GNewton}\,\frac{A}{r}
    = \frac{1}{4\pi\epszero}\,\frac{Q_1 Q_2}{r}
      - \GNewton\,\frac{m_1 m_2}{r}.
  \]
  Thus gravity and electrostatics arise as two channels of a single invariant 
  scalar form.

  \item We identify two vector-valued bilinear forms, 
  \(\bm B(\widetilde{\mathsf q}_1,\widetilde{\mathsf q}_2)\) and 
  \(\bm C(\widetilde{\mathsf q}_1,\widetilde{\mathsf q}_2)\), corresponding to 
  the symmetric and axial parts of the quaternion product.
  In the dressed setting these naturally describe current-like and vortical 
  channels.
  In particular, the vortical form \(\bm C\) reproduces the geometry of 
  magnetic and gravitomagnetic fields generated by moving charges and masses.

  \item We construct a quaternionic GEM field 
  \(\mathcal F_{\rm GEM}(\bm x)\) over the observable three-space by combining 
  dressed source quaternions with purely imaginary distance quaternions.
  Its scalar channel reproduces the gravitational and electrostatic potentials, 
  while its vortical channel yields a physically natural ``phase-vortical'' 
  field \(C_{\text{phys}}\) with the same dimension as \(\bm E\).
  The standard magnetic field \(\bm B\) in SI units then appears as
  \[
    \bm B = \frac{1}{\clight}\,C_{\text{phys}},
  \]
  so that the familiar \(\muzero\) and \(\epszero\) can be interpreted in terms 
  of linear and areal stiffness of the vacuum, combined into an effective 
  volumetric stiffness proportional to \(1/(\epszero \clight^3)\).

  \item We analyze the action of unimetrical D-rotations and ordinary spatial 
  rotors on dressed quaternions.
  Pure spatial rotations act in the usual way on the vector channels and leave 
  the scalar form \(A\) invariant, while D-rotations mix the scalar channel and 
  the longitudinal component of \(\bm B\) in a two-dimensional 
  ``energy--current'' plane.
  This provides a quaternionic encoding of relativistic kinematics in the GEM 
  setting, with Lorentz-consistent transformation properties of the fields.

  \item Finally, we outline a Hamiltonian and Lagrangian formulation of the 
  quaternionic GEM theory in terms of the self-form \(A\) and the norm-squares 
  of \(\bm B\) and \(\bm C\), and discuss how the standard Maxwell Lagrangian 
  and linearized GEM equations arise in appropriate limits.
\end{itemize}

\subsection{Structure of the paper}

The paper is organized as follows.
In \cref{sec:quat-algebra} we recall the basic quaternion algebra and introduce 
the bilinear forms \(A\), \(\bm B\), and \(\bm C\) that arise from the 
quaternion product, together with their matrix representation and geometric 
interpretation.
\Cref{sec:phase-protospace} provides a brief overview of the unimetrical phase 
proto-space, the phase potential, the flow vector, and the two calibrations of 
the flow that lead to kinematic and energetic interpretations.

In \cref{sec:dressed-quats} we introduce metrically dressed body quaternions and 
define the electric and gravitational lengths \(L_E\) and \(L_G\).
\Cref{sec:bilinear-static-gem} shows how the scalar form \(A\) for dressed 
quaternions reproduces the static Newton and Coulomb potentials.
In \cref{sec:quat-gem-field} we construct a quaternionic GEM field over the 
observable three-space and identify the scalar and vortical channels with 
gravitational, electric, and magnetic sectors.

\Cref{sec:drotors-dynamics} analyzes the action of spatial rotors and D-rotors 
on dressed quaternions and on the GEM field, clarifying the relativistic 
transformation properties of the scalar, current-like, and vortical channels.
\Cref{sec:calibration-EB-Cphys} is devoted to the calibration of \(\bm E\) and 
\(\bm B\), to the definition of the phase-vortical field \(C_{\text{phys}}\), 
and to the interpretation of \(\epszero\), \(\muzero\), and \(\clight\) in 
terms of vacuum stiffness.

In \cref{sec:hamiltonians-lagrangians} we outline Hamiltonian and Lagrangian 
formulations of quaternionic GEM, and in 
\cref{sec:maxwell-gem-comparison} we compare the resulting equations with the 
standard Maxwell and linearized GEM formalisms.
Finally, \cref{sec:discussion-conclusions} discusses limitations and open 
questions, and sketches possible extensions towards non-Abelian interactions 
and cosmological applications.

\subsection{Context and motivation}
% TODO: Explain the general idea of unimetry as a phase/flow reformulation of SR,
% and why it is natural to try to unify gravity and electromagnetism
% in a single quaternionic formalism.

\subsection{Relation to the base unimetry paper}
% TODO: Briefly describe that the core phase/flow structure
% is developed in ``Unimetry: A Phase-Space Reformulation of Special Relativity (rev.2)''
% (Paper~I). Clarify which elements are briefly recalled here, and which are only referenced.

\subsection{Main results}
% TODO: List the main technical results:
% - metrically dressed body quaternions;
% - forms A, B, C and their interpretation;
% - static GEM (Newton + Coulomb) from a single scalar form;
% - D-rotors and relativistic dynamics;
% - calibration of \bm E, \bm B and the object C_{\text{phys}} = \clight\,\mathbf B.

\subsection{Structure of the paper}
% TODO: Briefly outline the structure of the paper section by section.

%==================== 2. Quaternion algebra ====================

\section{Quaternion algebra and bilinear forms}

\subsection{Basic notation and conventions}
% TODO: Define a quaternion q = T \hat h + \bm S (scalar + vector part),
% conventions on signs, conjugation, norm, basis units,
% and the identification of \bm S with vectors in \mathbb R^3.

\subsection{Quaternion product and decomposition into A, B, C forms}
% TODO: Write explicitly
% q_1 q_2 = (T_1 T_2 - \bm S_1 \cdot \bm S_2)
%         + (T_1 \bm S_2 + T_2 \bm S_1 + \bm S_1 \times \bm S_2).
% Define
% A(q_1,q_2), \bm B(q_1,q_2), \bm C(q_1,q_2).

\subsection{Matrix representation of quaternions and forms}
% TODO: Give a 2x2 complex matrix representation (as in quats.md),
% and show how the product and the forms A,B,C appear in matrix language.

\subsection{Geometric interpretation of A, B, C}
% TODO: Explain A as a Minkowski-like invariant,
% \bm B as a ``current-like'' symmetric channel,
% and \bm C as a vorticity (axial) channel.

%==================== 3. Phase proto-space in unimetry ====================

\section{Phase proto-space and flow: brief unimetry overview}

\subsection{Proto-space, phase potential and 1-form}
% TODO: Introduce (\Espace,\langle\cdot,\cdot\rangle), the phase potential \Phi,
% the phase 1-form \alpha = d\Phi, and the gradient \nabla\Phi.

\subsection{Flow vector and normalization}
% TODO: Introduce \widehat{\boldsymbol\chi} = \nabla\Phi / \|\nabla\Phi\|,
% then \boldsymbol\chi = \clight\,\widehat{\boldsymbol\chi}, with \|\boldsymbol\chi\| = \clight.

\subsection{Interval, proper time and correspondence with SR}
% TODO: Briefly indicate how the flow \boldsymbol\chi gives rise to the interval,
% proper time, and how this corresponds to the Minkowski metric in SR.

\subsection{Notation table}
% TODO: Insert a compact table of main symbols (\Phi, \alpha, \boldsymbol\chi, \zeta, etc.),
% adapted from Paper I (rev.2).

%==================== 4. Calibrations of the flow and energy ====================

\section{Flow calibrations and energy-like functionals}

\subsection{Phase (kinematic) calibration}
% TODO: Describe calibration in terms of phase/velocity,
% the relation between \zeta and \beta = v/\clight,
% and how Lorentz boosts appear geometrically in this picture.

\subsection{Proper-time (energetic) calibration}
% TODO: Introduce calibration with respect to proper flow (proper time or phase parameter),
% and relate it to an energy-like functional.

\subsection{Energy of a flow and the self-form A(q,q)}
% TODO: Show that A(q,q) naturally plays the role of a free Hamiltonian
% / energy invariant for a single flow, and relate this to proper frequency
% and to an E \sim \kappa \clight^3 scaling in unimetry.

%==================== 5. Metrically dressed body quaternions ====================

\section{Metrically dressed body quaternions}

\subsection{Dimensionless flow quaternion and dressed body quaternion}
% TODO: Distinguish between a dimensionless flow quaternion \widehat{\mathsf q}
% and a metrically dressed body quaternion \widetilde{\mathsf q} with dimension of length.

\subsection{Lengths L\_E and L\_G: definitions and dimensional analysis}
% TODO: Define
% L_{E,i} = \sqrt{\GNewton/(4\pi\epszero \clight^4)}\,Q_i,
% L_{G,i} = (\GNewton/\clight^2)\,m_i,
% and discuss dimensions and physical meaning (``electric length'' and gravitational radius).

\subsection{Self-form of a dressed quaternion and rest energy}
% TODO: Consider A(\widetilde{\mathsf q}_i,\widetilde{\mathsf q}_i)
% and relate it to rest energy / the mass channel.

%==================== 6. Bilinear forms and static GEM ====================

\section{Bilinear forms for dressed bodies and static GEM}

\subsection{Forms A, B, C for dressed body quaternions}
% TODO: Write A,B,C explicitly for \widetilde{\mathsf q}_1,\widetilde{\mathsf q}_2
% and discuss the interpretation of each component in the dressed context.

\subsection{Static interaction energy of two bodies}
% TODO: Derive A_0 = A(\widetilde{\mathsf q}_1,\widetilde{\mathsf q}_2),
% define U(r) = (\clight^4/\GNewton)\,A_0 / r,
% and show that this reproduces the Coulomb + Newton potentials.

\subsection{Multi-body field and linear regime}
% TODO: Discuss superposition of many sources and the linear classical limit.

%==================== 7. Quaternionic GEM field in R^3 ====================

\section{Quaternionic GEM field in the observable three-space}

\subsection{Distance as a purely imaginary quaternion}
% TODO: Introduce R_j(\bm x) = \bm r_j and R_j^{-1} = -\widehat{\bm r}_j / r_j.

\subsection{Potential-like object \texorpdfstring{$\mathcal U_{\rm GEM}$}{U\_GEM}}
% TODO: Define \mathcal U_{\rm GEM}(\bm x) = \sum_j R_j^{-1}\,\widetilde{\mathsf q}_j.

\subsection{Field \texorpdfstring{$\mathcal F_{\rm GEM}$}{F\_GEM} and extraction of potentials}
% TODO: Define \mathcal F_{\rm GEM}(\bm x) = \sum_j R_j^{-1}\,\widetilde{\mathsf q}_j\,R_j^{-1},
% and explain how its A-channel yields the gravitational and electrostatic potentials.

%==================== 8. D-rotors and dynamics ====================

\section{D-rotors and relativistic dynamics of GEM}

\subsection{Definition of the D-rotation and its action on a flow}
% TODO: Introduce d(\zeta) = \cos(\zeta/2) + \widehat{\bm u}\sin(\zeta/2),
% and the action q' = d q d,
% with explicit formulas for T', S_u', \bm S_\perp'.

\subsection{Spatial rotors and invariance of A}
% TODO: Show that a purely spatial rotor R
% leaves A(q_1,q_2) invariant,
% while \bm B and \bm C transform as ordinary 3-vectors.

\subsection{Mixing of A and B channels under a D-rotation}
% TODO: Show how the pair (U,B_u) = (T_1T_2 - S_{u,1}S_{u,2}, B_u)
% rotates by an angle 2\zeta in the (energy–current) plane,
% and interpret this as the dynamical mixing of scalar and current-like channels.

\subsection{Example configurations}
% TODO: Discuss:
% - a purely gravitational channel (L_E = 0),
% - a purely electric channel (L_G \to 0),
% - transverse motion of a charge and the emergence of a vortical EM part.

%==================== 9. Calibration of E, B and C_phys ====================

\section{Calibration of fields \texorpdfstring{$\bm E$}{E}, \texorpdfstring{$\bm B$}{B}
and the phase-vortical object \texorpdfstring{$C_{\text{phys}}$}{Cphys}}

\subsection{Electric field from the A-channel}
% TODO: Derive \bm E(\bm r) = (1/(4\pi\epszero))\,Q\,\widehat{\bm r} / r^2
% from the scalar A-channel in the static limit.

\subsection{The vortical form C and the geometry of \texorpdfstring{$C_{\text{geom}}$}{Cgeom}}
% TODO: Show that C_{\text{geom}} \sim (Q/r^2)\,\beta\,(\widehat{\bm v} \times \widehat{\bm r})
% for a moving charge.

\subsection{Definition of \texorpdfstring{$C_{\text{phys}}$}{Cphys} and relation to \texorpdfstring{$\bm B$}{B}}
% TODO: Define C_{\text{phys}} := (1/(4\pi\epszero))\,C_{\text{geom}},
% note that its dimension matches that of \bm E,
% and write \bm B = C_{\text{phys}} / \clight in SI units.

\subsection{Interpretation of \texorpdfstring{$\epszero$}{eps0}, \texorpdfstring{$\muzero$}{mu0}
and volumetric stiffness of the vacuum}
% TODO: Discuss \muzero = 1/(\epszero \clight^2)
% and \muzero/\clight = 1/(\epszero \clight^3) as an effective ``volumetric stiffness''
% of the vortical channel of the vacuum.

%==================== 10. Hamiltonians and Lagrangians ====================

\section{Hamiltonians and Lagrangians of quaternionic GEM}

\subsection{Free-body Hamiltonian}
% TODO: Introduce H_{\rm free} = \kappa A(\widetilde{\mathsf q}_i,\widetilde{\mathsf q}_i),
% and relate it to the rest energy and proper frequency of a body.

\subsection{Interaction Hamiltonian}
% TODO: Introduce H_{\rm int} = (\clight^4/\GNewton)\,
% A(\widetilde{\mathsf q}_1,\widetilde{\mathsf q}_2)/r
% and decompose it into gravitational and EM contributions.

\subsection{Field Lagrangian in terms of A, B, C}
% TODO: Sketch a field Lagrangian in terms of A(Q,Q), \|B(Q,Q)\|^2, \|C(Q,Q)\|^2,
% and relate it to the standard EM Lagrangian and linearized GEM.

%==================== 11. Relation to Maxwell and GEM ====================

\section{Relation to Maxwell's equations and linearized GEM}

\subsection{Static equations and Poisson-type equations}
% TODO: Show how Poisson's equation for \phi_{\rm EM}
% and its gravitational analogue arise from the quaternionic field.

\subsection{Quasi-stationary regime and continuity equations}
% TODO: Show how, in the quasi-stationary regime,
% one recovers \nabla \cdot \bm E, \nabla \times \bm B,
% and continuity equations for charge and mass currents.

\subsection{Comparison with the classical GEM formalism}
% TODO: Discuss the limits L_E = 0 (pure gravity),
% L_G = 0 (pure EM), and correspondence with standard GEM formulations.

%==================== 12. Discussion and conclusions ====================

\section{Discussion and conclusions}

\subsection{Summary of results}
% TODO: Summarize the main achievements and emphasize
% the unified quaternionic language for gravity and EM.

\subsection{Limitations and open questions}
% TODO: Discuss assumptions (classical limit, linearity, etc.)
% and highlight open problems.

\subsection{Perspectives}
% TODO: Sketch possible extensions:
% inclusion of weak/strong interactions,
% quantization of the quaternionic field,
% links to cosmology and possible evolution of \GNewton, etc.

%==================== Appendices ====================

\appendix

\section{Quaternion algebra and matrix representation (details)}
% TODO: Provide more detailed algebraic and matrix derivations.

\section{Extended notation table}
% TODO: Full table of symbols used in the paper.

\section{Dimensional analysis and numerical estimates}
% TODO: Work out explicit examples for the electron, proton, etc.

\section{D-rotors, Lorentz transformations and Wigner--Thomas rotation}
% TODO: Show in detail how the D-rotor formalism reproduces Lorentz boosts
% and the Wigner--Thomas rotation.

\end{document}
