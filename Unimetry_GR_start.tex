\documentclass[12pt,a4paper]{article}
\usepackage{amsmath,amssymb,mathtools}
\usepackage[hidelinks]{hyperref}

% --- Notation (narrow tildes) ---
\newcommand{\tH}{\tilde{H}}
\newcommand{\tS}{\tilde{S}}
\newcommand{\tL}{\tilde{L}}
\newcommand{\tchi}{\tilde{\chi}}
\newcommand{\tzeta}{\tilde{\zeta}}

% convenience
\newcommand{\dd}{\mathrm{d}}
\newcommand{\R}{\mathbb{R}}
\newcommand{\T}{\mathbb{T}}

\begin{document}

% \title{~} % intentionally left empty for now
% \author{}
% \date{}
% \maketitle

\section*{1.\;Time normalization as a function of phase speed}

Consider the 3--sphere $S^3_R\subset\R^4$ in Hopf coordinates $(\xi_0,\xi_1,\xi_2)$:
\[
x_1=R\cos\xi_0\cos\xi_1,\quad
x_2=R\cos\xi_0\sin\xi_1,\quad
x_3=R\sin\xi_0\cos\xi_2,\quad
x_4=R\sin\xi_0\sin\xi_2.
\]
Fixing $\xi_0=\xi_0^\star$ yields a flat torus $\T^2$ with radii
\[
R_1=R\cos\xi_0^\star,\qquad R_2=R\sin\xi_0^\star,
\]
and metric $ds^2=R_1^2\,\dd\xi_1^2+R_2^2\,\dd\xi_2^2$.
Introduce arc--lengths along the two circles,
\[
\chi:=R_1\xi_1,\qquad \zeta:=R_2\xi_2,
\]
with $\dot\chi=R_1\dot\xi_1$ and $\dot\zeta=R_2\dot\xi_2$. 
Free (geodesic) motion on $\T^2$ has Lagrangian $L=\tfrac12(R_1^2\dot\xi_1^2+R_2^2\dot\xi_2^2)$, hence conserved momenta
\[
p_1=R_1^2\dot\xi_1=\text{const},\qquad p_2=R_2^2\dot\xi_2=\text{const}.
\]
Define the phase speed in local time $\omega_\chi:=\dd\chi/\dd\tau=\dot\chi$ and the inter--fiber (``proto'') speed $\tH:=\dd\zeta/\dd\chi$. Then
\[
\frac{p_1}{p_2}=\frac{R_1^2\dot\xi_1}{R_2^2\dot\xi_2}
=\frac{R_1}{R_2}\frac{\dot\chi}{\dot\zeta}
=\frac{R_1}{R_2}\frac{\omega_\chi}{\dot\zeta}
=\frac{R_1}{R_2}\frac{\omega_\chi}{\tH\,\omega_\chi}
=\frac{R_1}{R_2}\frac{1}{\tH}.
\]
Equivalently,
\begin{equation}\label{eq:omega_kH}
\boxed{\ \omega_\chi \;=\; k\,\tH,\qquad k:=\frac{p_1}{p_2}\frac{R_1}{R_2}\ }.
\end{equation}
The \emph{time--normalization factor} (a 1--form) is, therefore,
\begin{equation}\label{eq:Theta}
\boxed{\ \Theta_\chi:=\frac{\dd\tau}{\dd\chi}=\frac{1}{\omega_\chi}=\frac{1}{k\,\tH}\ }.
\end{equation}

\section*{2.\;Equivalence of gauges (no contradictions)}

There are two natural, but mutually exclusive, gauge choices:
\[
\text{(}\chi\text{-gauge)}\quad \tH\equiv c,
\qquad
(\tau\text{-gauge})\quad \dot H\equiv c\quad (H\equiv\zeta).
\]
They are related by the invariant identity
\begin{equation}\label{eq:dotH_master}
\boxed{\ \dot H \;=\; \tH\,\dot\chi \;=\; k\,\tH^2\ }.
\end{equation}
Thus one must not set $c$ for both $\tH$ and $\dot H$ simultaneously (except in the special $k=1$ case). From \eqref{eq:dotH_master}:
\[
\begin{cases}
\text{$\chi$-gauge: } \tH\equiv c \ \Rightarrow\ \dot H = k\,c^2,\\[1mm]
\text{$\tau$-gauge: } \dot H\equiv c \ \Rightarrow\ \tH = \sqrt{c/k}.
\end{cases}
\]

\section*{3.\;Flow speed and energy in both gauges}

Define the \emph{phase--space energy} as the gauge--invariant scalar
\begin{equation}\label{eq:Energy_def}
\boxed{\ E \;:=\; k\,\tH^2\ }.
\end{equation}
Using \eqref{eq:dotH_master} we have $E=\dot H$ precisely in the $\chi$-gauge, while in the $\tau$-gauge $E=k\,c^2$:
\[
\begin{cases}
\text{$\chi$-gauge: } \tH\equiv c \ \Rightarrow\ E=k\,c^2=\dot H,\\[1mm]
\text{$\tau$-gauge: } \dot H\equiv c \ \Rightarrow\ E=k\,c^2,\ \ \dot H=c.
\end{cases}
\]
Hence $E=kc^2$ in either gauge, and $E$ is independent of which variable is held fixed by convention.

\paragraph{Interpretation.}
$k$ plays the role of an invariant mass scalar (winding/structure on the Hopf torus), while $c$ is the fixed inter--fiber speed. The observable flux along the time fiber is $\dot H$; it equals $E$ only in the $\chi$-gauge.

\section*{4.\;Generalized energy in phase space}

In curved or inhomogeneous settings $k$ and $\tH$ may vary with position along the flow on $S^3$; the local energy field is
\begin{equation}\label{eq:E_field}
\boxed{\ E(x)=k(x)\,\tH(x)^2\ }.
\end{equation}
For simple (photon--like) flows one has $k=0$; the energy is then carried by the phase frequency $\omega_\chi=\dd\chi/\dd\tau$. Introducing an action constant $\sigma$ (to be calibrated empirically),
\begin{equation}\label{eq:photon_branch}
\boxed{\ E_\gamma \;=\; \sigma\,\omega_\chi \ },
\end{equation}
so that the massive and simple branches are recovered as the limits of the unified ansatz
\begin{equation}\label{eq:unified_energy}
\boxed{\ E \;=\; k\,\tH^2 \;+\; \sigma\,\omega_\chi \ }.
\end{equation}
In the massive regime the first term dominates; for $k=0$ we get the pure frequency law $E_\gamma=\sigma\omega_\chi$.

% Optional remark: geodesic Hamiltonian on \T^2
\paragraph{Remark.}
The geodesic Hamiltonian on $\T^2$ reads $\mathscr{H}=\tfrac12(\dot\chi^2+\dot\zeta^2)=\tfrac12(1+k^2)\tH^2$, which is a kinematic invariant of the free motion; it should not be confused with the physical energy scalar $E$ in \eqref{eq:Energy_def}.
\end{document}
