\documentclass[11pt,a4paper]{article}
\usepackage[T1]{fontenc}
\usepackage[utf8]{inputenc}
\usepackage[english]{babel}
\usepackage{amsmath,amssymb,amsthm,mathtools}
\usepackage{bm}
\usepackage{hyperref}
\usepackage{geometry}
\geometry{margin=1in}

\newcommand{\HH}{\mathbb{H}}
\newcommand{\ImH}{\mathrm{Im}\,\mathbb{H}}
\newcommand{\Scal}{\mathrm{Scal}}
\newcommand{\uu}{\mathbf{u}}        % unit imaginary axis
\newcommand{\Sand}[2]{#1\,#2\,#1}   % sandwich without inverse (D)
\newcommand{\Snd}[2]{#1\,#2\,#1^{-1}} % sandwich with inverse (ordinary)

\title{\textbf{Unimetry: Quaternion Phase Space}\\
{\large The D-Rotation and Tangent Parametrization}}
\author{}
\date{}

\begin{document}
\maketitle

\begin{abstract}
We formulate the kinematic core of Unimetry using the quaternion D-rotation $Z \mapsto rZr$ acting on the phase state $\chi$.
We adopt the \emph{tangent parametrization} $\beta = \tan\vartheta$, which maps the physical light cone to the geometric angle $\vartheta = \pi/4$.
We demonstrate that the D-rotation naturally induces \textbf{longitudinal contraction} (scaling the axis of motion by $\cos\vartheta$) and \textbf{transverse invariance}, providing an algebraic realization of relativistic kinematics within a Euclidean phase space substrate.
\end{abstract}

\section{Quaternion Phase State}
Let $\uu \in \ImH$ be a unit imaginary quaternion ($\uu^2 = -1$) representing the direction of motion.
We represent the phase state (proto-parameter) as a quaternion:
\begin{equation}
  \chi \;=\; \tilde S + \uu\,\tilde L \;\in\; \mathrm{span}\{1, \uu\} \subset \HH.
\end{equation}
We define the kinematic angle $\vartheta$ via the projective slope of the state components:
\begin{equation}
  \beta \;:=\; \tan\vartheta \;=\; \frac{\tilde L}{\tilde S}.
\end{equation}
In this parametrization, the ``speed of light'' corresponds to $\beta=1$ ($\vartheta=\pi/4$), consistent with the null cone of the projector-based metric $g = 2NN - \delta$.

\section{The D-Rotation}
We distinguish two actions of unit quaternions on the algebra:
\begin{enumerate}
    \item \textbf{Ordinary Rotation (Form B):} $X \mapsto q X q^{-1}$. This rotates vectors within $\ImH$ and preserves the scalar part.
    \item \textbf{D-Rotation (Form A):} $Z \mapsto r Z r$. This is a fundamental operation of the algebra that mixes scalar and vector parts.
\end{enumerate}

Let $r$ be the rotor encoding a boost of angle $\vartheta$:
\begin{equation}
  r(\vartheta) \;=\; e^{\frac{\vartheta}{2}\uu} \;=\; \cos\frac{\vartheta}{2} + \uu\sin\frac{\vartheta}{2}.
\end{equation}

\section{Action on Space: Longitudinal Contraction}
Consider a spatial vector $X \in \ImH$. Decompose it into longitudinal ($X_\parallel \parallel \uu$) and transverse ($X_\perp \perp \uu$) components.
Applying the D-rotation $Z' = rZr$:

\paragraph{1. Transverse Invariance}
For any vector $X_\perp$ orthogonal to $\uu$, the action simplifies to identity:
\begin{equation}
  r X_\perp r \;=\; \left(\cos\tfrac{\vartheta}{2} + \uu\sin\tfrac{\vartheta}{2}\right) X_\perp \left(\cos\tfrac{\vartheta}{2} + \uu\sin\tfrac{\vartheta}{2}\right) \;=\; X_\perp.
\end{equation}
(Using the anticommutation $\uu X_\perp = - X_\perp \uu$).
Thus, \textbf{transverse lengths are invariant}, matching standard relativity.

\paragraph{2. Longitudinal Squeeze}
For the parallel component $X_\parallel$ (commuting with $\uu$):
\begin{equation}
  r X_\parallel r \;=\; X_\parallel r^2 \;=\; X_\parallel e^{\vartheta\uu} \;=\; X_\parallel (\cos\vartheta + \uu\sin\vartheta).
\end{equation}
Separating the result into vector and scalar parts:
\begin{align}
  \Im(\Sand{r}{X_\parallel}) \;&=\; X_\parallel \cos\vartheta, \\
  \Scal(\Sand{r}{X_\parallel}) \;&=\; -\|X_\parallel\| \sin\vartheta.
\end{align}
The vector part undergoes \textbf{isotropic scaling} by $\cos\vartheta$.

\paragraph{Summary of Spatial Action}
The projection of the D-rotation onto the spatial section $\ImH$ is:
\begin{equation}
  \boxed{\ \Im(rXr) \;=\; X_\perp \;+\; X_\parallel \cos\vartheta\ }.
\end{equation}
This algebraically realizes \textbf{Lorentz contraction}: the dimension along the motion is squeezed, while perpendicular dimensions are preserved.

\section{Correspondence to Physical Kinematics}
The algebraic squeeze factor is $k = \cos\vartheta$.
Using the tangent parametrization $\beta = \tan\vartheta$:
\begin{equation}
  k \;=\; \cos\vartheta \;=\; \frac{1}{\sqrt{1+\beta^2}}.
\end{equation}
While standard SR predicts $k_{SR} = \sqrt{1-\beta^2}$, the D-rotation provides the correct \emph{geometric form} (longitudinal contraction).
The exact Lorentz factor $\gamma_{SR}$ is recovered by considering the induced metric invariant $ds^2 \propto \cos 2\vartheta$, which implies a physical time dilation of $\sqrt{1-\tan^2\vartheta}$ relative to the Euclidean background.

\end{document}