\documentclass[11pt,a4paper]{article}
\usepackage[T1]{fontenc}
\usepackage[utf8]{inputenc}
\usepackage[english]{babel}
\usepackage{amsmath,amssymb,amsthm,mathtools}
\usepackage{bm}
\usepackage{hyperref}

\newcommand{\HH}{\mathbb{H}}
\newcommand{\ImH}{\mathrm{Im}\,\mathbb{H}}
\newcommand{\Scal}{\mathrm{Scal}}
\newcommand{\uu}{\mathbf{u}}        % unit imaginary axis
\newcommand{\Sand}[2]{#1\,#2\,#1}   % sandwich without inverse (D)
\newcommand{\Snd}[2]{#1\,#2\,#1^{-1}} % sandwich with inverse (ordinary)

\title{\textbf{Unimetry: Quaternion Phase Space}\\
{\large Extension to quaternions and two rotation types}}
\author{}
\date{}

\begin{document}
\maketitle

\begin{abstract}
We introduce a quaternion extension of the unimetry phase formalism. An object's proto-parameter is represented by a quaternion
\(\chi=\tilde S+\uu\,\tilde L\in\HH\) with \(\|\uu\|=1\). We distinguish two sandwich actions: (i) the \emph{ordinary} rotation \(X\mapsto qXq^{-1}\), which rotates vectors within the imaginary quaternion 3-space \(\ImH\); and (ii) the \emph{real--imaginary} D-rotation \(Z\mapsto rZr\) in the plane \(\mathrm{span}\{1,\uu\}\). We show that D-rotation provides a simple model for how nonzero ``speed'' reshapes \(\ImH\): the subspace orthogonal to \(\uu\) is isotropically squeezed by a factor \(1/\gamma\) while the pair \((1,\uu)\) undergoes a standard planar rotation by angle \(\alpha\) with \(\tan\alpha=\beta\).
\end{abstract}

\section{Quaternion phase space}
Let \(\uu\in\ImH\) be a unit imaginary quaternion (\(\uu^2=-1\)). Represent the state (proto-parameter) of an object as
\begin{equation}
  \chi \;=\; \tilde S + \uu\,\tilde L \;\in\; \HH,\qquad
  \tilde H := |\chi| = \sqrt{\tilde S^2+\tilde L^2}.
\end{equation}
Projectivizing forgets the overall scale \(\tilde H\): use homogeneous coordinates \([\tilde S:\tilde L]\in\mathbb{RP}^1\) and define
\(\theta=\arctan(\tilde L/\tilde S)\). Convenient kinematic parameters are
\begin{equation}
  \beta := \sin\theta,\qquad
  \gamma := \frac{1}{\cos\theta}.
\end{equation}
It is also useful to set \(\tan\alpha=\beta\), so that \(\cos\alpha=1/\gamma\) and \(\sin\alpha=\beta/\gamma\).

\section{Ordinary rotation in \texorpdfstring{\(\ImH\)}{Im H}}
Let \(q\in SU(2)=\{q\in\HH\mid |q|=1\}\), and \(X\in\ImH\).
The ordinary (3D) rotation acts by the \emph{inverse} sandwich:
\begin{equation}
  X \mapsto X' = \Snd{q}{X} = q\,X\,q^{-1},\qquad |X'|=|X|.
\end{equation}
Its (infinitesimal) generator is the imaginary quaternion \(\Omega_H=2\,q^{-1}\dot q\in\ImH\), yielding \(\dot X = \Omega_H\times X\).
We include this branch only for contrast with the D-rotation below.

\section{D-rotation in \texorpdfstring{\(\mathrm{span}\{1,\uu\}\)}{span\{1,u\}}}
Define the \emph{real--imaginary} D-rotation as
\begin{equation}
  Z \mapsto Z' = \Sand{r}{Z} = r\,Z\,r,\qquad
  r:=e^{\frac{\alpha}{2}\,\uu}=\cos\frac{\alpha}{2}+\uu\sin\frac{\alpha}{2},\quad |r|=1.
\end{equation}
Since \(|r|=1\), the norm is preserved for any \(Z\in\HH\).

\medskip
\noindent\textbf{Action on imaginary vectors.}
Take \(X\in\ImH\) and decompose \(X=X_\parallel+X_\perp\), where
\(X_\parallel=(X\!\cdot\!\uu)\,\uu\) and \(X_\perp\perp\uu\).
Then
\begin{equation}\label{eq:D-on-imag}
  \boxed{\ \Im(\Sand{r}{X}) \;=\; X_\parallel \;+\; \cos\alpha\,X_\perp\ },\qquad
  \Scal(\Sand{r}{X}) \;=\; -\,\sin\alpha\ (X\!\cdot\!\uu).
\end{equation}
\emph{Sketch.} Verify \eqref{eq:D-on-imag} on the basis
\(\{\,\uu,\ \mathbf e_1,\ \mathbf e_2\,\}\) with \(\mathbf e_{1,2}\perp\uu\),
using linearity and the closed-form identities for \(r\,\uu\,r\) and \(r\,\mathbf e_i\,r\).
\(\square\)

\medskip
\noindent Thus the pair \((1,\uu)\) undergoes an ordinary planar rotation by \(\alpha\) (as in the complex case),
while the entire orthogonal plane \(\uu^\perp\subset\ImH\) is \emph{isotropically scaled} by \(\cos\alpha\).

\section{Relating D-angle to ``speed''}
With
\begin{equation}
  \tan\alpha=\beta,\qquad
  \cos\alpha=\frac{1}{\gamma},\qquad
  \sin\alpha=\frac{\beta}{\gamma},
\end{equation}
formula \eqref{eq:D-on-imag} becomes
\begin{equation}\label{eq:gamma-squeeze}
  \Im(\Sand{r}{X}) \;=\; X_\parallel \;+\; \frac{1}{\gamma}\,X_\perp.
\end{equation}
Hence D-rotation \emph{squeezes} the imaginary quaternion 3-space in directions orthogonal to \(\uu\) by the factor \(1/\gamma\).
This provides a minimal geometric model of how a nonzero ``speed'' reshapes \(\ImH\).

\section{Evolution of \texorpdfstring{\(\chi\)}{chi} and projective coordinates}
For \(\chi=\tilde S+\uu\tilde L\),
\begin{equation}
  \chi'=\Sand{r}{\chi} \quad\Longleftrightarrow\quad
  \begin{pmatrix}\tilde S'\\ \tilde L'\end{pmatrix}
  =
  \begin{pmatrix}
    \cos\alpha & -\sin\alpha\\
    \sin\alpha & \ \cos\alpha
  \end{pmatrix}
  \begin{pmatrix}\tilde S\\ \tilde L\end{pmatrix},\qquad
  \tilde H'=|\chi'|=\tilde H.
\end{equation}
In projective coordinates \([\tilde S:\tilde L]\in\mathbb{RP}^1\), this is a rotation by \(\alpha\):
it rebalances the temporal and spatial shares at fixed \(\tilde H\).

\section{Minimal example (axis \texorpdfstring{\(z\)}{z})}
Let \(\uu=\mathbf e_z\), and \(X=(\sin\Theta\cos\Phi,\sin\Theta\sin\Phi,\cos\Theta)\).
From \eqref{eq:gamma-squeeze},
\begin{equation}
  X'=\Im(\Sand{r}{X})
  = \Big(\frac{1}{\gamma}\sin\Theta\cos\Phi,\ \frac{1}{\gamma}\sin\Theta\sin\Phi,\ \cos\Theta\Big),
  \quad\text{followed by normalization } \frac{X'}{\|X'\|}.
\end{equation}
Special cases: (i) \(X\parallel\uu\Rightarrow X'=X\);
(ii) \(X\perp\uu\Rightarrow X'=\frac{1}{\gamma}X\) (before normalization).
This is precisely the ``aberration-like'' geometric effect in \(\ImH\) induced by D-rotation.

\section{Remarks}
\begin{itemize}
  \item Both actions, \(\Snd{q}{X}\) and \(\Sand{r}{Z}\), are norm-preserving (since \(|q|=|r|=1\)), but they act on different structures:
  the former rotates within \(\ImH\), the latter rotates the mixed plane \(\mathrm{span}\{1,\uu\}\) and scales \(\uu^\perp\).
  \item Choosing \(\tan\alpha=\beta\) is convenient because \(\cos\alpha=1/\gamma\) appears directly as the squeeze factor of \(\uu^\perp\).
  \item If desired, energy/momentum calibrations can be attached via \(\tilde S=\tilde H\cos\theta\) and \(\tilde L=\tilde H\sin\theta\) without introducing additional objects.
\end{itemize}

\end{document}
